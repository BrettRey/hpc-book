% OPTION A: Cross-linguistic Quotative Emergence
% Draft case study for Chapter 7

\section{One case in depth: the emergence of quotatives}
\label{sec:7:one-case}

Rather than enumerate mechanisms abstractly, let's trace the full stabilising story for one category: the quotative marker.

Quotatives introduce reported speech, thought, gesture, or sound. Every language needs them. But when we look across typologically unrelated languages, we find a striking convergence: in the late twentieth century, similar forms emerged, spread through similar populations, and stabilized through similar mechanisms. Japanese \mention{tte}, Turkish \mention{diye}, German \mention{so}, English \mention{be like} -- different sources, different structures, but the same functional niche, the same stabilising dynamics.

This is not coincidence. It's convergent maintenance.

\subsection{The cluster}

What properties co-occur in these new quotatives?

They introduce material that goes beyond words: gesture, facial expression, tone of voice, inner monologue. \mention{And I was like} doesn't just report what someone said; it enacts the experience. The same is true of German \mention{und ich so} and Japanese \mention{tte}.

They attach preferentially to first-person subjects and present-tense frames. When speakers narrate, they use these forms to bring the audience into the moment of the original event -- the historical present, the vivid first person.

They are associated with youth. In every documented case, younger speakers lead adoption. Young women, in particular, tend to be innovative adopters.

They are discourse-conditioned: common in storytelling, informal speech, peer conversation; rare in formal registers.

These properties cluster. A quotative that introduces vivid re-enactment, prefers first person and present tense, and spreads through young speakers in informal contexts -- this is a recognisable type, a cross-linguistically recurring profile.

The question is: why does this cluster cluster?

\subsection{Stabilisers}

\textbf{Processing economy.} Quotatives of this type are structurally light. Japanese \mention{tte} is a phonologically reduced form of \mention{to itte} (`saying that'); the reduction removes syllables while preserving function. German \mention{so} is a single syllable. English \mention{be like} is syntactically simple: the copula plus a predicative adjective, no complementizer, no overt speech verb. These forms are easy to produce in real-time narrative, where processing pressure is high. Processing economy is a stabiliser: forms that reduce production cost are used more, entrenched more deeply, and survive transmission.

\textbf{Expressive fit.} These quotatives do something that older forms don't: they introduce inner monologue, gesture, and attitude without committing the speaker to an exact utterance. \mention{She said} implies verbatim report; \mention{she was like} implies approximation. This looseness is functionally useful. Speakers need to convey the gist, the stance, the affect -- not the transcript. The expressive fit between form and function is a stabiliser: forms that serve communicative needs better than alternatives tend to spread.

\textbf{Acquisition pathways.} Children and adolescents acquire these forms in dense peer networks. The input is frequent, contextually salient, socially marked. Once acquired, the forms become entrenched as default options for a particular discourse function. The acquisition pathway is concentrated in a developmental window where peer influence is maximal and identity formation is active. This is not Chomskyan critical period; it's socially modulated sensitivity. The result is a cohort effect: each generation's linguistic inventory reflects what was frequent in their adolescent input.

\textbf{Social indexing.} New quotatives index youth, informality, in-group membership. Using \mention{be like} signals that you are a certain kind of speaker, in a certain kind of register, with a certain kind of audience. This social value is not epiphenomenal; it's a stabiliser. Speakers maintain the form partly because it does social work. As \textcite{Eckert2012} argues, social meaning is part of the mechanism.

\textbf{Transmission and spread.} Young speakers acquire the form; they use it into adulthood; their children hear it. The form spreads vertically (parent to child) and horizontally (peer to peer). The apparent-time trajectory -- younger speakers use it more -- becomes a real-time change within a generation. Forms that suit the demographic and interactional structure of transmission survive; forms that don't, die.

\subsection{Why the same profile across languages?}

This is the most telling fact. Japanese, Turkish, German, and English are not typologically close. They have different word-order properties, different morphological profiles, different sociolinguistic ecologies. Yet they converge on similar quotative innovations.

The convergence is not lexical; it's structural. The stabilising mechanisms are the same:

\begin{itemize}
   \item Processing pressure favours light forms.
   \item Discourse needs favour vague-reference quotatives.
   \item Social dynamics favour youth-indexed forms.
   \item Transmission dynamics favour high-frequency, contextually salient forms.
\end{itemize}

Wherever these mechanisms operate -- and they operate everywhere humans tell stories to each other -- the same type of quotative emerges. The category is not defined by a shared etymon or a universal grammar rule. It's stabilized by a convergent mechanism profile.

This is what `same category across languages' means in the maintenance view. Not shared essence. Not definitional identity. Convergent stabilisation.

\subsection{The wider ecology: colloquialisation and register shift}

There's a tempting further explanation. If similar quotatives emerged in multiple unrelated languages at roughly the same historical moment -- the late twentieth century -- perhaps something in the wider environment shifted. Perhaps a global trend toward informalization opened ecological space for these forms to flourish.

The hypothesis deserves scrutiny.

\textbf{Evidence for colloquialisation.} Corpus-based studies of English document a measurable drift. \textcite{biber1989} identified a shift in written registers toward oral styles: more contractions, more first-person pronouns, more active constructions. \textcite{leech2009} confirmed the pattern across the twentieth century, finding that colloquial features increased in fiction, journalism, and letters while academic prose remained relatively stable. The trend is attributed to external social factors -- marketization, media ecology, reduced formality in public discourse -- rather than internal linguistic pressure.

The timing fits. English \mention{be like} was first documented in the early 1980s; German \mention{so} emerged in the 1990s; both spread during the same decades when written and broadcast registers were absorbing more colloquial features. If the threshold for hearing and using informal speech lowered, forms that were once restricted to peer storytelling could leak into wider circulation. The quotative's stabilisers -- processing economy, expressive fit, youth indexing -- were always present. What changed, on this account, was the environment in which they operated.

\textbf{Counter-evidence and limits.} But the colloquialisation story oversimplifies if generalised incautiously.

First, it's not universal. The corpus evidence comes overwhelmingly from English, and similar trends are documented for some closely related varieties (Dutch, German, Swedish). But Japanese and Korean have seen \emph{increased} honorific elaboration in some workplace contexts, not decreased. Arabic and Turkish retain robust formal/informal distinctions with little evidence of erosion. The claim that languages are globally becoming less formal is not well-supported beyond the Anglophone and Northern European sphere.

Second, formality persists even where colloquialisation occurs. Academic prose, legal documents, medical communication, and official correspondence remain stubbornly formal. What Biber and Gray (2016) found was not that formality disappeared, but that the \emph{gap} between formal and informal registers widened: informal registers got more informal, while academic English actually increased in nominal complexity. Colloquialisation is register-specific, not language-wide.

Third, the quotative case doesn't \emph{require} the colloquialisation hypothesis. The mechanisms enumerated above -- processing economy, expressive fit, acquisition dynamics, social indexing, transmission -- are sufficient to explain convergence. These mechanisms operate wherever humans tell personal stories in peer contexts. They don't need a global social trend to do their work; they need only the stable features of narrative interaction that have existed for millennia.

\textbf{Selection environment, not mechanism.} The most defensible framing is this: colloquialisation is a \emph{shift in the selection environment}, not a distinct stabilising mechanism.

In evolutionary biology, climate change is not a mechanism of evolution -- mutation, selection, and drift are mechanisms. But climate change alters the selection pressures that determine which variants succeed. Likewise, colloquialisation is not itself a mechanism that maintains the quotative cluster. Rather, it changes the conditions under which the mechanisms operate: which registers are widely heard, which indexical values carry social weight, how quickly innovations spread from youth cohorts to the wider population.

If colloquialisation increased the visibility of informal registers, quotatives that were already stabilized in those registers gained access to new transmission pathways. The mechanisms did the stabilising work; the ecological shift expanded the territory into which the stabilized form could spread.

This framing makes a prediction: in environments where colloquialisation hasn't occurred (or has reversed), the same quotative mechanisms should still operate -- but in a narrower ecological niche. And this is what we observe. German \mention{so} is more register-restricted than English \mention{be like}, consistent with a more modest colloquialisation trend in German public discourse. Turkish \mention{diye} is well-established but hasn't displaced formal quotative constructions in contexts where formality is maintained. The mechanisms are the same; the scope of their product differs with the ecology.

\textbf{What colloquialisation doesn't explain.} This perspective also clarifies what colloquialisation \emph{can't} explain. It can't explain why \emph{these particular forms} won out -- why \mention{be like} and not some other light quotative. That requires the specific mechanisms: the processing advantage, the expressive fit, the social indexing, the transmission dynamics.

And it can't explain the \emph{internal structure} of the quotative category -- the cluster of properties that co-occur. Colloquialisation doesn't predict that successful quotatives will favour first-person subjects, tolerate vague reference, and resist past-tense framing. The mechanisms predict this. The ecology only determines how far the category spreads.

So: colloquialisation is real, where it occurs. It modulates the reach of stabilized forms. But it is not a named mechanism in the quotative story. It is the weather in which the mechanisms operate.

\subsection{Variation and activation}

Not all quotative tokens are equally stable. English \mention{be like} is deep in its basin for speakers under 50; for speakers over 70, \mention{say} may still dominate. This is not two categories; it's the same functional category in different activation states, conditioned by generational input.

Within a single speaker, \mention{be like} is activated in storytelling and suppressed in formal report. The form is there; the activation depends on discourse environment.

German \mention{so} is more restricted than English \mention{be like} -- still marked as adolescent, still register-bound. This is a shallower basin: fewer stabilisers, more sensitivity to environmental signal.

The activation-state picture predicts: forms in shallow basins should show higher inter-speaker variation, more register sensitivity, and faster historical flux. \mention{So} does; \mention{be like}, now entrenched across generations, doesn't.

\subsection{What if a mechanism were absent?}

If processing economy were the only stabiliser, we'd expect the shortest form to win regardless of discourse function. But \mention{say} is shorter than \mention{be like}, and it doesn't dominate.

If expressive fit were the only stabiliser, we'd expect any form that introduces vivid quotation to spread equally. But quotatives with pejorative associations, or unfashionable indexical value, don't spread even when they fit the discourse need.

If social indexing were the only stabiliser, we'd expect pure fashion effects: quotatives rising and falling with generational taste. But \mention{be like} has been stable for three decades, long past a typical fashion cycle.

If transmission were the only stabiliser, we'd expect all forms heard in childhood to survive. But archaic quotatives like \mention{quoth} or reduced variants that never gained social cachet don't persist.

The observed pattern -- cross-linguistic convergence on light, expressive, youth-indexed, narratively deployed quotatives -- requires the full braid. No single mechanism is sufficient.

