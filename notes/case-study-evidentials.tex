% OPTION B: Evidentials in Japanese vs Turkish
% Draft case study for Chapter 7

\section{One case in depth: evidentials in Turkish and Japanese}
\label{sec:7:one-case}

Rather than enumerate mechanisms abstractly, let's trace the full stabilising story for one category: evidentiality.

Evidentials encode where information comes from. Did the speaker witness the event directly? Infer it from effects? Hear about it from someone else? Many languages grammaticalise these distinctions; many others mark them only lexically. The same semantic space, different stabilisation paths.

Turkish and Japanese both need to mark information source. Both have forms that do this work. But Turkish has a fully grammaticalised evidential suffix; Japanese has a constellation of lexical and semi-grammatical forms. Why the difference?

The answer lies in the stabilising mechanisms.

\subsection{The cluster: Turkish}

Turkish marks indirect evidentiality with the suffix \mention{-(I)mIş}. This form signals that the speaker didn't directly witness what they're reporting -- they inferred it from traces, heard about it from others, or encountered it unexpectedly.

What properties cluster?

\mention{-(I)mIş} is a bound morpheme, obligatorily attached to the verb. It occupies a fixed slot in the verbal template. It contrasts paradigmatically with \mention{-DI}, the direct-experience past. The choice is forced: if you're using a past-tense verb, you mark whether your evidence is direct or indirect.

The form covers multiple evidence types: inference from results, hearsay, surprise (mirativity). These aren't distinguished morphologically; context disambiguates.

The category is acquired early. Turkish children use \mention{-(I)mIş} productively by age three. The input is frequent, obligatory, paradigmatically contrastive.

The form is prosodically integrated. It doesn't stand alone; it's part of the verbal complex.

\subsection{The cluster: Japanese}

Japanese marks similar evidential distinctions, but the forms are different.

\mention{Rashii} indicates something is probably true based on external indications -- report, inference, general knowledge. \mention{Sōda} (in one of its two forms) indicates hearsay: `I hear that...'. \mention{Yōda} and \mention{mitai da} indicate appearance-based inference: `it looks like...'.

What properties cluster?

These forms are syntactically semi-independent. They attach at the clause level, not inside the verbal morphology. They are not paradigmatically obligatory: a speaker can describe a past event without marking evidence source at all.

The forms have partially overlapping distributions. Where Turkish forces a choice, Japanese allows optionality and fine-grained selection among alternatives.

These forms retain traces of their lexical sources. \mention{Rashii} etymologically means `like'; \mention{sōda} comes from `looking thus' or `saying thus'. They are grammaticalised, but not fully bleached.

The category is acquired later. Japanese children master these forms somewhat later than Turkish children master \mention{-(I)mIş} -- the optionality and lexical residue make the target distribution harder to induce from input.

\subsection{Stabilisers: why the difference?}

Both languages need evidentiality. Both have forms that do the work. But the stabilisation paths differ, and the mechanisms explain the difference.

\textbf{Frequency and obligatoriness.} Turkish evidentiality is obligatory in certain tense contexts. Every utterance about a past event triggers a choice. This high frequency stabilises the form as a grammatical suffix: deeply entrenched, early acquired, paradigmatically integrated. Japanese evidentiality is optional. High-frequency use exists, but the optionality reduces per-utterance exposure and delays full entrenchment.

\textbf{Paradigmatic pressure.} Turkish \mention{-(I)mIş} contrasts directly with \mention{-DI}. The contrast is minimal, salient, and distributionally complementary. This paradigmatic structure scaffolds acquisition: learners receive clear evidence that the two forms exclude each other. Japanese evidentials compete with each other and with zero-marking. The paradigmatic structure is looser; the scaffolding is weaker.

\textbf{Prosodic integration.} Turkish evidential marking is prosodically bound. It's not a separate word; it's part of the verb. This integration promotes automatisation and retrieval as a unit. Japanese evidentials are prosodically more independent, which preserves their lexical flavour and slows the progression toward full grammaticalisation.

\textbf{Functional differentiation.} Japanese evidentials differentiate more finely: appearance-based vs report-based vs general-knowledge-based inference. This differentiation is useful communicatively but requires more input to acquire. Turkish bundles these under a single form, with context handling disambiguation. The bundling simplifies acquisition; the differentiation enriches expression.

\textbf{Community transmission norms.} Standard Turkish promotes \mention{-(I)mIş} usage; dialectal variation exists but the grammatical category is stable across registers. Japanese politeness norms interact with evidentiality: these forms are associated with less assertive speech, with hedging, with deference. The social function adds a stabilising layer but also complicates the acquisition target.

\subsection{Same semantic space, different stabilisation regimes}

This is the key insight. Turkish and Japanese occupy the same functional territory: marking information source. But they've built different category architectures.

Turkish has a single, deeply grammaticalised category: the indirect evidential. The stabilisers -- high frequency, obligatory paradigm, prosodic integration, early acquisition -- all reinforce grammaticalisation.

Japanese has a cluster of semi-grammaticalised forms. The stabilisers -- optionality, lexical residue, functional differentiation, politeness-linked social indexing -- maintain a looser, more articulated system.

Both are stable. Both are real kinds in their respective languages. But they illustrate that `the same category' cross-linguistically means convergence of function, not identity of architecture. The mechanism profile determines the shape of the category, not the semantic need alone.

\subsection{Variation and activation}

Within Turkish, \mention{-(I)mIş} varies in its mirative vs reportative flavour depending on context. The morpheme is the same; the activation differs. This is variation within a deeply stable category.

Within Japanese, speakers select among \mention{rashii}, \mention{sōda}, \mention{yōda}, and others depending on evidence type, formality, and epistemic commitment. The variation is wider because the category architecture is looser. More forms, more sensitivity to environmental signal.

The prediction: categories with tighter mechanism braids should show less intra-category variation. Turkish evidentiality is tighter; Japanese evidentiality is looser. The prediction holds.

\subsection{What if a mechanism were absent?}

If frequency were the only stabiliser, we'd expect high-frequency optional forms to grammaticalise at the same rate as obligatory ones. They don't: obligatoriness accelerates entrenchment.

If paradigmatic contrast were the only stabiliser, any two-way contrast would produce tight grammaticalisation. But loose paradigms with optional competitors -- like Japanese -- produce semi-grammaticalised clusters.

If prosodic integration were the only stabiliser, clitics would always become affixes. But many languages maintain clitics for centuries without further fusion, because other stabilisers don't push in that direction.

If functional need were the only stabiliser, all languages would grammaticalise evidentiality to the same degree. They don't: the mechanism profile varies, and so does the category architecture.

The comparison between Turkish and Japanese shows that the same semantic territory can be stabilised differently. The difference is not in what speakers need to express. It's in how the stabilising mechanisms interact.

