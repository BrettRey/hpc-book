\chapter*{Preface}

This book offers a new way of thinking about linguistic categories~-- what they are, how they persist, and why they have the structure they do. The central claim is that categories like \textsc{noun}, \textsc{definite}, and \textsc{countable} aren't defined by essential properties but maintained by mechanisms: frequency effects, functional pressures, transmission dynamics, and cognitive anchoring. Categories are real, but their reality is dynamic rather than definitional.

\section*{Framework and scope}

For English examples, I work within the framework of \textit{The Cambridge Grammar of the English Language} \citep{huddleston2002}~-- its categories, its terminology, and its analyses. This isn't a defense of \textit{CGEL} over competing grammars; it's a consistent baseline that allows claims to be tested against a well-articulated descriptive system. When I say that \mention{otherwise} straddles categories, I mean \textit{CGEL}'s categories. Readers who prefer a different framework can substitute accordingly; the theoretical argument doesn't depend on any particular grammatical analysis.

The theoretical claims, though, are intended to generalize beyond English. The mechanisms that maintain categories~-- entrenchment, transmission, functional anchoring~-- aren't language-specific. Where I draw on cross-linguistic data (from Welsh, Spanish, Korean, and others), I rely on the sources cited rather than claiming expertise in those languages.

\section*{What this book is not}

This isn't a grammar. It doesn't aim to describe English (or any other language) exhaustively. It isn't a typology, and it makes no claims about how many languages exhibit particular properties. It isn't a defense of any grammatical framework~-- generative, functional, cognitive, or otherwise.

What it is, instead, is an argument about \emph{what kind of thing} a linguistic category has to be for the evidence we have to make sense. The book sits at the intersection of linguistics, philosophy of science, and cognitive science. It asks what categories are made of, and it answers: mechanisms, not essences.

\section*{Audience}

I've tried to write for linguists who are curious about the metaphysics underlying their practice, and for philosophers who are curious about how the abstract questions play out in a concrete domain. Some background in linguistics will help~-- familiarity with basic grammatical terminology, with the idea that categories can be disputed, with the general shape of debates about universals and variation. But I've aimed to make the argument accessible to anyone willing to follow the examples carefully.

\section*{Conventions}

Linguistic examples are italicized; so are mentions of words and expressions~-- cases where the form itself is under discussion: \mention{cattle}, \mention{the}. Category labels and technical terms appear in small caps: \textsc{noun}, \textsc{verb}, \term{projectibility}. The glossary at the back collects key terms with brief definitions.

Unacceptable examples are marked with an asterisk (*); marginal or disputed examples with a question mark (?). Page references to \textit{CGEL} are given where relevant.


\bigskip

\noindent\textit{Toronto, January 2026}
