\chapter*{Typographical Conventions}
\addcontentsline{toc}{chapter}{Typographical Conventions}
\markboth{Typographical Conventions}{}

This book uses the following typographical conventions:

\subsection*{Linguistic examples}

\begin{itemize}
\item \textit{Italics} mark linguistic forms under discussion: \textit{the}, \textit{furniture}, \textit{who}.
\item \textsc{Small capitals} introduce technical terms at their point of definition: \textsc{designatum}, \textsc{homeostatic property cluster}.
\item \textbf{*} marks ungrammatical strings: *\textit{She have left}.
\item \textsuperscript{?} marks marginally acceptable or degraded strings: \textsuperscript{?}\textit{The team has scored on themselves}.
\item \textsuperscript{??} marks severely degraded strings.
\item \# marks semantically anomalous strings (pragmatically odd but not ungrammatical): \#\textit{Colourless green ideas sleep furiously}.
\end{itemize}

\subsection*{Quotation marks}

\begin{itemize}
\item Double quotes (\enquote{like this}) enclose direct quotations and reported speech.
\item Single quotes (`like this') enclose meanings or glosses: Spanish \textit{patatas} `potatoes'.
\item Scare quotes are used sparingly, to signal that a term is being used with reservations.
\end{itemize}

\subsection*{Examples}

Numbered examples follow standard linguistic convention. Sub-examples are lettered (a, b, c). Interlinear glosses follow the Leipzig Glossing Rules.

\subsection*{Bibliography}

Citations follow APA style. In-text citations use the author-date format: \textcite{huddleston2002} or (\cite{huddleston2002}). Full references appear in the References section at the end of the book.
