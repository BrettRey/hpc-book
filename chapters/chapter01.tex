% Chapter 1: Words That Won't Hold Still
% 
% SLOGAN: This chapter foreshadows "the maintenance view" in the impasse section.
% See notes/book-slogans.md for the book-wide slogan strategy.

\chapter{Words That Won't Hold Still}
\label{ch:words-wont-hold-still}

\label{sec:1:opening}
On March 2, 2008, at 3:13 in the morning, I got an email from Rodney Huddleston about a word he didn't understand.

The time stamp is misleading~-- Huddleston was writing from Australia, where it was a reasonable hour. But there's something fitting about the image of a linguist awake in the dark, wrestling with a single word. The word was \mention{otherwise}. I had asked him whether there were grounds for treating it as a preposition rather than an adverb. His reply:

\begin{quote}
I'm not proud of the adverb analysis, or confident about it, and don't intend it to cover all uses. Its classification is quite a puzzle. Dictionaries have it as adverb and adj (\mention{the truth is quite otherwise}) and some also as conjunction. There is something to be said for a prep analysis, which might cover adjunct and predicative complement uses. But I don't know how to handle \mention{this suggests otherwise} or \mention{the correctness or otherwise of the proposal}.
\end{quote}

\ixnq{Huddleston, Rodney}Huddleston was not a careless analyst. He was the lead author of \textit{The Cambridge Grammar of the English Language} \citep{huddleston2002}~-- 1,860 pages, seventeen years in the making, the most comprehensive descriptive grammar of English ever produced. If anyone knew what \ixl{otherwise} was, it should have been him.

But he didn't. Not because he had missed something. Because the word wouldn't hold still.

\bigskip

This book is about what that puzzle reveals.

The natural reading is that \mention{otherwise} is simply a hard case~-- an outlier, a word with an unusual history, a problem for specialists. Every grammar has its edge cases. You note them, flag the uncertainty, and move on.

But \mention{otherwise} is not alone. Each of the following behaves as if it were built from a different blueprint than the one our categories assume:

\ixl{Cattle} takes plural agreement (\mention{the cattle are grazing}) but has no singular. You can say \mention{many cattle} or \mention{three cattle}, but not \mention{a cattle} or \mention{one cattle}. Yet it readily functions as a modifier in compound nouns like \mention{a cattle ranch}, a position usually reserved for singulars (\mention{a toothbrush}, not \mention{*a teethbrush}). The usual singular--plural paradigm simply doesn't apply. And it has been this way for centuries~-- not drifting toward regularity, not acquiring a singular, just sitting there being strange.

\mention{The} marks definiteness~-- or so the textbooks say. But \mention{go to the hospital} needn't pick out a particular hospital. \mention{The tiger is endangered} doesn't refer to any individual tiger. These aren't rare or marginal uses; they're ordinary English, and they're well-known problem cases precisely because the standard definition of definiteness~-- that the referent is identifiable to the hearer~-- fails to cover them. Grammars describe these uses; they don't explain how they fit the category.

Reciprocals~-- \mention{each other} and \mention{one another}~-- sit at the boundary between pronouns and compound determinatives. Morphologically, they pattern with items like \mention{somebody} and \mention{anybody}: fused determiner--head structures that realize multiple grammatical functions in a single form. Syntactically and semantically, they behave like pronouns: they function as objects (\mention{they saw each other}), resist subject position, and refer to participants in a reciprocal relation. The tension is systematic. Their morphology pulls them toward the compound determinatives; their distribution and meaning pull them toward pronouns. \textit{The Cambridge Grammar} classifies them as pronouns, and that remains the standard analysis. But the fit is imperfect in predictable ways, and the imperfection reflects genuine mechanism competition rather than analytic failure \citep{reynolds2025reciprocals}.

By then I had already \mention{thought myself through that problem}~-- attested, interpretable, built by analogy to \mention{talk oneself through} and \mention{work oneself through}. But \mention{think} is intransitive; you think about a problem, you don't think it. Here the construction has coerced an intransitive verb into a frame that treats it as capable of moving an entity along a path. Speakers will disagree about whether the verb belongs there at all.

The boundary between /\ipa{ɪ}/ and /\ipa{ɛ}/ is categorical enough to distinguish \mention{pin} from \mention{pen} for most English speakers, but in parts of the American South, in many Australian and New Zealand varieties, the distinction collapses before nasals. The phoneme boundary is real enough to anchor lexical contrasts, variable enough to dissolve under specific phonological conditions.

Something similar holds at the social end of language. A form like \mention{gonna} can be ordinary in casual speech and jarring in academic prose. In languages with honorific systems, this isn't just a matter of style: the choice between plain and polite forms is part of what speakers have to track, and what hearers can reliably infer from.

These are not obscure examples chosen to embarrass a theory. They are common words, central constructions, patterns that learners have to acquire and speakers have to process every day. The categories we use to describe them~-- noun, article, pronoun, adverb~-- are the basic machinery of grammatical analysis. And the machinery keeps encountering parts it wasn't built for.

\bigskip

Two broad responses recur in the literature. Each captures something important; neither succeeds on its own.

The first says: categories are defined by necessary and sufficient conditions. A noun is whatever satisfies the necessary and sufficient conditions for nounhood; an adverb is whatever satisfies the conditions for adverbhood. Membership is binary. Boundaries are sharp. If a word doesn't fit, either the criteria are wrong or the word is exceptional. Refine the criteria, explain away the exceptions, and the system will be clean.

This is the \ixsq{essentialism}essentialist view~-- or at least the textbook version of it, sometimes called \enquote{classical} or \enquote{Aristotelian} (whether or not Aristotle himself held quite this position). It has the virtue of clarity: either something is an X or it isn't. That clarity has built the infrastructure of modern linguistics~-- great descriptive grammars, parsers, textbooks, annotation schemes all rest on the assumption that categories have definitions. The recurring difficulty is that the definitions keep encountering exceptions. Every set of criteria produces counterexamples. The counterexamples get handled by stipulation, by subclasses, by \enquote{special} readings that multiply until the exceptions rival the rules. Huddleston's email is the essentialist view confronting its limits: here is a word, here are the criteria, and no combination of criteria delivers a stable answer.

The second response says: categories are not defined by conditions at all. They are \ixs{prototype theory}prototypes~-- clusters of typical features, with central members and peripheral members and fuzzy boundaries all the way down. A robin is a better example of \textsc{bird} than a penguin is, but both are birds. \mention{Run} is a better example of \textsc{verb} than \mention{beware} is, but both are verbs. Stop expecting sharp edges. Gradience is the nature of the beast.

This view also captures something real~-- the empirical fact of gradience, the persistent failure of neat definitions. But as it is usually deployed in linguistics, it purchases descriptive adequacy at the cost of explanation. If \mention{cattle} is a \enquote{less central} noun, why does its non-centrality take exactly the form it takes~-- no singular, plural agreement, full compatibility with numerals above one? Why has it been stable for five hundred years instead of drifting toward the core or out of the category entirely? Why don't categories dissolve into chaos if their boundaries are genuinely fuzzy? Prototype descriptions record the gradience. They don't, by themselves, explain why the gradient structure holds still.

\bigskip

The essentialist locates boundaries in definitions that don't exist. The prototype theorist denies that boundaries are sharp at all. Both share an assumption so deep it's almost invisible: that these are the only options. Either categories have essences, or they're looser groupings~-- useful for description, perhaps, but not the kind of thing that could bear explanatory weight.

A clarification before we proceed. The problem with essentialism is not that it posits sharp boundaries and binary membership~-- this book will defend both, produced by mechanisms rather than definitions. Likewise, the virtue of prototype theory is not that it discovers genuine fuzziness in category structure~-- this book will argue that apparent fuzziness is epistemic rather than ontological. The virtue is that prototype theory took gradience seriously as data. The impasse we're diagnosing is not a disagreement about sharpness. It's a shared failure to ask what maintains the structure.

There is a third possibility.

What if categories are real, stable, and explanatorily powerful~-- but not because they have essences? What if their boundaries are sharp in structure but gradient in access~-- sharp enough to support induction, yet appearing fuzzy because we can't locate them precisely? What if the stability and the apparent fuzziness are both consequences of something else, something neither tradition has squarely addressed?

What if the mechanisms that usage-based linguists, variationists, and typologists already study are not merely causes of category behaviour but constitutive of category reality itself? What if \enquote{refactoring} that research around an explicit ontology of mechanism-maintained kinds would resolve puzzles that each tradition, working alone, has left open?

\bigskip

Huddleston's email sits in my files, seventeen years old now. \enquote{Its classification is quite a puzzle.} The puzzle was never just \mention{otherwise}. It was what kind of thing a grammatical category has to be for that sentence to be exactly the right thing to say.


\section{Essentialism examined}
\label{sec:1:essentialism}

\citet{chomsky1965}, in \textit{Aspects of the Theory of Syntax}, crystallized a picture many linguists had already been working with: a grammar as a finite stock of sharply bounded categories, each lexical item assigned to exactly one major class (or handled via homonymy), with syntactic theory tasked with discovering the conditions that fix those assignments once and for all. This picture brought order: it made syntactic theory look like a matter of discovery rather than stipulation, and for decades that's how linguistics proceeded.

This assumption~-- that each category has an \emph{essence}, a set of properties necessary and sufficient for membership~-- embodies what philosophers call \term{essentialism}, and it's been linguistics' default mode for millennia. On this view, \textsc{noun} and \textsc{verb} are fixed and universal because they track real essences, not because linguists have stipulated them. Essentialism is the natural view. It's where you start if no one has told you there's a problem.

The intuition runs deep. When a child learns that \mention{dog} picks out the class of dogs, we're tempted to think that there's some inner dogness~-- some cluster of necessary properties~-- that all and only dogs share. Borderline cases exist, of course, but we think of them as fringe matters. The real work of the concept is done by its core, and the core is fixed by an essence. The same picture quietly informs how linguists talk about \mention{phonemes}, about \mention{nouns} and \mention{adjectives}, about \mention{definiteness}, about \mention{subject} and \mention{object}. There are different kinds of sounds, different kinds of words, different kinds of grammatical functions, and the grammar cares about the difference.

The philosophical roots go back to Aristotle: a category is defined by properties that are individually necessary and jointly sufficient. No borderline triangles, no gradient membership. That framework remained stable across two millennia~-- from Dionysius Thrax's eight parts of speech through \ipa{Pāṇini}'s Sanskrit categories to Bloomfield's structural phonemes. When generative grammar arrived, it inherited rather than invented the assumption that each grammatical category is defined by an essence.

The structuralist inheritance brought this view into modern linguistics. Bloomfield's phoneme was a \enquote{minimum unit of distinctive sound-feature}~-- discrete, bounded, essential \citep[79]{bloomfield1933}. He treated morphemes and word classes the same way. Linguistic analysis became a matter of identifying and taxonomizing essential units.

Applied to language, this yields a picture that most working linguists absorb without being taught it explicitly. A noun (in English, at least) is whatever satisfies the criteria for nounhood~-- it inflects for number, it heads noun phrases, it can be modified by adjectives. A verb is whatever satisfies the criteria for verbhood~-- it inflects for tense, it takes arguments, it heads verb phrases. The criteria might be debated; the assumption that there \emph{are} criteria, that categories have definitions, is rarely questioned. Even linguists who reject essentialism in principle often write as if it were true, because the alternative is hard to operationalize.

The essentialist picture has genuine explanatory power. The great descriptive grammars~-- \citet{jespersen1924}, \citet{quirk1985}, \textit{The Cambridge Grammar} itself \citep[ch.~1]{huddleston2002}~-- proceed category by category, laying out the membership criteria, cataloguing the members, noting the exceptions. The exceptions are always there, but they're handled as exceptions: marginal cases, historical residue, items in transition. The core of each category is secure. Textbooks are organized around this architecture. Pedagogical grammars depend on it. Parsers are built on it. The infrastructure of linguistic analysis presupposes that categories have boundaries and that the boundaries can, in principle, be found.

The strongest modern version of essentialism grounds categories in substantive theoretical claims about what the categories \emph{are}. \citet{baker2003} provides an explicit defense of this position. Baker's goal is to redeem what he calls the \enquote{long-standing promissory note} of the feature system [+N] and [\ipa{−}V]~-- that is, to provide substantive syntactic definitions of the major lexical categories that hold universally \citep[17]{baker2003}. Nouns, on his account, bear referential indices; verbs license specifiers; adjectives predicate properties. These are not tendencies or prototypes but defining properties, and they hold because of what nouns, verbs, and adjectives essentially are. The features don't just describe the categories; they flow from the categories' natures.

The same essentialist logic extended to semantics. Jerrold Katz, working in the 1960s and 70s, defended the classical theory of lexical concepts: each word's meaning is defined by a set of necessary and sufficient semantic features \citep{katz1972}. The standard example was \mention{bachelor}, analyzed as [+adult] [+male] [+unmarried]. Anything satisfying those features is a bachelor; anything lacking one isn't. Katz treated such decompositions as cognitively real~-- part of what speakers know when they know a word's meaning. The category \textsc{bachelor} has an essence: unmarried adult maleness. The definition carves the concept at its joints.

But much work in formal linguistics uses feature bundles without Baker's or Katz's deeper commitments. Categories are specified as [+N, \ipa{−}V] or [\ipa{−}N, +V], but the features are treated as primitives~-- tools for stating generalizations, not reflections of underlying essences. On this view, the feature bundle \emph{is} the category; there's no further question about why these features cluster. Whether this counts as essentialism is less clear. The boundaries may be sharp, but the explanation stops at the features themselves. I'll return to this distinction in the next section, where it turns out to matter.

\bigskip

The problem with essentialism isn't that it's wrong about everything. It's that it keeps encountering cases where the machinery doesn't fit~-- and the response is always the same. Add a stipulation. Create a subclass. Mark the item as exceptional. The exceptions accumulate, and at some point you have to ask whether the machinery is doing explanatory work or merely recording the failures of a prior commitment.

Return to \mention{otherwise}. Huddleston's puzzlement wasn't a lapse but a report from the front lines. The word looks like an adverb (\mention{think otherwise}), looks like an adjective (\mention{the truth is quite otherwise}), and appears in constructions that resist any standard label (\mention{the correctness or otherwise of the proposal}). What are the necessary and sufficient conditions for adverbhood that include \mention{otherwise} in \mention{think otherwise} and exclude it in \mention{the correctness or otherwise}? There aren't any. Every definition either lets in too much or leaves out too much.

The essentialist response is predictable: \mention{otherwise} is polysemous, multiply listed, historically complex. Each use gets its own entry, its own feature specification, its own subcategorization frame. The grammar doesn't fail; it proliferates. But this is bookkeeping, not explanation. We wanted to know what makes something an adverb. We got a list of contexts where \mention{otherwise} behaves adverbially, and another list where it doesn't, and no principled account of why the word straddles the boundary instead of sitting on one side or the other.

Consider the broader pattern. In English, the major categories~-- noun, verb, adjective, adverb, preposition~-- are supposed to be definable by distributional and morphological criteria. But every category has items that meet some criteria and fail others:

\ixl{fun} looks like a noun (\mention{we had fun}) but increasingly takes degree modification like an adjective (\mention{that was very fun}). The change is ongoing, and speakers differ. What is \mention{fun}~-- noun, adjective, or in transit between them? The essentialist has to say it's one or the other, perhaps with a secondary listing for the emerging use. But the very fact of the apparent transition shows that the word is moving through a region that essentialism's static picture can't accommodate~-- a space between basins where competing mechanisms pull in different directions.

\mention{Near} takes objects without a preceding \mention{to} (\mention{near the house}), which makes it look like a preposition. But it also has comparative and superlative forms (\mention{nearer}, \mention{nearest}), which makes it look like an adjective. It can function as a predicative complement after \mention{become}~-- but only in its affectionate sense (\mention{become near and dear to someone}), not its locative sense (\mention{*become near the door}). Some grammars call it a preposition; some call it an adjective; some, including \textit{CGEL}, call it both~-- a word that belongs to two categories. But if categories are defined by necessary and sufficient conditions, how can a single word satisfy two incompatible definitions (unless distinct homonyms are posited)? The answer is that the definitions weren't necessary and sufficient after all. They were family portraits, not passport photos~-- and \mention{near} has the look of both families.

Compare \mention{pease}, which until Early Modern English was a mass noun like \mention{cattle}. Speakers reanalyzed the final \mention{-se} as plural \mention{-s}, back-formed a singular \mention{pea}, and the word regularized. No functional anchor blocked the analogy. \mention{Cattle} has \mention{cow}; \mention{pease} had nothing. The essentialist account says both words should have stable essential properties. But one regularized and one didn't, and the difference isn't in their essences~-- it's in whether another word was occupying the functional slot that analogy would have filled.

\bigskip

The deeper problem is that the failures aren't random. They cluster in predictable places.

The intersection zones between major categories are chronic trouble spots. The preposition--adverb boundary, the complement--modifier boundary, the agent--experiencer boundary~-- these are not marginal curiosities. They're systematic sites of instability. And the words that straddle these boundaries tend to be high-frequency items with broad distributions: \mention{otherwise}, \mention{fun}, \mention{near}. Frequency isn't a feature; distribution is a consequence of how a word behaves, not a defining criterion. These are precisely the patterns a theory of categories has to explain. These are precisely the zones where, on the view I'll defend, the forces maintaining the cluster are weakest or in competition.

Items undergoing historical change present a special problem. If categories have essences, then change should be abrupt: a language either has subject–auxiliary inversion or it doesn't. But English inversion changed gradually. \mention{Go you to church?} and \mention{I go not} gave way to \mention{Do you go to church?} and \mention{I don't go}~-- but not all at once. For centuries, both patterns coexisted, distributed by register, construction type, and verb class. Shakespeare inverts lexical verbs in questions; a generation later, the pattern is receding. Essentialist grammars have to treat each stage as a discrete system, missing the continuity that drove the change. The alternative~-- acknowledging that syntactic properties can be variably present across a grammar~-- is precisely what essentialism forbids.

None of this means essentialism is useless. For most words, most of the time, the category labels work. \mention{Dog} is a noun; \mention{run} is a verb; \mention{quickly} is an adverb. The labels support generalization. They organize grammars. They enable pedagogy. If every word were like \mention{otherwise}, linguistic analysis would be impossible.

But neither is every word like \mention{dog}. And the question is what to make of the words that aren't. The essentialist answer~-- treat them as exceptions, lexical idiosyncrasies, marginal cases~-- works as long as the exceptions stay marginal. When the exceptions multiply, when the boundaries fray systematically, when the core cases start looking like special instances of a more general pattern of variation, then the answer stops working. The framework designed to handle the clear cases has nothing to say about the unclear ones, except that they're problems to be solved later, with better definitions.

This is where essentialism stands. It handles the core. It fumbles the periphery. And it has no account of why the periphery exists at all~-- why categories should \emph{appear} to have fuzzy edges when nothing in the definition explains the gradient structure.

\bigskip
\noindent The examples so far have focused on word-level categories~-- nouns, verbs, prepositions~-- because they're familiar and the puzzles are readily visible. But the challenge generalizes across linguistic structure. Phonemes, construction types, inflectional classes, thematic roles, speech acts, information-structure categories: all present the same problem of clustering with variation, stability with gradience, projectible patterns that resist sharp definitions. This book examines linguistic categories across all levels of analysis.\footnote{I use \term{category} in the philosopher's sense: a causally sustained kind whose boundaries and internal structure are maintained by mechanisms. This flattens \textit{CGEL}'s distinctions among categories (noun, NP), systems (tense, number), and functions (subject, modifier). Those distinctions are analytically indispensable, but from the vantage point of asking what holds a kind together, all three pose the same question. Fellow \textit{CGEL} partisans: forgive the flattening.} The framework developed here applies equally to a phonological segment, a syntactic phrase type, or a discourse marker.

\subsection{Essentialism beyond formalism}

The impulse isn't confined to formalist or generative traditions. Even frameworks that foreground gradience and usage often smuggle essences back in. Cognitive grammarians like \citet{langacker1987foundations} reject autonomous syntax and embrace prototypes but treat the noun/verb opposition as a universal, notional distinction grounded in essential cognitive abilities for conceptualizing things versus processes. Functional grammarians in Dik's tradition \citep{dik1997} allow that not every language has a \mention{subject}, but hold that once a language does, the function comes with universal properties and obeys cross-linguistic hierarchies. Role and Reference Grammar, explicitly typological and semantically driven, posits exactly two macroroles~-- Actor and Undergoer~-- as a \enquote{fundamental opposition} underlying clause structure everywhere. The details differ, but the metaphysical shape is familiar: a small, privileged stock of basic categories whose natures are fixed in advance and which languages are assumed to instantiate.

Essentialism survives across theoretical divides. It continues to guide practice even where explicit talk of necessary and sufficient conditions has faded. It's still the default assumption: that categories, if they're to be real and explanatory, have to have sharp boundaries and that fuzziness, where it appears, is a defect in our descriptions.

But the pragmatic compromise runs deeper than most practitioners acknowledge. Working linguists proceed \emph{as if} categories have sharp boundaries even when they privately doubt that the boundaries are real. This isn't intellectual dishonesty; it's a rational response to the demands of description and analysis. Grammars need to be written. Students need to be taught. Parsers need to be built. All of these tasks become vastly simpler if you can assume that \mention{fun} is either a noun or an adjective, that \mention{near} is either a preposition or an adjective, that \mention{otherwise} has a stable category assignment. The alternative~-- treating every item as occupying a unique position in continuous category space~-- would make grammatical description intractable.

The compromise shows up in how grammarians talk about their own methods. They'll note that a particular classification is \enquote{not entirely satisfactory} or that \enquote{the boundaries are somewhat fluid}, then proceed to use the classification anyway because the architecture of the grammar requires it. They'll acknowledge gradient membership in one paragraph and then, two pages later, write rules that presuppose binary distinctions. Far from being sloppy, this practice recognizes that some idealizations are necessary for the work to proceed, even when the idealizations are known to be false.

Some grammarians make this tension explicit.\footnote{For example, Geoffrey K.\ Pullum (personal communication) writes: \enquote{I am well aware that I am (for pragmatic reasons) presupposing an essentialist mindset that assumes `Categories exist; items either belong or don't; our job is to discover the boundaries', and that such essentialism might well be wrong. I have thought that for quite a long time. The main pragmatic reason for maintaining the implausible essentialism in question is just that it's useful to be able to talk to linguists in a language they understand.}} They distinguish the metaphysical implausibility of sharp essences from the practical convenience of talking as if such essences existed. Essentialism becomes a shared fiction~-- useful for communication, indispensable for pedagogy, but not to be confused with a claim about what grammatical categories actually are.

The question this raises is whether there's an alternative. Can we build a picture of categories that accommodates the gradience, the flux, the systematic misbehavior of high-frequency items and boundary cases~-- but still manages to be tractable, explanatory, and usable in grammatical description? Or are we stuck with the choice between an essentialist picture we know to be false and a gradient picture too unwieldy to operationalize?

\section{Prototype theory examined}
\label{sec:1:prototype}

The prototype tradition began as a rebellion against definitions. Psychologists documented the gradient structure that classical categories were not supposed to have; linguists seized on their results as licence to relax sharp boundaries; and cognitive scientists built a research programme around the claim that fuzziness was not a defect in our descriptions but a fact about how concepts work.

In the early 1970s, the psychologist Eleanor Rosch ran a series of experiments that should have been impossible on the classical view \citep{rosch1975}. She asked subjects to rate how good an example each item was of everyday categories: \textsc{bird}, \textsc{fruit}, \textsc{furniture}, \textsc{vehicle}. If categories were defined by necessary and sufficient conditions, the question would be meaningless. A robin either is or isn't a bird; there's no sense in which it could be a better bird than a penguin.

But subjects had no trouble answering. They rated robins as better birds than penguins, apples as better fruit than olives, chairs as better furniture than rugs. The ratings were consistent across subjects and stable across time. People had robust intuitions about category structure that the classical theory said they shouldn't have.

Rosch's interpretation: categories aren't defined by conditions. They're organized around \term{prototypes}, central members that best exemplify the category. Other members are included by similarity to the prototype, with membership grading off toward the periphery. A robin is a better bird than a penguin because a robin is closer to the prototype; a penguin is still a bird, but a marginal one. The boundaries aren't sharp lines but fuzzy gradients.

The idea spread fast. By the 1980s, prototype theory had become a major framework in cognitive psychology, and linguists were importing it wholesale. George Lakoff's \textit{Women, Fire, and Dangerous Things} \citep{lakoff1987} argued that prototype structure was fundamental to human cognition, not a quirk of folk categories but the way concepts work. John Taylor's \textit{Linguistic Categorization} \citep{taylor2003} applied the framework systematically to grammatical categories. The cognitive linguistics movement, associated with Lakoff, Ronald Langacker, and others, built an entire research programme around the idea that linguistic categories are prototype-structured, gradient, and grounded in embodied experience.

The appeal was obvious. Here, finally, was a framework that took the evidence seriously. The messy cases that essentialism swept under the rug (the \mention{fun}s and \mention{near}s and \mention{otherwise}s) became data rather than noise. If category membership is gradient, then words can be more or less noun-like, more or less verbal, more or less adverbial. The boundaries don't have to be sharp because the theory doesn't predict sharp boundaries. The evidence that embarrassed essentialism confirmed prototype theory.

\subsection{Feature bundles: essentialism in disguise}

While cognitive psychologists were developing prototype theory, formal linguists were arriving at a similar place by a different route: encoding categories as feature matrices rather than distances from a prototype, treating the features as discrete rather than graded, but running into the same problem when items turned out to have some features of a category without having all of them.

The feature-bundle approach I flagged in the previous section~-- categories as specifications like [+N, \ipa{−}V]~-- looks essentialist on the surface. The features are discrete. Category membership appears binary: either an item has the feature or it doesn't. But once you allow that items can have \emph{some} features of a category without having all of them, gradience enters through the back door. A word that is [+N] for agreement but [\ipa{−}N] for pluralization is partly noun-like. The prototype theorist would say it's a marginal noun, distant from the prototype. The feature theorist would say it has an atypical feature specification. These are notational variants of the same observation.

The deeper convergence is methodological. Prototype theory says: categories are organized around central members, and membership is determined by similarity to those members. Feature-bundle approaches say: categories are defined by feature matrices, and membership is determined by how many features an item has. But where do the prototypes come from? And where do the features come from?

Neither framework has a principled answer. Prototypes are typically identified by introspection or experiment~-- these items \emph{feel} central. Features are typically posited to capture distributional patterns~-- items that pattern together get assigned shared features. Each approach takes its primitives as given.

What's more, the two enterprises inform each other in something like the reflective equilibrium that Nelson Goodman described for rules and judgments \citep{Goodman1955}. Features tell you what dimensions similarity should be computed over: if [±plural] is a feature, then similarity with respect to number inflection matters. Typicality judgments tell you which features are doing work: if speakers consistently rate \mention{dog} as a better noun than \mention{fun}, that's evidence that whatever features \mention{dog} has and \mention{fun} lacks are central to the category. Neither set of judgments is foundational. The features aren't derived from first principles; they're abduced from patterns in typicality and distribution. The typicality judgments aren't brute psychological facts; they're shaped by implicit feature analysis.

This is where I must locate my own past practice. For most of my career, I operated comfortably in the feature-bundle tradition.\footnote{See, for example, \citet{reynolds2021}, where I encode determinatives and pronouns as binary feature matrices and use clustering algorithms to validate \textit{CGEL}'s category assignments. The methodology treats property clustering as evidence for category validity without asking what maintains the clustering~-- exactly the explanatory gap that HPC addresses.} I was not committed to Platonic essences, but I took \textit{CGEL}'s categories to be real structural kinds that could be characterized by clusters of discrete properties, and I assumed that the clustering itself required no further explanation. If pressed, I would have acknowledged gradience; I would have granted that \mention{fun} was a marginal noun (or a marginal adjective). But I wouldn't have asked \emph{why} that particular configuration of properties, \emph{why} stable for decades, \emph{why} not drifting toward the core or out of the category entirely. The clustering was a fact to be recorded, not a phenomenon to be explained.

Prototype theory and feature-bundle approaches turn out to be two faces of a single tradition~-- one psychological, one formal, but sharing a crucial limitation. Both describe the gradience. Neither explains it.

\subsection{Prototypes in the grammar}

Consider parts of speech. The essentialist says: a word is a noun if and only if it has the necessary and sufficient properties that define nounhood. The prototype theorist says: \mention{nounhood} is a cluster of properties (reference to things, inflection for number, occurrence in certain syntactic frames) and words exhibit these properties to varying degrees. A word like \mention{dog} exhibits all of them (in its nominal use); it's a prototypical noun. A word like \mention{fun} exhibits some but not others; it's a marginal noun, or a noun-becoming-adjective, or a word in the gradient space between categories. The labels become approximations rather than verdicts.

Consider syntactic constructions. The essentialist says: a sentence is grammatical or it isn't. The prototype theorist says: constructions have central instances and extended instances, with acceptability shading off toward the margins. \mention{The dog bit the man} is a prototypical transitive; \mention{The bed slept two people} is an extended transitive, coerced by analogy to the prototype; \mention{The stone kicked the ball} is further out still, requiring more contextual support to be acceptable. Grammaticality becomes a gradient, not a boundary.

Consider meaning. The essentialist says a word has a definition~-- a set of conditions that determine what it applies to. The prototype theorist says word meanings are organised around exemplars, with extension to new cases governed by similarity rather than rule. \mention{Bird} doesn't mean \enquote{feathered bipedal vertebrate capable of flight}; it means something like \enquote{thing similar to robins, sparrows, eagles}~-- and penguins count because they're similar enough, while bats don't, despite meeting some featural criteria.

This framework captured something real. The gradient judgments Rosch documented are genuine: speakers do have intuitions about better and worse examples, and categories do have internal structure. The essentialist picture~-- clean definitions as the ground for sharp boundaries and binary membership~-- failed in its foundational claim, and prototype theory said so clearly.

\subsection{But there's a cost}

Neither approach tells you why the boundaries are \emph{there} rather than \emph{elsewhere}. Prototype theory tells you that categories have fuzzy boundaries; feature-bundle approaches tell you which properties cluster. Neither tells you why the periphery has the structure it has: why \mention{fun} is becoming adjective-like in one specific way and not another, why \mention{cattle} has exactly the peculiar profile it has, why \mention{otherwise} straddles exactly those categories and not others.

Return to \mention{cattle}. The prototype theorist can say: \mention{cattle} is a less prototypical noun than \mention{dog}. It lacks the singular–plural contrast that prototypical nouns have. It takes plural agreement but resists \mention{a} and \mention{one}. Fine, this describes its distance from the prototype. But if category membership is just similarity to a prototype~-- or just partial match to a feature bundle~-- then nothing in that story explains why exactly this configuration appears rather than countless other logically possible ones. A word could be noun-like in its agreement but verb-like in its argument structure and adjective-like in its modification patterns. The space of possible non-prototypical members is vast. But actual non-prototypical members cluster in predictable ways. \mention{Cattle} isn't random noise around the noun prototype. It's a stable configuration that has persisted for centuries.

The stability problem is general. If categories are gradient similarity structures, they should drift. Each generation learns from slightly different data. Each speaker has slightly different exemplars in memory. Small perturbations should accumulate. Over time, categories should dissolve into chaos, or at least into unrecognizable configurations. But they don't. The periphery of \textsc{noun} in English today looks broadly like the periphery of \textsc{noun} a hundred years ago. Marginal members stay marginal. Central members stay central. The gradient structure holds still.

Why?

\bigskip

Prototype theory, as it's usually deployed in linguistics, has no answer. Taylor's \textit{Linguistic Categorization} \citep{taylor2003}, perhaps the most influential application of the framework to grammar, describes prototype effects in detail but remains agnostic about what maintains them. It registers that categories have centres and edges. It doesn't say what keeps the edges from eroding.

\subsection{An analogy}

Imagine mapping the distribution of a species across a landscape. You find that the species is densest in certain habitats and thins out toward others. You could describe this as a prototype structure: prototypical members live in the core habitat; peripheral members live at the margins. The description would be accurate.

But it would miss the explanation. The species is distributed that way because of ecological mechanisms: resource availability, predation pressure, climate tolerance. The distribution isn't a brute fact about similarity to a prototype. It's the outcome of processes that concentrate the population in some places and thin it in others.

Prototype theory and feature-bundle approaches, applied to language, are like the distributional description without the ecology. They capture the shape. They miss the dynamics.

\subsection{The deeper issue}

Prototype theory, despite its rebellion against definitions, inherits a key assumption from the tradition it rejected: that categories are primarily synchronic structures.

The essentialist asks what a category \emph{is}. The prototype theorist asks what it \emph{looks like}. Neither asks what \emph{keeps it that way}. Both treat the category as something you can describe by examining its current state. Neither asks what maintains the category, what keeps it from collapsing, what generates its particular structure rather than some other structure.

This is a question about mechanisms. Not \enquote{what do the members of this category have in common?} but \enquote{what processes cause these properties to cluster and stay clustered?} Not \enquote{where is the boundary?} but \enquote{why is there a boundary there at all, and why does it persist?}

A strand of cognitive linguistics~-- associated with Bybee, Goldberg, and others~-- goes further than classical prototype theory. These researchers don't merely accept gradience; they study the mechanisms that produce it: frequency effects, entrenchment, analogical extension, constructional schemas. This usage-based tradition has many of the pieces this book will draw on. What it typically lacks is the explicit ontological claim that mechanisms are not merely \emph{causes} of category structure but \emph{constitutive} of category reality. That claim~-- that the mechanisms \emph{are} the category, not merely forces acting on it~-- is what the maintenance view adds. Variationists like Labov and Tagliamonte, typologists like Haspelmath and Croft, grammaticalisation theorists tracking the pathways by which forms change~-- all have been studying stabilising dynamics, often without foregrounding what that work implies about what categories fundamentally are.

Prototype theory and feature-bundle approaches were right that classical essentialism failed. They were right that categories exhibit gradient \emph{judgments}, apparent fuzziness at boundaries, better and worse members. They didn't go far enough: documenting the gradience is not the same as understanding it.

\section{The impasse}
\label{sec:1:impasse}

So we have two traditions, and neither works.

Essentialism offers a clear ontology~-- categories are discrete, membership is binary, definitions can in principle be found~-- and this clarity has built the infrastructure of modern linguistics. Grammars, parsers, textbooks, annotation schemes: all of it rests on the assumption that \mention{fun} is either a noun or an adjective (or perhaps a verb), that \mention{otherwise} belongs somewhere. The definitional grounding is false, but the discreteness is load-bearing. Take it away and the architecture sags.

Prototype theory offers an accurate phenomenology~-- categories \emph{appear} gradient, boundaries \emph{seem} fuzzy, typicality is real~-- and this accuracy has licensed a generation of work on the messiness that essentialism couldn't see. But accuracy about appearances is not explanation. To say that \mention{cattle} is a non-prototypical noun is to redescribe the problem, not to solve it. The gradient structure is stable. Prototype theory records the stability. It doesn't say what produces it.

We are left oscillating. In one mode, we write as if categories were crystalline: discrete, bounded, the kind of thing that could appear in a rule. In another mode, we acknowledge the gradience, note the exceptions, and quietly set them aside. The two modes don't communicate. The crystalline picture is useful but misgrounded. The gradient picture is phenomenologically apt but explanatorily inert.

\bigskip

Return one last time to Huddleston's email. \enquote{I don't know how to handle} certain uses of \mention{otherwise}.

The essentialist hears this as a problem to be solved: find better criteria, and the word will sort itself out. The prototype theorist hears it as a fact to be accepted: \mention{otherwise} is marginal, gradient, a fuzzy case in a fuzzy system.

Neither hears what Huddleston actually wrote.

He didn't say the criteria were unclear. He didn't say the word was marginal. He said he didn't know how to \emph{handle} it~-- how to think about a word that behaves one way in one construction and another way in another, that passes some tests and fails others, that has sat at the intersection of categories for centuries without drifting to either side. The question isn't which box the word belongs in. The question is why it won't stay in any box at all~-- and why that instability is itself stable.

That question is the one neither tradition can ask. Essentialism can't ask it because it assumes the boxes are real and the task is to find them. Prototype theory can't ask it because it assumes the gradience is primitive and there's nothing further to explain. Both traditions take the shape of the categories as given. Neither asks what gives them that shape.

This is the impasse. Not a choice between two theories, but a shared blind spot. Both traditions~-- at least in their textbook forms~-- take the shape of the categories as given. Usage-based and variationist work has begun to ask what maintains the clustering, but typically without foregrounding the ontological stakes: what it would mean for categories to \emph{be} the mechanisms that maintain them, rather than entities that mechanisms act upon.

\bigskip

A parallel from physics: for decades, foundational questions about quantum theory~-- what measurement means, how entanglement works, whether hidden variables exist~-- were dismissed as philosophy, not physics. Researchers who pursued them faced career obstacles; they were warned away from the questions that seemed merely conceptual. But the foundational work produced some of the most practically useful results in modern physics: Bell's inequality, the no-cloning theorem, teleportation protocols. Tools now used in quantum computing and cryptography began as questions about what quantum theory \emph{means}.

We don't know yet what the Bell inequality of linguistics will be. But the physics example suggests that foundational clarity can have unexpected payoffs. The question \enquote{what are linguistic categories, actually?} may look like philosophy~-- and in a sense it is. But so was the question that produced Bell's theorem. Getting the foundations right tends to matter eventually.

\bigskip

The next two chapters trace the impasse's consequences.

Chapter~\ref{ch:essentialism} examines why essentialism's fixes keep failing~-- why better definitions don't solve the problem, why the boundary cases aren't going away, why the search for necessary and sufficient conditions blocks the questions that would actually help.

Chapter~\ref{ch:what-we-havent-been-asking} asks what falls out of view when we try to escape. Some traditions have retreated to nominalism, treating categories as convenient fictions with no cross-linguistic reality. Others~-- usage-based linguistics, construction grammar, sociolinguistics, functionalism~-- have moved toward gradience, frequency, variation, and communicative pressure, glimpsing the mechanisms without quite making them central. These traditions have the right instincts. But none has asked what holds the clusters together, or why certain configurations persist, or what kind of thing a grammatical category has to be for \mention{otherwise} to sit at an intersection for three centuries without budging. And none has noticed that the hardest questions aren't about words at all.

Only then will we be in a position to ask what Huddleston's email has been waiting seventeen years to hear.

The chapters that follow don't reject prior work on language variation and change. They reorganise it. The variationist insight that patterns spread through populations, the cognitive linguist's insight that frequency entrains form, the typologist's insight that similar pressures produce similar categories~-- all of these are data for the maintenance view, not competitors to it. What changes is the framing: these are not independent explanations but braided strands of a single stabilising story.
