% Chapter 1: Words That Won't Hold Still

\chapter{Words That Won't Hold Still}

On March 2, 2008, at 3:13 in the morning, Rodney Huddleston sent me an email about a word he didn't understand.

The time stamp is misleading~-- Huddleston was writing from Australia, where it was a reasonable hour. But there's something fitting about the image of a linguist awake in the dark, wrestling with a single word. The word was \term{otherwise}. I had asked him whether there were grounds for treating it as a preposition rather than an adverb. His reply:

\begin{quote}
I'm not proud of the adverb analysis, or confident about it, and don't intend it to cover all uses. Its classification is quite a puzzle. Dictionaries have it as adverb and adj (\term{the truth is quite otherwise}) and some also as conjunction. There is something to be said for a prep analysis, which might cover adjunct and predicative complement uses. But I don't know how to handle \term{this suggests otherwise} or \term{the correctness or otherwise of the proposal}.
\end{quote}

Huddleston was not a careless analyst. He was the lead author of \textit{The Cambridge Grammar of the English Language}~-- 1,860 pages, seventeen years in the making, the most comprehensive descriptive grammar of English ever produced. If anyone knew what \term{otherwise} was, it should have been him.

But he didn't. Not because he had missed something. Because the word wouldn't hold still.

\section*{~}

This book is about what that puzzle reveals.

The natural reading is that \term{otherwise} is simply a hard case~-- an outlier, a word with an unusual history, a problem for specialists. Every grammar has its edge cases. You note them, flag the uncertainty, and move on.

But \term{otherwise} is not alone. Each of the following behaves as if it were built from a different blueprint than the one our categories assume:

\term{Cattle} takes plural agreement (\term{the cattle are grazing}) but has no singular. You can say \term{many cattle} or \term{three cattle}, but not \term{a cattle} or \term{one cattle}. The usual singular--plural paradigm simply doesn't apply. And it has been this way for centuries~-- not drifting toward regularity, not acquiring a singular, just sitting there being strange.

\term{The} marks definiteness~-- or so the textbooks say. But \term{go to the hospital} doesn't pick out a particular hospital. \term{The tiger is endangered} doesn't refer to any individual tiger. These aren't rare or marginal uses; they're ordinary English, and they're well-known problem cases precisely because the standard definition of definiteness~-- that the referent is identifiable to the hearer~-- fails to cover them. Grammars describe these uses; they don't explain how they fit the category.

Reciprocals~-- \term{each other} and \term{one another}~-- pass some pronoun diagnostics and fail others. They function as objects (\term{they saw each other}), but their possessive forms are restricted: \term{each other's} exists but is limited in distribution; \term{one another's} is rarer still. Are they pronouns? Something else? \textit{The Cambridge Grammar} classifies them as pronouns, but the fit is imperfect, and the imperfection is systematic.

These are not obscure examples chosen to embarrass a theory. They are common words, central constructions, patterns that learners must acquire and speakers must process every day. The categories we use to describe them~-- noun, article, pronoun, adverb~-- are the basic machinery of grammatical analysis. And the machinery keeps encountering parts it wasn't built for.

\section*{~}

Two broad responses recur in the literature, and both fail.

The first says: categories are defined by necessary and sufficient conditions. A noun is whatever satisfies the necessary and sufficient conditions for nounhood; an adverb is whatever satisfies the conditions for adverbhood. Membership is binary. Boundaries are sharp. If a word doesn't fit, either the criteria are wrong or the word is exceptional. Refine the criteria, explain away the exceptions, and the system will be clean.

This is the essentialist view. It has the virtue of clarity: either something is an X or it isn't. It has the vice of never quite working. Every set of criteria produces counterexamples. The counterexamples get handled by stipulation, by subclasses, by \enquote{special} readings that multiply until the exceptions rival the rules. Huddleston's email is the essentialist view confronting its limits: here is a word, here are the criteria, and no combination of criteria delivers a stable answer.

The second response says: categories are not defined by conditions at all. They are prototypes~-- clusters of typical features, with central members and peripheral members and fuzzy boundaries all the way down. A robin is a better example of \term{bird} than a penguin is, but both are birds. \term{Run} is a better example of \term{verb} than \term{beware} is, but both are verbs. Stop expecting sharp edges. Gradience is the nature of the beast.

This is the prototype view. It captures something real~-- the empirical fact of gradience, the persistent failure of neat definitions. But as it is usually deployed in linguistics, it purchases descriptive adequacy at the cost of explanation. If \term{cattle} is a \enquote{less central} noun, why does its non-centrality take exactly the form it takes~-- no singular, plural agreement, full compatibility with numerals above one? Why has it been stable for five hundred years instead of drifting toward the core or out of the category entirely? Why don't categories dissolve into chaos if their boundaries are genuinely fuzzy? Prototype descriptions record the gradience. They don't, by themselves, explain why the gradience holds still.

\section*{~}

The essentialist sees sharp boundaries that don't exist. The prototype theorist sees fuzzy boundaries and stops there. Both share an assumption so deep it's almost invisible: that these are the only options. Either categories have essences, or they are looser groupings~-- useful for description, perhaps, but not the kind of thing that could bear explanatory weight.

There is a third possibility.

What if categories are real, stable, and explanatorily powerful~-- but not because they have essences? What if their boundaries are genuinely fuzzy~-- but not because they are arbitrary? What if the stability and the fuzziness are both consequences of something else, something neither tradition has squarely addressed?

That something is \textit{mechanism}: the causal processes that hold category properties together. The forces that keep the clustering clustered. This idea will need to be made precise~-- \enquote{mechanism} can't be a black box with explanatory magic attributed to it. Specifying the mechanisms for grammatical categories is where the real work lies, and the next chapter begins that work. But the core insight can be stated now: categories hold together not because they have essences but because something is actively holding them together.

The framework has a name~-- homeostatic property cluster kinds~-- and a history. It was developed in philosophy of biology, where it offered a way through the species problem: the long failure to define \term{species} by necessary and sufficient conditions. A species, on this view, is not a type fixed by morphology or genetics or interbreeding capacity. It is a population whose properties cluster because gene flow, development, ecology, and selection maintain the clustering. The extension to grammar is not a loose analogy. Grammatical categories are populations too~-- distributed across speakers, transmitted across generations, stabilized by processes that can be identified and tested. A grammatical category is a linguistic population whose properties cluster because acquisition, entrenchment, interactive alignment, and iterated transmission maintain the clustering over time.

But before the solution can do its work, the problem needs to be felt more precisely. The next sections sharpen the critique: what essentialism actually claims and where exactly it breaks; what prototype theory offers and what it leaves unexplained; what it would take for a view of categories to do better. Only then will the third option look like what it is~-- not a compromise, but a different kind of answer to a different kind of question.

Huddleston's email sits in my files, seventeen years old now. \textit{Its classification is quite a puzzle.} The puzzle was never just \term{otherwise}. It was what kind of thing a grammatical category must be for that sentence to be exactly the right thing to say.
