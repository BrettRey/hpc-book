\chapter{Kinds without essences}
\label{ch:kinds-without-essences}

% This chapter introduces the homeostatic property cluster (HPC) framework.
% It is the first chapter of Part II: The Fix.

Part I raised a question: if categories aren't defined by essences, what makes them real? The answer this chapter develops is simple: categories are real because they are maintained. This is the \term{maintenance view}.

The idea comes from philosophy of biology. Species aren't defined by genetic essences~-- no checklist of necessary and sufficient conditions separates one species from another. But species are real. They support induction, figure in explanations, persist across time. What makes them real is not a shared essence but a shared causal history: mechanisms of reproduction, selection, and development that keep certain properties clustering together.

Richard Boyd called these \term{homeostatic property cluster kinds}~-- or HPC kinds \citep{boyd1991,boyd1999}. The name is a mouthful, but the idea is intuitive. A property cluster is homeostatic when mechanisms tend to restore it: disturb the cluster and it returns to form. Species are like this. Perturb the genome and selection pushes back. Isolate a population and reproductive barriers emerge. The clustering isn't accidental. It's maintained.

% TODO: Develop the parallel to grammatical categories
% - Grammatical categories as HPC kinds
% - What "mechanism" means in this context
% - How this resolves the essentialism/nominalism impasse

[Chapter content to be developed]

