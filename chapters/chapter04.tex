% SLOGAN: This chapter formally introduces "the maintenance view."
% Sticky sentence: "Categories are real because they are maintained."
% See notes/book-slogans.md for the book-wide slogan strategy.

\chapter{Kinds without essences}
\label{ch:kinds-without-essences}

\epigraph{You should consider that the essential art of civilization is maintenance.}{Pete Seeger, quoted in \citet{brand2024}}

Part I raised a question: if categories aren't defined by essences, what makes them real? The answer this chapter develops is simple: categories are real because they are maintained. This is the \term{maintenance view}.

The word \enquote{maintenance} is chosen deliberately. Stewart Brand's recent work on infrastructure, buildings, and software makes a point that applies equally to grammar: maintenance isn't just preventing breakdown; it's the whole process of keeping something going \citep{brand2024}. Monitoring, repair, renewal, adaptation~-- these are what keep bridges standing and cities functioning. Brand's insight is that we systematically undervalue maintenance because its successes are invisible: a well-maintained system just \emph{works}, and we notice only when it fails.

Grammatical categories are like this. When they function smoothly~-- when speakers agree on what counts as a noun, when learners acquire the same categories their parents have~-- the maintenance is invisible. We notice the mechanisms only at the boundaries, where the system creaks: the words that won't classify, the constructions that shift between generations, the gradience that essentialism can't explain. Brand's framing helps: think of the mechanisms not as defining the categories but as maintaining them. The categories are real because the maintenance is continuous.

The idea of maintenance as constitutive~-- not just preservative~-- comes from philosophy of biology. Species aren't defined by genetic essences~-- no checklist of necessary and sufficient conditions separates one species from another. But species are real. They support induction, figure in explanations, persist across time. What makes them real is not a shared essence but a shared causal history: mechanisms of reproduction, selection, and development that keep certain properties clustering together.

Richard Boyd called these \term{homeostatic property cluster kinds}~-- or HPC kinds \citep{boyd1991,boyd1999}. The name is technical but its parts are revealing. \emph{Stasis}: standing, position~-- the cluster \emph{stays} in place, maintained by ongoing processes. \emph{Homeo}: same, similar~-- when disturbed, the cluster tends to return to the \emph{same} configuration, not just any stable state. A homeostatic system doesn't merely persist; it self-corrects. Perturb a species' genome and selection pushes back toward the original distribution. Isolate a population and reproductive barriers emerge that restore the clustering. The name captures both the \emph{staying} and the \emph{sameness}~-- stability is not just inertia but active return.

Think of a spinning top. It stays upright not because it's rigid but because it's moving. The spin resists perturbation~-- push it slightly and gyroscopic forces bring it back. Stop the spin and the top falls. Homeostatic kinds are like this: their stability is dynamic, not static. What keeps the cluster clustered is not a fixed structure but an ongoing process.

And the maintenance needn't be entirely external. Some of the stability arises from reciprocal reinforcement among the clustered properties themselves. Recall \emph{Smilodon} and \emph{Thylacosmilus} from Chapter~\ref{ch:essentialism} (Figure~\ref{fig:convergent-morphology}): their massive canines, powerful forelimbs, solitary hunting, and apex-predator niche aren't accidentally co-present. They are causally interlocking~-- each makes the others more likely, more stable, more resistant to drift. Large canines require the musculature to wield them; ambush predation requires the solitude to stalk; apex status requires the whole package. The cluster is a self-stabilising network, embedded in broader developmental, ecological, and transmission dynamics. This is the reflexive dimension of homeostasis: the properties, by virtue of being those properties, participate in maintaining the cluster.

% TODO: Develop the parallel to grammatical categories
% - Grammatical categories as HPC kinds
% - What "mechanism" means in this context
% - How this resolves the essentialism/nominalism impasse

[Chapter content to be developed]

