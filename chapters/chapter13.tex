\chapter{The category zipper}
\label{ch:the-category-zipper}

\epigraph{If you have two complementary strands of DNA, they zip up. That's what they do.}{— Sri Kosuri, quoted in \textit{Harvard Gazette} (2019)}

%~--  ~--  ~--  ~--  ~--  ~--  ~--  ~--  ~--  ~--  ~--  ~--  ~--  ~--  ~--  ~--  ~-- 
\section{When errors go wrong differently}
\label{sec:13:hook}
%~--  ~--  ~--  ~--  ~--  ~--  ~--  ~--  ~--  ~--  ~--  ~--  ~--  ~--  ~--  ~--  ~-- 

You know the moment a zipper catches. The slider finds the teeth, they interlock, and the two sides move as one. You don't think about it. That's the point~-- when the coupling is tight, the mechanism disappears.

You also know the moment it doesn't catch. The slider moves but the teeth won't seat. Or it catches partway and then jams, leaving a gap where the fabric bulges through. The fix depends on where the failure is: at the bottom, you have to reseat the whole thing; in the middle, you can sometimes work it free; at the top, you're just short of done and the last few teeth won't close.

Language comprehension has all of these failure modes.

When you mishear a word, the error is a clean \term{slip}. You hear \mention{grape} as \mention{great}, not as a smear of sound. The perceptual system delivers discrete candidates even when the input is degraded. The mistake stays inside the system: one phoneme for another, one word for another. The zipper caught; it just caught on the wrong tooth.

When you misparse a sentence, the failure is a \term{jam}. You can get every word right and still get the structure wrong. \mention{The horse raced past the barn fell} is acoustically clear; the problem is that your parser zipped up the wrong configuration and only discovered the problem at \mention{fell}. The teeth were seating fine until they weren't.

And when you miss an implicature, the zipper looks closed but the fabric is \term{mis-tracked} underneath. \mention{Some students passed} registers as good news when the speaker meant it as bad. Every tooth is in place. The tracking is off.

\subsection{From failure modes to coupling regimes}

These three failure types diagnose three coupling regimes~-- and each regime licenses different diagnostic evidence.

\begin{enumerate}
    \item \textbf{Slip} (clean substitution) signals \term{transparent coupling}. Form and value are so tightly bound that errors stay within the contrast system. Perturbations shift category boundaries predictably where cues degrade; they don't produce extragrammatical gibberish. \emph{Diagnostic evidence}: perceptual confusion matrices cluster by featural similarity; mergers occur where acoustic cues weaken.

    \item \textbf{Jam} (late crash after locally coherent assembly) signals \term{architectural coupling}. Form templates pair with interpretive templates, but hidden commitments accumulate as parsing proceeds. The zipper catches partway, then fails when downstream requirements conflict. \emph{Diagnostic evidence}: garden-path signatures, revision points, cue-competition effects that show up in reading times or eye movements.

    \item \textbf{Mis-tracking} (surface OK, downstream mismatch) signals \term{loose coupling}. Multiple sources~-- lexical, constructional, pragmatic~-- contribute to interpretation, and they can align or misalign without local ill-formedness. \emph{Diagnostic evidence}: implicature failures, pragmatic infelicity in discourse, mismatches between what was said and what was understood that emerge only in interaction.
\end{enumerate}

The coupling regime reflects stabilizer weighting. Transparent coupling dominates where physical and perceptual constraints lock form to contrastive identity (phonemes). Architectural coupling dominates where conventional form--value pairings are compositionally assembled (constructions). Loose coupling dominates where contextual inference supplements or overrides encoded content (discourse). The zipper metaphor is a diagnostic instrument, not just an illustration: error shape tells you where to look for the stabilizers.

\subsection{What \enquote{value} means}

To say what's being coupled, we need a term that works across grains. We've already met the term \term{value}~-- what a unit counts as in the system when deployed, its contrastive role, recoverable by competent users. Here it does heavy lifting.

\term{Value} in this book is not evaluative worth, not teleological purpose, not \enquote{meaning} in the undifferentiated everyday sense. It is relational rather than intrinsic: a unit's value is constituted by what it contrasts with and what inferences it licenses. For phonemes, value is contrastive identity: /\ipa{k}/ counts as not-/\ipa{g}/, not-/\ipa{t}/, not-/\ipa{p}/, and that exhausts its contribution. For morphemes, value is conventional pairing: the form \mention{-ed} is associated with pastness by agreement, not resemblance. For constructions, value is interpretive template: a form configuration paired with a meaning configuration, compositionally and holistically.

The zipper, then, has form on one track and value on the other. What changes across grains is what kind of value, and how tightly form is coupled to it.

\subsection{The chapter's claims}

This chapter argues that error-shape diagnoses coupling regime; coupling regime reflects stabilizer weighting; and two diagnostics~-- projectibility and homeostasis~-- test whether a purported category earns kindhood.

\term{Projectibility} asks whether patterns learned from one sample generalize to another~-- can we infer properties of new instances from old ones? \term{Homeostasis} asks whether we can identify stabilizers that maintain the property cluster and find their signatures in the data. These tests are independent:

\begin{itemize}
    \item A category might project reliably without identifiable stabilizers. This is the \emph{black-box success} case: a classifier achieves cross-sample accuracy, but we can't say what maintains the clustering. The projectibility is real but unexplained~-- possibly spurious, possibly tracking a genuine kind whose stabilizers we haven't identified. Example: a neural network trained on part-of-speech tags might achieve high accuracy without any interpretable feature weights. The projection is real; the mechanism story is missing.
    
    \item A category might have plausible stabilizers that don't produce stable clustering. This is the \emph{broken homeostasis} case: we can tell a story about what should maintain the cluster, but the predictions don't hold across samples or perturbations. Example: one might propose that \enquote{long words} form a natural class stabilized by processing difficulty, but the class doesn't project~-- knowing a word is long doesn't predict its syntactic behaviour, its semantic type, or its distributional neighbourhood. The mechanism story is decorative.
\end{itemize}

Both tests have to pass for kindhood to be warranted. The independence matters: it means the framework can fail in two directions, not just one.

\paragraph{What would falsify this chapter's claims?}

Three failure conditions, tied to the three positive cases:

\begin{enumerate}
    \item \textbf{Phonemes}: If mergers showed no relationship to cue reliability~-- if the \mention{pin}/\mention{pen} merger occurred in environments where acoustic cues are robust rather than degraded~-- the homeostasis story would be wrong.
    
    \item \textbf{Morphemes}: If regularization probability showed no relationship to frequency~-- if high-frequency irregulars like \mention{went} were just as vulnerable as low-frequency ones like \mention{clove}~-- the entrenchment stabilizer would be decorative.
    
    \item \textbf{Constructions}: If cross-corpus transfer for constructions like \mention{let alone} performed at chance, or if ablating formal cues (parallelism, licensing context) produced no interpretable degradation pattern, the claim that constructions are HPCs would lack evidential support.
\end{enumerate}

\subsection{What this chapter will show}

If the HPC framework applies at a different grain, we expect to find:
\begin{enumerate}
    \item Properties that cluster without being definitionally linked.
    \item Stabilizers that maintain the clustering.
    \item Graded category membership at the boundaries.
    \item Historical mutability: perturbation degrades the cluster.
\end{enumerate}

The chapter applies these expectations to three positive cases (phonemes, morphemes, constructions) and three negative cases (academic register, Indo-European, polysynthetic). The positive cases show the framework scaling. The negative cases show it has teeth~-- it can say \emph{no} as well as \emph{yes}.

One caveat before we begin. The tiers I'll discuss~-- phoneme, morpheme, construction~-- are convenient labels, not universal strata. They describe how English (and languages of similar type) organizes its categories. Other languages draw the boundaries differently: §\ref{sec:13:crossling} stress-tests the framework against typological variation. The HPC explanatory strategy is domain-general. The coupling regime is language-specific.

%~--  ~--  ~--  ~--  ~--  ~--  ~--  ~--  ~--  ~--  ~--  ~--  ~--  ~--  ~--  ~--  ~-- 
\section{Transparent coupling: phonemes}
\label{sec:13:phonemes}
%~--  ~--  ~--  ~--  ~--  ~--  ~--  ~--  ~--  ~--  ~--  ~--  ~--  ~--  ~--  ~--  ~-- 

Say the word \mention{key}. Now say \mention{ski}. You probably think the /\ipa{k}/ sound is the same in both. It isn't.

In \mention{key}, your tongue is further forward, anticipating the front vowel. In \mention{ski}, it's further back. A spectrogram would show you: the burst frequencies differ, the formant transitions differ. They're acoustically distinct.

But in ordinary listening, you won't notice the difference, because English doesn't treat it as contrastive. Both sounds are filed under /\ipa{k}/. The phoneme is an abstraction over the acoustics~-- a category that groups variant tokens and opposes them to other categories (/\ipa{g}/, /\ipa{t}/, /\ipa{p}/).

This is transparent coupling. At the phoneme level, form directly realizes contrastive value. The acoustic pattern discharges the contrastive job without any intervening representational structure. There's no separable \enquote{meaning} to decouple from~-- the value of /\ipa{k}/ is simply \enquote{not /\ipa{g}/, not /\ipa{t}/, not /\ipa{p}/.}

\subsection{The cluster}

For a worked case, consider the vowels /\ipa{ɪ}/ (as in \mention{bit}) and /\ipa{ɛ}/ (as in \mention{bet}). For most English speakers, these anchor a lexical contrast: \mention{pin} versus \mention{pen}, \mention{tin} versus \mention{ten}, \mention{bit} versus \mention{bet}.

The contrast is maintained by a cluster of properties that travel together:

\begin{itemize}
    \item \textbf{Acoustic cues.} Distributions in F1/F2 formant space. For /\ipa{ɪ}/, typical values are F1 $\approx$ 400 Hz, F2 $\approx$ 2000 Hz; for /\ipa{ɛ}/, F1 $\approx$ 550 Hz, F2 $\approx$ 1800 Hz. These are statistical tendencies, not fixed targets~-- speakers vary, contexts vary, and cue trading allows one dimension to compensate for another.
    \item \textbf{Production routines.} Tongue height and advancement differ. The articulatory gesture is reliable enough to produce consistent acoustic output, but flexible enough to accommodate coarticulation.
    \item \textbf{Lexical contrast.} Minimal pairs (\mention{pin}/\mention{pen}, \mention{bit}/\mention{bet}) make the contrast communicatively consequential. Misproduction yields a different word.
    \item \textbf{Perceptual categorisation.} Listeners map continuous acoustic input onto discrete categories, with sharper boundaries in high-contrast regions.
\end{itemize}

\subsection{The stabilizers}

Three stabilizer types maintain phoneme categories:

\paragraph{Design-space constraints.} \textcite{stevens1989} showed that the articulatory-to-acoustic mapping is non-linear. Certain vocal-tract configurations produce stable acoustic outputs across a range of articulatory variation. The stable regions are \term{quantal}~-- natural \enquote{parking spots} where small articulatory changes produce minimal acoustic change, but crossing a threshold produces a jump. These constrain what phoneme contrasts are \emph{possible}.

\paragraph{Dispersion pressure.} \textcite{lindblom1990} argued that vowel inventories disperse in acoustic space to maximize discriminability. This is a community-level transmission pressure: inventories that fail to disperse are harder to transmit and drift toward dispersed configurations over time. Dispersion shapes which of the possible phonemes a language \emph{selects}.

\paragraph{Perceptual tuning.} \textcite{kuhl1992} demonstrated that infant perception is warped by early exposure. Category prototypes act as \term{perceptual magnets}: stimuli near the prototype are perceptually pulled toward it. This tunes the \emph{individual speaker} to the community standard.

The phoneme that emerges is not a universal category but a language-specific cluster, maintained by this layered stabilizer braid.

\subsection{Stress test: the \mention{pin}/\mention{pen} merger}

If phonemes are HPCs maintained by stabilizers, what happens when the stabilizers weaken?

The \mention{pin}/\mention{pen} merger, documented extensively by \textcite{labov2006}, provides the evidence. In much of the American South and parts of the Midland, /\ipa{ɪ}/ and /\ipa{ɛ}/ have merged before nasal consonants. Speakers produce the same vowel in \mention{pin} and \mention{pen}; they can't reliably distinguish the two words by ear.

The conditioning environment is revealing. Before nasals, the vowels are subject to nasalization, which smears the formant cues. F1 is raised and F2 is altered by velopharyngeal coupling. In exactly the environment where the acoustic cues are least reliable, the contrast collapses.

This is homeostasis failing in a predictable way. The merger is not random noise; it's a consequence of perturbing the stabilizers. Where cue reliability decreases, the category boundary destabilizes. In non-nasal contexts, the two vowel distributions remain separated in F1/F2 space; in pre-nasal contexts, they overlap substantially or merge completely. A listener confusion matrix in the merged dialect shows near-chance performance on \mention{pin}/\mention{pen} discrimination, while \mention{bit}/\mention{bet} discrimination remains intact.

\subsection{Diagnostic punchline}

\paragraph{Projectibility.} The /\ipa{ɪ}/--/\ipa{ɛ}/ contrast projects reliably across English dialects that maintain it. Knowing the contrast exists predicts minimal-pair discrimination, lexical organisation, and phonotactic patterns. The merger also projects: knowing a speaker has the merger predicts confusion on the relevant pairs.

\paragraph{Homeostasis.} The stabilizers~-- quantal regions, dispersion pressure, perceptual magnets~-- are independently motivated and make testable predictions. The \mention{pin}/\mention{pen} merger confirms them: where stabilizers weaken (pre-nasal cue degradation), clustering degrades.

%~--  ~--  ~--  ~--  ~--  ~--  ~--  ~--  ~--  ~--  ~--  ~--  ~--  ~--  ~--  ~--  ~-- 
\section{Opaque coupling: morphemes}
\label{sec:13:morphemes}
%~--  ~--  ~--  ~--  ~--  ~--  ~--  ~--  ~--  ~--  ~--  ~--  ~--  ~--  ~--  ~--  ~-- 

At the phoneme level, form directly realizes contrastive value. At the next level up, form and value come apart.

Consider \mention{went}. It's the past tense of \mention{go}. Why?

There's no phonological connection. No suffix, no ablaut pattern you could generalize, no rule that predicts \mention{went} from \mention{go}. The connection is brute memory: English speakers learn this pairing as a fact.

This is \term{opaque coupling}. The form carries information about the value~-- \mention{went} encodes pastness~-- but you can't read the value off the form. The acoustic shape [\ipa{wɛnt}] doesn't \emph{mean} past; it's \emph{associated with} past by convention.

\subsection{The cluster}

For a case with richer stabilizer signature, consider regular past-tense marking. The suffix \mention{-ed} has three phonologically conditioned allomorphs: [\ipa{t}] after voiceless consonants (\mention{walked}), [\ipa{d}] after voiced sounds (\mention{played}), [\ipa{ɪd}] after alveolar stops (\mention{wanted}).

The form cluster is maintained by phonological naturalness~-- the allomorphy follows from assimilation and syllable structure. The value cluster is maintained by temporal cognition~-- pastness is a coherent semantic category cross-linguistically.

What maintains the \emph{pairing}?

\subsection{The stabilizers}

\paragraph{Frequency entrenchment.} High-frequency verbs are heard and produced often enough that the pairing is densely attested. Speakers don't have to generalize; they retrieve.

\paragraph{Paradigm pressure.} The regular pattern (\mention{walk}--\mention{walked}, \mention{play}--\mention{played}) provides a template that extends to novel verbs (\mention{to google}~$\rightarrow$~\mention{googled}).

\paragraph{Analogical extension.} When speakers encounter unfamiliar verbs, they produce past tenses by analogy to similar forms. Wug-test results confirm the pattern: children and adults extend \mention{-ed} to nonce verbs.

\subsection{Stress test: regularisation and frequency}

If form--value pairings are maintained by frequency and analogy, we should see a specific pattern: high-frequency irregulars should resist regularization; low-frequency irregulars should be vulnerable.

\textcite{bybee2001} documents exactly this. High-frequency irregulars (\mention{go}--\mention{went}, \mention{have}--\mention{had}, \mention{be}--\mention{was}) show no regularization pressure. Mid-frequency irregulars (\mention{weave}--\mention{wove}, \mention{strive}--\mention{strove}, \mention{dive}--\mention{dove}) show variable forms, with regularized variants emerging. Low-frequency irregulars (\mention{cleave}--\mention{clove}, \mention{chide}--\mention{chid}, \mention{gird}--\mention{girt}) have largely regularized.

This is the stabilizer signature. Frequency protects the arbitrary pairing by ensuring dense memory traces. Analogy extends the regular pattern to underdetermined cases. The two stabilizers interact: frequency protects irregulars from analogy; analogy regularizes the unprotected.

\subsection{Diagnostic punchline}

\paragraph{Projectibility.} Knowing a verb's frequency predicts its regularization vulnerability. The pattern holds across verbs, across time (historical regularization follows the same curve), and across speakers (children regularize low-frequency irregulars earlier).

\paragraph{Homeostasis.} Frequency entrenchment and analogical pressure are independently motivated. The prediction~-- inverse relationship between frequency and regularization~-- is confirmed. The stabilizers are doing the work the framework claims.

%~--  ~--  ~--  ~--  ~--  ~--  ~--  ~--  ~--  ~--  ~--  ~--  ~--  ~--  ~--  ~--  ~-- 
\section{Architectural coupling: constructions}
\label{sec:13:constructions}
%~--  ~--  ~--  ~--  ~--  ~--  ~--  ~--  ~--  ~--  ~--  ~--  ~--  ~--  ~--  ~--  ~-- 

At the construction level, form and value are re-unified~-- but now with visible seams.

A construction is a conventional pairing of form and value, stored as a unit and deployed as a whole. Unlike morphemes, constructions can be internally complex: they can contain slots, span clauses, and impose constraints on their fillers. The form--value coupling is \emph{architectural}~-- built up from components, maintained by use, transmitted through interaction.

\subsection{The cluster: \mention{let alone}}

Consider:

\ea[]{\mention{I wouldn't live in Cleveland, let alone Detroit.}}\label{ex:letalone}
\z

The speaker is saying: if Cleveland is already below my threshold, Detroit is even further below. There's an implicit scale, and the second element is more extreme than the first.

Now try to break it:

\ea[*]{\mention{I would live in Cleveland, let alone Detroit.}}\label{ex:letalone-bad}
\z

The positive sentence is odd. \mention{Let alone} requires a negative or downward-entailing context.

The form cluster: fixed anchor phrase \mention{let alone} (not *\mention{leave alone}); syntactic parallelism between the two contrasted elements; licensing context requiring negation, doubt, or other downward-entailing operators.

The value cluster: scalar semantics (the second element ranks higher on an implied scale); presupposition (if the first is rejected, the second is a fortiori rejected); rhetorical force (emphasis through implicit comparison).

The coupling is tight. Speakers who encounter the phrase infer the scalar value; speakers who want to express a scalar comparison reach for the phrase.

\subsection{The stabilizers}

\paragraph{Entrenchment.} Repeated exposure strengthens the form--value pairing. High-frequency users have robust representations; low-frequency users may be uncertain about the licensing conditions.

\paragraph{Cue redundancy.} Multiple formal features converge on the same interpretation. The fixed phrase, the parallelism, the licensing context all point to scalar semantics. Redundancy makes the construction recognisable even when individual cues are noisy.

\paragraph{Interactional alignment.} Speakers coordinate their usage in interaction. If I use \mention{let alone} and you understand it, our representations converge. Miscommunication triggers repair.

\subsection{Stress test: licensing errors and cross-corpus transfer}

If \mention{let alone} is an HPC maintained by entrenchment and cue redundancy, we predict specific failure modes. Low-frequency users should show uncertainty about licensing~-- using \mention{let alone} in upward-entailing contexts where it doesn't belong, or avoiding it entirely. \textcite{fillmore1988} documents exactly these patterns: speakers who lack robust exposure produce errors like \mention{?I can speak French, let alone Spanish} (scalar direction reversed) or avoid the construction in favour of paraphrase.

The projectibility diagnostic is cross-corpus transfer: can cue patterns learned from one corpus identify the construction in another? For \mention{let alone} and the related scalar-additive \mention{or even}, a classifier trained on one UD English corpus (GUM) and tested on another (EWT) achieves PR-AUC of 0.73, substantially above the 0.31 baseline for anchor-string matching alone. Ablating individual cues degrades performance in predictable ways: removing parallelism produces the largest drop (PR-AUC falls to 0.58), consistent with parallelism serving as a dominant stabilizer. Full methodology and results appear in Appendix~\ref{app:construction-classifier}.

The framework's teeth show in the designed failure cases. Pooling all \enquote{resultative} patterns into a single category produces weak cross-corpus transfer (PR-AUC near baseline) and washed-out ablation signatures~-- exactly what the \enquote{too fat} diagnosis predicts when an umbrella groups distinct subtypes maintained by heterogeneous stabilizers.

\subsection{Diagnostic punchline}

\paragraph{Projectibility.} Cross-corpus transfer succeeds. Cue ablation produces interpretable degradation. The pattern learned from one sample predicts another.

\paragraph{Homeostasis.} Entrenchment, cue redundancy, and interactional alignment are independently motivated. The predicted error patterns (licensing violations in low-frequency users, comprehension failure under cue degradation) are attested.

%~--  ~--  ~--  ~--  ~--  ~--  ~--  ~--  ~--  ~--  ~--  ~--  ~--  ~--  ~--  ~--  ~-- 
\section{The stabilizer-weighting map}
\label{sec:13:map}
%~--  ~--  ~--  ~--  ~--  ~--  ~--  ~--  ~--  ~--  ~--  ~--  ~--  ~--  ~--  ~--  ~-- 

The three cases above~-- phonemes, morphemes, constructions~-- are not a ladder. They are regions in a stabilization space defined by how form couples to value and which stabilizers carry the load. Table~\ref{tab:13:stabilizer-map} summarizes the mapping.

\begin{table}[htbp]
\centering
\caption{Stabilizer-weighting profiles across levels. Each level shows a characteristic coupling regime, error type, and dominant stabilizers.}
\label{tab:13:stabilizer-map}
\begin{tabular}{@{}llll@{}}
\toprule
\textbf{Level} & \textbf{Coupling} & \textbf{Error type} & \textbf{Dominant stabilizers} \\
\midrule
Phoneme & Transparent & Slip & Quantal regions, dispersion, perceptual magnets \\
Morpheme & Opaque & — & Frequency entrenchment, paradigm pressure, analogy \\
Construction & Architectural & Jam & Entrenchment, cue redundancy, alignment, type frequency \\
Discourse & Loose & Mis-tracking & Pragmatic inference, common ground, accommodation \\
\bottomrule
\end{tabular}
\end{table}

This arrangement is English-specific. Other languages trace different paths through the space~-- a point the next section stress-tests.

%~--  ~--  ~--  ~--  ~--  ~--  ~--  ~--  ~--  ~--  ~--  ~--  ~--  ~--  ~--  ~--  ~-- 
\section{Cross-linguistic stress-test}
\label{sec:13:crossling}
%~--  ~--  ~--  ~--  ~--  ~--  ~--  ~--  ~--  ~--  ~--  ~--  ~--  ~--  ~--  ~--  ~-- 

The HPC explanatory strategy is domain-general. The coupling regime is not universal. Three typologically distinct languages show how the tier labels are conveniences while the diagnostic method remains constant.

\paragraph{Mohawk (polysynthetic).} A single Mohawk verb can encode what English requires a sentence to express. The word \mention{Washakotya'tawitsherahetkvhta'se'} means \enquote{he made the thing that one puts on one's body ugly for her} \citep{mithun1984}~-- a proposition English distributes across multiple words, Mohawk packs into one. The morpheme--word--clause boundaries that seem natural in English are blurred. The \enquote{morpheme tier} exists, but it's structured differently: incorporated nouns, complex verb templates, and applicative morphology reorganize the form--value couplings. What this means for stabilizer weighting: the entrenchment mechanisms that maintain English constructions operate in Mohawk at the level of verb-template positions. The \emph{diagnostic} (does frequency entrenchment predict resistance to change?) still applies; the \emph{locus} shifts.

\paragraph{Vietnamese (isolating).} Vietnamese has minimal bound morphology. The particle \mention{đã} marks past/perfective; \mention{sẽ} marks future; \mention{đang} marks progressive~-- but none is obligatory, and all are free-standing words. Tense, aspect, and mood are often inferred from context rather than marked overtly. Is there a \enquote{morpheme tier} in the English sense? Barely. The stabilizer weighting shifts: storage mechanisms operate on particles and constructions rather than on bound affixes. The \emph{diagnostic} (does cue redundancy predict construction identification?) still applies; the \emph{inventory of cues} changes.

\paragraph{!Xóõ (click-heavy).} With over 100 phonemes including multiple click types, !Xóõ's phoneme inventory is stabilized by a different mechanism mix. Click production requires distinct articulatory coordination~-- tongue-tip, tongue-blade, and back-of-tongue contacts with independent timing. Click perception involves different cue weighting: the release burst matters more than formant transitions. The quantal-regions story from English doesn't transfer directly; the \emph{diagnostic} (do mergers occur where cues degrade?) still applies, but the relevant cue dimensions are language-specific.

The lesson is methodological: where the English tier labels don't fit, the diagnostic method still applies. Each language has its own stabilizer-weighting profile. The framework tells you what to look for (clustering, stabilizers, perturbation signatures); it doesn't prescribe what you'll find.

%~--  ~--  ~--  ~--  ~--  ~--  ~--  ~--  ~--  ~--  ~--  ~--  ~--  ~--  ~--  ~--  ~-- 
\section{Negative cases: when the framework says no}
\label{sec:13:negative}
%~--  ~--  ~--  ~--  ~--  ~--  ~--  ~--  ~--  ~--  ~--  ~--  ~--  ~--  ~--  ~--  ~-- 

The HPC framework would be toothless if it said yes to everything. Here are three cases where it says no~-- each for a different reason.

\subsection{Academic register}

\paragraph{Temptation.} Academic writing has a recognizable flavour. Passive constructions. Nominalizations. Hedges like \mention{it has been argued} and \mention{the data suggest}. Author-effacing expressions. The features cluster; experienced readers identify academic prose instantly. Surely this is a natural category~-- a register, a style, maybe even a construction family.

\paragraph{Stress test.} Put the same researcher in front of a grant panel, a blog audience, and a conference poster. The passives evaporate. The hedges reweight. The nominalizations drop by half. Now hold the institutional context fixed but change the discipline: compare physics and literary criticism. The bundles differ substantially while both count as \enquote{academic.}

\paragraph{Failure signature.} The bundle doesn't resist perturbation; it dissolves and reconstitutes in a new configuration. The stabilizers are there~-- genre conventions, disciplinary gatekeeping, editorial norms~-- but they have high \term{looping intensity}: the category changes because it is classified. Peer reviewers enforce the \enquote{academic} sound; writers simulate it; the classification itself shifts the target. This reflexivity is the diagnostic signature: perturbations that leave institutional incentives intact produce different bundles; perturbations that change incentives rewire the category entirely.

\paragraph{What kind instead.} Academic register is a \emph{social institution} maintained by institutional policing, not a natural kind maintained by cognitive binding. It belongs in the same ontological category as \enquote{formal attire} or \enquote{professional demeanour}~-- real, consequential, but not projectible in the way phonemes and constructions are. The HPC framework correctly says no.

\subsection{Indo-European}

\paragraph{Temptation.} Indo-European is a language family~-- a genealogical grouping defined by historical descent. The cluster properties are real: cognate vocabulary (\mention{mother}/\mention{Mutter}/\mention{mater}/\mention{mātr}), shared sound correspondences (Grimm's Law), similar grammatical patterns. Languages are Indo-European because they share a common ancestor. Isn't common ancestry a kind of mechanism?

\paragraph{Stress test.} Ask what would push English \emph{back toward} Proto-Indo-European. Nothing. There's no homeostatic pressure maintaining Indo-European-ness. English drifts away from PIE continuously; contact with non-IE languages accelerates the drift.

\paragraph{Failure signature.} The mechanisms are historical (common origin), not homeostatic (ongoing maintenance). Modern English doesn't stay Indo-European because of stabilizers pushing it back toward the prototype. It's Indo-European because of where it came from~-- and it's becoming less Indo-European-like over time. The diagnostic that fails is homeostasis: there's no perturbation-resistance, no error-correction, no convergent pressure.

\paragraph{What kind instead.} Indo-European is a \emph{historical kind}, not a homeostatic kind. Historical kinds are defined by causal continuity with an origin; homeostatic kinds are defined by stabilizers that maintain covariance. Confusing them invites explanatory error: treating historical residue as if it required ongoing maintenance, or expecting convergent evolution where only descent explains the pattern.

\subsection{Polysynthetic}

\paragraph{Temptation.} Polysynthetic languages are characterized by high morpheme-to-word ratios, incorporating structures, and complex verb templates. Mohawk, Chukchi, Ainu, and various others across multiple families are standard examples. The properties seem to cluster: if a language has noun incorporation, it probably also has applicatives and head-marking. Isn't this an HPC?

\paragraph{Stress test.} Ask whether the same stabilizers maintain polysynthesis across families. Mohawk (Iroquoian) has noun incorporation for discourse-pragmatic reasons~-- incorporated nouns are non-referential, allowing the verb to function as a kind of lexical compound. Chukchi (Chukotko-Kamchatkan) has incorporation with different constraints and different discourse functions. The surface similarity~-- complex words~-- masks heterogeneous routes.

\paragraph{Failure signature.} The clustering is loose: languages are called polysynthetic if they score high on several dimensions, but the dimensions don't cohere tightly~-- a language can have noun incorporation without applicatives, or complex verb templates without head-marking. More importantly, there's no convergent stabilizer profile. Why Mohawk has noun incorporation is a different story from why Chukchi does. The diagnostic that fails is projectibility across the class: knowing a language is \enquote{polysynthetic} doesn't let you predict which specific features it will have or how they'll behave under change.

\paragraph{What kind instead.} \enquote{Polysynthetic} is a \emph{typological region label}~-- a region in morphological space that different languages arrive at by different routes. It's useful for descriptive purposes (it picks out languages with certain surface properties) but not for explanatory purposes (it doesn't name a mechanism-maintained kind). The HPC framework correctly says no: the heterogeneity of routes is exactly what distinguishes a typological region from a natural kind.

\subsection{What the negative cases show}

These three cases fail the HPC diagnostics for different reasons:

\begin{itemize}
    \item \textbf{Academic register}: Stabilizers present but reflexive. High looping intensity (the category changes because it's classified). Fails projectibility under perturbation.
    \item \textbf{Indo-European}: Maintained by historical continuity, not ongoing stabilizers. A different kind of kind. Fails homeostasis.
    \item \textbf{Polysynthetic}: A gradient label over heterogeneous structures. No convergent stabilizer profile. Fails projectibility across the class.
\end{itemize}

The HPC framework distinguishes these failure modes. That's the payoff of having explicit diagnostics: you can say \emph{why} something doesn't qualify, not just that it doesn't.

%~--  ~--  ~--  ~--  ~--  ~--  ~--  ~--  ~--  ~--  ~--  ~--  ~--  ~--  ~--  ~--  ~-- 
\section{Looking forward}
\label{sec:13:forward}
%~--  ~--  ~--  ~--  ~--  ~--  ~--  ~--  ~--  ~--  ~--  ~--  ~--  ~--  ~--  ~--  ~-- 

This chapter has argued that the HPC explanatory strategy scales across linguistic levels. Phonemes, morphemes, and constructions are all homeostatic property clusters maintained by stabilizers. What differs is the coupling regime~-- transparent at the sound level, opaque at the morpheme level, architectural at the construction level~-- and the stabilizer weighting that maintains each regime.

The zipper metaphor returns for the final turn. Error shape diagnoses coupling: slips reveal transparent coupling; jams reveal architectural coupling; mis-tracking reveals loose coupling. Each error type tells you where the teeth failed to seat and, therefore, where to look for the stabilizers that normally keep them seating.

But one level has special status.

Throughout the case studies in Part III~-- countability, definiteness, lexical categories~-- the categories clustered at the morphosyntactic level. That's where form--value coupling is both \emph{tight} and \emph{obligatory}. You can speak without using \mention{let alone}. In English, you can't speak without committing to number, tense, and definiteness.

Morphosyntax is the zone of maximum systematicity: the region of grammar where coupling is enforced, not optional. Chapter~\ref{ch:grammaticality-itself} asks what happens when we take this observation seriously. If grammaticality is what emerges when form--value coupling is obligatory, compositional, and learnable~-- then grammaticality itself is a kind.
