\chapter{Grammaticality itself}
\label{ch:grammaticality-itself}

% Draft status: Provisional intro for Metaphor Refinement phase
Every analysis of every construction presupposes a distinction between grammatical and ungrammatical. The asterisk is the most common symbol in syntax. But what is the asterisk diagnosing? If categories are not definitions but homeostatic property clusters, then grammaticality is not a binary switch. It is a measure of stability.

When we judge a sentence, we are not checking a rulebook. We are interrogating the standing wave~-- probing the system to see if the dynamic equilibrium holds or collapses under strain. A fully grammatical sentence sits at the bottom of a deep basin: perturb it slightly (change word order, shift stress, substitute a near-synonym) and the system corrects. An ungrammatical sentence sits outside the basin entirely: no perturbation will roll it back to stable ground. The marginal cases~-- the 3-star and 4-star judgments, the sentences that split informants~-- are poised on the rim. They are at the edge of a phase transition, where small changes in context or construal can tip the system one way or the other.

This reframing has consequences. If grammaticality is basin-stability, then the sharp boundary between well-formed and ill-formed is not a property of the grammar itself but an artifact of how we test. Gradient judgments are not noise to be smoothed away; they are signal. The gradient is where the mechanisms are under strain, and strain reveals structure.
