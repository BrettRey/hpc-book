\chapter{Grammaticality itself}
\label{ch:grammaticality-itself}

% Draft status: Expanded skeleton based on HPC framing (Jan 2026)

Every analysis of every construction presupposes a distinction between grammatical and ungrammatical. The asterisk is the most common symbol in syntax. But what is the asterisk diagnosing?

If categories are homeostatic property clusters, then grammaticality is not merely a binary switch. It is itself an HPC~-- a maintained coupling between form and value. And the feeling of ungrammaticality that guides our judgments is not an oracle; it is a noisy detector, shaped by entrenchment and processing ease, capable of illusions.

Chapter~\ref{ch:dynamic-discreteness} introduced a two-layer picture: discrete grammaticality filtered through processing noise to yield gradient acceptability. What we add here is the question of what \emph{kind} grammaticality is, and what maintains it.


\section{The HPC claim}
\label{sec:14:hpc-claim}

Grammaticality is a form--value pairing. But the ``value'' side is not compositional semantics~-- not truth conditions, not the full propositional content. It is something more basic: \term{structural meaning}.

Structural meaning is what grammar contributes before lexical content fills the slots:

\begin{itemize}
    \item \textbf{Modification relations}: What modifies what. The difference between \mention{old friend's car} and \mention{old car's friend}. Or the bracketing that distinguishes \mention{[the dogs'] house} (a house belonging to the dogs) from \mention{the [dogs' house]} (a type of house).
    \item \textbf{Argument structure}: Who did what to whom. The difference between \mention{the dog bit the man} and \mention{the man bit the dog}.
    \item \textbf{Tense and aspect}: Temporal anchoring. Whether the event is past, ongoing, or hypothetical.
    \item \textbf{Clause type}: Statement, question, command. The difference between \mention{she left} and \mention{did she leave?}
    \item \textbf{Information structure}: Topic, focus, given versus new. Where the emphasis falls.
\end{itemize}

This is \emph{subcompositional}~-- it is the instruction manual, not the assembled product. Grammar tells you how to wire the meanings; the lexicon provides what to wire. \mention{Colorless green ideas sleep furiously} has perfect wiring instructions and absurd components. That's why it feels strange but not \emph{ungrammatical}.

The HPC claim, then, is that grammaticality is the coupling between morphosyntactic form and this structural semantics. When the coupling holds~-- when the form reliably signals the structural meaning~-- the sentence sits in the grammatical basin. When it doesn't, the sentence is outside.


\section{The stabilizer: a feeling}
\label{sec:14:stabilizer}

What maintains this coupling? Partly the mechanisms we've discussed throughout: acquisition, entrenchment, alignment, transmission. But the proximate stabilizer~-- the thing that corrects deviations in real time~-- is a \term{feeling of ungrammaticality}.

When something is wrong with the form--value coupling, speakers notice. They hesitate, repair, rephrase. The feeling is what triggers the correction. It is the immune response of the grammar.

But the feeling is not a direct readout of the underlying structure. It is a \emph{detector}~-- and like all detectors, it is noisy. The feeling is informed by:

\begin{enumerate}
    \item \textbf{Entrenchment}: Frequent structures feel normal; rare ones feel marked, even when perfectly grammatical. The garden-variety English VP (\mention{ate the cake}) feels more grammatical than the rarer but equally well-formed double-object construction (\mention{baked her a cake}).
    \item \textbf{Processing ease}: Sentences that parse smoothly feel grammatical; sentences that induce garden-paths or reanalysis feel suspect. Processing difficulty is not the same as ungrammaticality, but the detector often conflates them.
    \item \textbf{Analogy to attested forms}: Speakers compare novel sentences to stored exemplars. If the novel form is distant from everything stored, it feels wrong~-- even if the grammar licenses it.
\end{enumerate}

This is why acceptability and grammaticality come apart. Acceptability is what the detector reports. Grammaticality is the underlying form--value coupling. The two usually align, but not always.


\section{Grammaticality illusions}
\label{sec:14:illusions}

The strongest evidence that the feeling is dissociable from the structure comes from \term{grammaticality illusions}~-- cases where the detector misfires.

\subsection{Feels ungrammatical, is grammatical}

The classic case is the garden-path sentence:

\begin{quote}
\mention{The horse raced past the barn fell.}
\end{quote}

This is grammatical~-- a reduced relative clause (\mention{the horse [that was] raced past the barn fell}). But the processing system commits to the main-clause parse (\mention{the horse raced past the barn}) and crashes when \mention{fell} appears. The detector reports ungrammaticality; the structure is fine.

\subsection{Feels grammatical, is ungrammatical}

The converse is the Escher sentence:

\begin{quote}
\mention{More people have been to Russia than I have.}
\end{quote}

Each local chunk parses. \mention{More people have been to Russia}~-- fine. \mention{Than I have}~-- fine as a comparative clause. But the whole doesn't compose: what is being compared? The sentence has no coherent interpretation, yet it \emph{feels} acceptable because each piece satisfies local constraints.

These illusions are exactly analogous to visual illusions. The Müller-Lyer lines are the same length, but the perceptual system reports otherwise. The illusion doesn't show that perception is broken; it shows that perception is a mechanism with characteristic failure modes. Grammaticality illusions show the same: the feeling is a detector, not an oracle.


\section{What doesn't count}
\label{sec:14:negative-space}

The form--value coupling is specifically \emph{morphosyntactic}. This predicts what \emph{won't} trigger ungrammaticality:

\begin{itemize}
    \item \textbf{Mispronunciation}: \mention{[fɪlɔzəfi]} for \mention{philosophy}. The phonetic form is wrong, but the morphosyntax is untouched. It sounds odd; it doesn't feel ungrammatical.
    \item \textbf{Wrong word} (lexical error): \mention{I drove my car to the \emph{airport}} when you meant \mention{train station}. The slot is filled correctly; the filler is wrong. The morphosyntax is fine.
    \item \textbf{Semantic anomaly}: \mention{Colorless green ideas sleep furiously.} The morphosyntax is impeccable; the violation is selectional. You can assign structural meaning~-- you know what modifies what, who does what~-- you just can't make sense of it.
\end{itemize}

These violations trigger \emph{different} responses. Phonetic errors trigger \mention{what?} Lexical errors trigger correction or confusion. Semantic anomaly triggers puzzlement. But none of them trigger the specific feeling of ungrammaticality~-- the sense that the wiring is broken.


\section{The gradient at the boundary}
\label{sec:14:gradient}

When we judge a sentence, we are not checking a rulebook. We are interrogating the standing wave~-- probing the system to see if the dynamic equilibrium holds or collapses under strain.

A fully grammatical sentence sits at the bottom of a deep basin: perturb it slightly (change word order, shift stress, substitute a near-synonym) and the system corrects. An ungrammatical sentence sits outside the basin entirely: no perturbation will roll it back to stable ground.

The marginal cases~-- the 3-star and 4-star judgments, the sentences that split informants~-- are poised on the rim. They are at the edge of a phase transition, where small changes in context or construal can tip the system one way or the other.

This reframing has consequences. If grammaticality is basin-stability, then the sharp boundary between well-formed and ill-formed is not a property of the grammar itself but an artifact of how we test. Gradient judgments are not noise to be smoothed away; they are signal. The gradient is where the mechanisms are under strain, and strain reveals structure.


\section{Grammaticality as mechanism and category}
\label{sec:14:dual-role}

Grammaticality plays a dual role in the framework.

As a \term{category}, it is an HPC~-- the maintained coupling between morphosyntactic form and structural meaning. It passes the two diagnostics: it is projectible (learning that one form--value pairing is grammatical lets you predict that similar pairings will be), and it is homeostatic (the feeling of ungrammaticality stabilizes the coupling, correcting deviations).

As a \term{mechanism}, it is itself a stabilizer for other HPCs. The category \term{noun} is maintained partly because nounhood is grammatical~-- because using a word in a noun slot, with noun morphology, in noun constructions, triggers no ungrammaticality. Grammaticality is the enforcement mechanism for the form--value couplings that constitute all other grammatical categories.

This dual role~-- category and mechanism, maintainer and maintained~-- is not paradoxical. It is characteristic of homeostatic systems. The immune system is both a system (a biological category) and a mechanism that maintains other systems (the body's integrity). Grammaticality is the immune system of the grammar.


% TODO: Add HPC-kind audit for grammaticality
% TODO: Add \textsc{Definiteness thread} box


