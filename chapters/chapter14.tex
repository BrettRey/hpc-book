\chapter{Grammaticality itself}
\label{ch:grammaticality-itself}
\epigraph{It is impossible to get rid of the illusion in spite of our better knowledge.}{— Hermann von Helmholtz, \textit{Treatise on Physiological Optics} (trans. J. P. C. Southall).}

% Draft status: Expanded skeleton based on HPC framing (Jan 2026)

Every analysis of every construction presupposes a distinction between grammatical and ungrammatical. The asterisk is the most common symbol in syntax. But what is the asterisk diagnosing?

Chapter~\ref{ch:what-we-havent-been-asking} asked a question that almost no one asks: is grammaticality itself a natural kind? Not whether this sentence is grammatical, but what kind of thing we're probing when we ask. This chapter provides the answer.

If categories are homeostatic property clusters, then grammaticality is not merely a binary switch. It is itself an HPC~-- a maintained coupling between form and value. And the feeling of ungrammaticality that guides our judgments is not an oracle; it is a noisy detector, shaped by entrenchment and processing ease, capable of illusions.

Chapter~\ref{ch:dynamic-discreteness} introduced a two-layer picture: discrete grammaticality filtered through processing noise to yield gradient acceptability. What we add here is the question of what \emph{kind} grammaticality is, and what maintains it.


\section{The HPC claim}
\label{sec:14:hpc-claim}

Grammaticality is a form--value pairing. But the ``value'' side is not compositional semantics~-- not truth conditions, not the full propositional content. It is something more basic: \term{structural meaning}.

Structural meaning is what grammar contributes before lexical content fills the slots. It includes, for example:

\begin{itemize}
    \item \textbf{Dependency relations}: The bracketing that distinguishes [\mention{the dogs'}] \mention{house} (a house belonging to the dogs) from \mention{the} [\mention{dogs' house}] (a type of house).
    \item \textbf{Argument structure}: The difference between \mention{the dog bit the man} and \mention{the man bit the dog}.
    \item \textbf{Tense and aspect}: Temporal anchoring. Whether the biting is past, ongoing, or hypothetical.
    \item \textbf{Clause type}: The difference between \mention{he was bit}, \mention{was he bit?}, and \mention{bite him.}
    \item \textbf{Information structure}: \mention{Him, he was bit} vs \mention{it was him who was bit}.
\end{itemize}

This is \emph{subcompositional}~-- it is the instruction manual, not the assembled product. Grammar tells you how to wire the meanings; the lexicon provides what to wire. \mention{Colorless green ideas sleep furiously} has perfect wiring instructions and absurd components. That's why it feels strange but not \emph{ungrammatical}.

A note on what structural meaning is \emph{not}. It is not semantic coherence (\mention{Colorless green ideas} shows grammaticality survives semantic absurdity). It is not contextual felicity (a sentence can be grammatical and pragmatically infelicitous). And it is not whatever the hearer infers (inference runs on many channels; grammaticality is specifically the morphosyntactic one). Structural meaning is the contribution of form to interpretation~-- the wiring diagram, not the current flowing through it.

The HPC claim, then, is that grammaticality is the coupling between morphosyntactic form and this structural semantics. When the coupling holds~-- when the form reliably signals the structural meaning~-- the sentence sits in the grammatical basin. When it doesn't, the sentence is outside.

This is the \term{zipper}: morphosyntax provides one set of teeth, structural meaning the other. Grammaticality is what it feels like when the zipper closes. An ungrammatical sentence is a zipper that won't mesh.

Two metaphors will recur. The zipper describes the \emph{relation}~-- the form--value alignment itself. The \term{immune system} (below) describes the \emph{maintenance mechanism}~-- what pushes the system back toward stable couplings when they slip. One is structure; the other is dynamics.

\section{The stabilizer: a feeling}
\label{sec:14:stabilizer}

What maintains this coupling? Partly the mechanisms we've discussed throughout: acquisition, entrenchment, alignment, transmission. But the proximate stabilizer~-- the thing that corrects deviations in real time~-- is a \term{feeling of ungrammaticality}.

When something is wrong with the form--value coupling, speakers notice. They hesitate, repair, rephrase. The feeling is what triggers the correction. It is the immune response of the grammar.

The feeling is a \emph{detector}~-- and like all detectors, it is noisy. It's not a direct readout of the underlying structure. Instead, it's informed by:

\begin{enumerate}
    \item \textbf{Probability}: The detector is Bayesian. A novel sentence is weighted by its likelihood under the distribution of attested forms~-- not just distance from the nearest exemplar, but probability given the entire pattern. Low-probability forms feel suspect even when grammatically licensed. The distribution itself is built by entrenchment: exposure tunes the prior.
    \item \textbf{Processing ease}: Sentences that parse smoothly feel grammatical; sentences that induce garden-paths or reanalysis feel suspect. Processing difficulty is not the same as ungrammaticality, but the detector often conflates them.
\end{enumerate}

This is why acceptability and grammaticality come apart. Acceptability is what the detector reports. Grammaticality is the category~-- the HPC whose core is the form--value coupling. The two usually align, but not always.

\subsection{Entrenchment in action}

But here's the puzzle: if grammaticality \emph{is} the maintained coupling, and entrenchment is what maintains it, then ``tuning the detector'' and ``shaping what counts as grammatical'' aren't separable. The detector isn't just tracking some pre-existing reality; the reality is constituted by what gets entrenched. The feeling tracks the coupling, and the coupling is entrenched. All the way down.

This is language-specific, construction-specific, even register-specific.

\begin{itemize}
    \item \textbf{Age expressions}: In French and Spanish, age is expressed with \mention{avoir}/\mention{tener}: \mention{J'ai vingt ans} (`I have twenty years'). In English, *\mention{I have twenty years} is ungrammatical as a statement of age~-- though perfectly fine as a statement of time (until retirement, on the job, etc.). The structural slot is the same; the construction-specific entrenchment differs.
    \item \textbf{Progressive aspect}: If you're describing an ongoing event while your fingers are moving across a keyboard, \mention{I write} mis-signals aspect in English~-- many speakers experience this as ``wrong'' in the specific sense of a form--value mismatch. The progressive (\mention{I'm writing}) is not merely preferred; it is required for the ongoing-action construal. In French, the simple present would be grammatical for the same construal; the progressive is marked or absent. What counts as a violation of the form--value coupling depends on what the community has entrenched.
    \item \textbf{Countability}: Whether \mention{furniture} takes count morphosyntax depends on entrenchment, not metaphysics. Furniture is as individuable as anything~-- you can count chairs and tables. But English has entrenched \mention{furniture} as noncount, so *\mention{I bought two furnitures} is ungrammatical. Other languages make different choices. The coupling is real; its contents are conventional.
    \item \textbf{Deitality}: What requires the definite article in English~-- \mention{go to the hospital} vs.~\mention{go to hospital} (British)~-- is shaped by entrenchment within each dialect. The structural meaning (institutionalized activity frame) is the same; the morphosyntactic realization differs.
\end{itemize}

These examples show that grammaticality is not a window onto universal logic. It is a window onto what a community has entrenched as the couplings between form and structural meaning. The feeling of ungrammaticality is tracking those couplings~-- but the couplings themselves are historical, contingent, and maintained.

And here is where the entrenchment story connects to the social: entrenchment is always entrenchment \emph{within a discourse community}. The grammar isn't floating free; it's tethered to the people who use it. When the community shifts~-- through contact, migration, register change~-- the entrenchment shifts with it. What feels ungrammatical in one community may be unremarkable in another.

But speakers flow between communities~-- sometimes instantaneously~-- and communities can be created on the fly. \mention{Do you lift?} licenses intransitive \mention{lift} for weightlifters but not for warehouse workers; the grammaticality of the construction depends on which community you're invoking, and you can switch in mid-conversation. Code-switching within a clause~-- \mention{Vamos a hacer shopping}~-- is governed grammar, not error, but only within the discourse community of the bilinguals who maintain it. The detector isn't calibrated once; it's calibrated to whoever you're talking to, right now.

One more piece. We could not correct grammatical errors if we could not derive the intended meaning without the grammar. When someone says *\mention{I have twenty years} for their age, we understand what they mean~-- and \emph{then} we notice the mismatch. The feeling of ungrammaticality arises from comparison: the grammar's value versus the intended value. We need both to detect the error. This means comprehension is partly independent of grammaticality~-- and grammaticality is partly a matter of checking the grammar's signal against other inference channels. When they align, we don't notice; when they mismatch, we do.


\section{Grammaticality illusions}
\label{sec:14:illusions}

The strongest evidence that the feeling is dissociable from the structure comes from \term{grammaticality illusions}~-- cases where the detector misfires.

\subsection{Feels ungrammatical, is grammatical}

The classic case is the garden-path sentence:

\begin{quote}
\mention{The horse raced past the barn fell.}
\end{quote}

This is grammatical~-- a reduced relative clause (\mention{the horse [that was] raced past the barn fell}). But the processing system commits to the main-clause parse (\mention{the horse raced past the barn}) and crashes when \mention{fell} appears. The detector reports ungrammaticality; the structure is fine.

\subsection{Feels grammatical, is ungrammatical}

The converse is the \term{comparative illusion}~-- sometimes called the \term{Escher sentence}:

\begin{quote}
\mention{More people have been to Russia than I have.}
\end{quote}

Each local chunk parses. \mention{More people have been to Russia}~-- fine. \mention{Than I have}~-- fine as a comparative clause. But the whole doesn't compose: what is being compared? The sentence has no coherent interpretation, yet it \emph{feels} acceptable because each piece satisfies local constraints.

These illusions are exactly analogous to visual illusions. The Müller-Lyer lines are the same length, but the perceptual system reports otherwise. The illusion doesn't show that perception is broken; it shows that perception is a mechanism with characteristic failure modes. Grammaticality illusions show the same: the feeling is a detector, not an oracle. \textcite{phillips2011grammatical} develop this point systematically, showing that the parser exhibits \term{selective fallibility}~-- impressive accuracy for some constraints (island effects, Principle C), systematic vulnerability for others (agreement attraction, NPI licensing). The pattern reveals the architecture: a retrieval system that is structure-sensitive when structural cues have temporal priority, and error-prone when they don't.


\section{What doesn't count}
\label{sec:14:negative-space}

The form--value coupling is specifically \emph{morphosyntactic}. This predicts what \emph{won't} trigger ungrammaticality:

\begin{itemize}
    \item \textbf{Mispronunciation}: \mention{[fɪlɔzəfi]} for \mention{philosophy}. The phonetic form is wrong, but the morphosyntax is untouched. It sounds odd; it doesn't feel ungrammatical.
    \item \textbf{Wrong word} (lexical error): \mention{I drove my car to the \emph{airport}} when you meant \mention{train station}. The slot is filled correctly; the filler is wrong. The morphosyntax is fine.
    \item \textbf{Semantic anomaly}: \mention{Colorless green ideas sleep furiously.} The morphosyntax is impeccable; the violation is selectional. You can assign structural meaning~-- you know what modifies what, who does what~-- you just can't make sense of it.
\end{itemize}

These violations trigger \emph{different} responses. Phonetic errors trigger \mention{what?} because the phoneme's value \emph{is} the contrast~-- that's the entire form--value coupling at that level, and when it breaks, comprehension stalls. Lexical errors trigger correction or confusion. Semantic anomaly triggers puzzlement. But none of them trigger the specific feeling of ungrammaticality~-- the sense that the wiring is broken.

A revealing contrast: *\mention{a apple}, *\mention{an banana}. These look phonological~-- wrong allomorph of the indefinite article. But they trigger ungrammaticality, not just \mention{what?} Why? Because the a/an alternation is morphophonologically conditioned. The selection rule (a before consonants, an before vowels) is entrenched as a grammatical requirement, part of the morphosyntactic zipper. A mispronunciation of \mention{banana} would trigger \mention{what?}; distributing the wrong allomorph of the article triggers the grammar's immune response. This is exactly what we'd expect from an HPC: the edge case reveals the boundary of the morphosyntactic coupling, and the feeling of ungrammaticality~-- however muted~-- is the trace of the immune system at work.

\subsection{Value at every grain}

What is ``value'' at each level of the grammar? The zipper closes differently depending on what you're zipping.

\begin{table}[h]
\centering
\small
\begin{tabular}{@{}lll@{}}
\toprule
\textbf{Grain} & \textbf{Form} & \textbf{Value} \\
\midrule
Phoneme & Sound segment & Contrast (minimal-pair distinctiveness) \\
Morpheme & Affix, stem & Grammatical or derivational modification \\
Word & Phonological shape & Lexical meaning + category membership \\
Construction & Syntactic pattern & Argument structure, information structure, meaning template \\
Clause type & Structural configuration & Illocutionary force (statement, question, command) \\
\bottomrule
\end{tabular}
\caption{Form--value coupling at different grains of grammar.}
\label{tab:form-value-grains}
\end{table}

At the phonemic level, value is nothing \emph{but} contrast. The /p/ in \mention{pat} contributes no semantic content; it just distinguishes the word from \mention{bat}, \mention{cat}, \mention{sat}. When contrast fails~-- when you can't tell which phoneme was intended~-- you say \mention{what?}

At the morphemic level, value is the contribution to word meaning or grammatical function. The \mention{-ed} in \mention{walked} signals past tense; the \mention{un-} in \mention{unhappy} reverses polarity. Break the coupling~-- use \mention{-ed} on a noun, or \mention{un-} on a verb that doesn't license it~-- and ungrammaticality results.

At the word level, value is lexical meaning plus category membership. \mention{Dog} contributes dog-ness \emph{and} nounhood. Substitute the wrong word and you get confusion; use a word in the wrong category slot and you get ungrammaticality.

At the constructional level, value is the structural meaning the pattern contributes. The double-object construction (\mention{gave her the book}) signals transfer plus affected recipient. The passive (\mention{was eaten}) demotes the agent. Break the construction's slot-filling requirements and the zipper won't close.

At the clause-type level, value is illocutionary force. Declarative structure signals assertion; interrogative structure signals question; imperative structure signals command. Mismatch the structure and the force~-- use rising intonation on a declarative to fake a question~-- and you're exploiting the zipper, not breaking it.

This table explains why grammaticality is primarily a \emph{syntactician's} category. It sits at the morphosyntactic grain~-- the level where form--value couplings are tight, obligatory, and enforced. Phoneticians have their own coupling (contrast), and when it breaks, the response is \mention{what?}, not ungrammaticality. Semanticists have theirs (compositionality, inference), and when it breaks, the response is puzzlement or absurdity. Grammaticality is what you feel when the morphosyntactic zipper won't close. That's why syntax owns it.


\section{The gradient at the boundary}
\label{sec:14:gradient}

When we judge a sentence, we are not checking a rulebook. We are interrogating the standing wave~-- probing the system to see if the dynamic equilibrium holds or collapses under strain.

A fully grammatical sentence sits at the bottom of a deep basin: perturb it slightly (change word order, shift stress, substitute a near-synonym) and the system corrects. An ungrammatical sentence sits outside the basin entirely: no perturbation will roll it back to stable ground.

The marginal cases~-- the 3-star and 4-star judgments, the sentences that split informants~-- are poised on the rim. They are at the edge of a phase transition, where small changes in context or construal can tip the system one way or the other.

This reframing has consequences. If grammaticality is basin-stability, then the sharp boundary between well-formed and ill-formed is not a property of the grammar itself but an artifact of how we test. Gradient judgments are not noise to be smoothed away; they are signal. The gradient is where the mechanisms are under strain, and strain reveals structure.


\section{Grammaticality as mechanism and category}
\label{sec:14:dual-role}

Grammaticality plays a dual role in the framework.

As a \term{category}, it is an HPC~-- the maintained coupling between morphosyntactic form and structural meaning. It passes the two diagnostics: it is projectible (learning that one form--value pairing is grammatical lets you predict that similar pairings will be), and it is homeostatic (the feeling of ungrammaticality stabilizes the coupling, correcting deviations).

As a \term{mechanism}, it is itself a stabilizer for other HPCs. The category \term{noun} is maintained partly because nounhood is grammatical~-- because using a word in a noun slot, with noun morphology, in noun constructions, triggers no ungrammaticality. Grammaticality is the enforcement mechanism for the form--value couplings that constitute all other grammatical categories.

This dual role~-- category and mechanism, maintainer and maintained~-- is not paradoxical. It is characteristic of homeostatic systems. The immune system is both a system (a biological category) and a mechanism that maintains other systems (the body's integrity). Grammaticality is the immune system of the grammar.

Or think of the \term{spinning top} from Chapter~\ref{ch:kinds-without-essences}. A top is both an object~-- a thing with shape, mass, material~-- and the spinning that keeps it upright. Remove the spin and the top falls. Grammaticality is the spin. It is both the pattern (the form--value coupling) and the force that maintains the pattern (the feeling that triggers correction). The top keeps spinning because something keeps pushing.

If grammaticality is a maintained coupling and acceptability is a noisy detector, then the evidential status of the asterisk changes. It is not a report from an oracle; it is a signal from a measuring instrument~-- informative, but fallible, and calibrated to the community in which it was trained.


\section{Empirical signatures}
\label{sec:14:empirical}

If the feeling of ungrammaticality is a noisy detector~-- not an oracle~-- then laboratory studies should reveal its characteristic failure modes. Three bodies of evidence confirm this prediction.

\subsection{Satiation}

\term{Syntactic satiation} is the finding that repeated exposure to a marginally acceptable sentence can improve its perceived acceptability. \textcite{snyder2000grammaticality} demonstrated that sentences violating subjacency constraints (island violations) become more acceptable after multiple exposures, while sentences violating Principle A (anaphor binding) do not. The effect is selective: some constraints are susceptible to satiation; others are robust.

\textcite{Snyder2022} argues that satiation reveals the architecture of the detector. Constraints that depend on probabilistic expectations~-- how often has this pattern occurred?~-- are susceptible to frequency-based updating. Constraints that depend on structural impossibility~-- can this parse at all?~-- are not. The detector's response to the first type is Bayesian; its response to the second is categorical. This is exactly what we would expect if the feeling of ungrammaticality arises from comparing the input against entrenched form--value pairings: the more entrenched the pairing, the less susceptible to satiation.

\subsection{Individual differences}

Speakers differ systematically in their grammaticality thresholds. \textcite{dabrowska2012} shows that grammatical knowledge is not uniform across the adult population: speakers with less formal education, less exposure to complex syntactic constructions, and less metalinguistic training produce reliably different acceptability judgments~-- not just noise, but systematic divergence.

This is the prediction from community-indexed entrenchment. If grammaticality is what a community has entrenched, then members with different exposure histories will have different detectors. The variation is not random; it tracks social and educational ecologies. The ``grammar'' is not a Platonic object that speakers approximate; it is a statistical regularity that speakers instantiate to varying degrees.

\subsection{Garden-path recovery and beyond}

Processing difficulty is not grammaticality, but the detector often conflates them. \textcite{phillips2011grammatical} show that the parser exhibits \term{selective fallibility}: it is exquisitely sensitive to some constraints (island effects, Principle C) and systematically vulnerable to others (agreement attraction, NPI licensing). The pattern reveals an architecture, not noise.

Consider the recovery data. When readers encounter a garden-path~-- a sentence that is grammatical but initially misparsed~-- they produce high unacceptability ratings at the point of reanalysis, even though the sentence is well-formed. The detector reports grammaticality failure because processing cost correlates, in normal use, with form--value mismatch. The illusion is systematic: it follows from the design of a mechanism optimized for rapid, good-enough parsing, not from a failure of the grammar itself.

These three signatures~-- satiation, individual differences, and processing-based illusions~-- converge on the same conclusion. The feeling of ungrammaticality is a detector, not an oracle. It tracks the coupling between form and value, but it does so noisily, shaped by exposure and processing constraints. The asterisk is informative, but it is not infallible.


\subsection{Relevance and the immune system}
\label{sec:14:relevance}

\textcite{scottphillips2024communication} offers a complementary account. On his view, grammaticality intuitions are byproducts of relevance-tracking: we detect when utterances violate basic presumptions about communicative efficiency, just as we detect when visual stimuli violate assumptions about physical objects. Unacceptability arises not from mere inefficiency, but from an inherent impossibility of interpreting an utterance consistently with those presumptions.

The MMMG framework shares several premises with this account. Both reject the need for an innate grammar faculty; both locate intuitions within general cognitive systems; both recognize that communicative pressures shape linguistic conventions. The frameworks differ in their explanatory mechanisms. Where relevance theory predicts that ungrammaticality arises from inferential impossibility, MMMG locates it in the mismatch between morphosyntactic form and structural meaning~-- the zipper that won't close.

The difference has empirical consequences. MMMG predicts that ungrammaticality will be construction-specific: *\mention{I have twenty years} fails as an age statement because the English age construction is \mention{be} + \mention{years old}, not because relevance-seeking fails. Scott-Phillips's account predicts more general inferential failure. Both approaches agree that the feeling is a detector, not an oracle; they disagree about what the detector is tracking.

The accounts may ultimately converge. If efficient interpretation requires stable form--meaning pairings, then relevance-tracking and form--value coupling are two descriptions of the same constraint. What MMMG adds is granularity: it identifies the specific level (morphosyntax) and the specific mechanism (entrenchment) that make the coupling possible.


\section{The HPC audit}
\label{sec:14:hpc-audit}

Chapter~\ref{ch:failure-modes} introduced the Two-Diagnostic Test for genuine HPC kinds: high projectibility and robust homeostasis. Grammaticality passes both.

\paragraph{Projectibility.} Learning that one form--value pairing is grammatical lets you predict that similar pairings will be. If \mention{the dog bit \emph{him}} is grammatical, then \mention{the cat scratched \emph{her}} will be too~-- not because speakers have memorized every possible sentence, but because they have abstracted the construction and project its properties to novel instances. This is the epistemic payoff of categories: they license induction. Grammaticality is projectible because the form--value coupling is systematic.

\paragraph{Homeostasis.} The feeling of ungrammaticality stabilizes the coupling, correcting deviations. Speakers hesitate, repair, rephrase. Hearers ask for clarification, offer corrections, signal non-understanding. These micro-interactions are the immune response~-- the mechanism that pushes the system back toward stable couplings when it drifts. Grammaticality is homeostatic because the feeling is a stabilizer.

Grammaticality is therefore not merely a binary switch, nor a primitive of the theory. It is an HPC kind in its own right~-- maintained by the same mechanisms that maintain countability, definiteness, gender, and lexical categories. It passes the tests. It earns its category status.


\bigskip
\noindent\textsc{Definiteness thread.} The form--value coupling that constitutes grammaticality operates within domains we have already examined. In Chapter~\ref{ch:definiteness-and-deitality}, we saw that definiteness and deitality are two HPCs~-- one semantic, one morphosyntactic~-- coupled by bidirectional inference. That coupling is itself subject to the grammaticality constraint: *\mention{a the book} is not merely infelicitous but ungrammatical, because the form--value pairing for determiner stacking is not entrenched. The definiteness system sits inside the grammaticality system. The immune response that rejects *\mention{a the book} is the same immune response that rejects *\mention{I have twenty years}. Both are form--value mismatch at the morphosyntactic level.


\section{Looking forward}
\label{sec:14:looking-forward}

We began with an asterisk. We end with a question about what the asterisk is probing.

Grammaticality is not a primitive. It is not a direct readout of structure. It is not a line in the sand between possible and impossible. It is a maintained coupling~-- form and value, zipper teeth that mesh or don't~-- and the feeling that signals the mismatch is a noisy detector shaped by what a community has entrenched.

This reframing has consequences. If grammaticality is a category, then we can ask: how projectible is it? how robust is its homeostasis? And if grammaticality is a mechanism~-- the immune system of the grammar~-- then we can ask: what maintains it? what degrades it? what illusions does it produce?

The dual role~-- category and mechanism, maintainer and maintained~-- is not paradoxical. It is characteristic of homeostatic systems. The spinning top is both the object and the spin that keeps it upright. Grammaticality is the spin.

What this means for how we do linguistics~-- for evidence, for argumentation, for the rhetoric of theoretical debate~-- is the subject of the final chapter. The asterisk is a probe into the standing wave. Chapter~\ref{ch:what-changes} asks what happens when we take that probe seriously.

