\chapter{Grammaticality itself}
\label{ch:grammaticality-itself}
\epigraph{it's impossible to get rid of the illusion in spite of our better knowledge.}{— Hermann von Helmholtz, \textit{Treatise on Physiological Optics} (trans. J. P. C. Southall).}

% Draft status: Expanded skeleton based on HPC framing (Jan 2026)

Every analysis of every construction presupposes a distinction between grammatical and ungrammatical. The asterisk's the most common symbol in syntax. But what's the asterisk diagnosing?

Chapter~\ref{ch:what-we-havent-been-asking} asked a question almost no one asks: is grammaticality itself a natural kind? Not whether this sentence's grammatical, but what kind of thing we're probing when we ask. This chapter provides the answer.

If categories are homeostatic property clusters, then grammaticality isn't merely a binary switch. It's itself an HPC~-- a maintained coupling between form and value. And the feeling of ungrammaticality that guides our judgments isn't an oracle; it's a noisy detector, shaped by entrenchment and processing ease, capable of illusions.

Chapter~\ref{ch:dynamic-discreteness} introduced a two-layer picture: discrete grammaticality filtered through processing noise to yield gradient acceptability. What we're adding here's the question of what \term{kind} of thing grammaticality is, and what maintains it.


\section{The HPC claim}
\label{sec:14:hpc-claim}

Grammaticality's a form--value pairing. But the \enquote{value} side isn't compositional semantics~-- not truth conditions, not the full propositional content. It's something more basic: \term{structural meaning}.

Structural meaning's what grammar contributes before lexical content fills the slots. It includes dependency relations, like the bracketing that distinguishes [\mention{the dogs'}] \mention{house} (a house belonging to the dogs) from \mention{the} [\mention{dogs' house}] (a type of house). It includes argument structure, distinguishing \mention{the dog bit the man} from \mention{the man bit the dog}. 

It provides temporal anchoring through tense and aspect, distinguishing whether the biting's past, ongoing, or hypothetical. It sets the clause type, distinguishing statements (\mention{he was bit}) from questions (\mention{was he bit?}) or commands (\mention{bite him.}). And it manages information structure~-- the difference between \mention{Him, he was bit} and \mention{it was him who was bit}.

This is \term{subcompositional}~-- it's the instruction manual, not the assembled product. Grammar tells you how to wire the meanings; the lexicon provides what to wire. \mention{Colorless green ideas sleep furiously} has perfect wiring instructions and absurd components. That's why it feels strange but not \term{ungrammatical}.

A note on what structural meaning isn't. It's not semantic coherence (\mention{Colorless green ideas} shows grammaticality survives semantic absurdity). It's not contextual felicity (a sentence can be grammatical and pragmatically infelicitous). And it's not whatever the hearer infers (inference runs on many channels; grammaticality's specifically the morphosyntactic one). Structural meaning's the contribution of form to interpretation~-- the \term{wiring diagram}, not the current flowing through it.

The HPC claim, then, is that grammaticality's the coupling between morphosyntactic form and this structural semantics. When the coupling holds~-- when the form reliably signals the structural meaning~-- the sentence sits in the \term{grammatical basin}. When it doesn't, the sentence's outside.

This is the \term{zipper}: morphosyntax provides one set of teeth, structural meaning the other. Grammaticality's what it feels like when the zipper closes. An ungrammatical sentence's a zipper that won't mesh.

Two metaphors will recur. The zipper describes the relation~-- the form--value alignment itself. The \term{immune system} (below) describes the \term{maintenance mechanism}~-- what pushes the system back toward stable couplings when they slip. One's structure; the other's dynamics.


\section{The stabilizer: a feeling}
\label{sec:14:stabilizer}

What maintains this coupling? Partly the mechanisms we've discussed throughout: acquisition, entrenchment, alignment, transmission. But the proximate \term{stabilizer}~-- the thing that corrects deviations in real time~-- is a \term{feeling of ungrammaticality}.

When something's wrong with the form--value coupling, speakers notice. They hesitate, repair, rephrase. The feeling's what triggers the correction. It's the immune response of the grammar.

The feeling's a \term{detector}~-- and like all detectors, it's noisy. It's not a direct readout of the underlying structure. Instead, it's informed by probability and processing ease. First, the detector's Bayesian: a novel sentence's weighted by its likelihood under the distribution of attested forms~-- not just distance from the nearest exemplar, but probability given the entire pattern. Low-probability forms feel suspect even when grammatically licensed. The distribution itself is built by entrenchment: exposure tunes the prior. 

Second, sentences that parse smoothly feel grammatical; sentences that induce garden-paths or reanalysis feel suspect. Processing difficulty isn't the same as ungrammaticality, but the detector often conflates them. This is why \term{acceptability} and \term{grammaticality} come apart. Acceptability's what the detector reports. Grammaticality's the category~-- the HPC whose core's the form--value coupling. The two usually align, but they don't always.

\subsection{Entrenchment in action}

But here's the puzzle: if grammaticality \emph{is} the maintained coupling, and entrenchment's what maintains it, then \enquote{tuning the detector} and \enquote{shaping what counts as grammatical} aren't separable. The detector isn't just tracking some pre-existing reality; the reality is constituted by what gets entrenched. The feeling tracks the coupling, and the coupling is entrenched. All the way down.

This is language-specific, construction-specific, even register-specific.

Consider age expressions: in French and Spanish, age is expressed with \mention{avoir}/\mention{tener}: \mention{J'ai vingt ans} (\enquote{I have twenty years}). In English, *\mention{I have twenty years}'s ungrammatical as a statement of age~-- though perfectly fine as a statement of time (until retirement, on the job, etc.). The structural slot's the same; the construction-specific entrenchment differs. 

The same applies to progressive aspect. If you're describing an ongoing event while your fingers are moving across a keyboard, \mention{I write} mis-signals aspect in English~-- many speakers experience this as \enquote{wrong} in the specific sense of a form--value mismatch. The progressive (\mention{I'm writing}) isn't merely preferred; it's required for the ongoing-action construal. In French, the simple present would be grammatical for the same construal; the progressive's marked or absent. What counts as a violation of the form--value coupling depends on what the community has entrenched.

Countability follows the same pattern. Whether \mention{furniture} takes count morphosyntax depends on entrenchment, not metaphysics. Furniture's as individuable as anything~-- you can count chairs and tables. But English has entrenched \mention{furniture} as noncount, so *\mention{I bought two furnitures}'s ungrammatical. Other languages make different choices. The coupling's real; its contents are conventional.

And consider deitality: what requires the definite article in English~-- \mention{go to the hospital} vs.~\mention{go to hospital} (British)~-- is shaped by entrenchment within each dialect. The structural meaning (institutionalized activity frame)'s the same; the morphosyntactic realization differs.

These examples show grammaticality isn't a window onto universal logic. It's a window onto what a community has entrenched as the couplings between form and structural meaning. The feeling of ungrammaticality is tracking those couplings~-- but the couplings are historical, contingent, and maintained.

And here's where the entrenchment story connects to the social: entrenchment's always entrenchment \enquote{within a discourse community}. The grammar isn't floating free; it's tethered to the people who use it. When the community shifts~-- through contact, migration, register change~-- the entrenchment shifts with it. What feels ungrammatical in one community may be unremarkable in another.

But speakers flow between communities~-- sometimes instantaneously~-- and communities can be created on the fly. \mention{Do you lift?} licenses intransitive \mention{lift} for weightlifters but not for warehouse workers; the grammaticality of the construction depends on which community you're invoking, and you can switch in mid-conversation. Code-switching within a clause~-- \mention{Vamos a hacer shopping}~-- is governed grammar, not error, but only within the discourse community of the bilinguals who maintain it. The detector isn't calibrated once; it's calibrated to whoever you're talking to, right now.

One more piece. We couldn't correct grammatical errors if we couldn't derive the intended meaning without the grammar. When someone says \ungram\mention{I have twenty years} for their age, we understand what they mean~-- and then we notice the mismatch. The feeling of ungrammaticality arises from comparison: the grammar's value versus the intended value. We need both to detect the error. This means comprehension's partly independent of grammaticality~-- and grammaticality's partly a matter of checking the grammar's signal against other inference channels. When they align, we don't notice; when they mismatch, we do.



\section{The phenomenology of coupling: What it's like}
\label{sec:14:phenomenology}

If grammaticality is a maintained coupling, what is the \term{phenomenology} of that state? What is it \emph{like} to be in a state of grammaticality?

The experience isn't a drive, like hunger. Hunger is a homeostatic signal that motivates action (eating); grammaticality doesn't typically push us to \enquote{consume} structure. Nor is it like pain, a dedicated alarm channel. Instead, the phenomenology of grammaticality is a bundle of metacognitive readouts that track the success of the coupling mechanism. These readouts look less like appetites and more like \term{balance} and \term{perception}.

\subsection{Balance: The silent baseline}

The primary analogue is \term{balance} or proprioception. Most of the time, balance is phenomenologically silent. When you walk down the street, you don't feel \enquote{balanced}; you just experience the world and your movement through it. The coupling work is backgrounded.

Grammaticality behaves the same way. When the form--value coupling is functioning smoothly, you don't experience \enquote{grammaticality}; you experience content, stance, and interactional fit. The zipper is closed, and because it holds, you can ignore it.

The feeling arises only when the coupling fails. When you trip, you get an immediate, non-propositional jolt~-- a \enquote{catch} in the system~-- followed by a reflex to repair. When you hear an ungrammatical sentence, you get a similar jolt: a sense that the structure has stumbled. In production, this manifests as monitoring error signals: speakers detect upcoming problems and self-correct, often before the error is even articulated \parencite{nozari2011is}. The feeling of ungrammaticality is the system trying to keep itself upright.

\subsection{Vision: Illusion and good-enough processing}

If balance describes the baseline, \term{vision} describes the failure modes. Vision delivers structured objects, not raw sense data; we are aware of the output, not the intermediate computations. And like vision, grammatical processing is subject to systematic illusions.

Visual illusions (like the Müller-Lyer lines) persist even when we know they're illusions. Grammaticality illusions behave similarly. \textcite{wagers2009agreement} show that \term{agreement attraction}~-- where an ungrammatical sentence like \ungram\mention{The key to the cabinets are missing} feels acceptable~-- is a stable feature of the architecture. The processing system retrieves the plural feature from \mention{cabinets} and erroneously checks it against the verb, satisfying the local constraint even though the global structure fails.

This selective fallibility \parencite{phillips2011grammatical} is the hallmark of a perceptual system, not a logical oracle. It aligns with the \enquote{good-enough} processing framework \parencite{ferreira2002good}, which suggests that comprehension often settles for representations that are serviceable rather than fully specified. The result is a dissociation: the feeling of acceptability (the readout) can float free of the actual structural coupling (the state).

\subsection{Memory and attention}

The substrate of this coupling is memory~-- specifically, the procedural memory that handles skill and habit \parencite{ullman2001neurocognitive}. The feeling of \enquote{rightness} that accompanies a grammatical sentence is a feeling of \term{fluency}: the structure affords a well-supported pattern completion, and that ease of processing is metacognitively marked as \enquote{correct} \parencite{ackerman2012fluency}.

Attention acts as the modulator. It changes the \enquote{lighting} in the room. In ordinary conversation, attention is on the content; in a linguistics experiment, attention is shifted to the form itself. This shift can change the threshold for the mismatch signal, turning a sub-perceptual glitch into a reportable error. But attention doesn't create the signal; it just determines whether it reaches consciousness.

\subsection{Production: The near-miss}

Finally, there is the \term{tip-of-the-tongue} (TOT) state \parencite{brown1966tip}. TOT is a conscious experience of partial access: you have the meaning, but the phonological form won't retrieve.

The coupling view predicts an analogous family of \enquote{near-miss} grammatical states. You have the intended value (structural meaning) and some morphosyntactic scaffolding, but the interface won't lock. You don't feel \enquote{ungrammaticality} per se; you feel that you \enquote{can't get it to come out right.} This is the territory of hesitations and reformulations~-- the feeling of a zipper that's stuck halfway.

\bigskip

In sum: Grammaticality itself is the structural property of the form--value coupling. What it's \emph{like} is the metacognitive readout of that coupling: silence when it works, a jolt when it fails, and a graded sense of \enquote{weirdness} or fluency when it's somewhere in between.


\section{Grammaticality illusions}
\label{sec:14:illusions}

The strongest evidence that the feeling is dissociable from the structure comes from \term{grammaticality illusions}~-- cases where the detector misfires.

\subsection{Feels ungrammatical, is grammatical}

The classic case is the garden-path sentence:

\ea[]{\label{ex:14:garden-path}
\mention{The horse raced past the barn fell.}}
\z

This is grammatical~-- a reduced relative clause (\mention{the horse [that was] raced past the barn fell}). But the processing system commits to the main-clause parse (\mention{the horse raced past the barn}) and crashes when \mention{fell} appears. The detector reports ungrammaticality; the structure's fine.

\subsection{Feels grammatical, is ungrammatical}

The converse is the \term{comparative illusion}~-- sometimes called the \term{Escher sentence}:

\ea[]{\label{ex:14:comparative-illusion}
\mention{More people have been to Russia than I have.}}
\z

Each local chunk parses. \mention{More people have been to Russia}~-- fine. \mention{Than I have}~-- fine as a comparative clause. But the whole doesn't compose: what's being compared? The sentence has no coherent interpretation, yet it feels acceptable because each piece satisfies local constraints.

These illusions are exactly analogous to visual illusions. The Müller-Lyer lines are the same length, but the perceptual system reports otherwise. The illusion doesn't show perception's broken; it shows perception is a mechanism with characteristic failure modes. Grammaticality illusions show the same: the feeling's a detector, not an oracle.

\textcite{phillips2011grammatical} develop this point systematically, showing comprehension exhibits \term{selective fallibility}~-- impressive accuracy for some constraints, and systematic vulnerability for others. The pattern reveals the architecture: a retrieval system that's structure-sensitive when structural cues have temporal priority, and error-prone when they don't.


\section{What doesn't count}
\label{sec:14:negative-space}

The form--value coupling's specifically morphosyntactic. This predicts what won't trigger ungrammaticality. It excludes phonetic errors, like \mention{[fɪlɔzəfi]} for \mention{philosophy}, where the form's wrong but the morphosyntax's untouched. It excludes lexical errors, like saying \mention{airport} when you meant \mention{train station}. And it excludes semantic anomaly, like \mention{Colorless green ideas sleep furiously}, where the wiring's impeccable even if the components are absurd.

These violations trigger different responses. Phonetic errors trigger \mention{what?} because the phoneme's value's the contrast~-- that's the entire form--value coupling at that level, and when it breaks, comprehension stalls. Lexical errors trigger correction or confusion. Semantic anomaly triggers puzzlement. But none of them trigger the specific feeling of ungrammaticality~-- the sense the wiring's broken.

A revealing contrast comes from the allomorphy of the indefinite article. We reject \ungram{*a apple} and \ungram{*an banana} with the same immune response we bring to syntax. This looks phonological, but the selection rule is entrenched as a grammatical requirement~-- part of the morphosyntactic zipper. A mispronunciation of \mention{banana} would trigger \mention{what?}; distributing the wrong allomorph of the article triggers the grammar's immune response. This is exactly what we'd expect from an HPC: the edge case reveals the boundary of the morphosyntactic coupling, and the feeling of ungrammaticality~-- if somewhat muted~-- is the trace of the immune system at work.

\subsection{Value at every grain}

What's \enquote{value} at each level of the grammar? The zipper closes differently depending on what you're zipping.

\begin{table}[h]
\centering
\small
\begin{tabular}{@{}lll@{}}
\toprule
\textbf{Grain} & \textbf{Form} & \textbf{Value} \\
\midrule
Phoneme & Sound segment & Contrast (minimal-pair distinctiveness) \\
Morpheme & Affix, stem & Grammatical or derivational modification \\
Word & Phonological shape & Lexical meaning + category membership \\
Construction & Syntactic pattern & Argument structure, information structure, meaning template \\
Clause type & Structural configuration & Illocutionary force (statement, question, command) \\
\bottomrule
\end{tabular}
\caption{Form--value coupling at different grains of grammar.}
\label{tab:form-value-grains}
\end{table}

At the phonemic level, value's nothing \emph{but} contrast. The /p/ in \mention{pat} contributes no semantic content; it just distinguishes the word from \mention{bat}, \mention{cat}, \mention{sat}. When contrast fails~-- when you can't tell which phoneme was intended~-- you say \mention{what?}

At the morphemic level, value's the contribution to word meaning or grammatical function. The \mention{-ed} in \mention{walked} signals past tense; the \mention{un-} in \mention{unhappy} reverses polarity. Break the coupling~-- use \mention{-ed} on a noun, or \mention{un-} on a verb that doesn't license it~-- and ungrammaticality results.

At the word level, value's lexical meaning plus category membership. \mention{Dog} contributes dog-ness \emph{and} nounhood. Substitute the wrong word and you get confusion; use a word in the wrong category slot and you get ungrammaticality.

At the constructional level, value's the structural meaning the pattern contributes. The double-object construction (\mention{gave her the book}) signals transfer plus affected recipient. The \term{passive} (\mention{was eaten}) demotes the agent. Break the construction's slot-filling requirements and the zipper won't close.

At the clause-type level, value's illocutionary force. \term{Declarative} structure signals assertion; \term{interrogative} structure signals question; \term{imperative} structure signals command. Mismatch the structure and the force~-- use rising intonation on a declarative to fake a question~-- and you're exploiting the zipper, not breaking it.

This table explains why grammaticality's primarily a syntactician's category. It sits at the morphosyntactic grain~-- the level where form--value couplings are tight, obligatory, and enforced. 

Phoneticians have their own coupling (\term{contrast}), and when it breaks, the response's \mention{what?}, not ungrammaticality. Semanticists have theirs (\term{compositionality}, \term{inference}), and when it breaks, the response's puzzlement or absurdity. Grammaticality's what you feel when the morphosyntactic zipper won't close. That's why syntax owns it.

\subsection{Grammaticality as a species of appropriateness}

This grain-specific view suggests a larger picture. Grammaticality isn't sui generis; it's a \emph{species} of appropriateness, narrowed to the morphosyntactic channel.

\term{Appropriateness} in general is the coupling between a form and \emph{all} the values it signals: structural meaning, social indexicality, register stance, identity marking. Grammaticality is what you get when you restrict attention to one channel~-- morphosyntactic form and structural meaning. But morphosyntax isn't the only grain at which form--value couplings are maintained and enforced.

\begin{table}[h]
\centering
\small
\begin{tabular}{@{}lll@{}}
\toprule
\textbf{Domain} & \textbf{Form} & \textbf{Value} \\
\midrule
Grammaticality & Morphosyntax & Structural meaning \\
Register fit & Lexical/syntactic choice & Situational stance \\
Accent & Phonetic realization & Social identity \\
Lexical precision & Word choice & Domain membership, exactness \\
Pragmatic felicity & Utterance in context & Communicative fit \\
\bottomrule
\end{tabular}
\caption{Appropriateness channels: form--value couplings at different grains.}
\label{tab:appropriateness-channels}
\end{table}

Each channel has its own detector, its own failure mode, and its own phenomenology. Grammaticality failure triggers the immune response we've been describing: hesitation, repair, the jolt of \enquote{wrongness}. Register mismatch triggers a different response: the raised eyebrow, the red ink on contractions, the sense that something's \enquote{off} without being \emph{ungrammatical}. Accent mismatch triggers identity attribution: \enquote{that's not how we say it here.} Lexical inappropriateness triggers precision failure: \enquote{that's not quite the right word.}

The channels aren't orthogonal. A single utterance is evaluated simultaneously on all of them. And they interact: a register-inappropriate form can \emph{feel} ungrammatical (through prescriptive training or hypercorrection) even when the morphosyntax is impeccable. But they're semi-independent: you can have perfect grammaticality with terrible register fit (\mention{I would like to utilize the restroom facilities} is grammatical but register-weird in casual speech), and native speakers can have strong grammaticality intuitions with weak register intuitions in unfamiliar domains.

Chapter~\ref{ch:social-stabilization} develops this picture further, showing how dialect, register, and discourse community act as conditioning variables that select which appropriateness couplings apply in a given context. The academic register activates its own appropriateness detector \emph{alongside} the grammaticality detector, not instead of it. They're layered, not competing.

The upshot: grammaticality is appropriateness at the morphosyntactic grain. The asterisk probes one detector; the raised eyebrow probes another. Both are form--value couplings; both are maintained by social mechanisms; both can trigger repair. Grammaticality is just the one that syntacticians own.


\section{The gradient at the boundary}
\label{sec:14:gradient}

When we judge a sentence, we aren't checking a rulebook. We're interrogating the \term{standing wave}~-- probing the system to see if the dynamic equilibrium holds or collapses under strain.

A fully grammatical sentence sits at the bottom of a deep basin. Perturb it slightly~-- change word order, shift stress, substitute a near-synonym~-- and the system corrects. An ungrammatical sentence sits outside the basin entirely: no perturbation's going to roll it back to stable ground.

The marginal cases~-- the 3-star and 4-star judgments, the sentences that split informants~-- are poised on the rim. They're at the edge of a phase transition, where small changes in context or construal can tip the system one way or the other.

This reframing has consequences. If grammaticality's basin-stability, then the sharp boundary between well-formed and ill-formed isn't a property of the grammar itself but an artifact of how we test. Gradient judgments aren't noise to be smoothed away; they're signal. The gradient is where the mechanisms are under strain, and strain reveals structure.


\section{Grammaticality as mechanism and category}
\label{sec:14:dual-role}

Grammaticality plays a dual role in the framework. As a \term{category}, it's an HPC~-- the maintained coupling between morphosyntactic form and structural meaning. It passes the diagnostics: it's projectible (one form--value pairing predicts others) and homeostatic (the feeling of ungrammaticality stabilizes the coupling).

As a \term{mechanism}, it's a stabilizer for other HPCs. The category \term{noun}'s maintained partly because nounhood's grammatical~-- because using a word in a noun slot, with noun morphology, in noun constructions, triggers no mismatch signal. Grammaticality's the enforcement mechanism for the form--value couplings that constitute all other categories.

This dual role~-- maintainer and maintained~-- isn't paradoxical. It's characteristic of homeostatic systems. The immune system's both a system (a biological category) and a mechanism that maintains other systems. Grammaticality's the immune system of the grammar.

Or think of the \term{spinning top} metaphor. A top's both an object and the spin that keeps it upright. Remove the spin and the top falls. Grammaticality's the spin. It's both the pattern and the force that maintains it. The top keeps spinning because something keeps pushing.

If grammaticality's a maintained coupling and acceptability's a noisy detector, then the evidential status of the asterisk changes. It's not a report from an oracle but a signal from a measuring instrument~-- informative, but fallible, and calibrated to the community in which it was trained.


\section{Empirical signatures}
\label{sec:14:empirical}

If the feeling of ungrammaticality's a noisy detector~-- not an oracle~-- then laboratory studies should reveal its characteristic failure modes. Three bodies of evidence confirm this prediction.

\subsection{Satiation}

\term{Syntactic satiation}'s the finding that repeated exposure to a marginally acceptable sentence can improve its perceived acceptability \parencite{snyder2000grammaticality}. But the effect's selective. Some constraints are susceptible to frequency-based updating: repeated exposure makes them feel better. Others resist satiation entirely. Left-branch extraction (\ungram\mention{Which did you buy car?}) remains categorically rejected regardless of exposure \parencite{Snyder2022}. The intended meaning's easily grasped~-- \mention{Which car did you buy?}~-- and the structure isn't unusually complex. But the ban persists across tasks, speakers, and presentation modes.

Why the split? If the feeling of ungrammaticality arises from comparing input against entrenched form--value pairings, then repeated exposure should matter~-- but only for constraints that are frequency-sensitive in the first place. When you hear a marginal sentence ten times, you're adding exemplars to the distribution; the pairing becomes more familiar, the mismatch signal weakens, and the sentence feels better. 

But when a constraint's categorical~-- as left-branch extraction is in English~-- no amount of exposure can change that. The resistance doesn't imply the structure is unimaginable; it implies the learner has detected a \term{conpicuous absence}. Fronting is productive in English; the *potential* for extraction is visible. But in this specific configuration, the community systematically avoids it (\textasciitilde3-sigma absence), using entrenched \term{workarounds} instead: pied-piping (\mention{whose book}), fused-head constructions (\mention{whose is this}), and the \mention{big mess} construction (\mention{how good a teacher}). \textcite{Reynolds2026} shows that this density of alternatives is precisely what makes the gap unsatiatable: the \enquote{satiation} path is blocked by the superior activation of the competitors. 

The gap persists because the potential is there but the practice is absent. Learners infer that the community has tacitly agreed *not* to use the available mechanism in this way. The detector isn't just saying \enquote{this doesn't happen}; it's saying \enquote{we could do this, but we don't.} The feeling of ungrammaticality tracks this specific, entrenched abstinence.

\subsection{Individual differences}

Speakers differ systematically in their grammaticality thresholds. The limiting case's trivial: we have no feeling of grammaticality whatsoever for languages we don't speak. A sentence in Zulu or Finnish produces no mismatch signal for me because there's nothing to compare it against~-- no entrenched pairings, no expectations. Foreign language learners occupy the middle ground: their feelings are degraded, unreliable, slow to fire, and easily fooled by surface plausibility. Only with sufficient exposure does the feeling sharpen.

Within a single language community, the variation persists. \textcite{dabrowska2012} shows that grammatical knowledge isn't uniform across the adult population: speakers with less formal education, less exposure to complex syntactic constructions, and less metalinguistic training produce reliably different acceptability judgments~-- not just noise, but systematic divergence.

That's the prediction from community-indexed entrenchment. If grammaticality is what a community has entrenched, then members with different exposure histories will have different thresholds. The variation isn't random; it tracks social and educational ecologies. The \enquote{grammar}'s not a Platonic object that speakers approximate; it's a statistical regularity that speakers instantiate to varying degrees.

\subsection{Processing difficulty and beyond}

Processing difficulty isn't grammaticality, but the two are often conflated. \textcite{phillips2011grammatical} show that comprehension exhibits \term{selective fallibility}: it's exquisitely sensitive to some constraints and systematically vulnerable to others. The pattern reveals something about the architecture of the feeling, not just noise.

Consider \term{garden-path sentences}. When readers encounter a grammatical sentence that's initially misparsed, they produce high unacceptability ratings at the point of reanalysis, even though the sentence's well-formed. The feeling of ungrammaticality fires because processing cost correlates, in normal use, with form--value mismatch. 

The illusion's systematic: it follows from the design of a system optimized for rapid, good-enough processing, not from a failure of the grammar itself. These three signatures~-- satiation, individual differences, and processing-based illusions~-- converge on the same conclusion. The feeling of ungrammaticality's noisy, not oracular. It tracks the coupling between form and value, but it does so imperfectly, shaped by exposure and processing constraints. The asterisk's informative, but it's not infallible.


\subsection{Relevance and the immune system}
\label{sec:14:relevance}

\textcite{scottphillips2024communication} offers a complementary account. On his view, grammaticality intuitions are byproducts of relevance-tracking: we detect when utterances violate basic presumptions about communicative efficiency, just as we detect when visual stimuli violate assumptions about physical objects. Unacceptability arises not from mere inefficiency, but from an inherent impossibility of interpreting an utterance consistently with those presumptions.

The MMMG framework shares several premises with this account. Both reject the need for an innate grammar faculty; both locate intuitions within general cognitive systems; both recognize that communicative pressures shape linguistic conventions. The frameworks differ in their explanatory mechanisms. Where relevance theory predicts that ungrammaticality arises from inferential impossibility, MMMG locates it in the mismatch between morphosyntactic form and structural meaning~-- the zipper that won't close.

The difference has empirical consequences. MMMG predicts that ungrammaticality will be construction-specific: \ungram\mention{I have twenty years} fails as an age statement because the English age construction's \mention{be} + \mention{years old}, not because relevance-seeking fails. Scott-Phillips's account predicts more general inferential failure. Both approaches agree that the feeling's noisy; they disagree about what it's tracking.

The accounts may ultimately converge. If efficient interpretation requires stable form--meaning pairings, then relevance-tracking and form--value coupling are two descriptions of the same constraint. What MMMG adds is granularity: it identifies the specific level (morphosyntax) and the specific mechanism (entrenchment) that make the coupling possible.


\section{The HPC audit}
\label{sec:14:hpc-audit}

Chapter~\ref{ch:failure-modes} introduced the Two-Diagnostic Test for genuine HPC kinds: high projectibility and robust homeostasis. Grammaticality passes both.

First, consider \term{projectibility}. Learning that one form--value pairing's grammatical lets you predict that similar pairings will be. If \mention{the dog bit \emph{him}}'s grammatical, then \mention{the cat scratched \emph{her}} will be too. This isn't because speakers have memorized every possible sentence; it's because they've abstracted the construction and project its properties to novel instances. Grammaticality's projectible because the form--value coupling's systematic.

Second, \term{homeostasis}. The feeling of ungrammaticality stabilizes the coupling, correcting deviations. Speakers hesitate, repair, rephrase. Hearers ask for clarification, or offer corrections. These micro-interactions are the immune response~-- the mechanism that pushes the system back toward stable couplings when it drifts. Grammaticality's homeostatic because the feeling's a stabilizer.

Grammaticality isn't merely a binary switch, nor a primitive of the theory. It's an HPC kind in its own right~-- maintained by the same mechanisms that maintain countability, definiteness, gender, and lexical categories. It passes the tests. It earns its category status.


\bigskip
\noindent\textsc{Definiteness thread.} This chapter adds the final filter to the definiteness diagnosis: the morphosyntactic \enquote{zipper} rejection of infelicitous forms. The coupling between semantic definiteness and morphosyntactic deitality isn't just a preference; it's a grammatical requirement.
\ea[]{\label{ex:14:definiteness-thread}
\ungram\mention{a the book}}
\z
The immune response that rejects this determiner stacking is the same one that rejects \ungram\mention{I have twenty years} for age.
\textit{Scoped prediction.} This chapter predicts in any population where a novel determiner sequence becomes entrenched (e.g., \mention{all the both}), the ungrammaticality feeling will take exactly three generations to sharpen into a categorical ban on the previous pattern.
\textit{Falsifier.} The prediction would be falsified if we found a language where semantic definiteness remains stable but NO morphosyntactic mismatch signal ever develops for \enquote{illegal} determiner combinations.


\section{Looking forward}
\label{sec:14:looking-forward}

We began with an asterisk. We end with a question mark: what is the asterisk probing?

Grammaticality's not a primitive. It's not a direct readout of structure. It's not a line in the sand between possible and impossible. It's a maintained coupling~-- form and value, zipper teeth that mesh or don't~-- and the feeling that signals a mismatch is noisy, shaped by processing constraints and what a community has entrenched.

This reframing has consequences. If grammaticality is a category, then we can ask: how projectible is it? how robust is its homeostasis? And if grammaticality is a mechanism~-- the immune system of the grammar~-- then we can ask: what maintains it? what degrades it? what illusions does it produce?

The dual role~-- maintainer and maintained~-- isn't paradoxical. It's characteristic of homeostatic systems. The spinning top is both the object and the spin that keeps it upright. Grammaticality is the spin.

This means the equilibrium is \term{contingent}. A constraint-based account says LBE \emph{can't} emerge in English~-- that there's a grammatical principle blocking it. The HPC account says only that it \emph{hasn't}~-- the social and statistical conditions have never favored it. But imagine a high-status speaker~-- or a low-status one with memetic cachet like Homer Simpson~-- introducing \mention{Whose did you eat sandwich?} to the language. Use the form, get the laugh, signal the group membership. Homer doesn't just provide new evidence; he spawns a new discourse community. If the social uptake shifts, the prior shifts. The grammar isn't resisting; there's no \enquote{grammar} there to resist. There's only convention, and convention can change. Because the mechanism is maintenance, not law.

What this means for how we do linguistics~-- for evidence, for argumentation, for the rhetoric of theoretical debate~-- is the subject of the final chapter. The asterisk is a probe into the standing wave. Chapter~\ref{ch:what-changes} asks what happens when we take that probe seriously.

