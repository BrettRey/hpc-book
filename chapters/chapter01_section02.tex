% Chapter 1, Section 1.2: Essentialism examined

\section{Essentialism examined}

Chomsky, in his 1965 \textit{Aspects of the Theory of Syntax}, characterizes traditional universal grammar as holding that \enquote{certain fixed syntactic categories (Noun, Verb, etc.) can be found in the syntactic representations of the sentences of any language} \citep[28]{chomsky1965}. He presents this not as a thesis requiring defence but as a starting point~-- what any adequate theory must accommodate. And for decades, that's how linguistics proceeded.

Essentialism is the natural view. It's where you start if no one has told you there's a problem.

The intuition runs deep. When a child learns that \term{dog} picks out dogs and not cats, she has grasped something about boundaries. Dogs are one kind of thing; cats are another; the boundary isn't arbitrary. When a student learns that \term{run} is a verb and \term{quick} is an adjective, she's learning something similar~-- or so it seems. Verbs are one kind of thing; adjectives are another; the grammar cares about the difference.

The philosophical roots go back to Aristotle. A category, on the classical view, is defined by a set of properties that are individually necessary and jointly sufficient. To be a triangle is to be a closed plane figure with three straight sides. Anything that satisfies those conditions is a triangle; anything that doesn't, isn't. There are no borderline triangles, no gradient membership, no triangles that are more triangular than others. The definition carves nature at its joints.

That framework remained remarkably stable. In the second century BCE, Dionysius Thrax codified Greek grammar into eight exhaustive parts of speech, each defined by necessary features~-- nouns signifying substances, verbs carrying tense. Medieval Modistae sharpened this into explicit ontology: nouns denoted permanent \textit{modi essendi}, verbs temporal ones. Renaissance humanists and the Port-Royal grammarians carried the program forward, treating word classes as reflections of universal thought. When Bloomfield wrote his 1933 \textit{Language}, he was not inventing a taxonomy but formalizing a two-thousand-year-old assumption.


The structuralist inheritance brought this view into modern linguistics. Leonard Bloomfield, writing in 1933, characterized the phoneme as \enquote{a minimum unit of distinctive sound-feature{\ldots} The speaker has been trained to make sound-producing movements in such a way that the phoneme features will be present in the sound waves, and he has been trained to respond only to these features} \citep[79]{bloomfield1933}. The phoneme, on this view, is a real unit with identifiable acoustic properties~-- discrete, bounded, essential. Bloomfield treated morphemes and word classes the same way: each a fixed category with clear membership conditions. His program made linguistic analysis a matter of identifying and taxonomizing these essential units.

Applied to language, this yields a picture that most working linguists absorb without being taught it explicitly. A noun is whatever satisfies the criteria for nounhood~-- it inflects for number, it heads noun phrases, it can be modified by adjectives. A verb is whatever satisfies the criteria for verbhood~-- it inflects for tense, it takes arguments, it heads verb phrases. The criteria might be debated; the assumption that there \textit{are} criteria, that categories have definitions, is rarely questioned. Even linguists who reject essentialism in principle often write as if it were true, because the alternative is hard to operationalize.

The essentialist picture has done real work. The great descriptive grammars~-- Jespersen, Quirk, the Cambridge Grammar itself~-- proceed category by category, laying out the membership criteria, cataloguing the members, noting the exceptions. The exceptions are always there, but they're handled as exceptions: marginal cases, historical residue, items in transition. The core of each category is secure. Textbooks are organized around this architecture. Pedagogical grammars depend on it. Parsers are built on it. The infrastructure of linguistic analysis presupposes that categories have boundaries and that the boundaries can, in principle, be found.

The strongest modern version of essentialism isn't naive. Generative grammar, in particular, has developed sophisticated accounts of category structure. Categories are defined by bundles of features~-- [+N], [−V], and so on~-- and the features are claimed to be universal, part of the innate language faculty. On this view, the category system isn't learned; it's given in advance. What children acquire is which words go in which slots, not the slots themselves. The boundaries are sharp because they're hardwired.

Mark Baker's \textit{Lexical Categories} (2003) provides a recent, explicit defense of this position. Baker argues that Noun, Verb, and Adjective are not arbitrary or language-specific groupings but natural categories with essences rooted in syntactic roles. His approach is, in his words, \enquote{formal, syntax-oriented, and universal, as opposed to the functionalist, semantic, and relativist approaches} \citep[ix]{baker2003}. Nouns, on his account, bear referential indices; verbs license specifiers; adjectives predicate properties. These defining properties hold across all languages. Surface variation gets explained via parameterized syntax, but the underlying category types remain constant.

The same essentialist logic extended to semantics. Jerrold Katz, working in the 1960s and 70s, defended the classical theory of lexical concepts: each word's meaning is defined by a set of necessary and sufficient semantic features. The standard example was \term{bachelor}, analyzed as [+adult] [+male] [+unmarried]. Anything satisfying those features is a bachelor; anything lacking one isn't. Katz treated such decompositions as cognitively real~-- part of what speakers know when they know a word's meaning. The category \textsc{bachelor} has an essence: unmarried adult maleness. The definition carves the concept at its joints.

This is a powerful picture. It explains why categories are universal~-- every language has nouns and verbs~-- and why children converge on the same system despite variable input. It makes predictions: if categories are feature bundles, then certain combinations should be impossible, and certain patterns of acquisition should be observed. The predictions have been tested, debated, refined. Whatever its ultimate fate, generative essentialism is a serious research program, not a folk prejudice dressed up in jargon.

And yet.

The problem with essentialism isn't that it's wrong about everything. It's that it keeps encountering cases where the machinery doesn't fit~-- and the response is always the same. Add a stipulation. Create a subclass. Mark the item as exceptional. The exceptions accumulate, and at some point you have to ask whether the machinery is doing explanatory work or merely recording the failures of a prior commitment.

Return to \term{otherwise}. Huddleston's puzzlement wasn't a lapse. It was a report from the front lines. The word functions as an adverb (\term{think otherwise}), as an adjective (\term{the truth is quite otherwise}), as something conjunction-like (\term{do it or otherwise face the consequences}), and in constructions that resist any standard label (\term{the correctness or otherwise of the proposal}). What are the necessary and sufficient conditions for adverbhood that include \term{otherwise} in \term{think otherwise} and exclude it in \term{the correctness or otherwise}? There aren't any. Every definition either lets in too much or leaves out too much.

The essentialist response is predictable: \term{otherwise} is polysemous, multiply listed, historically complex. Each use gets its own entry, its own feature specification, its own subcategorization frame. The grammar doesn't fail; it proliferates. But this is bookkeeping, not explanation. We wanted to know what makes something an adverb. We got a list of contexts where \term{otherwise} behaves adverbially, and another list where it doesn't, and no principled account of why the word straddles the boundary instead of sitting on one side or the other.

Consider the broader pattern. In English, the major categories~-- noun, verb, adjective, adverb, preposition~-- are supposed to be definable by distributional and morphological criteria. But every category has items that meet some criteria and fail others:

\term{Fun} looks like a noun (\term{we had fun}) but increasingly takes degree modification like an adjective (\term{that was very fun}). The change is ongoing, and speakers differ. What is \term{fun}~-- noun, adjective, or in transit between them? The essentialist must say it's one or the other, perhaps with a secondary listing for the emerging use. But the very fact of the transition shows that the boundary isn't sharp. The word is moving across a space that essentialism says doesn't exist.

\term{Near} takes objects without a preceding \term{to} (\term{near the house}), which makes it look like a preposition. But it also has comparative and superlative forms (\term{nearer}, \term{nearest}), which makes it look like an adjective. Some grammars call it a preposition; some call it an adjective; some, including CGEL, call it both~-- a word that belongs to two categories. But if categories are defined by necessary and sufficient conditions, how can a single word satisfy two incompatible definitions? The answer is that the definitions weren't necessary and sufficient after all. They were rough guides, and \term{near} falls in the gap between them.

\term{Like} has migrated from verb (\term{I like it}) to preposition (\term{people like you}) to conjunction (\term{like I said}), with each use attracting different degrees of acceptance depending on register and generation. The prescriptive tradition treats conjunction \term{like} as an error; the descriptive tradition documents its spread; neither can say what \term{like} is, because \term{like} is not one thing. It's a word with multiple category memberships, or a word in category flux, or a word that exposes the fiction of stable category boundaries. What it's not is a tidy example of how essentialism works.

The deeper problem is that the failures aren't random. They cluster in predictable places.

Items with high frequency and broad distribution are more likely to have mixed category profiles. This makes sense if category membership is reinforced by consistent behavior in consistent contexts: a word that shows up everywhere accumulates conflicting evidence about what it is. But essentialism has no account of this. Frequency isn't a feature. Distribution is a symptom, not a cause. The theory lacks the resources to explain why some words misbehave and others don't.

Items at the intersection of major categories are more likely to be disputed. The adjective--adverb boundary, the preposition--conjunction boundary, the noun--adjective boundary~-- these are chronic trouble spots, not because linguists are careless but because the boundaries themselves are unstable. Essentialism predicts sharp lines; the evidence shows gradients. The response is always the same: posit more subcategories, more features, more exceptions. But a theory that can accommodate any pattern by adding epicycles is a theory that predicts nothing.

Items undergoing historical change present a special problem. If categories have essences, then change should be abrupt: a word is one thing, then it's another. But the actual pattern is gradual. \term{While} was a noun (meaning 'period of time'), became an adverb, became a conjunction. Each stage left traces in the next. Essentialist grammars must treat each stage as a separate entry, missing the continuity that drives the change. The alternative~-- acknowledging that category membership is something a word can have more or less of~-- is precisely what essentialism forbids.

None of this means essentialism is useless. For most words, most of the time, the category labels work. \term{Dog} is a noun; \term{run} is a verb; \term{quickly} is an adverb. The labels support generalization. They organize grammars. They enable pedagogy. If every word were like \term{otherwise}, linguistic analysis would be impossible.

But not every word is like \term{dog}. And the question is what to make of the words that aren't. The essentialist answer~-- treat them as exceptions, lexical idiosyncrasies, marginal cases~-- works as long as the exceptions stay marginal. When the exceptions multiply, when the boundaries fray systematically, when the core cases start looking like special instances of a more general pattern of variation, then the answer stops working. The framework designed to handle the clear cases has nothing to say about the unclear ones, except that they're problems to be solved later, with better definitions.

This is where essentialism stands. It handles the core. It fumbles the periphery. And it has no account of why the periphery exists at all~-- why categories should have fuzzy edges if they're defined by sharp conditions.

The prototype tradition arose precisely to address this gap. Its answer: categories don't have sharp conditions. They have central tendencies, typical members, gradient structure. The fuzziness is real, and the theory should accommodate it rather than explain it away.

That answer has its own problems, which the next section addresses. But the prototype theorists saw something the essentialists missed: the boundaries weren't going to sharpen up with better definitions. The boundaries were evidence of something the definitions couldn't capture.
