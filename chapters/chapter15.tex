\chapter{What changes}
\label{ch:what-changes}

% TODO: Write chapter content
% NOTE: Chapter 6 (sec:6:what-framework-offers) references this chapter for agent-based modelling
%       as a methodological consequence of the maintenance view. See notes/chapter-feedback-deferred.md
%       for planned ABM content: testing basin dynamics, boundary cases, convergence vs homology, etc.

\section{The status of the framework}

What \emph{kind} of kind is an \term{HPC category}? It is a \term{second-order explanatory kind}: a kind whose members are themselves kinds, unified not by shared first-order properties but by a shared explanatory role and stabilization pattern. Its members are things like \term{species}, \term{chemical elements}, \term{phonemes}, and \term{constructions}~-- categories that support projectible generalizations, are stabilized by multiple partially independent mechanisms, and tolerate property variation without collapse.

What makes \term{HPC category} itself a kind is that these properties cluster reliably across domains. It is not a natural kind in the same sense as \term{gold}, nor a merely stipulative classificatory label, nor a purely philosophical artifact. It is a meta-level homeostatic cluster over explanatory practices.

This recursion is \emph{typed}, not flat. Compare \term{gene} (a kind whose instances are molecular entities) with \term{functional gene} (a kind whose instances are ways of carving interactions for explanation), or \term{model organism} (a kind whose instances are organisms plus institutional practices). None of these collapses because they occupy different explanatory grains. Likewise, \term{noun} is an HPC kind in English; \term{lexical category} is an HPC kind across languages; and \term{HPC category} is an HPC kind across scientific taxonomies.

Crucially, the higher-order kind is not stabilized by the same mechanisms as its members. HPC categories \emph{in the world} are stabilized by causal mechanisms, developmental pathways, and communicative coordination. The \term{HPC category} kind is stabilized by repeated success of explanation, methodological convergence across sciences, robustness under theory change, and the survival of the concept under critical scrutiny (against eliminativism or essentialism). Its homeostasis is epistemic and methodological rather than biological or physical.

This framing distinguishes the theory from a mere framework or model. Frameworks can be swapped without residue; HPC categories resist that. Once the pattern is identified~-- why strict definitions fail, why exceptions cluster~-- we can make reliable predictions about where gradience will appear, where sharp boundaries will re-emerge, and where eliminativist arguments will systematically overreach. That predictive success is exactly what licenses kindhood.

A natural objection is that if \term{HPC category} is itself an HPC kind, the theory is self-validating or circular. This would be true only if HPC theory claimed \emph{a priori} necessity. It does not. It makes a fallible empirical claim about the structure of successful scientific kinds. If HPC categories stopped supporting prediction, coordination, or explanation, the kind would dissolve~-- by its own lights. That is not circularity; it is reflexive risk.
