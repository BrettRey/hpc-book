% Chapter 7: The Stabilisers
% Complete rewrite 2025-12-08: biological explanatory style throughout

\chapter{The Stabilisers}
\label{ch:stabilisers}

\epigraph{The general rule would establish itself insensibly, and by slow degrees, in consequence of that love of analogy and similarity of sound, which is the foundation of by far the greater part of the rules of grammar.}{Adam Smith, \textit{Considerations Concerning the First Formation of Languages} (1761)}


\section{The cluster}
\label{sec:7:the-cluster}

Start where immunologists start: with the cluster.

When a biologist asks what a macrophage is, they don't look for a definition. They look for properties that tend to co-occur. A macrophage is picked out by a constellation: typical functions (phagocytosis, antigen presentation), marker profiles, morphologies, response tendencies, developmental origins. None of these is strictly necessary and sufficient. Some macrophages don't phagocytose; some share markers with dendritic cells; morphology varies with tissue context. But the properties cluster reliably enough to make the category useful -- to make predictions about new instances, to design experiments, to coordinate research programmes.

This is exactly the situation for grammatical categories.

Take nouns. What properties tend to co-occur? Items we call nouns typically take determinatives (\mention{the dog}, \mention{a problem}), pluralise (\mention{dogs}, \mention{problems}), function as heads of phrases in argument positions, refer to entities or entity-like abstractions, and enter into modification relations with adjectives. None of these is strictly necessary: \mention{cattle} doesn't pluralise; \mention{honesty} rarely takes \mention{a}; names resist determinatives in English; some nouns never appear in argument positions. And similar properties cluster under different labels in different languages -- Mandarin nouns don't pluralise the way English nouns do, and yet typologists recognise something noun-like in both systems.

The question is not: what is the essence of noun-ness? We tried that. It generated boundary disputes and competing definitions but no stable resolution.

The question is: what keeps the cluster clustered?


\section{Stabilisers at multiple scales}
\label{sec:7:stabilisers-at-scales}

When immunologists answer the analogous question for cell types, they don't point to a single cause. They narrate a multi-level story.

\textbf{Gene regulatory networks} produce relatively robust ``attractor'' states -- configurations the cell tends to settle into and return to after perturbation. These are not definitions; they are dynamical basins. The cell can be pushed around within a basin, but it takes a strong signal to tip it into a different one.

\textbf{Developmental lineage} constrains which basins are reachable. A cell's history matters: what precursor states it passed through, what signals it received at critical windows. The present state is not intelligible without the trajectory that led to it.

\textbf{Microenvironmental signalling} -- local cytokines, tissue context -- pushes cells into different regions of the same basin, producing variation within the type. A classically activated macrophage and an alternatively activated macrophage are both macrophages; the difference is which signals they've received and which region of the state space they occupy.

\textbf{Functional feedback}: what the cell does alters its microenvironment, which in turn sustains or shifts its state. The stabilisation is not one-way. The system is reciprocal.

Port this directly to language.

\textbf{Cognitive architectures and processing biases} are the analogue of gene regulatory networks. Some configurations of form-meaning pairings are easier to learn, store, and retrieve than others. Processing constraints produce attractor-like states: patterns that the system tends to fall into and return to. A construction that is frequent enough becomes entrenched -- retrieved as a unit rather than computed by rule. This is not definitional; it's dynamical.

\textbf{Acquisition pathways and learning biases} are the analogue of developmental lineage. A speaker's knowledge of English nouns was not installed by fiat; it was built up through exposure, starting with high-frequency exemplars and generalising outward. The order of acquisition matters. Early-learned patterns anchor later ones. Critical windows exist -- not strict Chomskyan critical periods, but sensitive phases where input has disproportionate effect.

\textbf{Discourse context and pragmatic ecology} are the analogue of microenvironmental signalling. A noun in subject position behaves differently from a noun in vocative position or in a compound. Same category, different activation state. The discourse environment pushes the token into a particular region of the distributional space. Variation is not noise to be explained away; it's expected, given the mechanisms.

\textbf{Functional feedback}: what speakers do with categories alters the input the next generation receives, which alters what categories stabilise. Usage entrenches representations; entrenched representations shape usage. The maintenance is not one-way. The system is reciprocal.

This is already a richer picture than \enquote{a noun is defined by such-and-such properties}. The definition tells you what the cluster looks like at a moment. The stabilising story tells you why it persists.


\section{Variation as activation states}
\label{sec:7:variation-activation-states}

The immunologist's key move: variation within a category is not embarrassing fuzziness to be explained away. It's the expected signature of a kind that is maintained by context-sensitive mechanisms.

A macrophage in inflamed tissue shows different markers than a macrophage in healthy tissue. Both are macrophages. The difference is activation state -- which signals the cell has received, which region of the phenotypic space it currently occupies. The category is real; the variation is real; both are explained by the stabilising dynamics.

Apply this to linguistic categories.

Consider the English word \mention{fun}. Is it a noun or an adjective? It takes determinatives like a noun (\mention{the fun we had}), but it also modifies nouns like an adjective (\mention{a fun party}) and resists pluralisation (*\mention{funs}). Textbook debates ask: which category is it really?

The immunologist's framing reinterprets the question. \mention{Fun} is a lexeme whose current state occupies a region of distributional space that overlaps two basins. It shows noun-like activation in some contexts, adjective-like activation in others. The category boundaries are not failed definitions; they are regions where multiple attractor states are close enough to produce mixed behaviour under different environmental signals.

This is not a dodge. It's a prediction: items in overlap regions should show higher inter-speaker variation, more sensitivity to discourse context, and faster historical change. And they do. \mention{Fun} is precisely the kind of word that shows register-dependent category behaviour and whose syntactic possibilities have shifted within living memory.

The same framing applies to aspect in Polish. As Chapter~\ref{ch:projectibility} showed, 90\% of verbs strongly prefer one aspect -- they're deep in a single basin. The remaining verbs occupy shallower regions where contextual signals matter more. The verbs that require explicit contextual cues to disambiguate are the ones where entrenchment is weaker, where the tense-frame signal has more work to do. That's the activation-state picture applied to grammatical aspect.


\section{One case in depth: the emergence of new quotatives}
\label{sec:7:one-case}

Rather than enumerate mechanisms abstractly, let's trace the full stabilising story for one category: the quotative marker.

Quotatives introduce reported speech, thought, gesture, or sound. Every language has had them for as long as there has been narrative. But when we look across typologically unrelated languages, we find a striking convergence in recent history: in the late twentieth century, new quotative forms emerged or expanded~-- forms with a distinctive profile that spread through similar populations and stabilised through similar mechanisms.

The English case is well documented. \textcite{butters1982} first noted \mention{be like} in American English in 1982, observing its use to report unuttered thoughts: \mention{I was like, `Let me live, Lord'}. \textcite{ferrarabell1995} hypothesised that the form originated with first-person subjects and internal dialogue before expanding to third-person subjects and direct speech. By the mid-1990s, the form had spread rapidly. \textcite{tagliamontedarcy2004} tracked apparent-time data from Canadian English: in 1995, speakers aged 18--27 used \mention{be like} for 13\% of quotative tokens; by 2003, speakers aged 17--19 used it for 63\%~-- a fivefold increase in under a decade. The same cohort, re-sampled seven years later, showed \mention{be like} rising from 13\% to 31\% as they aged into their thirties \autocite{tagliamontedarcy2007}. This is not age-grading; it's real-time change, with each generation carrying its quotative inventory into adulthood.

German shows a parallel trajectory. \textcite{golato2000} analysed \mention{und ich so} (`and I'm like') in video-recorded conversations, documenting its function in storytelling: introducing punchlines, sound effects, gestures, and verbal enactments~-- the same profile as English \mention{be like}. The form was attested in youth speech by the late 1990s \autocite{androutsopoulos1998} and has since become established among younger German speakers, though it remains more register-restricted than its English counterpart.

Japanese \mention{tte} (reduced from \mention{to itte}, `saying that') functions similarly: a phonologically light form that introduces reported speech, thought, or stance in informal narrative. The pattern extends to Turkish \mention{diye} (converb of \mention{demek}, `to say'), which has expanded from its older quotative functions into increasingly colloquial discourse contexts.

\textcite{buchstaller2014} synthesises the cross-linguistic evidence: the emergence of these innovative quotatives is not media-driven borrowing but independent development under similar functional pressures. Different sources, different structures, same functional niche, same stabilising dynamics.

This is not coincidence. It's convergent maintenance.

\subsection{The cluster}

What properties co-occur in these new quotatives? The distributional studies reveal a consistent profile.

\textbf{Non-lexical content.} These forms introduce material that goes beyond words: gesture, facial expression, tone of voice, inner monologue. \mention{And I was like} doesn't just report what someone said; it enacts the experience. \textcite{ferrarabell1995} noted that early \mention{be like} tokens disproportionately introduced internal states and non-lexicalised sounds~-- groans, sighs, exclamations~-- before expanding to verbatim-style direct speech. \textcite{golato2000} documented the same pattern for German \mention{so}: it turns reported speech into performance.

\textbf{First-person and present-tense preference.} The forms attach preferentially to first-person subjects and historical-present tense. \textcite{tagliamontedarcy2004} found that in their Toronto data, first-person contexts strongly favoured \mention{be like} over \mention{say}. When speakers narrate, they use these forms to bring the audience into the moment of the original event~-- the vivid first person, the historical present.

\textbf{Youth association.} In every documented case, younger speakers lead adoption. The apparent-time data are consistent across studies: the highest rates of \mention{be like} appear among speakers under 30; rates decline sharply for speakers over 50 \autocite{tagliamontedarcy2004,tagliamontedarcy2007}. Young women, in particular, tend to be innovative adopters~-- a pattern consistent with the broader sociolinguistic finding that women often lead linguistic change from below \autocite{labov2001}.

\textbf{Register restriction.} The forms are discourse-conditioned: common in storytelling, informal speech, peer conversation; rare in formal registers. Even as \mention{be like} has spread across age cohorts, it remains suppressed in academic prose, institutional speech, and written genres. German \mention{so} is still more restricted~-- marked as adolescent, rarely encountered in broadcast media.

These properties cluster. A quotative that introduces vivid re-enactment, prefers first person and present tense, spreads through young speakers in informal contexts~-- this is a recognisable type, a cross-linguistically recurring profile.

The question is: why does this cluster cluster?

\subsection{Stabilisers}

Why does this cluster persist? The answer lies in a braid of mechanisms, each contributing to the category's stability.

\textbf{Processing economy.} Quotatives of this type are structurally light. Japanese \mention{tte} is a phonologically reduced form of \mention{to itte} (`saying that'); the reduction removes syllables while preserving function. German \mention{so} is a single syllable. English \mention{be like} is syntactically minimal: copula plus predicative, no complementiser, no overt speech verb. Forms that reduce production cost under narrative pressure are used more frequently, entrenched more deeply, and survive transmission. This is a general principle from usage-based linguistics: high-frequency forms resist analogical pressure and attract new tokens \autocite{bybee2006}.

\textbf{Expressive fit.} These quotatives do something older forms don't: they introduce inner monologue, gesture, and attitude without committing the speaker to verbatim accuracy. \mention{She said} implies exact report; \mention{she was like} implies approximation, enactment, stance. \textcite{ferrarabell1995} documented this as the form's original semantic territory~-- internal states, non-lexicalised sounds~-- before it expanded to canonical direct speech. The looseness is functionally adaptive. Speakers need to convey gist, stance, affect~-- not transcript. Forms that serve communicative needs better than alternatives spread.

\textbf{Acquisition and cohort effects.} Children and adolescents acquire these forms in dense peer networks where the input is frequent, contextually salient, and socially marked. \textcite{tagliamontedarcy2007} provide the critical evidence: the same speakers, tracked over seven years, showed \mention{be like} rising from 13\% to 31\% of quotative tokens as they aged from their twenties into their thirties. This is not age-grading (using youth forms only when young); it's lifespan change. Speakers carry their adolescent inventory into adulthood, and their children hear it as ambient input. The result is a cohort effect: each generation's linguistic inventory reflects what was frequent in their formative years.

\textbf{Social indexing.} New quotatives index youth, informality, in-group membership. Using \mention{be like} signals that you are a certain kind of speaker, in a certain kind of register, with a certain kind of audience. This social value is not epiphenomenal; it's a stabiliser. As \textcite{Eckert2012} argues, social meaning is part of linguistic structure, not a by-product. Speakers maintain the form partly because it does identity work~-- marking solidarity, casualness, narrative immediacy. \textcite{buchstaller2014} found that attitudes toward \mention{be like} track age and gender: listeners perceive its users as younger and more socially attractive, even while rating them lower on status dimensions. The form is maintained partly \emph{because} of this indexical profile.

\textbf{Transmission dynamics.} The apparent-time pattern~-- younger speakers use \mention{be like} more~-- becomes a real-time change as each cohort ages. \textcite{tagliamontedarcy2004} showed that Canadian speakers aged 17--19 in 2003 used \mention{be like} for 63\% of quotative tokens, compared to 13\% for speakers of the same age in 1995. That's not two populations differing in age-graded behaviour; it's the S-curve of language change in progress. Forms that suit the demographic and interactional structure of transmission~-- peer-to-peer in adolescence, parent-to-child thereafter~-- survive and spread.


\subsection{Why the same profile across languages?}

This is the most telling fact. Japanese, Turkish, German, and English are not typologically close. They have different word-order properties, different morphological profiles, different sociolinguistic ecologies. Yet they converge on similar quotative innovations.

The convergence is not lexical; it's structural. The stabilising mechanisms are the same:

\begin{itemize}
   \item Processing pressure favours light forms.
   \item Discourse needs favour vague-reference quotatives.
   \item Social dynamics favour youth-indexed forms.
   \item Transmission dynamics favour high-frequency, contextually salient forms.
\end{itemize}

Wherever these mechanisms operate -- and they operate everywhere humans tell stories to each other -- the same type of quotative emerges. The category is not defined by a shared etymon or a universal grammar rule. It's stabilised by a convergent mechanism profile.

This is what `same category across languages' means in the maintenance view. Not shared essence. Not definitional identity. Convergent stabilisation.

\subsection{The wider ecology: colloquialisation and register shift}

There's a tempting further explanation. If similar quotatives emerged in multiple unrelated languages at roughly the same historical moment -- the late twentieth century -- perhaps something in the wider environment shifted. Perhaps a global trend toward informalization opened ecological space for these forms to flourish.

The hypothesis deserves scrutiny.

\textbf{Evidence for colloquialisation.} Corpus-based studies of English document a measurable drift. \textcite{biber1989} identified a shift in written registers toward oral styles: more contractions, more first-person pronouns, more active constructions. \textcite{leech2009} confirmed the pattern across the twentieth century, finding that colloquial features increased in fiction, journalism, and letters while academic prose remained relatively stable. The trend is attributed to external social factors -- marketization, media ecology, reduced formality in public discourse -- rather than internal linguistic pressure.

The timing fits. English \mention{be like} was first documented in the early 1980s; German \mention{so} emerged in the 1990s; both spread during the same decades when written and broadcast registers were absorbing more colloquial features. If the threshold for hearing and using informal speech lowered, forms that were once restricted to peer storytelling could leak into wider circulation. The quotative's stabilisers -- processing economy, expressive fit, youth indexing -- were always present. What changed, on this account, was the environment in which they operated.

\textbf{Counter-evidence and limits.} But the colloquialisation story oversimplifies if generalised incautiously.

First, it's not universal. The corpus evidence comes overwhelmingly from English, and similar trends are documented for some closely related varieties (Dutch, German, Swedish). But Japanese and Korean have seen \emph{increased} honorific elaboration in some workplace contexts, not decreased. Arabic and Turkish retain robust formal/informal distinctions with little evidence of erosion. The claim that languages are globally becoming less formal is not well-supported beyond the Anglophone and Northern European sphere.

Second, formality persists even where colloquialisation occurs. Academic prose, legal documents, medical communication, and official correspondence remain stubbornly formal. What \textcite{bibergray2016} found was not that formality disappeared, but that the \emph{gap} between formal and informal registers widened: informal registers got more informal, while academic English actually increased in nominal complexity. Colloquialisation is register-specific, not language-wide.

Third, the quotative case doesn't \emph{require} the colloquialisation hypothesis. The mechanisms enumerated above -- processing economy, expressive fit, acquisition dynamics, social indexing, transmission -- are sufficient to explain convergence. These mechanisms operate wherever humans tell personal stories in peer contexts. They don't need a global social trend to do their work; they need only the stable features of narrative interaction that have existed for millennia.

\textbf{Selection environment, not mechanism.} The most defensible framing is this: colloquialisation is a \emph{shift in the selection environment}, not a distinct stabilising mechanism.

In evolutionary biology, climate change is not a mechanism of evolution -- mutation, selection, and drift are mechanisms. But climate change alters the selection pressures that determine which variants succeed. Likewise, colloquialisation is not itself a mechanism that maintains the quotative cluster. Rather, it changes the conditions under which the mechanisms operate: which registers are widely heard, which indexical values carry social weight, how quickly innovations spread from youth cohorts to the wider population.

If colloquialisation increased the visibility of informal registers, quotatives that were already stabilised in those registers gained access to new transmission pathways. The mechanisms did the stabilising work; the ecological shift expanded the territory into which the stabilised form could spread.

This framing makes a prediction: in environments where colloquialisation hasn't occurred (or has reversed), the same quotative mechanisms should still operate -- but in a narrower ecological niche. And this is what we observe. German \mention{so} is more register-restricted than English \mention{be like}, consistent with a more modest colloquialisation trend in German public discourse. Turkish \mention{diye} is well-established but hasn't displaced formal quotative constructions in contexts where formality is maintained. The mechanisms are the same; the scope of their product differs with the ecology.

\textbf{What colloquialisation doesn't explain.} This perspective also clarifies what colloquialisation \emph{can't} explain. It can't explain why \emph{these particular forms} won out -- why \mention{be like} and not some other light quotative. That requires the specific mechanisms: the processing advantage, the expressive fit, the social indexing, the transmission dynamics.

And it can't explain the \emph{internal structure} of the quotative category -- the cluster of properties that co-occur. Colloquialisation doesn't predict that successful quotatives will favour first-person subjects, tolerate vague reference, and resist past-tense framing. The mechanisms predict this. The ecology only determines how far the category spreads.

So: colloquialisation is real, where it occurs. It modulates the reach of stabilised forms. But it is not a named mechanism in the quotative story. It is the weather in which the mechanisms operate.

\subsection{Variation and activation}

Not all quotative tokens are equally stable. English \mention{be like} is deep in its basin for speakers under 50; for speakers over 70, \mention{say} may still dominate. This is not two categories; it's the same functional category in different activation states, conditioned by generational input.

Within a single speaker, \mention{be like} is activated in storytelling and suppressed in formal report. The form is there; the activation depends on discourse environment.

German \mention{so} is more restricted than English \mention{be like} -- still marked as adolescent, still register-bound. This is a shallower basin: fewer stabilisers, more sensitivity to environmental signal.

The activation-state picture predicts: forms in shallow basins should show higher inter-speaker variation, more register sensitivity, and faster historical flux. \mention{So} does; \mention{be like}, now entrenched across generations, doesn't.

\subsection{What if a mechanism were absent?}

If processing economy were the only stabiliser, we'd expect the shortest form to win regardless of discourse function. But \mention{say} is shorter than \mention{be like}, and it doesn't dominate.

If expressive fit were the only stabiliser, we'd expect any form that introduces vivid quotation to spread equally. But quotatives with pejorative associations, or unfashionable indexical value, don't spread even when they fit the discourse need.

If social indexing were the only stabiliser, we'd expect pure fashion effects: quotatives rising and falling with generational taste. But \mention{be like} has been stable for three decades, long past a typical fashion cycle.

If transmission were the only stabiliser, we'd expect all forms heard in childhood to survive. But archaic quotatives like \mention{quoth} or reduced variants that never gained social cachet don't persist.

The observed pattern -- cross-linguistic convergence on light, expressive, youth-indexed, narratively deployed quotatives -- requires the full braid. No single mechanism is sufficient.


\subsection{A contrast: same semantic territory, different architecture}

The quotative case shows convergent maintenance: different languages, same mechanism profile, same category architecture. But convergent function doesn't always produce convergent form.

Consider evidentiality -- the grammatical encoding of information source. Turkish and Japanese both mark the distinction between direct and indirect evidence. But they've built different architectures for it.

Turkish has a single, tightly grammaticalised suffix: \mention{-(I)mIş} marks indirect evidentiality (inference, hearsay, surprise), contrasting paradigmatically with \mention{-DI} for direct experience. The contrast is obligatory, prosodically integrated, and acquired early.

Japanese has a looser constellation: \mention{rashii} (inference from external signs), \mention{sōda} (hearsay), \mention{yōda} (appearance-based). These forms are syntactically semi-independent, optional rather than obligatory, and retain traces of their lexical sources. They are recruited not only for evidentiality but also for hedging and politeness -- softening assertion in a culture where directness risks face-loss.

Why the difference? The mechanisms are the same: frequency, paradigmatic contrast, prosodic integration, functional need. But the \emph{ecology} differs. Turkish sits within a ``Eurasian evidential belt'' -- the Balkans, Caucasus, Central Asia -- where areal contact has amplified evidential grammaticalisation for centuries. Japanese evidentials do double duty, serving politeness as much as information source, which preserves their semi-lexical status.

Same semantic territory. Different category architecture. The mechanisms don't differ; the selection environment does.

This is the general lesson. Convergent stabilisation produces similar categories when mechanism profiles align. Divergent architecture emerges when ecological factors -- areal pressure, pragmatic function, transmission pathways -- push the same mechanisms toward different equilibria. The ecology is the wind; the mechanisms are the sails.


\section{How to test whether a mechanism is real}
\label{sec:7:robustness-tests}

The biological approach gives us a natural framework for testing mechanistic claims. Not hand-waving that \enquote{entrenchment maintains the category}, but operational tests that distinguish genuine causal structure from convenient labels.

\textbf{Learning transfer.} If entrenchment is real, speakers who learn aspect marking from some verbs should generalise to new verbs in predictable ways. This is exactly what the Polish models show: training on one subset, testing on another, the predictions transfer. Learning transfer is evidence that the category boundaries track causal structure, not just filing conventions.

\textbf{Intervention stability.} If frequency-based entrenchment stabilises irregular forms, then manipulating frequency should shift the pattern. We can't easily do this experimentally for established languages, but we see it in acquisition studies: children under-exposed to irregulars regularise more. We see it in contact situations: languages in intense contact show accelerated regularisation as input frequencies change.

\textbf{Cross-context generalisation.} If the category is maintained by stable mechanisms, predictions should hold across contexts -- different registers, different tasks, different populations. Labels without mechanisms should fragment: what works in careful speech should fail in casual speech; what works for Standard Polish should fail for dialectal Polish. The Polish aspect models show robustness across tense frames and lemma contexts. That's cross-context generalisation. That's mechanism.

These tests are what distinguish \enquote{I found a category} from \enquote{I found a mechanism that maintains a category}. The former is descriptive; the latter is explanatory.


\section{Degrees of projectibility}
\label{sec:7:degrees-projectibility}

The stabilising story explains why projectibility comes in degrees.

A category deep in a single basin -- maintained by entrenchment, transmission, functional pressure, and social reinforcement, all pulling in the same direction -- is strongly projectible. You can learn about nouns from a few exemplars and generalise reliably to new nouns. The mechanisms reinforce each other across timescales.

A category in an overlap region -- where mechanisms pull in different directions, or where entrenchment is weak -- is weakly projectible. Predictions work for typical cases but fail for edge cases. The degree of projectibility tracks the degree of mechanistic support.

A label with no mechanisms behind it is not projectible at all. A wastebasket category defined by what it's not, a traditional term inherited from earlier analyses, a filing convenience without causal grounding -- these should fragment under the robustness tests. They do.

This is the operational content of \enquote{mechanism-maintained kinds}. Not a metaphor. A measurable property: how strongly do predictions transfer across novel instances, contexts, and populations?


\section{What this commits us to}
\label{sec:7:commitments}

If this picture is right, we're committed to some substantive claims.

\textbf{Process ontology.} Categories are not static objects. They're dynamically sustained patterns -- standing waves, not sculptures. What exists is the stabilising process; the category is what the process makes legible.

\textbf{Interventionist realism.} Kinds are real to the extent that tracking or manipulating them changes expectations. This is stronger than description: it says that category distinctions track causal structure, not just organise files.

\textbf{Reciprocal realism.} Mechanisms and categories co-construct each other. Categories shape what gets entrenched; entrenchment shapes what categories survive. This is not a vicious circle; it's a self-organising dynamic. The same style of feedback that immunologists model for cell states applies to grammatical categories.

\textbf{Variation as signal.} Differences across contexts, speakers, and registers are not noise. They're diagnostic -- evidence about which region of the state space a token occupies, which activation state is active. The gradient nature of projectibility is a feature of maintenance, not a defect of the category.

\textbf{Cross-level coherence.} A category theory should deliver compatible predictions whether we analyse a phenomenon at the level of subpatterns, the construction, the category, or the wider system. If treating X as a bundle of micro-regularities yields different predictions than treating it as a member of a broader category, we've found something real: a heterogeneous grouping, a mislocated mechanism, or a label whose scope is historically inherited rather than causally grounded.

\textbf{Measurable metaphysics.} These claims have operational teeth. Entropy reduction, cross-context generalisation accuracy, inter-speaker agreement as a function of frequency and entrenchment -- these are not hand-waving gestures toward mechanism. They're measurable properties of stabilised categories.


\section{Redemption, not replacement}
\label{sec:7:redemption}

A word for the essentialists.

Nothing here says that traditional grammatical description was wrong. The definitions and classifications that fill reference grammars clustered the phenomena well enough that we can now ask: what maintains the clustering?

The definitional work was preparatory. It mapped the landscape. The mechanistic work explains why the landscape holds together -- why certain hills and valleys recur across languages, why some boundaries are sharp and others fuzzy, why predictions fail where they do and succeed where they do.

This is not \enquote{we've been wrong about categories}. It's \enquote{we finally have a way to explain why the categories that work keep working}.


\section{The most telling facts}
\label{sec:7:failure-modes-preview}

A grammatical category is not a thing you find; it's a regime you maintain. Arguments over definitions are, at bottom, arguments over stabilisers.

And the most telling facts about categories live in their failure modes -- where boundaries blur, where judgments diverge, where the stabilising dynamics show their seams.

But if categories are maintained, they can be undermaintained. The mechanisms can fail to cluster, or cluster too loosely, or cluster in ways that don't project.

The next chapter asks: how do we know when we don't have a kind?
