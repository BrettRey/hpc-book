% Chapter 7: The Stabilisers
% Complete rewrite 2025-12-08: biological explanatory style throughout

\chapter{The Stabilisers}
\label{ch:stabilisers}

\epigraph{The general rule would establish itself insensibly, and by slow degrees, in consequence of that love of analogy and similarity of sound, which is the foundation of by far the greater part of the rules of grammar.}{Adam Smith, \textit{Considerations Concerning the First Formation of Languages} (1761)}


\section{The cluster}
\label{sec:7:the-cluster}

Start where immunologists start: with the cluster.

When a biologist asks what a macrophage is, they don't look for a definition. They look for properties that tend to co-occur. A macrophage is picked out by a constellation: typical functions (phagocytosis, antigen presentation), marker profiles, morphologies, response tendencies, developmental origins. None of these is strictly necessary and sufficient. Some macrophages don't phagocytose; some share markers with dendritic cells; morphology varies with tissue context. But the properties cluster reliably enough to make the category useful -- to make predictions about new instances, to design experiments, to coordinate research programmes.

This is exactly the situation for grammatical categories.

Take nouns. What properties tend to co-occur? Items we call nouns typically take determinatives (\mention{the dog}, \mention{a problem}), pluralise (\mention{dogs}, \mention{problems}), function as heads of phrases in argument positions, refer to entities or entity-like abstractions, and enter into modification relations with adjectives. None of these is strictly necessary: \mention{cattle} doesn't pluralise; \mention{honesty} rarely takes \mention{a}; names resist determinatives in English; some nouns never appear in argument positions. And similar properties cluster under different labels in different languages -- Mandarin nouns don't pluralise the way English nouns do, and yet typologists recognise something noun-like in both systems.

The question is not: what is the essence of noun-ness? We tried that. It generated boundary disputes and competing definitions but no stable resolution.

The question is: what keeps the cluster clustered?


\section{Stabilisers at multiple scales}
\label{sec:7:stabilisers-at-scales}

When immunologists answer the analogous question for cell types, they don't point to a single cause. They narrate a multi-level story.

\textbf{Gene regulatory networks} produce relatively robust ``attractor'' states -- configurations the cell tends to settle into and return to after perturbation. These are not definitions; they are dynamical basins. The cell can be pushed around within a basin, but it takes a strong signal to tip it into a different one.

\textbf{Developmental lineage} constrains which basins are reachable. A cell's history matters: what precursor states it passed through, what signals it received at critical windows. The present state is not intelligible without the trajectory that led to it.

\textbf{Microenvironmental signalling} -- local cytokines, tissue context -- pushes cells into different regions of the same basin, producing variation within the type. A classically activated macrophage and an alternatively activated macrophage are both macrophages; the difference is which signals they've received and which region of the state space they occupy.

\textbf{Functional feedback}: what the cell does alters its microenvironment, which in turn sustains or shifts its state. The stabilisation is not one-way. The system is reciprocal.

Port this directly to language.

\textbf{Cognitive architectures and processing biases} are the analogue of gene regulatory networks. Some configurations of form-meaning pairings are easier to learn, store, and retrieve than others. Processing constraints produce attractor-like states: patterns that the system tends to fall into and return to. A construction that is frequent enough becomes entrenched -- retrieved as a unit rather than computed by rule. This is not definitional; it's dynamical.

\textbf{Acquisition pathways and learning biases} are the analogue of developmental lineage. A speaker's knowledge of English nouns was not installed by fiat; it was built up through exposure, starting with high-frequency exemplars and generalising outward. The order of acquisition matters. Early-learned patterns anchor later ones. Critical windows exist -- not strict Chomskyan critical periods, but sensitive phases where input has disproportionate effect.

\textbf{Discourse context and pragmatic ecology} are the analogue of microenvironmental signalling. A noun in subject position behaves differently from a noun in vocative position or in a compound. Same category, different activation state. The discourse environment pushes the token into a particular region of the distributional space. Variation is not noise to be explained away; it's expected, given the mechanisms.

\textbf{Functional feedback}: what speakers do with categories alters the input the next generation receives, which alters what categories stabilise. Usage entrenches representations; entrenched representations shape usage. The maintenance is not one-way. The system is reciprocal.

This is already a richer picture than \enquote{a noun is defined by such-and-such properties}. The definition tells you what the cluster looks like at a moment. The stabilising story tells you why it persists.


\section{Variation as activation states}
\label{sec:7:variation-activation-states}

The immunologist's key move: variation within a category is not embarrassing fuzziness to be explained away. It's the expected signature of a kind that is maintained by context-sensitive mechanisms.

A macrophage in inflamed tissue shows different markers than a macrophage in healthy tissue. Both are macrophages. The difference is activation state -- which signals the cell has received, which region of the phenotypic space it currently occupies. The category is real; the variation is real; both are explained by the stabilising dynamics.

Apply this to linguistic categories.

Consider independent relative \mention{whose}~-- forms like \mention{a woman whose was straight} or \mention{those whose are not}. Is this construction grammatical in English? Textbook accounts say no: Hankamer and Postal's (1973) rejection was widely accepted. But attestations exist, stretching back to Middle English.

The immunologist's framing reinterprets the question. The construction may be grammatical~-- maintained by the same filler-gap mechanism that serves interrogatives and other relative constructions~-- but only under specific licensing conditions: contrastive parallelism, deictic anchoring, structural integration. In other conditions, the possessum isn't recoverable, and the construction crashes. The category boundary isn't failed syntax; it's the region where pragmatic licensing determines whether the mechanism can succeed.

This is not a dodge. It's a prediction: constructions in licensing-dependent regions should show high inter-speaker variation, sensitivity to discourse context, and apparent paradigm gaps that aren't really gaps. Section~\ref{sec:7:filler-gap-whose} develops the case.

The same framing applies to aspect in Polish. As §\ref{sec:6:mechanistic-alternative} showed, 90\% of verbs strongly prefer one aspect -- they're deep in a single basin. The remaining verbs occupy shallower regions where contextual signals matter more. The verbs that require explicit contextual cues to disambiguate are the ones where entrenchment is weaker, where the tense-frame signal has more work to do. That's the activation-state picture applied to grammatical aspect.


\section{One case in depth: the emergence of new quotatives}
\label{sec:7:one-case}

Rather than enumerate mechanisms abstractly, let's trace the full stabilising story for one category: the quotative marker.

Quotatives introduce reported speech, thought, gesture, or sound. Every language has had them for as long as there has been narrative. But when we look across typologically unrelated languages, we find a striking convergence in recent history: in the late twentieth century, new quotative forms emerged or expanded~-- forms with a distinctive profile that spread through similar populations and stabilised through similar mechanisms.

The English case is well documented. \textcite{butters1982} first noted \mention{be like} in American English in 1982, observing its use to report unuttered thoughts: \mention{I was like, `Let me live, Lord'}. \textcite{ferrarabell1995} hypothesised that the form originated with first-person subjects and internal dialogue before expanding to third-person subjects and direct speech. By the mid-1990s, the form had spread rapidly. \textcite{tagliamontedarcy2004} tracked apparent-time data from Canadian English: in 1995, speakers aged 18--27 used \mention{be like} for 13\% of quotative tokens; by 2003, speakers aged 17--19 used it for 63\%~-- a fivefold increase in under a decade. The same cohort, re-sampled seven years later, showed \mention{be like} rising from 13\% to 31\% as they aged into their thirties \autocite{tagliamontedarcy2007}. This is not age-grading; it's real-time change, with each generation carrying its quotative inventory into adulthood.

German shows a parallel trajectory. \textcite{golato2000} analysed \mention{und ich so} (`and I'm like') in video-recorded conversations, documenting its function in storytelling: introducing punchlines, sound effects, gestures, and verbal enactments~-- the same profile as English \mention{be like}. The form was attested in youth speech by the late 1990s \autocite{androutsopoulos1998} and has since become established among younger German speakers, though it remains more register-restricted than its English counterpart.

Japanese \mention{tte} (reduced from \mention{to itte}, `saying that') functions similarly: a phonologically light form that introduces reported speech, thought, or stance in informal narrative. The pattern extends to Turkish \mention{diye} (converb of \mention{demek}, `to say'), which has expanded from its older quotative functions into increasingly colloquial discourse contexts.

\textcite{buchstaller2014} synthesises the cross-linguistic evidence: the emergence of these innovative quotatives is not media-driven borrowing but independent development under similar functional pressures. Different sources, different structures, same functional niche, same stabilising dynamics.

This is not coincidence. It's convergent maintenance.

\subsection{The cluster}

What properties co-occur in these new quotatives? The distributional studies reveal a consistent profile.

\textbf{Non-lexical content.} These forms introduce material that goes beyond words: gesture, facial expression, tone of voice, inner monologue. \mention{And I was like} doesn't just report what someone said; it enacts the experience. \textcite{ferrarabell1995} noted that early \mention{be like} tokens disproportionately introduced internal states and non-lexicalised sounds~-- groans, sighs, exclamations~-- before expanding to verbatim-style direct speech. \textcite{golato2000} documented the same pattern for German \mention{so}: it turns reported speech into performance.

\textbf{First-person and present-tense preference.} The forms attach preferentially to first-person subjects and historical-present tense. \textcite{tagliamontedarcy2004} found that in their Toronto data, first-person contexts strongly favoured \mention{be like} over \mention{say}. When speakers narrate, they use these forms to bring the audience into the moment of the original event~-- the vivid first person, the historical present.

\textbf{Youth association.} In every documented case, younger speakers lead adoption. The apparent-time data are consistent across studies: the highest rates of \mention{be like} appear among speakers under 30; rates decline sharply for speakers over 50 \autocite{tagliamontedarcy2004,tagliamontedarcy2007}. Young women, in particular, tend to be innovative adopters~-- a pattern consistent with the broader sociolinguistic finding that women often lead linguistic change from below \autocite{labov2001}.

\textbf{Register restriction.} The forms are discourse-conditioned: common in storytelling, informal speech, peer conversation; rare in formal registers. Even as \mention{be like} has spread across age cohorts, it remains suppressed in academic prose, institutional speech, and written genres. German \mention{so} is still more restricted~-- marked as adolescent, rarely encountered in broadcast media.

These properties cluster. A quotative that introduces vivid re-enactment, prefers first person and present tense, spreads through young speakers in informal contexts~-- this is a recognisable type, a cross-linguistically recurring profile.

The question is: why does this cluster cluster?

\subsection{Stabilisers}

Why does this cluster persist? The answer lies in a braid of mechanisms, each contributing to the category's stability.

\textbf{Processing economy.} Quotatives of this type are structurally light. Japanese \mention{tte} is a phonologically reduced form of \mention{to itte} (`saying that'); the reduction removes syllables while preserving function. German \mention{so} is a single syllable. English \mention{be like} is syntactically minimal: copula plus predicative, no complementiser, no overt speech verb. Forms that reduce production cost under narrative pressure are used more frequently, entrenched more deeply, and survive transmission. This is a general principle from usage-based linguistics: high-frequency forms resist analogical pressure and attract new tokens \autocite{bybee2006}.

\textbf{Expressive fit.} These quotatives fill a functional niche that older forms leave empty: introducing inner monologue, gesture, and attitude without committing the speaker to verbatim accuracy. \mention{She said} implies exact report; \mention{she was like} implies approximation, enactment, stance. \textcite{ferrarabell1995} documented this as the form's original semantic territory~-- internal states, non-lexicalised sounds~-- before it expanded to canonical direct speech. The looseness is functionally adaptive. Speakers need to convey gist, stance, affect~-- not transcript. Forms that serve communicative needs better than alternatives spread.

\textbf{Acquisition and cohort effects.} Children and adolescents acquire these forms in dense peer networks where the input is frequent, contextually salient, and socially marked. \textcite{tagliamontedarcy2007} provide the critical evidence: the same speakers, tracked over seven years, showed \mention{be like} rising from 13\% to 31\% of quotative tokens as they aged from their twenties into their thirties. This is not age-grading (using youth forms only when young); it's lifespan change. Speakers carry their adolescent inventory into adulthood, and their children hear it as ambient input. The result is a cohort effect: each generation's linguistic inventory reflects what was frequent in their formative years.

\textbf{Social indexing.} New quotatives index youth, informality, in-group membership. Using \mention{be like} signals that you are a certain kind of speaker, in a certain kind of register, with a certain kind of audience. This social value is not epiphenomenal; it's a stabiliser. As \textcite{Eckert2012} argues, social meaning is part of linguistic structure, not a by-product. Speakers maintain the form partly because it does identity work~-- marking solidarity, casualness, narrative immediacy. \textcite{buchstaller2014} found that attitudes toward \mention{be like} track age and gender: listeners perceive its users as younger and more socially attractive, even while rating them lower on status dimensions. The form is maintained partly \emph{because} of this indexical profile.

\textbf{Transmission dynamics.} The apparent-time pattern~-- younger speakers use \mention{be like} more~-- becomes a real-time change as each cohort ages. \textcite{tagliamontedarcy2004} showed that Canadian speakers aged 17--19 in 2003 used \mention{be like} for 63\% of quotative tokens, compared to 13\% for speakers of the same age in 1995. That's not two populations differing in age-graded behaviour; it's the S-curve of language change in progress. Forms that suit the demographic and interactional structure of transmission~-- peer-to-peer in adolescence, parent-to-child thereafter~-- survive and spread.


\subsection{How deep do mechanisms go?}

The stabilisers just described~-- processing economy, expressive fit, social indexing, transmission dynamics~-- might look like endpoint explanations. But mechanisms have mechanisms. Each stabiliser can be decomposed causally (what underlies it?) and mereologically (what are its parts?). The decomposition reveals the multi-level architecture of category maintenance.

\subsubsection{Causal depth}

Take the robust finding that young women lead quotative innovation \autocite{labov2001,tagliamontedarcy2004}. This isn't a brute fact. It bottoms out in social-psychological structure.

Young women's peer networks tend to be denser and more multiplex than young men's \autocite{milroy1987}. Dense networks means more linguistic input, more accommodation pressure, faster entrenchment of shared forms. Multiplex ties~-- relationships serving multiple social functions~-- means more stakes in each interaction, more motivation to coordinate. When your interlocutor is also your classmate, neighbour, and confidante, you align more strongly than with a stranger.

But why dense networks in adolescence at all? Developmental psychology provides the next level down. Adolescence is the period of separation-individuation, when peer orientation overtakes parent orientation \autocite{erikson1968}. The social structure of schooling concentrates age-cohorts in prolonged daily contact. Identity construction becomes primary: you need to differentiate from your parents while affiliating with your peers. Linguistic innovation serves both goals simultaneously~-- it marks in-group solidarity and out-group distinction.

Why these developmental facts? Evolutionary pressures on coalition formation. Cultural transmission of gender norms that make social reputation more costly for women. Economic structure that extends education and delays labour-market entry, prolonging the adolescent peer-intensive phase. At this level we've left linguistics for psychology, sociology, economics~-- but the causal chain is continuous. \enquote{Young women lead quotative change} isn't a stipulation; it's the surface manifestation of a causal cascade that bottoms out in human developmental and social architecture.

How far down should linguistic explanation go? Far enough to explain the clustering at the level we care about. For the quotative case, the clustering is grammatical: first-person preference, youth association, non-verbatim content. The mechanisms are social and cognitive: network density, identity work, processing pressure, entrenchment dynamics. We \emph{could} trace further~-- to neuroscience, to evolutionary biology~-- but the grammatical clustering is already explained. The framework is level-agnostic: it says look for mechanisms that produce clustering, without dictating which level to bottom out at.

\subsubsection{Mereological structure}

The same decomposition applies to parts, not just causes. \enquote{Transmission} sounds like a single mechanism, but it's a composite.

Production processes come first: lexical selection (choosing \mention{be like} over \mention{say}), which involves activating competitors, weighting by frequency, and filtering by context~-- register, addressee, narrative function. Then syntactic planning: slotting the copula and predicative into a quotative frame. Then phonetic realisation: the actual articulation of \mention{like}.

The signal bridges production and perception: acoustic properties, prosodic packaging (the characteristic intonation contour that marks enactment).

Perception processes follow: segmenting the stream, recognising words, parsing the construction, assigning category membership. Each is a sub-mechanism with its own structure~-- word recognition draws on frequency-weighted lexical access, construction parsing draws on stored templates, category assignment draws on distributional learning.

Memory processes update the listener's representation: encoding this token in this context (acoustic trace, syntactic frame, pragmatic function, social co-occurrence~-- who said it, to whom, with what stance), storing it as an exemplar, adjusting prototype weights, strengthening form-function associations. These updates feed future production: the listener becomes a speaker whose lexical selection is now shifted slightly toward the form just heard.

Social processes interleave throughout: accommodation (matching the interlocutor), identity projection (signaling youth and informality), stance-taking (performing enactment rather than merely reporting). \textcite{pickering2004} model alignment as automatic priming; \textcite{Eckert2012} situates it in identity construction. Both are part of what \enquote{transmission} means.

The mereological lesson: even the mechanisms introduced in Chapter~\ref{ch:kinds-without-essences}~-- acquisition, entrenchment, alignment, functional pressure~-- are not atoms. They're composites with internal structure. When we say \enquote{transmission maintains the category}, we're compressing a multi-part process into a single term. The term is still explanatory~-- it picks out a real causal structure~-- but the structure is articulated, not monolithic.

\subsubsection{What this means for explanation}

The depth and structure of mechanisms matter for two reasons.

First, they reveal where intervention is possible. If transmission depends on particular memory processes, then anything that disrupts those processes~-- reduced input frequency, competing forms, register restriction~-- will weaken the mechanism. Knowing the parts tells you where the leverage points are. This is why Chapter~\ref{ch:failure-modes} on failure modes will matter: category dissolution happens when sub-mechanisms fail, and understanding which sub-mechanisms are failing requires knowing what they are.

Second, they connect linguistic explanation to the broader sciences. The causal depth that runs from \enquote{young women lead change} through network density to developmental psychology to evolutionary pressures is not a detour from linguistics; it's where linguistics fits into the larger picture. The maintenance view doesn't require every linguist to become a psychologist or evolutionary biologist. But it does make explicit that grammatical categories are maintained by mechanisms that ultimately ground in facts about human cognition, social structure, and history. The framework is continuous with the rest of science, not sealed off from it.

\subsubsection{Mechanisms as categories, categories as mechanisms}

A deeper point emerges from this decomposition.

If categories are projectible clusters stabilised by mechanisms, and if the mechanisms themselves are projectible clusters stabilised by further mechanisms, then mechanisms are categories. \enquote{Processing economy} is a category~-- it bundles properties (reduced forms, high frequency, resistance to analogical pressure) that cluster reliably because of underlying neural and memory constraints. \enquote{Social indexing} is a category~-- it bundles properties (youth association, in-group marking, register restriction) maintained by identity construction and network dynamics. The mechanisms we invoke to explain quotative stability are themselves HPC kinds, maintained by their own braids of stabilisers.

And the stabilisation is reciprocal.

Quotatives don't just \emph{use} processing economy~-- they \emph{maintain} it. Every time a speaker produces \mention{be like}, that token is input to the cognitive architecture that grounds processing economy. The form's high frequency reinforces the memory representations that make high-frequency forms easier to access. The mechanism depends on the category for its continued operation just as the category depends on the mechanism.

The same reciprocity holds for social indexing. Social indexing isn't just a mechanism \emph{for} quotatives; quotatives are part of what \emph{constitutes} social indexing. The indexical field that makes youth-associated forms socially meaningful exists because forms like \mention{be like} circulate with those associations. Remove the quotatives and the indexical structure weakens. The category and the mechanism co-construct each other.

This is not a vicious circle. It's a self-organising dynamic~-- precisely the homeostatic structure that gives HPC kinds their stability. Categories shape what gets entrenched; entrenchment shapes which categories survive. Each stabilises the other.

The quotative category is also a mechanism in the larger kinds that contain it. Quotatives are part of what maintains \term{narrative discourse structure}~-- they enable the vivid re-enactment that makes storytelling work. They are part of what maintains \term{youth register}~-- the forms that index adolescent identity and solidarity. They are part of what maintains \term{informal speech} as a recognisable variety~-- the cluster of features that signals casualness, immediacy, peer context. And again, the stabilisation is reciprocal: narrative structure provides the discourse ecology in which quotatives function; youth register provides the social meaning that quotatives carry; informal speech provides the register niche that quotatives occupy.

So the hierarchy runs both ways. Downward: quotatives are maintained by mechanisms (processing, acquisition, social indexing, transmission), which are themselves maintained by deeper mechanisms (memory architecture, developmental psychology, network structure). Upward: quotatives are mechanisms in larger categories (narrative structure, youth register, informal speech), which are themselves mechanisms in still larger kinds (discourse genres, sociolinguistic repertoires, speech communities). And at every level, the stabilisation is bidirectional: each element both depends on and sustains the elements above and below it.

The distinction between \enquote{category} and \enquote{mechanism} is perspectival, not ontological. What counts as a category at one scale is a mechanism at another. What stabilises quotatives is itself stabilised by quotatives; what quotatives stabilise is itself what stabilises them. The recursion is not infinite~-- it bottoms out in physics and tops out in the largest social formations~-- but within the range linguistics cares about, the structure is fractal: categories all the way down, mechanisms all the way up, and reciprocal maintenance all the way through.


\subsection{Why the same profile across languages?}

This is the most telling fact. Japanese, Turkish, German, and English are not typologically close. They have different word-order properties, different morphological profiles, different sociolinguistic ecologies. Yet they converge on similar quotative innovations.

The temporal pattern is striking. In English, \mention{be like} was first attested in the US in 1982 \autocite{butters1982}, at least seven years before it appeared in any other variety \autocite{buchstaller2014}. It reached the UK by 1994 \autocite{andersen1996}, New Zealand and Canada by 1995 \autocite{baird2001,tagliamonteH1999}, Australia by 1997 \autocite{winter2002}. The form then appeared in outer-circle Englishes: Singapore, Hong Kong, Kenya, Jamaica, the Philippines. By the 2000s, it had global attestation across typologically diverse contact varieties.

One obvious hypothesis is media transmission. The `Valley Girl' stereotype, propelled into popular culture by Frank Zappa's 1982 song and subsequent Hollywood films, made \mention{be like} audible to millions. But \textcite{buchstaller2014} finds little evidence for straightforward media-driven borrowing. Speakers rarely associate \mention{be like} with media stereotypes; the form's linguistic conditioning varies across localities in ways inconsistent with uniform transmission; and the development pattern shows `transformation under transfer'~-- local adaptation of a form that arrived from elsewhere~-- rather than passive copying.

The stronger hypothesis is convergent development under similar functional pressures. The semantic-pragmatic sources for innovative quotatives~-- deictic markers, movement verbs, approximation expressions~-- recur across typologically unrelated languages \autocite{guldemann2008}. German recruits \mention{so} (a deictic), Japanese recruits \mention{tte} (a reduced quotative verb), Turkish recruits \mention{diye} (a converb), English recruits \mention{like} (an approximator). The sources differ, but the functional territory is the same: introducing vivid, non-verbatim re-enactment in informal narrative.

The convergence is not lexical; it's structural. The stabilising mechanisms are the same:

\begin{itemize}
   \item Processing pressure favours light forms.
   \item Discourse needs favour vague-reference quotatives.
   \item Social dynamics favour youth-indexed forms.
   \item Transmission dynamics favour high-frequency, contextually salient forms.
\end{itemize}

Wherever these mechanisms operate~-- and they operate everywhere humans tell stories to each other~-- the same type of quotative emerges. The category is not defined by a shared etymon or a universal grammar rule. It's stabilised by a convergent mechanism profile.

This is what `same category across languages' means in the maintenance view. Not shared essence. Not definitional identity. Convergent stabilisation.


\subsection{The wider ecology: colloquialisation and register shift}

There's a tempting further explanation. If similar quotatives emerged in multiple unrelated languages at roughly the same historical moment -- the late twentieth century -- perhaps something in the wider environment shifted. Perhaps a global trend toward informalization opened ecological space for these forms to flourish.

The hypothesis deserves scrutiny.

\textbf{Evidence for colloquialisation.} Corpus-based studies of English document a measurable drift. \textcite{biber1989} identified a shift in written registers toward oral styles: more contractions, more first-person pronouns, more active constructions. \textcite{leech2009} confirmed the pattern across the twentieth century, finding that colloquial features increased in fiction, journalism, and letters while academic prose remained relatively stable. The trend is attributed to external social factors -- marketization, media ecology, reduced formality in public discourse -- rather than internal linguistic pressure.

The timing fits. English \mention{be like} was first documented in the early 1980s; German \mention{so} emerged in the 1990s; both spread during the same decades when written and broadcast registers were absorbing more colloquial features. If the threshold for hearing and using informal speech lowered, forms that were once restricted to peer storytelling could leak into wider circulation. The quotative's stabilisers -- processing economy, expressive fit, youth indexing -- were always present. What changed, on this account, was the environment in which they operated.

\textbf{Counter-evidence and limits.} But the colloquialisation story oversimplifies if generalised incautiously.

First, it's not universal. The corpus evidence comes overwhelmingly from English, and similar trends are documented for some closely related varieties (Dutch, German, Swedish). But Japanese and Korean have seen \emph{increased} honorific elaboration in some workplace contexts, not decreased. Arabic and Turkish retain robust formal/informal distinctions with little evidence of erosion. The claim that languages are globally becoming less formal is not well-supported beyond the Anglophone and Northern European sphere.

Second, formality persists even where colloquialisation occurs. Academic prose, legal documents, medical communication, and official correspondence remain stubbornly formal. What \textcite{bibergray2016} found was not that formality disappeared, but that the \emph{gap} between formal and informal registers widened: informal registers got more informal, while academic English actually increased in nominal complexity. Colloquialisation is register-specific, not language-wide.

Third, the quotative case doesn't \emph{require} the colloquialisation hypothesis. The mechanisms enumerated above -- processing economy, expressive fit, acquisition dynamics, social indexing, transmission -- are sufficient to explain convergence. These mechanisms operate wherever humans tell personal stories in peer contexts. They don't need a global social trend to do their work; they need only the stable features of narrative interaction that have existed for millennia.

\textbf{Selection environment, not mechanism.} The most defensible framing is this: colloquialisation is a \emph{shift in the selection environment}, not a distinct stabilising mechanism.

In evolutionary biology, climate change is not a mechanism of evolution -- mutation, selection, and drift are mechanisms. But climate change alters the selection pressures that determine which variants succeed. Likewise, colloquialisation is not itself a mechanism that maintains the quotative cluster. Rather, it changes the conditions under which the mechanisms operate: which registers are widely heard, which indexical values carry social weight, how quickly innovations spread from youth cohorts to the wider population.

If colloquialisation increased the visibility of informal registers, quotatives that were already stabilised in those registers gained access to new transmission pathways. The mechanisms did the stabilising work; the ecological shift expanded the territory into which the stabilised form could spread.

This framing makes a prediction: in environments where colloquialisation hasn't occurred (or has reversed), the same quotative mechanisms should still operate -- but in a narrower ecological niche. And this is what we observe. German \mention{so} is more register-restricted than English \mention{be like}, consistent with a more modest colloquialisation trend in German public discourse. Turkish \mention{diye} is well-established but hasn't displaced formal quotative constructions in contexts where formality is maintained. The mechanisms are the same; the scope of their product differs with the ecology.

\textbf{What colloquialisation doesn't explain.} This perspective also clarifies what colloquialisation \emph{can't} explain. It can't explain why \emph{these particular forms} won out -- why \mention{be like} and not some other light quotative. That requires the specific mechanisms: the processing advantage, the expressive fit, the social indexing, the transmission dynamics.

And it can't explain the \emph{internal structure} of the quotative category -- the cluster of properties that co-occur. Colloquialisation doesn't predict that successful quotatives will favour first-person subjects, tolerate vague reference, and resist past-tense framing. The mechanisms predict this. The ecology only determines how far the category spreads.

So: colloquialisation is real, where it occurs. It modulates the reach of stabilised forms. But it is not a named mechanism in the quotative story. It is the weather in which the mechanisms operate.

\subsection{Variation and activation}

Not all quotative tokens are equally stable. English \mention{be like} is deep in its basin for speakers under 50; for speakers over 70, \mention{say} may still dominate. This is not two categories; it's the same functional category in different activation states, conditioned by generational input.

Within a single speaker, \mention{be like} is activated in storytelling and suppressed in formal report. The form is there; the activation depends on discourse environment.

German \mention{so} is more restricted than English \mention{be like} -- still marked as adolescent, still register-bound. This is a shallower basin: fewer stabilisers, more sensitivity to environmental signal.

The activation-state picture predicts: forms in shallow basins should show higher inter-speaker variation, more register sensitivity, and faster historical flux. \mention{So} does; \mention{be like}, now entrenched across generations, doesn't.

\subsection{What if a mechanism were absent?}

If processing economy were the only stabiliser, we'd expect the shortest form to win regardless of discourse function. But \mention{say} is shorter than \mention{be like}, and it doesn't dominate.

If expressive fit were the only stabiliser, we'd expect any form that introduces vivid quotation to spread equally. But quotatives with pejorative associations, or unfashionable indexical value, don't spread even when they fit the discourse need.

If social indexing were the only stabiliser, we'd expect pure fashion effects: quotatives rising and falling with generational taste. But \mention{be like} has been stable for three decades, long past a typical fashion cycle.

If transmission were the only stabiliser, we'd expect all forms heard in childhood to survive. But archaic quotatives like \mention{quoth} or reduced variants that never gained social cachet don't persist.

The observed pattern -- cross-linguistic convergence on light, expressive, youth-indexed, narratively deployed quotatives -- requires the full braid. No single mechanism is sufficient.


\subsection{A contrast: same semantic territory, different architecture}

The quotative case shows convergent maintenance: different languages, same mechanism profile, same category architecture. But convergent function doesn't always produce convergent form.

Consider evidentiality -- the grammatical encoding of information source. Turkish and Japanese both mark the distinction between direct and indirect evidence. But they've built different architectures for it.

Turkish has a single, tightly grammaticalised suffix: \mention{-(I)mIş} marks indirect evidentiality (inference, hearsay, surprise), contrasting paradigmatically with \mention{-DI} for direct experience. The contrast is obligatory, prosodically integrated, and acquired early.

Japanese has a looser constellation: \mention{rashii} (inference from external signs), \mention{sōda} (hearsay), \mention{yōda} (appearance-based). These forms are syntactically semi-independent, optional rather than obligatory, and retain traces of their lexical sources. They are recruited not only for evidentiality but also for hedging and politeness -- softening assertion in a culture where directness risks face-loss.

Why the difference? The mechanisms are the same: frequency, paradigmatic contrast, prosodic integration, functional need. But the \emph{ecology} differs. Turkish sits within a ``Eurasian evidential belt'' -- the Balkans, Caucasus, Central Asia -- where areal contact has amplified evidential grammaticalisation for centuries. Japanese evidentials do double duty, serving politeness as much as information source, which preserves their semi-lexical status.

Same semantic territory. Different category architecture. The mechanisms don't differ; the selection environment does.

This is the general lesson. Convergent stabilisation produces similar categories when mechanism profiles align. Divergent architecture emerges when ecological factors -- areal pressure, pragmatic function, transmission pathways -- push the same mechanisms toward different equilibria. The ecology is the wind; the mechanisms are the sails.


\section{A second case: filler-gap and independent relative \mention{whose}}
\label{sec:7:filler-gap-whose}

The quotative case shows mechanisms maintaining a category from above~-- from the functional and social pressures that keep the clustering clustered. A second case shows mechanisms maintaining a construction from below~-- from the cognitive processes that make it processable.

\subsection{The filler-gap mechanism}

When you hear \mention{Which book did Mary say that John read \_\_?}, you track a displaced constituent: \mention{which book} is the filler, and the gap after \mention{read} is where it's interpreted. This filler-gap dependency spans clause boundaries, requires memory, and operates across surface variation~-- open interrogatives, relative clauses, clefts.

As mentioned in Chapter~\ref{ch:projectibility}, \citet{boguraev2025} used causal interventions to test whether language models learn a shared mechanism across these construction types. They do: a mechanism learned on embedded open interrogatives transfers to produce filler-gap behaviour in clefts. The structural dependency projects even when the surface constructions differ.

This is what mechanistic kindhood predicts. The filler-gap mechanism applies across surface variation~-- relative clauses, interrogatives, clefts. It is exercised every time a speaker produces or comprehends one of these constructions. Each construction reinforces the mechanism; the mechanism in turn makes each construction processable. The stabilisation is reciprocal.

\subsection{Independent relative \mention{whose}: a gap that isn't}

A case study deepens the point. \citet{hankamer1973whose} claimed that English lacks independent relative \mention{whose}~-- forms like \mention{a woman whose was straight} where the possessum is elided. Their constructed examples are indeed ungrammatical: \mention{*The guy whose you saw banging at the window...} fails. For fifty years, the construction was treated as a syntactic gap.

But the gap may be illusory. \citet{reynolds2024whose} documents attestations spanning seven centuries, from Middle English manuscripts to contemporary academic prose: \mention{a friend of whose had told us of the accident}; \mention{those whose are not}. The construction is rare~-- perhaps once in hundreds of millions of words~-- but attested in edited sources across registers.\footnote{The grammaticality judgments remain disputed. \textcite{reynolds2024whose} is a preprint, and native-speaker intuitions vary. The epistemic status is accordingly tentative: suggestive evidence that the mechanism is there, not conclusive proof.}

What maintains this construction despite its extreme rarity? The same mechanisms that maintain other filler-gap constructions~-- but operating under unusually stringent conditions.

\subsection{Stabilisers at multiple scales}

\textbf{Filler-gap processing.} Independent relative \mention{whose} is a filler-gap construction. \mention{Whose} fills a gap in subject or complement position: \mention{a woman whose \_\_ was straight}. The displaced constituent must be tracked across clause structure and resolved at the gap site. This is the same cognitive operation that underlies relative clauses, open interrogatives, and clefts~-- the mechanism that Boguraev et al. showed transfers across construction types.

The filler-gap mechanism is maintained by processing pressures. Incremental parsing requires tracking dependencies as they unfold; memory constraints favour local resolution; prediction mechanisms anticipate gap sites. These pressures apply every time a speaker processes a relative clause or an interrogative. The mechanism is exercised constantly, even if independent \mention{whose} is exercised rarely.

\textbf{Information structure.} But filler-gap processing alone isn't sufficient. Independent \mention{whose} faces an additional burden: the elided possessum must be recoverable. This is what \citet{reynolds2024whose} calls the ``double anaphora'' requirement~-- the hearer must simultaneously recover the possessor (from the relative clause antecedent) and the possessum (from discourse context).

Three information-structural configurations license this recovery:

\begin{itemize}
\item \textbf{Contrastive parallelism.} The possessum is established in a preceding clause, and the \mention{whose} clause contrasts with it: \mention{I knew someone whose greatest love affair was with objects, another whose was with books, and a third whose was with ideas.} The parallel structure keeps the possessum type (\mention{love affair}) maximally accessible.

\item \textbf{Deictic anchoring.} A demonstrative directly points to the possessum: \mention{The man whose these are hath gotten me with child.} Deixis removes the need for discourse-level recovery.

\item \textbf{Structural integration.} The possessum appears as head of the phrase containing \mention{whose}: \mention{a friend of whose had told us of the accident.} The oblique genitive construction provides the possessum explicitly.
\end{itemize}

These aren't arbitrary conditions. They're the configurations that satisfy the recoverability constraint~-- the same constraint that governs ellipsis generally. The information-structural mechanism is independent of the filler-gap mechanism, but both must be satisfied for the construction to succeed.

\textbf{Discourse ecology.} Why did identity-of-sense stranding emerge only in the 1990s, when identity-of-reference stranding has existed since Middle English? Not because the grammar changed, but because the discourse ecology changed.

Academic and journalistic prose favours list-like enumerations and explicit contrast sets: \mention{patients whose symptoms improved versus those whose didn't}. These registers build the contrastive parallelism that licenses possessum recovery. The construction became possible when the discourse patterns that license it became conventionalised in particular genres. The mechanism was always there; the licensing environment wasn't.

\textbf{Cross-linguistic convergence.} German and Japanese~-- despite radically different syntactic structures~-- show identical information-structural constraints on independent possessive relatives. German \mention{dessen} (whose) strands only under contrastive parallelism; Japanese independent genitives require the same discourse accessibility conditions. The constraints are not language-specific accidents. They're consequences of how filler-gap processing interacts with discourse accessibility~-- and those interactions are universal.

\subsection{Why the same constraints across languages?}

The convergence demands explanation. Why should German, Japanese, and English all require contrastive parallelism for possessum recovery?

The answer is that the mechanisms operate at a level deeper than language-specific syntax. Filler-gap processing is a consequence of incremental parsing under memory constraints~-- and all human languages are parsed incrementally under memory constraints. Discourse accessibility is a consequence of how information flows through working memory~-- and working memory operates the same way across languages. The recoverability constraint is a consequence of communicative pressure~-- hearers must be able to reconstruct what speakers elide, and that pressure applies everywhere.

The surface structures differ. German has case marking; Japanese is head-final; English has neither. But the underlying mechanisms~-- the processing pressures, the memory constraints, the communicative needs~-- are constants. The constraints on independent possessive relatives track those constants, not the surface variation.

This is the same pattern we saw for quotatives. The surface forms differ across languages (\mention{be like}, \mention{so}, \mention{って}), but the mechanism profile converges (youth-indexed, first-person bias, non-verbatim). Convergent stabilisation produces similar categories when the mechanism profile aligns.

\subsection{Reciprocal maintenance}

The filler-gap case shows the same reciprocal structure as quotatives.

The filler-gap mechanism maintains the constructions by making them processable. Without the ability to track displaced constituents and resolve them at gap sites, relative clauses and interrogatives couldn't exist. The mechanism is a precondition for the constructions.

But the constructions maintain the mechanism by exercising it. Every relative clause, every interrogative, every cleft is input to the cognitive architecture that sustains filler-gap processing. The mechanism is real because it's kept alive by the constructions it serves.

The same applies to the information-structural constraints. Contrastive parallelism is a discourse pattern that exists independently of \mention{whose}. It's used for contrast sets generally, for enumeration, for rhetorical balance. But when speakers use \mention{whose} in contrastive contexts, they reinforce the association between contrastive parallelism and possessum recovery. The discourse pattern maintains the construction; the construction maintains the discourse pattern.

And the mechanism itself is a category~-- a cluster of properties (incremental parsing, memory retrieval, gap-filling) maintained by cognitive constraints. The filler-gap mechanism is maintained by sub-mechanisms (working memory, predictive processing, syntactic expectation), which are themselves maintained by further mechanisms. The recursion we traced for quotatives applies equally here: categories all the way down, mechanisms all the way up, reciprocal maintenance all the way through.


\section{How to test whether a mechanism is real}
\label{sec:7:robustness-tests}

The biological approach gives us a natural framework for testing mechanistic claims. Not hand-waving that \enquote{entrenchment maintains the category}, but operational tests that distinguish genuine causal structure from convenient labels.

\textbf{Learning transfer.} If entrenchment is real, speakers who learn aspect marking from some verbs should generalise to new verbs in predictable ways. This is exactly what the Polish models show: training on one subset, testing on another, the predictions transfer. Learning transfer is evidence that the category boundaries track causal structure, not just filing conventions.

\textbf{Intervention stability.} If frequency-based entrenchment stabilises irregular forms, then manipulating frequency should shift the pattern. We can't easily do this experimentally for established languages, but we see it in acquisition studies: children under-exposed to irregulars regularise more. We see it in contact situations: languages in intense contact show accelerated regularisation as input frequencies change.

\textbf{Cross-context generalisation.} If the category is maintained by stable mechanisms, predictions should hold across contexts -- different registers, different tasks, different populations. Labels without mechanisms should fragment: what works in careful speech should fail in casual speech; what works for Standard Polish should fail for dialectal Polish. The Polish aspect models show robustness across tense frames and lemma contexts. That's cross-context generalisation. That's mechanism.

These tests are what distinguish \enquote{I found a category} from \enquote{I found a mechanism that maintains a category}. The former is descriptive; the latter is explanatory.


\section{Degrees of projectibility}
\label{sec:7:degrees-projectibility}

The stabilising story explains why projectibility comes in degrees.

A category deep in a single basin -- maintained by entrenchment, transmission, functional pressure, and social reinforcement, all pulling in the same direction -- is strongly projectible. You can learn about nouns from a few exemplars and generalise reliably to new nouns. The mechanisms reinforce each other across timescales.

A category in an overlap region -- where mechanisms pull in different directions, or where entrenchment is weak -- is weakly projectible. Predictions work for typical cases but fail for edge cases. The degree of projectibility tracks the degree of mechanistic support.

A label with no mechanisms behind it is not projectible at all. A wastebasket category defined by what it's not, a traditional term inherited from earlier analyses, a filing convenience without causal grounding -- these should fragment under the robustness tests. They do.

This is the operational content of \enquote{mechanism-maintained kinds}. Not a metaphor. A measurable property: how strongly do predictions transfer across novel instances, contexts, and populations?


\section{What this commits us to}
\label{sec:7:commitments}

If this picture is right, we're committed to some substantive claims.

\textbf{Process ontology.} Categories are not static objects. They're dynamically sustained patterns -- standing waves, not sculptures. What exists is the stabilising process; the category is what the process makes legible.

\textbf{Interventionist realism.} Kinds are real to the extent that tracking or manipulating them changes expectations. This is stronger than description: it says that category distinctions track causal structure, not just organise files.

\textbf{Reciprocal realism.} Mechanisms and categories co-construct each other. Categories shape what gets entrenched; entrenchment shapes what categories survive. This is not a vicious circle; it's a self-organising dynamic. The same style of feedback that immunologists model for cell states applies to grammatical categories.

\textbf{Variation as signal.} Differences across contexts, speakers, and registers are not noise. They're diagnostic -- evidence about which region of the state space a token occupies, which activation state is active. The gradient nature of projectibility is a feature of maintenance, not a defect of the category.

\textbf{Cross-level coherence.} A category theory should deliver compatible predictions whether we analyse a phenomenon at the level of subpatterns, the construction, the category, or the wider system. If treating X as a bundle of micro-regularities yields different predictions than treating it as a member of a broader category, we've found something real: a heterogeneous grouping, a mislocated mechanism, or a label whose scope is historically inherited rather than causally grounded.

\textbf{Measurable metaphysics.} These claims have operational teeth. Entropy reduction, cross-context generalisation accuracy, inter-speaker agreement as a function of frequency and entrenchment -- these are not hand-waving gestures toward mechanism. They're measurable properties of stabilised categories.


\section{Refactoring, not replacing}
\label{sec:7:refactoring}

A reminder of the pluralist posture established in §\ref{sec:4:payoffs}.

The quotative case study isn't meant to supersede Tagliamonte's variationist work or Buchstaller's cross-variety comparisons or Ferrara and Bell's functional analysis. That work provides the data~-- the apparent-time percentages, the first-person preferences, the global attestation timeline. The maintenance view provides a conceptual frame that makes the data cohere.

What the refactoring adds: an answer to ``why this cluster?''~-- why these properties (first-person, youth, non-verbatim, informal) hang together rather than some other set. An answer to ``what licenses comparison?''~-- why German \mention{so} and English \mention{be like} can be treated as instances of the same phenomenon despite different etyma. An answer to ``what makes the pattern stable?''~-- why the quotative profile persists across generations rather than dissolving into noise.

The sociolinguists already had the patterns. The cognitivists already had the mechanisms. The typologists already had the parallels. What was missing was the ontology~-- the account of what grammatical categories actually \emph{are} that explains why the patterns, mechanisms, and parallels hold together. That's what the maintenance view offers. Not replacement. Refactoring.


\section{The most telling facts}
\label{sec:7:failure-modes-preview}

A grammatical category is not a thing you find; it's a regime you maintain. Arguments over definitions are, at bottom, arguments over stabilisers.

And the most telling facts about categories live in their failure modes -- where boundaries blur, where judgments diverge, where the stabilising dynamics show their seams.

But if categories are maintained, they can be undermaintained. The mechanisms can fail to cluster, or cluster too loosely, or cluster in ways that don't project.

The next chapter asks: how do we know when we don't have a kind?
