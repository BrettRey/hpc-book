% Chapter 7: Homeostasis—why categories hold together
% Stub: See /Users/brettreynolds/Documents/LLM-CLI-projects/HPC book/notes/chapter07_mechanisms.md

\chapter{Homeostasis—why categories hold together}\label{ch:homeostasis}

% TODO: Develop full chapter based on notes/chapter07_mechanisms.md
% 
% Key sections to include:
% - Multi-timescale framework (millisecond, year, decade mechanisms)
% - Internal vs. external mechanisms (Miller 2021)
% - Bidirectional inference (comprehension and production)
% - Evidence: acquisition, diachronic, cross-linguistic
% - Case studies from phonemes, words, constructions
%
% NOTE: Entrenchment is introduced in Ch6 (Polish aspect example: 90% of verbs
% strongly prefer one aspect). Develop the mechanism here: what entrenchment is,
% how it works, evidence from other domains beyond aspect.
%
% NOTE: Ch6 uses terrain/constraint-landscape language (Chater-compliant). This
% chapter should explain what "carves the valleys" — the forces that shape the
% constraint surface:
% - Acquisition: carves initial landscape from input
% - Entrenchment: deepens valleys through frequency
% - Alignment: maintains consistency across speakers (bidirectional)
% - Transmission: filters what survives across generations
% - Functional pressure: determines which valleys persist
%
% NOTE: Tense-conditioning introduced in Ch6 (F1 0.95/0.90). Explain why tense
% predicts aspect — grammatical co-selection, acquisition of tense-aspect pairs.
%
% NOTE: Bidirectionality is key (discussed 2024-12-06). Aspect→verb and verb→aspect
% are both active; production and comprehension draw on same shaped landscape.
%
% For detailed mechanism inventory and synthesis, see:
% /Users/brettreynolds/Documents/LLM-CLI-projects/HPC book/notes/chapter07_mechanisms.md

[Chapter content to be developed]
