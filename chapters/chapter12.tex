\chapter{Gender and the maintenance spectrum}
\label{ch:gender}

% Chapter 12: Gender systems as HPC case study
% Target: ~3,600 words
% Core argument: Gender systems illustrate the full maintenance spectrum—
% from semantically transparent (English) to purely entrenched (French).

%--- --- --- --- --- --- --- --- --- --- --- --- --- --- --- --- --
\section{The ship and the table}
\label{sec:12:hook}
%--- --- --- --- --- --- --- --- --- --- --- --- --- --- --- --- --

A sailor calls her ship \mention{she}. Not because the word \mention{ship} carries a feminine tag---it doesn't, not in English---but because in that moment, the vessel is more than cargo capacity and hull stress. She has moods. She responds to handling. She rewards care and punishes neglect. When a sailor calls a ship \mention{she}, the grammar is tracking a relationship, not a lexical entry.

Now consider a French table. It is \mention{la table}, feminine. It will always be feminine. If I refer to it later, I must say \mention{elle}---the same word a French speaker uses for a woman, a goddess, or a ship. But the table has no moods. There is no relationship to track. The word simply wears a tag, inherited from Latin, maintained by centuries of repetition, and entirely independent of any property of the object.

The difference is not superficial. It reveals two utterly distinct mechanisms holding together what we call \enquote{gender.}

In English, the sailor's choice between \mention{she} and \mention{it} is live. It responds to how the speaker perceives the referent---as an agent worthy of personhood or as an object in the world. The link between form and meaning is transparent: change the construal, change the pronoun. Call the ship scrap metal, and \mention{she} becomes \mention{it}. The grammar is tuning to the world.

In French, the choice between \mention{elle} and \mention{il} is dead---or rather, it has been exapted for a different purpose. It no longer tracks the world. Instead, it tracks the word. The link between form and meaning has been severed; what remains is a link between form and form. Knowing that the referent is a table tells you nothing about its gender. Knowing the word \mention{table} tells you everything. The grammar is checking an ID card.

This chapter makes a simple claim: \textbf{what linguists call \enquote{grammatical gender} is not one thing.} The same label covers systems held together by semantic congruence and systems held together by morphological entrenchment. These are not variations on a theme; they are different ontological kinds, maintained by different mechanisms, producing different empirical signatures.

We will examine three cases. English represents the transparent end of the spectrum---a system where gender tracks attributed personhood. French represents the entrenched end---a system where gender tracks lexical inheritance. German represents the conflict zone---a hybrid where the two mechanisms compete for control.

The payoff is diagnostic. If you know the maintenance mechanism, you can predict acquisition curves, diachronic trajectories, processing profiles, and the behavior of novel coinages. The label \enquote{gender} obscures these differences; the maintenance spectrum reveals them.

%--- --- --- --- --- --- --- --- --- --- --- --- --- --- --- --- --
\section{English: the tuning fork}
\label{sec:12:english}
%--- --- --- --- --- --- --- --- --- --- --- --- --- --- --- --- --

English gender doesn't look like gender. It lacks the apparatus familiar from Latin primers: no noun classes, no adjectival concord, no determiner agreement. Ask students what English gender is, and they'll say \mention{he}, \mention{she}, \mention{it}---three pronouns, vaguely correlated with biological sex.

But the system runs deeper than pronouns. It includes the relative pronoun split between \mention{who} and \mention{which}. It includes the interrogative split between \mention{who} and \mention{what}. It includes the locative and temporal pro-forms: \mention{where} and \mention{when}. Once you see the pattern, you cannot unsee it.

\citet{reynolds2025proform} proposes a hierarchical structure:

\begin{itemize}
    \item \textbf{Personal}: The referent is construed as a person---an intentional agent, a participant in social exchange. Exponents: \mention{who}, \mention{he}, \mention{she}, \mention{they}, \mention{someone}.
    \item \textbf{Non-personal}: The referent is construed as an object, a location, or a time. Exponents: \mention{which}, \mention{what}, \mention{it}, \mention{where}, \mention{when}, \mention{something}.
\end{itemize}

Within the personal category, a further distinction emerges:

\begin{itemize}
    \item \textbf{Epicene}: Gender-neutral or unknown. Exponents: \mention{who}, \mention{they}, \mention{someone}.
    \item \textbf{Sexual}: Specified for sex/gender. Exponents: \mention{he}, \mention{she}.
\end{itemize}

The system is designatum-driven. The choice of pro-form depends not on the grammatical features of the antecedent noun phrase, but on how the speaker construes the referent. A doctor can be \mention{he}, \mention{she}, or \mention{they}---not because the word \mention{doctor} is ambiguous, but because doctors can be male, female, or of unspecified gender. The pro-form tracks the person, not the word.

This produces a striking diagnostic. Metonymic reference works cleanly in English:

\begin{quote}
The French fries is waiting. She's upset.
\end{quote}

The speaker has used a non-personal noun phrase (\mention{the French fries}) to refer to a person---the customer who ordered them. The pronoun \mention{she} tracks the designatum (the upset woman), not the antecedent (the fries). The syntax accommodates; the grammar tunes to the world.

This is what I mean by \textbf{transparent maintenance}. The cluster of properties we call \enquote{English gender} is held together by a live semantic check: is this referent a person? The mechanism is cognitive---it runs through the speaker's model of the situation---and the stability of the system derives from the stability of that cognitive distinction. As long as humans distinguish persons from things, English gender will persist.

The historical trajectory confirms this. Old English had grammatical gender on the French model: three classes, arbitrary assignment, full concord. As morphological markers eroded, the system didn't collapse into chaos. It reorganized around the most robust semantic signal available---personhood. English gender is not a relic of the old system. It is a new system that emerged when the scaffolding fell away.

%--- --- --- --- --- --- --- --- --- --- --- --- --- --- --- --- --
\section{French: the ID card}
\label{sec:12:french}
%--- --- --- --- --- --- --- --- --- --- --- --- --- --- --- --- --

French gender is the textbook case. Every noun belongs to one of two classes: masculine or feminine. The class determines the form of the article (\mention{le}/\mention{la}, \mention{un}/\mention{une}), the adjective (\mention{petit}/\mention{petite}), the past participle (\mention{parti}/\mention{partie}), and the pronoun (\mention{il}/\mention{elle}). The clustering is dense. The reinforcement is everywhere.

But the clustering is not semantic. A table is feminine; a desk is masculine. A key is feminine; a lock is masculine. There is no property of tables, keys, or locks that predicts their class. The assignment happened centuries ago, inherited from Latin, and the only thing maintaining it now is the system itself.

This is entrenchment-driven maintenance. The mechanism is morphological memory: speakers memorize the gender of each lexical item, reinforced by every encounter with the item in context. The redundancy is functional---the same gender feature surfaces in article, adjective, and pronoun---but the redundancy stabilizes form, not meaning.

The metonymy test fails cleanly:

\begin{exe}
    \ex[*]{\textit{Les patates frites} espère. \textit{Il} est fâché.\\
    \upshape [Intended: The French fries is waiting. He is upset.]}
\end{exe}

\noindent The grammar refuses. The pronoun \mention{il} cannot track the male customer because the antecedent \mention{patates frites} is grammatically feminine and plural. To refer to the waiter, the speaker must break the anaphoric chain and introduce a new noun phrase: \mention{le monsieur} or \mention{le client}. The grammar is checking ID cards, not tuning to the world.

Is the semantic link entirely dead? Here \citet{millikan1999} offers a refinement: the link is not dead but \textit{exapted}. French gender no longer functions to classify world-objects. Instead, it functions to track discourse tokens---a parity bit for reference coherence.

When a French speaker uses \mention{elle} for \mention{la table}, the utterance succeeds not because the hearer learns anything about the table's properties, but because the hearer can reliably identify which discourse entity is being continued. The system provides redundancy against noise: if you miss the noun, the pronoun narrows the field. This is not world-tracking; it is code-tracking.

The acquisition signature differs accordingly. English children make conceptual errors---calling a stuffed bear \mention{he} because they construe it as a person. French children make memory errors---mismatch between article and noun, gradually repaired by exposure. The errors reveal the mechanism. In English, the check runs through the mental model of the referent. In French, the check runs through the mental lexicon.

%--- --- --- --- --- --- --- --- --- --- --- --- --- --- --- --- --
\section{German: the conflict zone}
\label{sec:12:german}
%--- --- --- --- --- --- --- --- --- --- --- --- --- --- --- --- --

German combines both mechanisms and suffers for it.

A girl is \mention{das Mädchen}---grammatically neuter, because the diminutive suffix \mention{-chen} forces neuter agreement. The determiner is \mention{das}. The relative pronoun should be \mention{das}. But in the next sentence, if you refer to the girl, you say \mention{sie}---the feminine personal pronoun---because the referent is female.

\begin{exe}
    \ex \textit{Das Mädchen ist nett. Sie lacht.}\\
    \upshape [The girl is nice. She laughs.]
\end{exe}

\noindent The conflict is predictable once you know the mechanisms. Inside the noun phrase, where local dependencies are dense and morphological entrenchment is strong, the grammatical gender wins. Across sentences, where the link to the antecedent is looser and the referent is salient, the semantic gender takes over.

\citet{corbett1991} describes this as an \textit{agreement hierarchy}: the further you move from the noun (from attributive adjective to predicate to relative pronoun to personal pronoun), the more likely agreement is to track the referent rather than the controller. German lives in the middle of this hierarchy, with visible seams.

From the \abbr{HPC} perspective, German is a system under homeostatic tension. Two distinct stabilizing mechanisms---morphological entrenchment and semantic congruence---are operating on the same property cluster. Each pulls in a different direction. The system holds together because the mechanisms partition: entrenchment wins locally, semantics wins at a distance. But the partition is never clean, and the mismatches are notorious.

The acquisition data confirm the competition. German children show hybrid error patterns: sometimes they fail to retrieve the grammatical gender (memory error, like French); sometimes they override grammatical gender with semantic intuition (conceptual intrusion, like English). The mixture reflects the mixture of mechanisms.

%--- --- --- --- --- --- --- --- --- --- --- --- --- --- --- --- --
\section{The maintenance spectrum}
\label{sec:12:spectrum}
%--- --- --- --- --- --- --- --- --- --- --- --- --- --- --- --- --

We can now draw the spectrum.

\begin{center}
\begin{tabular}{lll}
\toprule
& \textbf{Transparent} & \textbf{Entrenched} \\
\midrule
System & English & French \\
Mechanism & Semantic congruence & Morphological memory \\
Causal arrow & World $\rightarrow$ Word & Word $\rightarrow$ Word \\
Learning & Rule-like, early & Rote, gradual \\
Errors & Conceptual & Memory-based \\
Novelty & Assign on the fly & Requires formal rule \\
\bottomrule
\end{tabular}
\end{center}

\citet{kirby2015compression} offers a transmission-level interpretation. Systems must survive the bottleneck of acquisition: learners see only a fraction of the language but must reconstruct the whole. To survive, patterns must be compressible.

English gender is highly compressed. The rule---\enquote{pro-forms track personhood}---applies to an open class of referents. A child who grasps the concept \mention{person} can deploy the system productively. The compression is semantic: the world provides the storage.

French gender is uncompressed. There is no rule that predicts why \mention{table} is feminine. The child must memorize each lexical item along with its arbitrary tag. The compression is syntagmatic: the redundancy of article-noun-adjective chunks reinforces the correct assignment, but the chunks themselves must be stored.

German occupies an unstable middle. It has partial rules (sex-based for animate nouns) and partial lists (arbitrary for inanimates). The child must learn both, and the partial rules create interference.

This predicts diachronic trajectories. When transmission noise increases---through contact, creolization, or rapid population turnover---the entrenched system is vulnerable. Arbitrary features that cannot be inferred from meaning are the first to erode. The transparent features, grounded in stable cognitive distinctions, survive.

This is why English looks the way it does. It didn't \enquote{lose} gender; it \textit{traded entrenchment for inference}. The new system is optimized for a language with a history of contact-induced simplification. It is leaner, semantically motivated, and computationally cheaper---at the cost of expressive redundancy.

%--- --- --- --- --- --- --- --- --- --- --- --- --- --- --- --- --
\section{Implications}
\label{sec:12:implications}
%--- --- --- --- --- --- --- --- --- --- --- --- --- --- --- --- --

Three consequences follow.

\textbf{First}: the return of singular \mention{they} is not a disruption of English grammar. It is the grammar completing its own logic.

English has long had a gap at the \enquote{personal-epicene} node---a pro-form for persons of unspecified or non-binary gender. Historical \mention{he} filled this role poorly; singular \mention{you} (originally plural) shows the system is tolerant of number mismatch when personhood is at stake. Singular \mention{they} fills the slot by the same logic: personhood trumps number, just as it did for \mention{you} centuries ago.

Critics who claim \mention{they} is \enquote{ruining grammar} have misidentified the system. English gender is not about counting heads; it is about identifying persons. Singular \mention{they} is a feature, not a bug.

\textbf{Second}: typologists must distinguish the comparative concept from the descriptive category.

\citet{croft2001} and \citet{haspelmath2010} have argued that cross-linguistic comparison requires analyst-created yardsticks---\textit{comparative concepts}---distinct from the language-particular constructions they describe. \enquote{Gender} is a comparative concept; English personhood-based pro-form classification and French morphological noun-class are descriptive categories. They share a functional domain (nominal classification, reference tracking) but not an ontological substrate.

The \abbr{HPC} framework sharpens this distinction. Asking \enquote{Does Language X have gender?} is the wrong question. The right question is: \enquote{What mechanism stabilizes nominal classification in Language X?} The answers differ, and the differences matter for acquisition, processing, and diachrony.

\textbf{Third}: the maintenance spectrum maps onto neural architecture.

\citet{Fedorenko2024naturalkind} has shown that the Language Network---left frontal and temporal regions specialized for linguistic processing---is functionally distinct from networks handling general reasoning and social cognition. If English gender is designatum-driven, the computation requires querying the Theory-of-Mind network: is this referent a person? French gender, by contrast, is a local syntactic check: does the pronoun match the noun's stored feature?

This predicts dissociable breakdown. Global aphasia, which devastates the Language Network, should obliterate French gender (loss of morphological memory) while potentially sparing residual English gender (the concept of personhood is preserved in intact right-hemisphere and ToM circuits). The prediction is testable.

%--- --- --- --- --- --- --- --- --- --- --- --- --- --- --- --- --
\section{What the spectrum reveals}
\label{sec:12:summary}
%--- --- --- --- --- --- --- --- --- --- --- --- --- --- --- --- --

We began with a ship and a table. The ship's gender is alive---a measure of how the speaker relates to the vessel. The table's gender is dead in one sense (it tells you nothing about the object) but alive in another (it knits the discourse together).

The label \enquote{gender} covers both. That is the problem. It suggests a single phenomenon varying in degree, when in fact we have distinct phenomena stabilized by different mechanisms.

English gender is a thin \abbr{HPC}---a small cluster of properties (pro-form selection, relativizer choice) held together by a single, robust semantic check. French gender is a thick \abbr{HPC}---a large cluster of properties (article, adjective, participle, pronoun) held together by dense morphological redundancy. German is a site of homeostatic conflict, with two mechanisms fighting for the same channel.

The spectrum is not a value judgment. Entrenchment has its virtues: predictive speed, redundancy against noise, resistance to innovation. Transparency has its virtues: learnability, flexibility, accommodation to the world.

What the spectrum reveals is the \textit{method}. Ask not \enquote{What is gender?} but \enquote{What maintains gender?} The answer varies by language, by construction, by distance from the antecedent. Once you know the mechanism, you can predict the behavior. That is the payoff of the maintenance view.

