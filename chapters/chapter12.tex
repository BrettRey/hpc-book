\chapter{Pro-form Gender}
\label{ch:proform-gender}

% Chapter 12: Third Part III case study
% Target: ~6,000 words
% Structure parallels Ch 9 (Countability) and Ch 10 (Definiteness)

\epigraph{\textit{There appears to be a certain amount of learning in losing one's tail; researchers in California found that once a young skink has had a close encounter of the near-fatal kind it seems to be more cautious.}}{— Simon Anderson, \textit{Blink of a Lizard} (2008)}

%--- --- --- --- --- --- --- --- --- --- --- --- --- --- --- --- ---
\section{The pet puzzle}
\label{sec:12:hook}
%--- --- --- --- --- --- --- --- --- --- --- --- --- --- --- --- ---

Consider two sentences about the same dog:

\begin{exe}
    \ex \label{ex:dog-it} \textit{The dog wagged its tail.}
    \ex \label{ex:dog-who} \textit{Who's a good boy? Yes, you are!}
\end{exe}

In (\ref{ex:dog-it}), the dog takes the non-personal pronoun \mention{it}. In (\ref{ex:dog-who}), addressed directly by its owner, the same animal takes the personal pronoun \mention{who} and receives \mention{you}. Nothing about the dog has changed. What changed is how the speaker conceptualized the referent.

This is the puzzle of English gender. Traditional accounts describe a three-way distinction~-- masculine, feminine, neuter~-- realized mainly on pronouns (\mention{he}, \mention{she}, \mention{it}). That description captures something, but it misses the organizing principle. The primary distinction in English gender isn't sex. It's personhood.

The term \term{gender} in linguistics denotes a system of grammatically relevant contrasts wherein certain semantic concepts are divided into a small number of categories. English gender is typically described as referential rather than noun-class: there's no arbitrary assignment to nouns (unlike German \mention{der Tisch}, \mention{die Lampe}), and the choice of pronoun tracks properties of the referent, not grammatical properties of the antecedent \autocite{corbett1991,huddleston2002}. That much is right. But the usual account keeps the scope narrow~-- \mention{he}/\mention{she}/\mention{it}~-- when the system is far wider.

\textcite{siemund2013} model dialectal variation in English pronominal gender as different thresholds on an individuation hierarchy; \textcite{audring2009} shows how pronominal systems resemanticize when agreement is borne primarily by pronouns; \textcite{dolberg2019} traces the transition from lexical to referential gender in the \textit{Anglo-Saxon Chronicle}. These accounts share a common architecture: English gender is referential, hierarchically organized, and driven by properties of the designatum rather than the antecedent. What they also share is a common limitation: they keep the system's scope largely within third-person pronouns.

This chapter shows why that restriction is too narrow. The same personhood-based logic that governs \mention{he}/\mention{she}/\mention{it} also governs \mention{who}/\mention{which}, \mention{somebody}/\mention{something}, and \mention{when}/\mention{where}. The system extends across the entire class of items that function semantically as pro-forms~-- a claim that requires defense, since these items belong to different lexical categories. I'll address that question after laying out the evidence.

I'll argue that English gender, like countability (Chapter~\ref{ch:countability}) and definiteness (Chapter~\ref{ch:definiteness-and-deitality}), exhibits the HPC architecture: two clusters~-- one semantic (personhood), one morphosyntactic (pro-form inventory)~-- coupled by designatum-driven inference and maintained by overlapping mechanisms. Violations aren't just semantic infelicities; they're grammatical errors. The system has teeth.


%--- --- --- --- --- --- --- --- --- --- --- --- --- --- --- --- ---
\section{The \mentionhead{who}/\mentionhead{which} puzzle}
\label{sec:12:who-which}
%--- --- --- --- --- --- --- --- --- --- --- --- --- --- --- --- ---

Consider the relative pronouns:

\begin{exe}
    \ex \label{ex:doctor-who} \textit{The doctor who I saw was helpful.}
    \ex \label{ex:doctor-which} \ungram{\textit{The doctor which I saw was helpful.}}
    \ex \label{ex:book-which} \textit{The book which I read was helpful.}
    \ex \label{ex:book-who} \ungram{\textit{The book who I read was helpful.}}
\end{exe}

The constraint is robust. \mention{Who} for persons, \mention{which} for non-persons. The violations in (\ref{ex:doctor-which}) and (\ref{ex:book-who}) aren't merely odd~-- they're ungrammatical, comparable to agreement errors or subcategorization violations. This isn't pragmatic markedness; it's grammar.

The same split appears in interrogatives:

\begin{exe}
    \ex \label{ex:int-who} \textit{Who is coming to the party?} \hfill [asking about persons]
    \ex \label{ex:int-what} \textit{What is on the table?} \hfill [asking about non-persons]
    \ex \label{ex:int-who-bad} \ungram{\textit{Who is on the table?}} \hfill [asking about objects]
    \ex \label{ex:int-what-bad} \ungram{\textit{What is coming to the party?}} \hfill [asking about persons]
\end{exe}

Examples (\ref{ex:int-who-bad}) and (\ref{ex:int-what-bad}) are grammatical only if the speaker expects a person on the table or an object attending the party. The choice of interrogative reveals the speaker's conceptualization of the potential referent.

This is the \mention{who}/\mention{which} puzzle. Relative pronouns are not usually discussed under the heading of ``gender,'' but their distribution obeys exactly the same personhood-based constraint that governs \mention{he}/\mention{she}/\mention{it}. The puzzle dissolves when we recognize that both are manifestations of a single hierarchical system.

%--- --- --- --- --- --- --- --- --- --- --- --- --- --- --- --- ---
\section{The hierarchy}
\label{sec:12:hierarchy}
%--- --- --- --- --- --- --- --- --- --- --- --- --- --- --- --- ---

The English pro-form gender system is hierarchically organized. At the top level, the distinction is between \term{personal} and \term{non-personal}. Within personal, there's a further split between \term{epicene} (unmarked for sex) and \term{sexual} (masculine/feminine). Within non-personal, productive subtypes include \term{locative} and \term{temporal}.

\begin{figure}[htbp]
\centering
\begin{tabular}{l}
\textsc{pro-form gender}\\
\quad \textsc{personal}\\
\qquad \textsc{epicene}\\
\qquad \textsc{sexual}\\
\qquad\quad \textsc{masculine}\\
\qquad\quad \textsc{feminine}\\
\quad \textsc{non-personal}\\
\qquad \textsc{general}\\
\qquad \textsc{locative}\\
\qquad \textsc{temporal}\\
\end{tabular}
\caption{Hierarchy of gender values for English pro-forms.}
\label{fig:12:hierarchy}
\end{figure}

Each subtype inherits the properties of its parent. A word with feminine gender is necessarily sexual, which is necessarily personal. A word with locative gender is necessarily non-personal.

Table~\ref{tab:12:inventory} shows the full inventory. The leftmost column lists gender-neutral pro-forms~-- items that impose no presuppositional constraint on the designatum. The remaining columns list gender-sensitive items, distinguished by the personhood constraint they encode.

\begin{table}[htbp]
\caption{The pro-form gender system of English.}
\label{tab:12:inventory}
\small
\setlength{\tabcolsep}{4pt}
\centering
\begin{tabular}{@{}p{2.2cm}p{2.0cm}p{1.5cm}p{1.5cm}p{2.0cm}p{1.0cm}p{1.0cm}@{}}
\toprule
\multicolumn{1}{c}{Gender-neutral} & \multicolumn{6}{c}{Gender-sensitive} \\
\cmidrule(l){1-1} \cmidrule(l){2-7}
& \multicolumn{3}{c}{Non-personal} & \multicolumn{3}{c}{Personal} \\
\cmidrule(lr){2-4} \cmidrule(l){5-7}
& General & Temporal & Locative & Epicene & \multicolumn{2}{c}{Sexual} \\
\cmidrule(l){2-2} \cmidrule(l){3-3} \cmidrule(l){4-4} \cmidrule(l){5-5} \cmidrule(l){6-7}
& & & & & Masc & Fem \\
\midrule
\mention{they}\textsubscript{pl} & \mention{it} & \mention{then} & \mention{there} & \mention{I}, \mention{we}, \mention{you}, \mention{they}\textsubscript{sg}, \mention{one} & \mention{he} & \mention{she} \\
\addlinespace
\mention{whose}\textsubscript{rel} & \mention{what}, \mention{which} & \mention{when} & \mention{where} & \mention{who}, \mention{whom}, \mention{whose}\textsubscript{int} \\
\addlinespace
\mention{this}, \mention{that} & & \mention{yesterday}, \mention{today}, \mention{tomorrow} & \mention{here} \\
\addlinespace
& \mention{-thing} & \mention{once}, \mention{twice} & \mention{-where} & \mention{-body}, \mention{-one} \\
\bottomrule
\end{tabular}
\end{table}

The table reveals a system far larger than the traditional \mention{he}/\mention{she}/\mention{it} triad. The compound determinatives~-- \mention{somebody}/\mention{something}, \mention{everyone}/\mention{everything}, \mention{anywhere}/\mention{anyone}~-- participate in the same personhood partition. The relative pro-forms \mention{where} and \mention{when} encode non-personal subtypes parallel to \mention{which}. Even temporal deictics like \mention{yesterday} and \mention{tomorrow} are part of the system: they designate times, which are non-personal.

%--- --- --- --- --- --- --- --- --- --- --- --- --- --- --- --- ---
\section{Are pro-forms a category?}
\label{sec:12:proform-category}
%--- --- --- --- --- --- --- --- --- --- --- --- --- --- --- --- ---

This inventory raises a question. Pronouns, determinatives, wh-words, and pro-adverbs belong to different lexical categories. They appear in different syntactic positions, take different complements, and show different agreement patterns. What justifies treating them as a unified class?

The objection has force. If `pro-form' were a morphosyntactic category~-- defined by shared distributional behaviour~-- the heterogeneity would be fatal. There is no frame that selects for ``any pro-form'' as opposed to pronouns specifically, or wh-words specifically. The items don't cluster distributionally the way, say, nouns do.

But the claim isn't morphosyntactic. It's semantic. Pro-forms share a \term{substitution function}: they designate by standing for something else. A pronoun stands for an NP; a pro-adverb stands for an adverb phrase; a wh-word stands for a constituent in a question or relative clause. What unifies them is what they do, not where they appear.

Three considerations support treating this as a genuine category rather than a descriptive convenience.

\textbf{First, the substitution function generates shared discourse properties.} All pro-forms require antecedent or designatum resolution. They create coherence relations across sentences; they participate in binding and coreference; they impose presuppositional constraints. These aren't accidental overlaps. They follow from the substitution function itself: if a form designates by standing for something else, it must have something to stand for.

\textbf{Second, the gender constraint provides positive evidence for unity.} The same personhood-based hierarchy~-- personal/non-personal, with masculine/feminine and locative/temporal as subtypes~-- applies across pronouns (\mention{he}/\mention{it}), determinatives (\mention{somebody}/\mention{something}), wh-words (\mention{who}/\mention{which}), and pro-adverbs (\mention{where}/\mention{when}). This isn't a stipulation; it's an empirical observation. The constraint doesn't respect lexical-category boundaries. Whatever property licenses \mention{who} also licenses \mention{somebody} and blocks \mention{something}. The gender system treats these items as instances of a single variable.

\textbf{Third, the boundary violations are symmetric.} Using \mention{which} for a human referent is ungrammatical (*\mention{the doctor which I saw}). Using \mention{who} for a non-human referent is ungrammatical (*\mention{the book who I read}). Using \mention{something} for a human referent is ungrammatical (*\mention{something came to the party} when you mean a person). The violations have the same character across lexical categories: they're not category-mismatches (like using an adjective where a noun belongs); they're personhood-mismatches. The constraint is semantic, and it applies wherever the substitution function does.

This is exactly the HPC signature. Pro-forms are unified by a semantic function (substitution) and a semantic constraint (personhood). The morphosyntactic heterogeneity is real, but it doesn't prevent the semantic cluster from being maintained. Compare the definiteness cluster from Chapter~\ref{ch:definiteness-and-deitality}: \mention{the}, demonstratives, and genitives belong to different determiner subclasses, but they share the deitality profile because they inherit it from a common grammaticalization source. Pro-forms inherit their gender sensitivity from a common semantic source: the requirement that a designatum be resolved.

The pro-form category, then, is a semantic HPC. Its properties cluster not because of shared morphosyntax but because of shared function. What maintains it is the discourse--grammar interface: the constant pressure to resolve what a pro-form stands for, and the constant constraint that the resolution track personhood.


%--- --- --- --- --- --- --- --- --- --- --- --- --- --- --- --- ---
\section{Designatum-driven selection}
\label{sec:12:designatum}
%--- --- --- --- --- --- --- --- --- --- --- --- --- --- --- --- ---

The English system differs from grammatical-gender languages in a crucial way. In Spanish, French, or German, gender assignment is fixed on nouns and agreement spreads through NP-internal morphology. The antecedent controls the pro-form.

In English, it's the \term{designatum}~-- how the referent is conceptualized~-- that controls selection. The source provides the constraint; the pro-form satisfies it.

Consider a case of referential metonymy:

\begin{exe}
    \ex \label{ex:fries} \textit{The French fries is waiting. She's upset.} \hfill \autocite{wu2025}
\end{exe}

An NP with a default non-personal reading (\mention{the French fries}) is used to invoke a customer; subsequent anaphora tracks that metonymic designatum, yielding personal-feminine \mention{she}. The antecedent's form (plural, non-personal) doesn't control the pro-form; the conceptualized source does.

This is impossible in Spanish:

\begin{exe}
    \ex[*] \label{ex:spanish-fries} \gll \textit{Las patatas fritas} espera. \textit{Él} está enfadado.\\
    \textsc{def.f.pl} potato.\textsc{f.pl} fried.\textsc{f.pl} wait.\textsc{3sg.pres} \textsc{3sg.m} be.\textsc{3sg.pres} angry.\textsc{m.sg}\\
    \glt Intended: `The French fries is waiting. He's upset.'
\end{exe}

Spanish requires grammatical agreement with the antecedent (\textit{patatas}, feminine plural), not semantic agreement with the metonymic designatum. The speaker would need a rephrasing like \textit{el señor} (`the gentleman') to achieve the intended reference.

English permits categorical shifts based on conceptualization. This is the signature of a designatum-driven system.

%--- --- --- --- --- --- --- --- --- --- --- --- --- --- --- --- ---
\section{Beyond pronouns}
\label{sec:12:beyond-pronouns}
%--- --- --- --- --- --- --- --- --- --- --- --- --- --- --- --- ---

Standard accounts keep gender within personal pronouns. \textcite[486]{huddleston2002} state that English gender is ``based purely on pronoun agreement.'' But the evidence shows the system extends across all pro-forms.

\subsection{Determinatives}

The compound determinatives with \mention{-body} and \mention{-one} are personal:

\begin{exe}
    \ex \label{ex:somebody} \textit{Somebody left their coat.} \hfill [personal]
    \ex \label{ex:somebody-bad} \ungram{\textit{Somebody fell off the table.}} \hfill [referring to an object]
\end{exe}

Example (\ref{ex:somebody-bad}) is ungrammatical unless \mention{somebody} refers to a person who fell. These are epicene~-- they don't encode sex~-- but they do encode the personal/non-personal distinction.

The compound determinatives with \mention{-thing} and \mention{-where} are non-personal:

\begin{exe}
    \ex \label{ex:something} \textit{Something is on the table.} \hfill [non-personal]
    \ex \label{ex:something-bad} \ungram{\textit{Everything enjoyed themselves.}} \hfill [personal referents]
\end{exe}

Example (\ref{ex:something-bad}) requires personification to be grammatical. The \mention{-thing} compounds cannot normally designate persons.

\subsection{Relative pro-forms}

The distinction between \mention{where}/\mention{when} and \mention{which} parallels the distinction between \mention{who} and \mention{which}:

\begin{exe}
    \ex \label{ex:room-where} \textit{the room where the painting was done} \hfill [room as location]
    \ex \label{ex:room-which} \textit{the room which was painted} \hfill [room as patient]
\end{exe}

\begin{exe}
    \ex \label{ex:2010-when} \textit{2010, when I left} \hfill [as time]
    \ex \label{ex:2010-which} \textit{2010, which has 365 days} \hfill [as entity]
\end{exe}

The same source (\mention{room}, \mention{2010}) takes different pro-forms depending on how it's conceptualized. This is designatum-driven selection applied to locative and temporal subtypes.

%--- --- --- --- --- --- --- --- --- --- --- --- --- --- --- --- ---
\section{Chain coherence}
\label{sec:12:chain-coherence}
%--- --- --- --- --- --- --- --- --- --- --- --- --- --- --- --- ---

Once a designatum is construed as personal or non-personal within a reference chain, mixing is disfavored. Consider:

\begin{exe}
    \ex \label{ex:dog-chain-a} \textit{That's the dog who attacked his owner.} \hfill [personal chain]
    \ex \label{ex:dog-chain-b} \textit{That's the dog which attacked its owner.} \hfill [non-personal chain]
    \ex[?] \label{ex:dog-chain-c} \textit{That's the dog which attacked his owner.} \hfill [mixed]
    \ex[?] \label{ex:dog-chain-d} \textit{That's the dog who attacked its owner.} \hfill [mixed]
\end{exe}

Examples (\ref{ex:dog-chain-a}) and (\ref{ex:dog-chain-b}) are fully acceptable~-- coherent personal and non-personal chains. Examples (\ref{ex:dog-chain-c}) and (\ref{ex:dog-chain-d}) are degraded~-- the chain starts in one gender and shifts to another mid-sentence.

This is \term{chain-internal coherence}. The constraint isn't absolute~-- discourse can independently reclassify a designatum (personification, metonymy)~-- but absent such cues, speakers prefer consistency within a reference chain.

Chain coherence is a maintenance mechanism. It stabilizes the personhood assignment across discourse, reinforcing the clustering of pro-forms around a shared construal.

%--- --- --- --- --- --- --- --- --- --- --- --- --- --- --- --- ---
\section{The coupling}
\label{sec:12:coupling}
%--- --- --- --- --- --- --- --- --- --- --- --- --- --- --- --- ---

We now have the architecture. On the semantic side sits the \term{personhood cluster}: the conceptual properties that make a referent construable as a person~-- intentionality, agency, consciousness, reciprocal treatment. These are the properties Dennett's philosophical account identifies \autocite{dennett1976}. They cluster because personhood is a natural cognitive category, grounded in Theory of Mind and social cognition \autocite{waytz2010}.

On the morphosyntactic side sits the \term{pro-form cluster}: the inventory of gender-sensitive lexical items and their distributional constraints~-- \mention{who}/\mention{which} in relative clauses, \mention{-body}/\mention{-thing} in compound determinatives, designated slots for personal and non-personal reference.

The two clusters are coupled by \term{designatum-driven inference}. When speakers produce a pro-form, they signal how they're construing the referent. When hearers process a pro-form, they infer the construal. The inference runs bidirectionally: form cues construal; construal constrains form.

This parallels the bidirectional inference that couples individuation to count morphosyntax (Chapter~\ref{ch:countability}) and identifiability to deitality (Chapter~\ref{ch:definiteness-and-deitality}). The coupling is what produces the HPC signature: properties that statistically co-occur, maintained by causal mechanisms, with graded membership at the edges.

\subsection{The machinery of maintenance}

In Chapter~\ref{ch:stabilisers}, we surveyed the general machinery that maintains linguistic kinds. For pro-form gender, three mechanisms do the heaviest work.

\term{Cognitive grounding} provides the anchor. Personhood attribution is cognitively basic~-- grounded in Theory of Mind, face processing, and social cognition \autocite{waytz2010}. Infants distinguish agents from non-agents within months of birth; the person/non-person boundary is among the earliest conceptual distinctions humans make. This gives the personhood cluster a stability that purely grammatical categories lack: it's rooted in pre-linguistic cognition, not just in distributional patterns.

\term{Semantic transparency} reinforces the coupling. The mapping from personhood to pro-form is largely predictable: if you know a referent is construed as a person, you can predict the pro-form inventory it will take. This transparency makes the system easy to learn and stable under transmission. Compare countability, where object-mass nouns like \mention{furniture} create opaque exceptions, or definiteness, where weak definites break the form--meaning correspondence. Gender has fewer such mismatches: the \mention{who}/\mention{which} alternation tracks personhood with high fidelity.

\term{Error and repair} operates through chain coherence. When a speaker shifts from \mention{who} to \mention{it} mid-chain, hearers notice. The mismatch triggers puzzlement, clarification requests, or mental correction. Unlike mismatches in, say, definiteness, gender mismatches are socially charged~-- calling a person \mention{it} is an insult; calling a ship \mention{who} is an eccentricity. This asymmetry in repair pressure keeps the personal/non-personal boundary sharp even as individual items (pets, robots) shift position.

These three~-- cognitive grounding, semantic transparency, error-and-repair~-- account for the system's distinctive stability. Standard mechanisms (acquisition, entrenchment, transmission) also apply, but they don't explain why gender is more stable than, say, the count/non-count distinction for \mention{data}. What explains the stability is the convergence: personhood is cognitively basic, the mapping is transparent, and violations are repaired.

%--- --- --- --- --- --- --- --- --- --- --- --- --- --- --- --- ---
\section{Passing the tests}
\label{sec:12:tests}
%--- --- --- --- --- --- --- --- --- --- --- --- --- --- --- --- ---

Chapter~\ref{ch:failure-modes} introduced the Two-Diagnostic Test for genuine HPC kinds: high projectibility and robust homeostasis.

\subsection{Projectibility}

Knowing a pro-form's gender predicts its distribution. If you know \mention{somebody} is personal, you can predict it will combine with personal predicates, take personal reflexives (\mention{themselves}), and resist non-personal contexts. If you know \mention{which} is non-personal, you can predict it cannot take human antecedents without personification.

Crucially, knowing the designatum's construal predicts pro-form selection. If you know the speaker is construing a pet as personal, you can predict \mention{who} over \mention{which}, \mention{he}/\mention{she} over \mention{it}. The predictions are bidirectional: from form to construal, from construal to form.

\subsection{Homeostasis}

The clustering is maintained by mechanisms. Acquisition transmits the system as a unit; entrenchment preserves high-frequency patterns; alignment provides real-time stabilization; transmission filters for learnability. Perturb any mechanism~-- expose children to non-standard input, isolate speakers from conversational feedback~-- and the cluster should degrade. But it doesn't scatter randomly; the degradation should track the mechanisms involved.

The clustering isn't just a statistical regularity. It's causally maintained.

%--- --- --- --- --- --- --- --- --- --- --- --- --- --- --- --- ---
\section{Where the clusters slip}
\label{sec:12:slippage}
%--- --- --- --- --- --- --- --- --- --- --- --- --- --- --- --- ---

Like countability and definiteness, the pro-form gender system shows characteristic slippage. These aren't anomalies; they're evidence of the underlying dual-cluster architecture.

\subsection{Pets and anthropomorphism}

Pets occupy the boundary. \textcite{shirvertesh2012} demonstrates that pets are often treated as persons, but flexibly~-- the attribution can shift with situation. This flexibility appears in pronoun usage:

\begin{exe}
    \ex \textit{The cat licked its paw.} \hfill [non-personal default]
    \ex \textit{Poor Whiskers! She's not feeling well.} \hfill [personified]
\end{exe}

Same referent, different construals. The slippage is systematic: speakers shift between personal and non-personal based on discourse framing, emotional involvement, and naming.

\subsection{Collectives}

Collective nouns present a related puzzle:

\begin{exe}
    \ex \textit{The team has scored its first goal.} \hfill [singular, non-personal]
    \ex \textit{The team have scored their first goal.} \hfill [plural, personal]
\end{exe}

The variation tracks construal: team-as-entity versus team-as-members. But singular personal construal is blocked:

\begin{exe}
    \ex[*] \textit{The team has scored on themself.}
\end{exe}

Collectives can be singular non-persons or plural persons, but not singular persons. The constraint reveals the structure of the personhood cluster.

\subsection{Infants}

Human infants, paradigmatic persons, can take non-personal \mention{it}:

\begin{exe}
    \ex \textit{The baby was crying because it was hungry.}
\end{exe}

This isn't a denial of personhood; it reflects temporary withholding of person-status~-- often when sex is unknown or when the discourse foregrounds non-personal properties. The slippage reveals that personhood is an attribution, not a biological fact.

%--- --- --- --- --- --- --- --- --- --- --- --- --- --- --- --- ---
\section{Looking forward}
\label{sec:12:transition}
%--- --- --- --- --- --- --- --- --- --- --- --- --- --- --- --- ---

Pro-form gender is a localized system. The relevant constraints are carried by a specific class of lexical items~-- those that function semantically as pro-forms~-- and enforced across anaphoric chains. The coupling between personhood and morphosyntax is tight within that domain.

The final case study pushes to the widest scope. Lexical categories like \term{noun} and \term{verb} look strikingly stable cross-linguistically, while \term{adjective} and \term{adposition} vary dramatically. If these are HPC kinds, the mechanisms that maintain them must be far more diffuse~-- operating across the entire grammar, not just within a circumscribed inventory. Chapter~\ref{ch:lexical-categories} asks what happens when the coupling is weak, the mechanisms are partial, and the cluster boundaries are contested.
