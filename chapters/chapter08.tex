\chapter{Failure modes}
\label{ch:failure-modes}

% =============================================================================
% AUTHORIAL NOTE: RECONSIDER CHAPTER ORDER
% =============================================================================
% At this point in drafting, consider whether the current order (Ch6 Projectibility 
% → Ch7 Mechanisms) is optimal, or whether reversing it (Mechanisms → Projectibility) 
% would make the argument tighter. 
%
% The concern: The Polish aspect example in Ch6 shows that distributional learning 
% beats semantic definitions, but the reader doesn't yet know what HPC mechanisms 
% *are*. Does the bridging paragraph (added 2024-12-06) adequately ground the example 
% in HPC, or does it still feel like a non sequitur? If the example would be stronger 
% with mechanisms already in hand, reorder. If the current order maintains momentum 
% (payoff first, then explanation), keep it.
%
% Test: After completing Ch7, read Ch6-7 in sequence and then reversed. Which order 
% makes the HPC framework feel more like a unified argument?
% =============================================================================

% TODO: Write chapter content
