\chapter{Definiteness and Deitality}
\label{ch:definiteness-and-deitality}

% Chapter 10: Second Part III case study
% Target: ~5,700 words
% Structure parallels Ch 9 (Countability)

\epigraph{\textit{She wears the veil.}}{— Semantically indefinite, grammatically definite}

%--- --- --- --- --- --- --- --- --- --- --- --- --- --- --- --- --
\section{The puzzle of the article}
\label{sec:10:hook}
%--- --- --- --- --- --- --- --- --- --- --- --- --- --- --- --- --

Consider two sentences:

\begin{exe}
    \ex \label{ex:veil} \textit{She wears the veil.}
    \ex \label{ex:hat} \textit{?She wears the hat.}
\end{exe}

The first is natural. The second is marked — it demands a context: which hat? But the curious thing is that (\ref{ex:veil}) doesn't demand any such context. We don't ask \enquote{which veil?} We understand, without prompting, that the sentence is about a practice, not a particular piece of cloth.

This is the \term{weak definite} puzzle. Standard accounts tell us that \mention{the} signals uniqueness or familiarity — there's exactly one contextually salient referent, or the referent has already been introduced in discourse. But in \mention{She wears the veil}, there's no unique veil and no prior mention. The article is doing something else.

Compare the related cases:

\begin{exe}
    \ex \textit{play the piano} — no particular piano
    \ex \textit{go to the hospital} (BrE) — not a specific hospital
    \ex \textit{The tiger is endangered} — the species, not an individual
    \ex \textit{The Pope visited Canada} — unique by role, not by discourse
\end{exe}

Each uses \mention{the}, but none involves the standard uniqueness or familiarity. These aren't marginal exceptions; they're high-frequency, fully productive patterns. A theory of \mention{the} that treats them as anomalies is a theory that can't explain the data.

The problem runs deeper than weak definites. Proper names like \mention{Kim} and \mention{London} are semantically definite — they identify unique, familiar referents — yet they take no article in English. Bare plurals like \mention{dogs bark} achieve generic reference without any marking at all. The mapping between form and function is systematic, but it isn't one-to-one.

This chapter argues that we've been conflating two categories. One is morphosyntactic: a cluster of grammatical properties that travel together. The other is semantic: a cluster of interpretive properties related to referent identifiability. They correlate strongly — most definite referents get marked as such, and most marked forms signal definiteness. But they aren't identical. They're two HPCs maintained by different mechanisms, and their imperfect alignment produces exactly the puzzles that have troubled the literature.

%--- --- --- --- --- --- --- --- --- --- --- --- --- --- --- --- --
\section{One form, two functions}
\label{sec:10:two-functions}
%--- --- --- --- --- --- --- --- --- --- --- --- --- --- --- --- --

The standard view has good reasons: \mention{the} overwhelmingly correlates with identifiable referents, and most semantic theories of definiteness derive the correlation from compositional semantics. The puzzle isn't that the correlation exists — it's that it admits systematic exceptions.

The split we need parallels what we saw with countability. Chapter~\ref{ch:countability} distinguished the \term{individuation cluster} (semantic) from the \term{count cluster} (morphosyntactic). The two were coupled by bidirectional inference but maintained by different mechanisms. The same architecture applies here.

On the semantic side sits the \term{definiteness cluster}: the interpretive properties that make a referent identifiable. These include uniqueness (there's only one candidate), familiarity (the referent is discourse-old), and anaphoric recoverability (the referent can be picked up by a pronoun). When all three align, we have a prototypically definite referent — \mention{the cat we discussed yesterday}.

On the morphosyntactic side sits what I'll call the \term{form cluster} — the grammatical properties that travel together in English determiners. These include resistance to existential \mention{there} under neutral prosody, eligibility as the complement of partitive \mention{of}, and suitability as hosts for nonrestrictive modification. When a determiner shows all these properties, it's at the centre of the form cluster — \mention{the}, \mention{this}, \mention{those}.

The crucial observation is that these clusters can decouple. Weak definites like \mention{go to the hospital} show the form-cluster properties — they resist \mention{*There's the hospital down the street}, they work in \mention{one of the hospitals} — but they lack the definiteness-cluster properties: no unique hospital, no familiar referent. Proper names show the reverse pattern: fully definite semantically, but lacking the form-cluster marking entirely.

This decoupling is what we'd expect if the two clusters are distinct HPCs. They overlap substantially because the mechanisms that maintain them are historically and functionally linked. But they're not identical, and the gaps reveal the underlying structure.

The next sections characterize each cluster in turn: first definiteness (semantic), then the form cluster (morphosyntactic). Only then will we ask what couples them and what makes them come apart.

%--- --- --- --- --- --- --- --- --- --- --- --- --- --- --- --- --
\section{The definiteness cluster}
\label{sec:10:definiteness-cluster}
%--- --- --- --- --- --- --- --- --- --- --- --- --- --- --- --- --

Definiteness, as a semantic category, is maintained by discourse-pragmatic mechanisms: common ground management, bridging, and topic continuity. The cluster comprises several properties that typically travel together.

\term{Identifiability} is the core: a definite referent is one the hearer can pick out. This doesn't mean they already know it — first-mention definites like \mention{the first person on Mars} work perfectly well. It means the description is sufficient for identification.

\term{Uniqueness} is closely related: in the relevant context, exactly one entity satisfies the description. \textcite{russell1905} formalized this as part of the logical form of definite descriptions. Later work refined it to allow pragmatic restriction — uniqueness holds relative to a shared frame of reference \citep{hawkins1978}.

\term{Familiarity} comes from a different tradition. \textcite{heim1982} argued that definite noun phrases must refer to entities already in the discourse model. Indefinites introduce new referents; definites presuppose existing ones. This explains the felicity of \mention{I met a student. The student was tired} — the definite picks up what the indefinite introduced.

\term{Anaphoric recoverability} is the practical consequence: definite referents can be tracked across discourse. They're the natural antecedents for pronouns, the topics of subsequent sentences, the entities we can keep talking about.

These properties cluster because of how discourse works. Topics are identifiable; identifiable referents get tracked; tracked referents become unique within the conversation. The mechanisms are cognitive and interactive: speakers assume shared knowledge, hearers accommodate, and both converge on a common ground where definite reference succeeds.

But the properties don't always co-occur. A cataphoric definite like \mention{The idea that she quit surprised me} achieves uniqueness without prior familiarity — the complement clause provides the identifying content. Role definites like \mention{the Pope} are unique by social role, not by discourse history. The cluster has a prototype (all properties present) and a periphery (some properties missing).

This is exactly the HPC signature: a family of properties that statistically co-occur, maintained by causal mechanisms, with graded membership at the edges. Definiteness is a semantic HPC.

%--- --- --- --- --- --- --- --- --- --- --- --- --- --- --- --- --
\section{The form cluster}
\label{sec:10:form-cluster}
%--- --- --- --- --- --- --- --- --- --- --- --- --- --- --- --- --

The morphosyntactic cluster is defined by distributional diagnostics. These are grammatical tests, not semantic ones — they target structural behaviour, not meaning. Four diagnostics converge to define the cluster.

\subsection{Existential \mention{there}}

The definiteness effect is well established \citep{milsark1977}: under neutral prosody, certain determiners resist appearing in the pivot position after existential \mention{there}.

\begin{exe}
    \ex \label{ex:there-bad} \ungram{\textit{There is the key on the table.}}
    \ex \label{ex:there-good} \textit{There is a key on the table.}
\end{exe}

The constraint is sensitive to prosody. List intonation can rescue otherwise unacceptable pivots: \mention{Well, there's THE KEY, the wallet...} But under neutral prosody, the constraint is robust.

\subsection{Partitive \mention{of}}

In true partitive constructions with subset semantics, the complement must come from the form cluster:

\begin{exe}
    \ex \textit{Two of the students left.}
    \ex \textit{Several of these books are damaged.}
    \ex \label{ex:part-bad} \ungram{\textit{Two of some students left.}}
\end{exe}

This selectional restriction doesn't yield to prosodic manipulation. It's grammatical, not discourse-pragmatic.

\subsection{Identificational hosting}

Determiners from the form cluster are natural hosts for nonrestrictive modification, topics, and specificational subjects:

\begin{exe}
    \ex \textit{The book, which I bought yesterday, is excellent.}
    \ex \textit{As for this report, I've already filed it.}
    \ex \textit{The culprit is John.}
\end{exe}

Non-form-cluster determiners are degraded in these contexts:

\begin{exe}
    \ex \odd{\textit{A book, which I bought yesterday, is excellent.}}
    \ex \odd{\textit{As for a report, I've filed it.}}
\end{exe}

\subsection{\mention{One}-substitution}

Non-form-cluster determiners allow bare anaphoric \mention{one}:

\begin{exe}
    \ex \textit{I bought a red pen and you bought one too.}
\end{exe}

Form-cluster determiners resist it:

\begin{exe}
    \ex \ungram{\textit{I bought the red pen and you bought one too.}}
\end{exe}

This test is narrower than the others and shows more dialectal variation, but it corroborates the pattern.

\subsection{Convergence}

Table~\ref{tab:10:diagnostics} summarizes the diagnostic profile. The prototypical form-cluster determiners — \mention{the}, demonstratives, genitives — show all four properties. Indefinite determiners show none. And some items show mixed profiles, exactly as the HPC framework predicts.

\begin{table}[htbp]
\centering
\caption{Diagnostic profile for English determiners. \checkmark\ = exhibits form-cluster behaviour.}
\label{tab:10:diagnostics}
\begin{tabular}{lcccc}
\toprule
& \mention{there} & Partitive & Hosting & \mention{one} \\
\midrule
\mention{the} & \checkmark & \checkmark & \checkmark & \checkmark \\
\mention{this/that} & \checkmark & \checkmark & \checkmark & \checkmark \\
Genitives & \checkmark & \checkmark & \checkmark & \checkmark \\
\mention{each/every} & \checkmark & — & — & \checkmark \\
\mention{a/an} & — & — & — & — \\
\mention{some} & — & — & — & — \\
\bottomrule
\end{tabular}
\end{table}

No single diagnostic defines the cluster. No property is necessary, no set sufficient. What matters is convergence — the homeostatic pattern that essentialist accounts would need to treat as exceptional rather than predicted.

%--- --- --- --- --- --- --- --- --- --- --- --- --- --- --- --- --
\section{The coupling}
\label{sec:10:coupling}
%--- --- --- --- --- --- --- --- --- --- --- --- --- --- --- --- --

We now have two clusters: definiteness (semantic) and the form cluster (morphosyntactic). Why do they correlate? And why isn't the correlation perfect?

The correlation arises because the clusters share a common origin. Cross-linguistically, definite articles arise primarily from demonstratives \citep{diessel1999,greenberg1978}. Demonstratives are inherently deictic — they point — and pointing presupposes a referent that both speaker and hearer can identify. As demonstratives grammaticalize into articles, this functional association persists. The article inherits a strong bias toward identifiable referents even as its semantic contribution generalizes.

This is what \textcite{millikan1984} would call a \term{proper function}: the article's job is to signal definiteness, and that job description gets transmitted through the mechanisms we'll examine next. Most of the time, the form cluster and the definiteness cluster align because speakers use form-cluster determiners for definite referents and vice versa. The correlation is maintained by high-frequency co-occurrence and pragmatic inference: hearers assume identifiability when they hear \mention{the}.

But the correlation isn't identity. The clusters can decouple for two reasons.

First, \term{grammaticalization preserves formal residue}. When a demonstrative becomes an article, it drags its distributional properties along — what \textcite{himmelmann1997} calls \enquote{context expansion.} The form-cluster diagnostics (partitive licensing, \mention{there}-resistance) are inherited from the demonstrative source even when the semantic contribution has shifted. Weak definites represent conventionalized frames where the morphosyntactic pattern persisted after the semantic link loosened.

Second, \term{constructional autonomy} allows the clusters to develop independently. Once a morphosyntactic pattern is established, it can be recruited for new functions. The narrative \mention{this} construction — \mention{There was this guy...} — uses demonstrative morphology for specificity and immediacy rather than definiteness. The form-cluster properties are intact, but the definiteness-cluster properties are absent.

The result is a stable misalignment. Most definite referents are marked with form-cluster determiners. Most form-cluster determiners signal definiteness. But some don't — and the exceptions are systematic, not random.

%--- --- --- --- --- --- --- --- --- --- --- --- --- --- --- --- --
\section{Five mechanisms}
\label{sec:10:mechanisms}
%--- --- --- --- --- --- --- --- --- --- --- --- --- --- --- --- --

The form cluster is maintained by five mechanisms operating at different timescales. Each contributes to the cluster's stability; together they explain why the pattern persists despite constant pressure from analogy and variation.

\subsection{Grammaticalization (centuries)}

As demonstratives become articles, they carry their distributional properties along. English \mention{the} descends from a Germanic demonstrative and retains demonstrative-like behaviour: resistance to existential pivots, partitive licensing, preferential hosting. The cluster wasn't stipulated; it was inherited.

The shared ancestry even leaves phonological traces. All /ð/-initial determiners in English — \mention{the}, \mention{this}, \mention{that}, \mention{these}, \mention{those} — are form-cluster items derived from demonstrative sources. The phonological clustering and the distributional clustering have the same origin.

\subsection{Acquisition (childhood)}

Each generation relearns the cluster. Children acquire \mention{the} early and overgeneralize it — using it where adults would use \mention{a} \citep{maratsos1976,rozendaalbaker2008}. Crucially, they learn the distributional frame before mastering the full semantic range. They know where \mention{the} appears (determiner slot, partitive complement) before they know exactly when referents count as identifiable.

This creates a bootstrapping dynamic: the distributional frame helps learners infer semantic function, and vice versa. The coupling is learned as a package, not as independent facts.

\subsection{Alignment (seconds to minutes)}

In conversation, speakers converge on syntactic frames through interactive alignment \citep{pickeringgarrod2004}. When someone uses \mention{the} in a weak-definite frame, their interlocutor accommodates. When someone violates a form-cluster constraint — producing \mention{*There's the solution} under neutral prosody — hearers flag the anomaly through hesitation or repair. This real-time feedback stabilizes the cluster within conversations.

\subsection{Prestige selection (decades)}

Some variants spread faster because they're associated with high-status speakers \citep{labov2001}. This shapes which weak-definite frames become conventionalized. British \mention{in hospital} persists as a prestige variant; American \mention{in the hospital} is the local norm. The mechanism doesn't create the cluster, but it determines which variants survive community transmission.

\subsection{Transmission (generations)}

The multi-generational bottleneck filters for learnability \citep{kirby2014}. Patterns that are too complex or inconsistent don't survive. This explains why structural constraints converge across dialects: all English dialects reject \mention{*Which did you buy car?} because the constraint is simple and stable. Weak-definite frames vary more because they're conventionalized idioms, learnable but not structurally forced.

These mechanisms interlock. Grammaticalization creates the bundle; acquisition transmits it; alignment maintains it in real time; prestige shapes which variants survive; transmission filters for stability. Perturb any mechanism, and the equilibrium shifts — but the cluster persists because the other mechanisms compensate.

%--- --- --- --- --- --- --- --- --- --- --- --- --- --- --- --- --
\section{When the clusters slip}
\label{sec:10:slippage}
%--- --- --- --- --- --- --- --- --- --- --- --- --- --- --- --- --

The explanatory power of the HPC framework shows when we examine cases where the two clusters dissociate. These aren't anomalies to be explained away — they're the natural consequence of having two distinct HPCs with imperfect correlation.

\subsection{Weak definites}

Expressions like \mention{take the bus}, \mention{listen to the radio}, and \mention{go to the hospital} have troubled semantic theories for decades \citep{carlsonsussman2005,aguilarguevarazwarts2010}. The puzzle is how \mention{the} appears without unique or familiar referents.

The HPC framework dissolves the puzzle. Weak definites show the full form-cluster profile — they resist neutral-prosody \mention{there}-pivots, they pattern with other form-cluster items in partitives. But they lack the definiteness-cluster properties: no unique hospital, no familiar referent, no anaphoric recoverability.

What they have is \term{institutionalization}. The morphosyntactic frame — verb + \mention{the} + noun in this conventionalized construction — persists as a grammaticalized fossil. The semantic contribution has shifted to an activity type rather than a specific referent. \mention{Go to the hospital} means something like \enquote{seek medical treatment}; the hospital itself is not identified.

This is exactly what the mechanism story predicts. Grammaticalization preserves formal residue. The weak-definite frames are fossils: morphosyntactically form-cluster, semantically non-definite, pragmatically conventionalized.

\subsection{Generic definites}

Sentences like \mention{The tiger is endangered} or \mention{The computer changed the world} pose a different challenge. The definite singular is used for kind reference, not individual reference \citep{carlson1977,krifka2004}.

Again, the HPC framework provides clarity. Generic definites are form-cluster items: they resist neutral existential contexts, they pattern distributionally with \mention{the}. But their semantics is kind-denoting, not individual-referring. Individual-level uniqueness doesn't apply; kind-level properties do.

The form cluster is intact. The definiteness cluster is orthogonal — neither satisfied nor violated, simply inapplicable at the kind level. The syntax generates a form-cluster phrase; semantics assigns it a generic interpretation that sidesteps the definiteness question.

\subsection{Proper names}

Proper names present the mirror image. \mention{Kim} and \mention{London} are semantically definite — they identify unique, familiar referents — yet they lack form-cluster marking in English.

This follows from the HPC architecture. Names show definiteness-cluster properties (identifiability, uniqueness, anaphoric recoverability) without the form-cluster properties. They don't resist existential pivots: \mention{There's a Kim here to see you} is natural. They don't license partitives in the standard way.

The clusters have decoupled in the opposite direction from weak definites. Where weak definites have form without function, proper names have function without form.

English even reveals the underlying logic when names collide with constructions that have indefinite defaults. Bare plurals are normally indefinite: \mention{dogs}, \mention{books}. Family names need pluralization but remain definite: \mention{the Smiths}, \mention{the Johnsons}. The article overrides the constructional default to preserve definiteness.

The same dissociation appears when we push names toward generic readings. Predicativists analyse bare singular names like \mention{Ruth} as containing a null definite determiner — structurally parallel to \mention{the tiger}. If so, we'd expect bare singular names to allow generic readings, just as \mention{the tiger} does. \textcite{gasparri2025} shows they can: \mention{Italian Andrea is generally male}; \mention{Ruth has good grades in biology} (in statistical contexts). But the generics are characterizing, not kind-level — \mention{*John became common} fails without quotation. The form (null-definite + name) decouples from canonical referential function, but only within limits maintained by the naming practice itself.

\subsection{Indefinite \mention{this}}

Colloquial English uses \mention{this} with discourse-new referents in narrative contexts:

\begin{exe}
    \ex \textit{There was this guy at the door...}
    \ex \textit{So I'm walking home when this dog starts following me...}
\end{exe}

This isn't an error — it's a productive, high-frequency pattern. The HPC account explains it as \term{constructional recruitment}. The narrative-presentational construction selects form-cluster morphology for specificity and immediacy while supplying indefinite semantics. The form-cluster properties remain: partitive restrictions still hold (\mention{*two of this guy}); \mention{one}-substitution is still blocked. But the definiteness-cluster properties are absent.

The construction itself has grammaticalized. It requires demonstrative form; it supplies non-definite meaning. This is exactly the form-meaning mismatch that two-HPC architecture predicts.

%--- --- --- --- --- --- --- --- --- --- --- --- --- --- --- --- --
\section{Passing the tests}
\label{sec:10:tests}
%--- --- --- --- --- --- --- --- --- --- --- --- --- --- --- --- --

Chapter~\ref{ch:failure-modes} introduced the Two-Diagnostic Test for genuine HPC kinds: high projectibility and robust homeostasis. Both clusters pass.

\subsection{Projectibility}

The definiteness cluster supports induction. Knowing a referent is semantically definite lets you predict its discourse behaviour: it can be picked up by pronouns, it can be topicalized, it can be presupposed. The form cluster supports induction too: knowing a determiner is form-cluster lets you predict structural behaviour across contexts.

Critically, the predictions are \term{dissociated}. You can know a noun phrase's form-cluster status without knowing its definiteness status — and the predictions you make will differ. Weak definites are form-cluster but not definiteness-cluster; proper names are definiteness-cluster but not form-cluster. Each category supports distinct inductions.

\subsection{Homeostasis}

Both clusters are maintained by mechanisms. The definiteness cluster is sustained by discourse-pragmatic processes: common ground management creates correlations among identifiability, uniqueness, and anaphoric recoverability. The form cluster is sustained by the five mechanisms detailed in §\ref{sec:10:mechanisms}.

The mechanism difference explains the decoupling. Discourse pressure cannot alter distributional restrictions; grammaticalization pressure doesn't change reference. The clusters drift independently because their maintenance is independent.

\subsection{Falsifiable predictions}

The HPC account generates specific predictions about how the clusters behave under manipulation:

\textbf{Prosodic rescue}: Prosody should selectively rescue existential pivots (a discourse-level constraint) but not partitive complements (a selectional restriction). This is testable and falsifiable.

\textbf{Dialectal preservation}: Dialects that drop \mention{the} in institutional frames (British \mention{in hospital}) should preserve form-cluster patterning when definiteness is supplied by demonstratives (\mention{in this hospital}).

\textbf{Acquisition asymmetry}: Children should master distributional restrictions before semantic ones, because distributional restrictions lack prosodic repair paths.

These predictions target the interaction between morphosyntax, prosody, semantics, and development — exactly what the mechanism story predicts.

%--- --- --- --- --- --- --- --- --- --- --- --- --- --- --- --- --
\section{The term: Deitality}
\label{sec:10:deitality}
%--- --- --- --- --- --- --- --- --- --- --- --- --- --- --- --- --

We need a name for the form cluster. The term \term{definiteness} is already taken — it names the semantic cluster. Calling both clusters \enquote{definiteness} would perpetuate exactly the conflation we've been trying to undo.

I propose \term{deitality}. The term derives from the deictic origins of the cluster: demonstratives grammaticalize into articles, and the distributional profile reflects that deictic source. A determiner is \term{deital} if it shows the form-cluster properties; it is \term{definite} if it contributes definiteness-cluster semantics.

The term is ugly — deliberately so. Its unfamiliarity forces a break with the assumption that form and function are identical. When you say \mention{deital}, you cannot accidentally mean \mention{definite}. The conceptual separation becomes lexicalized.

The utility is in the clarity. We can now say: weak definites are deital but indefinite; proper names are definite but non-deital; generic definites are deital and semantically orthogonal. Each statement is precise in a way the traditional terminology doesn't allow.

%--- --- --- --- --- --- --- --- --- --- --- --- --- --- --- --- --
\section{Cross-linguistic scope}
\label{sec:10:cross-linguistic}
%--- --- --- --- --- --- --- --- --- --- --- --- --- --- --- --- --

Is deitality an English-specific category or a universal potential?

The form cluster as described here — the specific diagnostics, the specific determiners — is English-specific. It reflects the grammaticalization history of English demonstratives, the selectional restrictions of English constructions, the conventions of English discourse.

But the \term{architecture} generalizes. Languages with demonstrative-derived articles (French \mention{le}, German \mention{der}, Greek \mention{o}) show similar clusters: partitive restrictions, information-structural constraints, imperfect correlation with semantic definiteness. The diagnostics differ; the pattern of convergent distributional properties maintained by grammaticalization persists.

Classifier languages show a different realization of the same underlying dynamic. Japanese lacks articles but has a rich demonstrative system (\mention{kono}/\mention{sono}/\mention{ano}) with its own distributional profile. The semantic definiteness cluster exists cross-linguistically; the morphosyntactic form cluster takes language-specific shapes.

The prediction is that wherever demonstratives grammaticalize into articles, the form cluster should emerge — because grammaticalization drags distributional properties along. The cluster is not stipulated; it's an emergent consequence of how grammaticalization works.

%--- --- --- --- --- --- --- --- --- --- --- --- --- --- --- --- --
\section{Looking forward}
\label{sec:10:transition}
%--- --- --- --- --- --- --- --- --- --- --- --- --- --- --- --- --

Countability and definiteness share an architecture: two HPCs — one semantic, one morphosyntactic — maintained by different mechanisms, coupled by convention and inference, capable of systematic decoupling. The semantic cluster tracks a cognitive or conceptual distinction (individuation, identifiability); the morphosyntactic cluster tracks a distributional profile inherited from historical sources and maintained by social and developmental mechanisms.

But countability and definiteness differ in tightness. The count cluster and the individuation cluster are closely coupled — most count morphosyntax signals individuation, and the bidirectional inference is strong. The form cluster (deitality) and the definiteness cluster are more loosely coupled — weak definites, generic definites, and proper names represent stable, productive dissociations.

The next case study pushes further. Lexical categories — noun, verb, adjective — show even more complex patterns. The noun/verb contrast is strikingly stable cross-linguistically; adjectives vary dramatically in both inventory and behaviour. If these are HPC kinds, the mechanisms maintaining them must themselves vary. Some may be deep and universal (cognitive, developmental); others may be local and reconfigurable (constructional, sociolinguistic).

The HPC framework accommodates this variation. The question is not \enquote{Is adjective a real category?} but \enquote{What maintains adjective-hood in this language, and how tightly does the cluster cohere?} Chapter~\ref{ch:lexical-categories} takes up that question.
