\chapter{Definiteness and Deitality}
\label{ch:definiteness-and-deitality}

% Chapter 10: Second Part III case study
% Target: ~5,700 words
% Structure parallels Ch 9 (Countability)

\epigraph{\textit{Not only was it difficult for him to comprehend that the generic symbol dog embraces so many unlike individuals of diverse size and form; it bothered him that the dog at three-fourteen (seen from the side) should have the same name as the dog at three-fifteen (seen from the front).}}{— Borges, \textit{Funes the Memorious} (1942)}

%--- --- --- --- --- --- --- --- --- --- --- --- --- --- --- --- --
\section{The puzzle of \mentionhead{the}}
\label{sec:10:hook}
%--- --- --- --- --- --- --- --- --- --- --- --- --- --- --- --- --

Consider two sentences:

\ea[]{\label{ex:veil}\mention{She wears the veil.}}
\z
\ea[]{\label{ex:hat}\mention{?She wears the hat.}}
\z

The first is natural. The second is marked~-- it demands a context: which hat? But the curious thing is that (\ref{ex:veil}) doesn't demand any such context. We don't ask \enquote{which veil?} We understand, without prompting, that the sentence is about a practice, not a particular piece of cloth.

This is the \term{weak definite} puzzle. Standard accounts tell us that \mention{the} signals uniqueness or familiarity~-- there's exactly one contextually salient referent, or the referent has already been introduced in discourse. But in \mention{She wears the veil}, there's no unique veil and no prior mention. The article is doing something else.

Compare the related cases:

\ea[]{\mention{play the piano}~-- no particular piano}
\z
\ea[]{\mention{go to the hospital} (AmE)~-- not a specific hospital}
\z
\ea[]{\mention{The tiger is endangered}~-- the species, not an individual}
\z
\ea[]{\mention{The Pope visited Canada}~-- unique by role, not by discourse}
\z

Each uses \mention{the}, but none involves the standard uniqueness or familiarity. The bus is never the same bus, and yet we keep giving it the same article. These aren't marginal exceptions; they're high-frequency, fully productive patterns \citep{carlsonsussman2005,poesio1994}. A theory of \mention{the} that treats them as anomalies risks missing the systematicity of the pattern.

The problem runs deeper than weak definites. Proper names like \mention{Kim} and \mention{London} are semantically definite~-- they identify unique, familiar referents~-- but they take no article in English. Bare plurals like \mention{dogs bark} achieve generic reference without any marking at all. The mapping between form and function is systematic, but it isn't one-to-one. Definite without the article; the article without definiteness.

This chapter argues that we've been conflating two categories. One is morphosyntactic: a cluster of grammatical properties that travel together. The other is semantic: a cluster of interpretive properties related to referent identifiability. They correlate strongly~-- most definite referents get marked as such, and most marked forms signal definiteness. But they aren't identical. They're two HPCs maintained by different mechanisms, and their imperfect alignment produces exactly the puzzles that have troubled the literature. Definiteness is a different basin~-- shallower than countability, with slippage at the edges where the form--value coupling loosens.

%--- --- --- --- --- --- --- --- --- --- --- --- --- --- --- --- --
\section{One form, two values}
\label{sec:10:two-functions}
%--- --- --- --- --- --- --- --- --- --- --- --- --- --- --- --- --

The standard view has good reasons: \mention{the} overwhelmingly correlates with identifiable referents, and most semantic theories of definiteness derive the correlation from compositional semantics. The puzzle isn't that the correlation exists~-- it's that it admits systematic exceptions.

The literature offers several competing diagnoses. Uniqueness-based accounts \citep{russell1905,heim1991} treat \mention{the} as encoding a presupposition that exactly one entity satisfies the description; exceptions become pragmatic accommodation or domain narrowing. If you have to keep saying \enquote{accommodation,} you're probably just squinting. Familiarity-based accounts \citep{heim1982,christophersen1939} require prior discourse introduction; apparent counterexamples involve bridging or situational salience. Domain-restriction accounts \citep{hawkins1978,roberts2003} focus on how context delimits the set of candidates; definiteness becomes identifiability within a shared frame. Constructional accounts \citep{carlsonsussman2005,aguilarguevarazwarts2010} treat systematically non-unique uses as conventionalized frames with distinct semantics. Each preserves something~-- compositionality, presuppositional uniformity, pragmatic flexibility, lexical economy~-- but none explains why the same distributional profile surfaces across such different semantic profiles.

Here is the decision criterion: any account that keeps definiteness purely semantic still needs to explain why \mention{the}, demonstratives, and genitives~-- but not \mention{a} or \mention{some}~-- share a distributional profile (existential resistance, partitive selection, hosting) that has nothing obvious to do with uniqueness or familiarity. Conversely, any account that treats weak definites as construction-specific semantics still owes an explanation for why the construction recruits exactly this morphological class. The two-cluster architecture isn't an alternative semantics; it's a demand that any semantics meet.

The split we need parallels what we saw with countability. Chapter~\ref{ch:countability} distinguished the \term{individuation cluster} (semantic) from the \term{count cluster} (morphosyntactic). The two were coupled by bidirectional inference but maintained by different mechanisms. The same architecture applies here.

On the semantic side sits the \term{definiteness cluster}: the interpretive properties that make a referent identifiable. These include \term{familiarity} (the referent is discourse-old), \term{uniqueness} (there's only one candidate in the relevant domain), and \term{identifiability} (the hearer can pick it out).\footnote{Specificity~-- whether the speaker has a particular referent in mind~-- is sometimes included, but it crosscuts the definite/indefinite distinction: indefinites can be specific (\mention{I'm looking for a book~-- it's on my shelf}) or non-specific (\mention{I'm looking for a book~-- any book}). Specificity tracks speaker knowledge; definiteness tracks hearer knowledge.} When all three align, we have a prototypically definite referent~-- \mention{the cat we discussed yesterday}. \term{anaphoric recoverability}~-- the capacity to be picked up by a pronoun~-- is a downstream discourse affordance, not a core property.

On the morphosyntactic side sits what I'll call the \term{form cluster}~-- or, in terminology we'll earn later, the \term{deitality cluster}. These are the grammatical properties that travel together in English determiners: resistance to existential \mention{there} under neutral prosody, eligibility as the complement of partitive \mention{of}, and suitability as hosts for nonrestrictive modification. When a determiner shows all these properties, it's at the centre of the form cluster~-- \mention{the}, \mention{this}, \mention{those}, \mention{my}.

The key observation is that these clusters can decouple. Weak definites like \mention{go to the hospital} show the form-cluster properties~-- they resist \mention{*There's the hospital down the street}, they work in \mention{one of the hospitals}~-- but they lack the definiteness-cluster properties: no unique hospital, no familiar referent. Proper names show the reverse pattern: fully definite semantically, but lacking the form-cluster marking entirely.

This decoupling is what we'd expect if the two clusters are distinct HPCs. They overlap substantially because the mechanisms that maintain them are historically and functionally linked. But they're not identical, and the gaps reveal the underlying structure.

The next sections characterize each cluster in turn: first definiteness (semantic), then the form cluster (morphosyntactic). Only then will we ask what couples them and what makes them come apart.

\begin{figure}[htbp]
    \centering
    \begin{tabular}{r|p{5cm}|p{5cm}|}
        \multicolumn{1}{c}{} & \multicolumn{1}{c}{\textbf{+ Definiteness}} & \multicolumn{1}{c}{\textbf{-- Definiteness}} \\
        & \multicolumn{1}{c}{\footnotesize(Identifiability, Uniqueness)} & \multicolumn{1}{c}{\footnotesize(Semantic Indefiniteness)} \\
        \cline{2-3}
        \textbf{+ Deitality} & \textbf{Core Coupled Items} \newline \textit{the cat}, demonstratives, genitives & \textbf{Form Without Meaning} \newline Weak definites (\textit{go to the hospital}), Generics (\textit{the tiger}) \\
        \footnotesize(Form Cluster) & & \\
        \cline{2-3}
        \textbf{-- Deitality} & \textbf{Meaning Without Form} \newline Proper names (\textit{Kim}, \textit{London}), definite bare plurals & \textbf{Non-Members} \newline \textit{a cat}, \textit{some cats}, indefinite bare plurals \\
        \footnotesize(No markings) & & \\
        \cline{2-3}
    \end{tabular}
    \caption{The decoupling of form and value. The diagnostics of the form cluster (deitality) and the semantic properties of the definiteness cluster usually align, but systematic dissociations reveal the underlying dual-cluster architecture.}
    \label{fig:10:decoupling}
\end{figure}

%--- --- --- --- --- --- --- --- --- --- --- --- --- --- --- --- --
\section{The definiteness cluster}
\label{sec:10:definiteness-cluster}
%--- --- --- --- --- --- --- --- --- --- --- --- --- --- --- --- --

Definiteness, as a semantic category, is maintained by discourse-pragmatic mechanisms operating at multiple timescales. The cluster comprises several properties that typically travel together~-- but acquisition research reveals they are learned sequentially, not as a bundle. First you track what's been mentioned; then you restrict the domain; only then do you model what your addressee can pick out.

\term{familiarity} emerges first. \textcite{heim1982} argued that definite noun phrases refer to entities already in the discourse model. Indefinites introduce new referents; definites presuppose existing ones. This explains the felicity of \mention{I met a student. The student was tired}~-- the definite picks up what the indefinite introduced. Children master this by age 2--3: they use \mention{the} appropriately for discourse-old referents before mastering when to withhold it for new ones \citep{rozendaalbaker2008}.

\term{uniqueness} develops next: in the relevant context, exactly one entity satisfies the description. \textcite{russell1905} formalized this as part of the logical form of definite descriptions; later work refined it to allow pragmatic restriction~-- uniqueness holds relative to a shared frame of reference \citep{hawkins1978}. But 3-year-olds who correctly use \mention{the} for familiar referents don't yet enforce uniqueness: \textcite{brockmann2018} found that children often fail to distinguish \mention{the} from \mention{a} when uniqueness is at stake. The uniqueness component stabilizes later.

\term{identifiability} is the hardest: a definite referent is one the hearer can pick out. This doesn't mean they already know it~-- first-mention definites like \mention{the first person on Mars} work perfectly well. It means the description is sufficient for identification \emph{from the hearer's perspective}. This requires Theory of Mind: the speaker has to represent what the hearer knows. Children show \enquote{egocentric definiteness}~-- using \mention{the} when they know the referent but the hearer does not~-- until around age 5--6, and the error correlates with ToM development \citep{decat2013}. Even adults with intact ToM make egocentric errors~-- participants in Director Task studies frequently select objects unknown to their interlocutor \citep{keysar2000}~-- suggesting that identifiability requires not just representational capacity but executive resources to suppress privileged knowledge in real time. Theory of Mind helps, but it's not a full-time employee.

\term{anaphoric recoverability} is the practical consequence: definite referents can be tracked across discourse. They're the natural antecedents for pronouns, the topics of subsequent sentences, the entities we can keep talking about.

These properties cluster because of how discourse works. Topics are identifiable; identifiable referents get tracked; tracked referents become unique within the conversation. But the developmental sequence~-- familiarity, then uniqueness, then identifiability~-- reveals that the components are cognitively dissociable. Three distinct mechanisms bind them:

\begin{itemize}
    \item \textbf{Discourse tracking} (seconds to minutes) binds familiarity by maintaining the given/new distinction.
    \item \textbf{Domain restriction} (seconds) binds uniqueness by narrowing the quantificational domain to the relevant set.
    \item \textbf{Theory of Mind} (years) binds identifiability by distinguishing speaker knowledge from hearer knowledge.
\end{itemize}

The properties don't always co-occur. A cataphoric definite like \mention{The idea that she quit surprised me} achieves uniqueness without prior familiarity~-- the complement clause provides the identifying content. Role definites like \mention{the Pope} are unique by social role, not by discourse history. The cluster has a prototype (all properties present) and a periphery (some properties missing).

This matches the HPC signature: a family of properties that statistically co-occur, maintained by causal mechanisms operating at different timescales, with graded membership at the edges, and projectibility across the cluster. The definiteness cluster is a semantic HPC~-- and like the form cluster, it's learned in stages, with the full adult system emerging only when all the binding mechanisms are in place. Children acquiring articleless languages (Japanese, Mandarin) show comparable developmental trajectories for definiteness interpretation~-- familiarity before uniqueness before identifiability~-- despite lacking the morphosyntactic form cluster, suggesting the semantic cluster is cognitively prior.

%--- --- --- --- --- --- --- --- --- --- --- --- --- --- --- --- --
\section{The form cluster}
\label{sec:10:form-cluster}
%--- --- --- --- --- --- --- --- --- --- --- --- --- --- --- --- --

The morphosyntactic cluster in English is defined by distributional diagnostics. These are grammatical tests, not semantic ones~-- they target structural behaviour, not meaning. Three diagnostics converge to define the cluster.

\subsection{Existential \mentionhead{there}}

The definiteness effect is well established \citep{milsark1977}: under neutral prosody, certain determiners resist appearing in the pivot position after existential \mention{there}.

\ea[*]{\label{ex:there-bad}\mention{There is the key/my key on the table.}}
\z
\ea[]{\label{ex:there-good}\mention{There is a key on the table.}}
\z

The constraint is sensitive to prosody. List intonation can rescue otherwise unacceptable pivots: \mention{Well, there's THE KEY, the wallet...} But under neutral prosody, the constraint is robust.

\subsection{Partitive \mentionhead{of}}

In true partitive constructions with subset semantics, the complement has to come from the form cluster:

\ea[]{\mention{Two of the students left.}}
\z
\ea[]{\mention{Several of these books are damaged.}}
\z
\ea[*]{\label{ex:part-bad}\mention{Two of some students left.}}
\z

This structural constraint doesn't yield to prosodic manipulation.

\subsection{Identificational hosting}

Determiners from the form cluster are natural hosts for nonrestrictive modification, topics, and specificational subjects. \mention{The book, which I bought yesterday, is excellent} is standard; \mention{?A book, which I bought yesterday, is excellent} typically requires specific licensing or marks the referent as unique despite the indefinite form.

\ea[]{\label{ex:host-good}\mention{The/This/My book, which I bought yesterday...}}
\z
\ea[]{\label{ex:host-bad}\mention{?A/Some book, which I bought yesterday...}}
\z

\subsection{Convergence}

Table~\ref{tab:10:diagnostics} summarizes the diagnostic profile. The prototypical form-cluster determiners~-- \mention{the}, demonstratives, genitives~-- show all three properties. Indefinite determiners show none. And some items show mixed profiles, exactly as the HPC framework predicts.

\begin{table}[htbp]
\centering
\caption{Diagnostic profile for English determiners. \checkmark\ = exhibits form-cluster behaviour;~-- = does not. Diagnostics: \mention{there} = resists existential \mention{there} under neutral prosody; Partitive = licenses partitive complement; Hosting = hosts nonrestrictive modification.}
\label{tab:10:diagnostics}
\begin{tabular}{lccc}
\toprule
& \mention{there} & Partitive & Hosting \\
\midrule
\multicolumn{4}{l}{\textit{Core Members}} \\
\mention{the} & \checkmark & \checkmark & \checkmark \\
\mention{this/that}\textsuperscript{*} & \checkmark & \checkmark & \checkmark \\
Genitives & \checkmark & \checkmark & \checkmark \\
\midrule
\multicolumn{4}{l}{\textit{Mixed/Peripheral}} \\
\mention{each/every} & \checkmark &~-- &~-- \\
\midrule
\multicolumn{4}{l}{\textit{Non-Members}} \\
\mention{a/an} &~-- &~-- &~-- \\
\mention{some} &~-- &~-- &~-- \\
\bottomrule
\end{tabular}

\vspace{2pt}
\footnotesize
*In deictic uses; narrative-presentational \mention{this} (\mention{There was this guy...}) is a derived-function construction that recruits demonstrative morphology for non-deictic use.
\end{table}

No single diagnostic defines the cluster. No property is necessary, no set sufficient. What matters is convergence~-- if the cluster is homeostatic, each diagnostic should be noisy; if each is noisy, none will be necessary; if none is necessary, convergence is what you should look for. The homeostatic pattern is exactly what essentialist accounts would need to treat as exceptional rather than predicted.

%--- --- --- --- --- --- --- --- --- --- --- --- --- --- --- --- --
\section{The coupling}
\label{sec:10:coupling}
%--- --- --- --- --- --- --- --- --- --- --- --- --- --- --- --- --

We now have two clusters: definiteness (semantic) and the form cluster (morphosyntactic). Why do they correlate? And why isn't the correlation perfect?

The correlation arises because the clusters share a common origin. Cross-linguistically, definite articles arise primarily from demonstratives \citep{diessel1999,greenberg1978}. Demonstratives are inherently deictic~-- they point~-- and pointing presupposes a referent that both speaker and hearer can identify. As demonstratives grammaticalize into articles, this functional association persists. The article inherits a strong bias toward identifiable referents even as its semantic contribution generalizes.

\textcite{millikan1984}'s \term{proper function} framework clarifies the logic. The article's proper function~-- what it was selected for, what makes utterances containing it reproductively successful~-- is signalling identifiability. Speakers use \mention{the} to direct hearers to a particular referent; when hearers recover that referent, communication succeeds; success sustains the convention. Under \term{Normal conditions}~-- the conditions under which the device performs its proper function~-- speaker and hearer share a discourse model, the referent is uniquely identifiable within that model, and the hearer can retrieve it.

But proper function isn't the whole story. Grammaticalization drags along more than meaning. As demonstratives become articles, they inherit distributional properties~-- what \textcite{himmelmann1997} calls \enquote{context expansion.} The form-cluster diagnostics (partitive licensing, \mention{there}-resistance) were not what the article was \emph{for}; they were side effects of the historical process. Once these side effects stabilize, they become available for recruitment.

This is where \term{derived proper function}s arise. Derived functions exploit stable features of a device for purposes other than what it was selected for. Weak definites exploit the article's form-cluster membership~-- its distributional profile~-- without performing the identifiability function. Generic definites exploit the article's capacity to pick out a domain-restricted entity, but shift the domain from individuals to kinds. In both cases, the derived function is \emph{parasitic}: it depends on the Normal function remaining intact for the convention to persist.

The parasitism explains why decoupling is tolerable. The overwhelming majority of \mention{the}-tokens still perform the proper function~-- signalling identifiability. Weak definites and generic definites are minority uses. As long as the Normal function is performed often enough to sustain the convention, the derived functions can persist without undermining the system. The slippage is systematic, not accidental: it occurs precisely where stable side effects offer something to exploit.

\begin{figure}[t]
\centering
\begin{tikzpicture}[
    % Node styles
    rootbox/.style={
        rectangle, draw, rounded corners=3pt,
        minimum width=4.5cm, minimum height=1.2cm,
        align=center, font=\small, text width=4.3cm
    },
    pathbox/.style={
        rectangle, draw, rounded corners=3pt,
        minimum width=3cm, minimum height=0.8cm,
        align=center, font=\small
    },
    derivedbox/.style={
        rectangle, draw, rounded corners=3pt,
        minimum width=3.4cm, minimum height=0.95cm,
        align=center, font=\small, text width=3.2cm
    },
    sideeffect/.style={
        rectangle, draw, dashed, rounded corners=2pt,
        minimum width=2.8cm, minimum height=1.4cm,
        align=center, font=\footnotesize, text width=2.6cm
    },
    % Arrow styles
    arrow/.style={->, >=stealth, thick},
    parasitic/.style={->, >=stealth, thick, dashed},
    % Label style
    edgelabel/.style={font=\footnotesize, fill=white, inner sep=2pt}
]

% === ROW 1: Demonstrative and Proper Function ===
\node[pathbox] (demo) {Demonstrative};
\node[rootbox, right=3.5cm of demo] (normal) {%
    \textbf{Proper function}\\[2pt]
    Signal identifiability\\[-1pt]
    {\footnotesize (under Normal conditions)}
};

% === ROW 2: Article ===
\node[pathbox, below=1.6cm of demo] (article) {Article (\textit{the})};

% === ROW 3: Side effects (below Article, aligned) ===
\node[sideeffect, below=1.2cm of article] (sideeffects) {%
    Distributional\\
    side effects\\[4pt]
    {\scriptsize \textit{there}-resistance}\\[-1pt]
    {\scriptsize partitive \textit{of}}\\[-1pt]
    {\scriptsize NR hosting}
};

% === Derived functions (to the right of side effects) ===
\node[derivedbox, right=1.8cm of sideeffects] (generic) {%
    \textbf{Generic definite}\\[-1pt]
    {\footnotesize \textit{The tiger is endangered}}\\[-1pt]
    {\footnotesize kind-level reference}
};
\node[derivedbox, above=0.5cm of generic] (weak) {%
    \textbf{Weak definite}\\[-1pt]
    {\footnotesize \textit{go to the hospital}}\\[-1pt]
    {\footnotesize activity-type meaning}
};
\node[derivedbox, below=0.5cm of generic] (narrative) {%
    \textbf{Narrative} \textit{this}\\[-1pt]
    {\footnotesize \textit{There was this guy...}}\\[-1pt]
    {\footnotesize file-opening}
};

% === ARROWS ===

% Grammaticalization (vertical, label on RIGHT)
\draw[arrow] (demo) -- node[edgelabel, right] {grammaticalization} (article);

% Function inherited (horizontal)
\draw[arrow] (demo.east) -- node[edgelabel, above=2pt] {function inherited} (normal.west);

% Article to side effects (vertical)
\draw[arrow] (article) -- node[edgelabel, right] {carries} (sideeffects);

% Side effects to derived functions (symmetric fan)
\draw[arrow] (sideeffects.east) -- ++(0.6,0) |- (weak.west);
\draw[arrow] (sideeffects.east) -- (generic.west);
\draw[arrow] (sideeffects.east) -- ++(0.6,0) |- (narrative.west);

% Parasitic dependency - stubs from each derived function (no arrows)
\draw[dashed, thick] (weak.east) -- ++(0.6, 0) coordinate (p1);
\draw[dashed, thick] (generic.east) -- ++(0.6, 0) coordinate (p2);
\draw[dashed, thick] (narrative.east) -- ++(0.6, 0) coordinate (p3);

% Vertical line connecting the stubs (no arrow)
\draw[dashed, thick] (p1) -- (p3);

% Line up and into Proper function (single arrow at destination)
\draw[parasitic] (p1) -| (normal.south);

% Label positioned on the vertical segment
\path (p1) -- (p1 |- normal.south) node[edgelabel, midway, right] {parasitic dependency};

\end{tikzpicture}
\caption{Derived proper functions of the English article. The demonstrative's function---signalling identifiability---was inherited by the article through grammaticalization. The process also carried distributional side effects (form-cluster properties) that were not what the article was \emph{for}. Derived functions exploit these stable side effects for purposes other than identifiability. The dashed arrow indicates parasitic dependency: derived functions persist only because the Normal function remains robust.}
\label{fig:derived-functions}
\end{figure}

%--- --- --- --- --- --- --- --- --- --- --- --- --- --- --- --- --
\section{The machinery of maintenance}
\label{sec:10:mechanisms}
%--- --- --- --- --- --- --- --- --- --- --- --- --- --- --- --- --

In Chapter~\ref{ch:stabilizers}, we saw the general machinery that maintains linguistic kinds. Here, we can see that machinery applied to the specific idiosyncrasy of the English article. The form cluster (deitality) isn't just a list of properties; it's a dynamic equilibrium maintained by at least five distinct pressures.

One mechanism is \term{grammaticalization}, which operates over decades to centuries. As demonstratives become articles, they carry their distributional properties along. English \mention{the} descends from a Germanic demonstrative and retains demonstrative-like behaviour: resistance to existential pivots, partitive licensing, preferential hosting. The cluster wasn't stipulated; it was inherited. This provides the cluster's \textbf{inertia}.

The shared ancestry even leaves phonological traces. All /ð/-initial determiners in English~-- \mention{the}, \mention{this}, \mention{that}, \mention{these}, \mention{those}~-- are form-cluster items derived from demonstrative sources. The phonological clustering and the distributional clustering have the same origin.\footnote{The phonological coherence itself likely acts as an additional stabilizing mechanism: a phonestheme that cues category membership and facilitates acquisition.}

Another mechanism is \term{acquisition}, which occurs in childhood. Each generation relearns the cluster. Children acquire \mention{the} early and overgeneralize it~-- using it where adults would use \mention{a} \citep{maratsos1976,rozendaalbaker2008}. But what matters most is that they learn the distributional frame before mastering the full semantic range. They know where \mention{the} appears (determiner slot, partitive complement) before they know exactly when to deploy it. This creates a \textbf{bootstrap}: the form cluster serves as the scaffolding on which the definiteness concept is constructed.

Third, \term{alignment} stabilizes the cluster in real time (seconds to minutes). In conversation, speakers converge on syntactic frames through interactive alignment \citep{pickeringgarrod2004}. When someone uses \mention{the} in a weak-definite frame, their interlocutor accommodates. When someone violates a form-cluster constraint~-- producing \mention{*There's the solution} under neutral prosody~-- hearers flag the anomaly through hesitation or repair. This real-time feedback provides \textbf{error-correction}, stabilizing the cluster within conversations.

Fourth, \term{prestige selection} determines survival over years to decades. Some variants spread faster because they're associated with high-status speakers \citep{labov2001}. This shapes which weak-definite frames become conventionalized. British \mention{in hospital} persists as a prestige variant; American \mention{in the hospital} is the local norm. The mechanism provides \textbf{protection} for arbitrary variants, shielding them from regularization.

A fifth mechanism is \term{transmission}, which acts as a filter over generations. The multi-generational bottleneck filters for learnability \citep{kirby2014}. Patterns that are too complex or inconsistent don't survive. This explains why structural constraints converge across dialects: all English dialects reject \mention{*Which did you buy car?} because the constraint is simple and stable. Weak-definite frames vary more because they're conventionalized idioms. This mechanism acts as a \textbf{filter}, setting the outer bounds of what the cluster can contain.

The list is not exhaustive. 

These mechanisms interlock. Grammaticalization creates the bundle; acquisition transmits it; alignment maintains it in real time; prestige shapes which variants survive; transmission filters for stability. Metalinguistic feedback adds another layer: grammarians' labels and diagnostics can feed back into pedagogy and prescriptive norms, potentially stabilizing or destabilizing particular constructions. Perturb any mechanism, and the equilibrium shifts~-- but the cluster persists because the other mechanisms compensate.

%--- --- --- --- --- --- --- --- --- --- --- --- --- --- --- --- --
\section{When the clusters slip}
\label{sec:10:slippage}
%--- --- --- --- --- --- --- --- --- --- --- --- --- --- --- --- --

The explanatory power of the HPC framework shows when we examine cases where the two clusters dissociate. These aren't anomalies to be explained away~-- they're instances of derived-function exploitation. Each case follows the same logic: a stable side effect of the article's history gets recruited for a purpose other than its proper function.

The logic parallels Gricean pragmatics. Just as flouting the maxim of Quality (\enquote{You're a genius} said ironically) exploits the norm to generate new meaning, flouting the definiteness norm (using \mention{the} without a specific referent) exploits the form to signal a different kind of content. The convention persists because the violation is systematic.

\subsection{Weak definites}

Weak definites look like a failure of definiteness. Expressions like \mention{take the bus}, \mention{listen to the radio}, and \mention{go to the hospital} have troubled semantic theories for decades \citep{carlsonsussman2005,aguilarguevarazwarts2010}. The puzzle is how \mention{the} appears without unique or familiar referents.

The proper-function framework dissolves the puzzle. Weak definites exploit the article's form-cluster membership~-- its distributional profile~-- without performing its proper function (signalling identifiability). The morphosyntactic frame~-- verb + \mention{the} + noun in this conventionalized construction~-- persists because the form-cluster properties are independently stable. What the construction does is semantic: it shifts the contribution from a specific referent to an activity type. \mention{Go to the hospital} means something like \enquote{seek medical treatment}; the hospital itself is not identified.

This is a derived function. The article's distributional behaviour was inherited from the demonstrative source~-- a side effect of grammaticalization, not what the article is \emph{for}. Once that side effect stabilized, speakers could recruit it for institutionalized activity frames. But weak definites aren't merely fossils: the construction is productive. We \mention{take the Uber}, \mention{check the app}, \mention{get on the WiFi}. The pattern will happily adopt any new ritual, especially if it comes with a subscription. New forms enter the pattern whenever a stereotypical activity arises. What persists is the frame~-- verb + \mention{the} + role-denoting noun~-- recruiting form-cluster membership for activity-type semantics.

\subsection{Generic definites}

Generic definites look like a failure of reference. Sentences like \mention{The tiger is endangered} or \mention{The computer changed the world} pose a different challenge: the definite singular is used for kind reference, not individual reference \citep{carlson1977,krifka2004}.

This is the puzzle that troubled Funes in the epigraph. The \enquote{generic symbol dog} embraces so many unlike individuals~-- the dog at three-fourteen seen from the side, the dog at three-fifteen seen from the front~-- and yet a single word holds them together. What Funes couldn't accept was exactly what generic definites achieve: a label that picks out not an individual but a kind. The mechanisms that maintain \textsc{dog} as a category are the same mechanisms that let \mention{the dog} do its work.

Again, the HPC framework provides clarity. Generic definites are form-cluster items: they resist neutral existential contexts, they pattern distributionally with \mention{the}. But their semantics is kind-denoting, not individual-referring. Individual-level uniqueness doesn't apply; kind-level properties do.

The form cluster is intact. The definiteness cluster is orthogonal~-- neither satisfied nor violated, merely inapplicable at the kind level. The syntax generates a form-cluster phrase; semantics assigns it a generic interpretation that sidesteps the definiteness question.

This too is a derived function. Generic definites exploit the article's capacity for domain restriction~-- inherited from demonstrative pointing~-- but shift the domain from individuals to kinds. The proper function (signalling identifiability of an individual) isn't performed; what persists is the form-cluster profile and the slot for domain specification.

\subsection{Proper names}

Proper names look like a failure of marking in English. \mention{Kim} and \mention{London} are semantically definite~-- they identify unique, familiar referents~-- but they typically lack form-cluster marking. This isn't universal: in Greek (\mention{\textit{o Petros}} `the Peter') or with titles in Spanish (\mention{\textit{el señor Costa}} `the Mr. Costa'), the form and meaning clusters align. But English dissociates them.

This dissociation follows from the HPC architecture. In English, names show definiteness-cluster properties (identifiability, uniqueness, anaphoric recoverability) without the form-cluster properties. They don't resist existential pivots: \mention{There's a Kim here to see you} is natural. They don't license partitives in the standard way.

The clusters have decoupled in the opposite direction from weak definites. Where weak definites have form without value, proper names have value without form.

English even reveals the underlying logic when names collide with constructions that have indefinite defaults. Bare plurals are normally indefinite: \mention{dogs}, \mention{books}. Family names need pluralization but remain definite: \mention{the Smiths}, \mention{the Johnsons}. The article overrides the constructional default to preserve definiteness.

The same dissociation appears when we push names toward generic readings. Predicativists analyse bare singular names like \mention{Ruth} as containing a null definite determiner~-- structurally parallel to \mention{the tiger}. If so, we'd expect bare singular names to allow generic readings, just as \mention{the tiger} does. \textcite{gasparri2025} shows they can: \mention{Italian Andrea is generally male}; \mention{Ruth has good grades in biology} (in statistical contexts). But the generics are characterizing, not kind-level~-- \mention{*John became common} fails without quotation.

Why the restriction? Names don't pick out natural kinds; they pick out what \textcite{dupre1993} calls \enquote{social-practice-unified collections}~-- the set of individuals who bear a name in virtue of naming conventions. Characterizing generics quantify over instances: \mention{Italian Andrea is generally male} says most Italians named Andrea are male. Kind-level generics require a kind as argument: \mention{The tiger is endangered} takes the species as subject. But \textsc{Ruth} isn't a species; it's a practice-maintained cluster of individuals. The form (null-definite + name) decouples from canonical referential function, but only within limits imposed by what names are for.

\subsection{Indefinite \mentionhead{this}}

Narrative-presentational \mention{this} (\mention{There was this guy...}) is the mirror image: an indefinite meaning recruiting a form-cluster item \citep{prince1981}. \textcite{prince1981} showed that indefinite \mention{this} introduces referents that will become topics~-- it uses the form cluster's \enquote{hosting} potential to signal future prominence. The item is semantically indefinite (new) but morphosyntactically deital (partitive-licensing, nonrestrictive-hosting). The mismatch is the mechanism: using a pointer form for a new referent forces the hearer to open a file that expects updates.

%--- --- --- --- --- --- --- --- --- --- --- --- --- --- --- --- --
\section{Passing the tests}
\label{sec:10:tests}
%--- --- --- --- --- --- --- --- --- --- --- --- --- --- --- --- --

Chapter~\ref{ch:failure-modes} introduced the Two-Diagnostic Test for genuine HPC kinds: high projectibility and robust homeostasis. Both clusters pass.

\subsection{Projectibility}

The definiteness cluster supports induction. Knowing a referent is semantically definite lets you predict its discourse behaviour: it can be picked up by pronouns, it can be topicalized, it can be presupposed. The form cluster supports induction too: knowing a determiner is form-cluster lets you predict structural behaviour across contexts.

Most importantly, the predictions are \term{dissociated}. You can know a noun phrase's form-cluster status without knowing its definiteness status~-- and the predictions you make will differ. Weak definites are form-cluster but not definiteness-cluster; proper names are definiteness-cluster but not form-cluster. Each category supports distinct inductions.

\subsection{Homeostasis}

Both clusters are maintained by mechanisms. The definiteness cluster is sustained by discourse-pragmatic processes: common ground management creates correlations among identifiability, uniqueness, and anaphoric recoverability. The form cluster is sustained by mechanisms such as those detailed in §\ref{sec:10:mechanisms}.

The mechanism difference explains the decoupling. Discourse pressure can't alter distributional restrictions; grammaticalization pressure doesn't change reference. The clusters drift independently because their maintenance is independent.

\subsection{Falsifiable predictions}

The HPC account generates specific predictions about how the clusters behave under manipulation:

\textbf{Prosodic rescue}: Prosody should selectively rescue existential pivots (a discourse-level constraint) but not partitive complements (a more deeply grammaticalized structural constraint). This is testable and falsifiable.

\textbf{Dialectal preservation}: Dialects that drop \mention{the} in institutional frames (British \mention{in hospital}) should preserve form-cluster patterning when definiteness is supplied by demonstratives (\mention{in this hospital}).

\textbf{Acquisition asymmetry}: Children should master distributional restrictions before semantic ones, because distributional restrictions lack prosodic repair paths.

These predictions target the interaction between morphosyntax, prosody, semantics, and development~-- exactly what the mechanism story predicts.

What would falsify the account? If the clusters fragment completely rather than cohering into two families. If \mention{there}-resistance, partitive licensing, and hosting requirements turned out to be independent rather than correlated, there would be no form cluster to explain. If identifiability, uniqueness, and familiarity showed no statistical tendency to co-occur, there would be no definiteness cluster. The HPC architecture predicts two cohesive families that occasionally decouple, not a swarm of independent features. The cohesion is the explanandum; the mechanisms are the explanans.

%--- --- --- --- --- --- --- --- --- --- --- --- --- --- --- --- --
\section{The term: Deitality}
\label{sec:10:deitality}
%--- --- --- --- --- --- --- --- --- --- --- --- --- --- --- --- --

We need a name for the form cluster. The term \term{definiteness} is already taken~-- it names the semantic cluster. Calling both clusters \enquote{definiteness} would perpetuate exactly the conflation we have been trying to undo.

The conflation has a history. When Russell analyzed \mention{the F} descriptions as quantificational (existence plus uniqueness), the article's distributional diagnostics were not the object of analysis~-- the truth-conditional contribution was. Later work refined the semantics, but the conflation persisted, converging with a pre-Russell grammatical tradition that treated \enquote{definiteness} as naming both the meaning and the marking. The slide was reinforced from other directions too~-- grammatical description, language pedagogy, typological comparison~-- often treating form and value as a single package. The term did double duty, and its equivocation became invisible. Weak definites that didn't fit the semantic story typically became exceptions to be explained rather than evidence that two categories had been conflated.

I propose \term{deitality} as a diagnostic remedy. The term derives from the deictic origins of the cluster: demonstratives grammaticalize into articles, and the distributional profile reflects that deictic source. A determiner is \term{deital} if it shows the form-cluster properties; it's \term{definite} if it contributes definiteness-cluster semantics.

The term is ugly~-- methodologically ugly. If it sounds like a minor villain in a low-budget sci-fi, that's partly the point. Its unfamiliarity forces a break with the assumption that form and value are identical. When you say \mention{deital}, you can't accidentally mean \mention{definite}. The conceptual separation becomes lexicalized. This is a terminological intervention, and it has costs: new labels require uptake. But the alternative~-- continued equivocation on \mention{definiteness}~-- has higher costs still.

The utility is in the clarity. We can now say: weak definites are deital but indefinite; proper names are definite but non-deital; generic definites are deital and semantically orthogonal. Each statement is precise in a way the traditional terminology doesn't allow.

%--- --- --- --- --- --- --- --- --- --- --- --- --- --- --- --- --
\section{Cross-linguistic scope}
\label{sec:10:cross-linguistic}
%--- --- --- --- --- --- --- --- --- --- --- --- --- --- --- --- --

Is deitality an English-specific category or a universal potential?

The form cluster as described here~-- the specific diagnostics, the specific determiners~-- is English-specific. It reflects the grammaticalization history of English demonstratives, the selectional restrictions of English constructions, the conventions of English discourse.

But the \term{architecture} generalizes \citep{lyons1999}. Languages with demonstrative-derived articles (French \mention{le}, German \mention{der}, Greek \mention{o}) show similar clusters: partitive restrictions, information-structural constraints, imperfect correlation with semantic definiteness. The diagnostics differ; the pattern of convergent distributional properties maintained by grammaticalization persists. Greek illustrates the recombination: proper names \emph{require} the definite article (\mention{o Giannis}, not \mention{*Giannis}), unlike English, where names lack form-cluster marking entirely \citep{lyons1999}. Same definiteness-cluster semantics; different form-cluster mapping. The architecture isn't an English projection~-- it's a framework for seeing where the pieces recombine differently.

Classifier languages show a different realization of the same underlying dynamic. Japanese lacks articles but has a rich demonstrative system (\mention{kono}/\mention{sono}/\mention{ano}) with its own distributional profile. The semantic definiteness cluster exists cross-linguistically; the morphosyntactic form cluster takes language-specific shapes.

The prediction is that wherever demonstratives grammaticalize into articles, the form cluster should emerge~-- because grammaticalization drags distributional properties along. The cluster isn't stipulated; it's an emergent consequence of how grammaticalization works.

%--- --- --- --- --- --- --- --- --- --- --- --- --- --- --- --- --
\section{Audit output}
\label{sec:10:audit}
%--- --- --- --- --- --- --- --- --- --- --- --- --- --- --- --- --

\paragraph{Target and scope.} The deitality cluster (form cluster) and the definiteness cluster (semantic cluster) in contemporary English; register-general; determiner-level grain.

\paragraph{Profile and stabilizers.} The form cluster comprises \mention{there}-resistance, partitive licensing, and nonrestrictive hosting. It is maintained by grammaticalization (demonstrative inheritance), acquisition (distributional frame learned before full semantics), alignment (real-time repair), prestige selection (dialectal variants), and transmission (multi-generational filtering). The definiteness cluster comprises familiarity, uniqueness, and identifiability. It is maintained by discourse tracking, domain restriction, and Theory of Mind development.

\paragraph{Boundary behaviour.} Weak definites (\mention{go to the hospital}) show form-cluster without definiteness-cluster. Proper names show definiteness-cluster without form-cluster. Generic definites are form-cluster and semantically orthogonal. These decouplings are predicted by the two-cluster architecture, not anomalies.

\paragraph{Failure-mode gate.} The mechanism story is not analyst convenience: grammaticalization pathways are historically documented; acquisition asymmetries are experimentally attested. Classification: \textbf{two coupled HPC kinds}.

\paragraph{Stress tests.} (i) If weak definites cluster with token-referring definites in embedding space rather than forming a distinct subcluster~-- tracking frequency or genre rather than the form-cluster diagnostics (\mention{there}-resistance, partitive licensing)~-- reclassify deitality as a thin pattern within definiteness. (ii) If the form-cluster diagnostics fragment completely~-- \mention{there}-resistance, partitive licensing, and hosting showing no correlation~-- there is no form cluster to explain. (iii) In acquisition data, if children show no asymmetry between learning the distributional frame (deitality) and the semantic licensing conditions (definiteness), the two-cluster architecture is falsified~-- they would be learning one category, not two.

%--- --- --- --- --- --- --- --- --- --- --- --- --- --- --- --- --
\section{Looking forward}
\label{sec:10:transition}
%--- --- --- --- --- --- --- --- --- --- --- --- --- --- --- --- --

We have now examined two variations on the HPC architecture. With countability, the fit between form and meaning is tight: the count cluster hugs the individuation cluster so closely that they are often mistaken for a single category. With definiteness, we found a fissure. Deitality tracks definiteness, but the coupling is loose enough to permit systematic, productive slippage. The \enquote{exceptions} that have plagued the literature~-- weak definites, generics, proper names~-- are simply the visible evidence of play in the joint.

The next case study tests a different configuration. English gender is usually described as a vestigial system~-- a three-way pronoun distinction (\mention{he}/\mention{she}/\mention{it}) that lacks the NP-internal concord of French or German. But the same personhood-based logic that governs pronouns also governs \mention{who}/\mention{which}, \mention{somebody}/\mention{something}, and \mention{when}/\mention{where}. Chapter~\ref{ch:proform-gender} shows that English gender is a robust HPC system once we recognize its proper scope: not pronouns alone, but the entire semantic class of pro-forms.
