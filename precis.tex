% Précis: Why Words Won't Hold Still
% A brief summary for the general reader

\chapter*{Précis}
\addcontentsline{toc}{chapter}{Précis}

At 3~AM one night, I got an email about the word \mention{otherwise} from Rodney Huddleston~-- a mentor of mine and one of the world's foremost grammarians. \enquote{Its classification is quite a puzzle,} he wrote.

The puzzle is easy to see. \mention{Otherwise} fits where you expect an adverb: \mention{I think otherwise}. But it also fits where you don't: \mention{The truth is quite otherwise}. And it fits where almost nothing fits: \mention{the correctness or otherwise of the proposal}. Three uses, three syntactic environments, one common word~-- and no clean label.

Here was a man who had spent decades cataloguing English grammar in exquisite detail. But a common word, one that appears in unremarkable sentences every day, was keeping him up at night. Why?

The honest answer is that we don't have a settled account of what grammatical categories \emph{are}~-- one that explains both their stability and their edge-cases. We learn in school that \mention{dog} is a noun and \mention{quickly} is an adverb, and we imagine that somewhere, in the back rooms of linguistics departments, experts have sorted it all out. They haven't~-- not in the way we assume. There is no master filing cabinet, only better and worse heuristics and occasional insomnia.

Two temptations have shaped the field.

\textbf{The essentialist temptation} says that categories have definitions. A noun is a word that names a person, place, or thing. An adjective describes. This works in simple cases. But the definitions leak everywhere. Consider \mention{action}: it names a kind of thing, but it's an \enquote{action word}, which is supposed to be a verb. Consider \mention{the} in \enquote{I took the bus}. \mention{The} is the \term{definite article}, but it doesn't identify a specific bus. Every rule has \enquote{exceptions} that aren't exceptions~-- they're evidence.

\textbf{The nominalist temptation} swings the other way. If definitions fail, maybe categories are just convenient labels~-- filing systems we invented, with no deeper reality. But this can't be right either. Some big contrasts recur across unrelated languages, often in altered form, and children converge on them quickly. Something is keeping \term{noun} and \term{verb} stable across centuries and continents.

Neither answer explains how categories can be \emph{real} without having \emph{essences}.

\bigskip

Biology faced the same puzzle. Darwin demolished the idea that species have essences~-- fixed, unchanging forms~-- but species are obviously real, and nobody has ever been mauled by a merely convenient label.

The philosopher Richard Boyd offered a solution: \emph{homeostatic property clusters}. A species isn't defined by a checklist of necessary properties. It's a bundle of features that tend to co-occur~-- fur patterns, behaviours, genetic markers~-- held together by causal mechanisms like interbreeding, shared ecology, and gene flow. The boundaries are fuzzy at the edges, but the cluster is real because \emph{something maintains it}.

Think of a spinning top. It stays upright not because it's rigid but because it's \emph{moving}. The spin resists perturbation. Push it slightly, and gyroscopic forces bring it back. Stop the spin, and the top falls.

Grammatical categories are like this. What keeps \term{noun} stable isn't a definition~-- it's a tangle of mechanisms:

\begin{itemize}
    \item \textbf{Acquisition:} Children hear millions of words, but they filter out noise and converge on the same categories. This filtering is itself a mechanism: only patterns that survive learning persist.
    \item \textbf{Entrenchment:} High-frequency items (\mention{the}, \mention{go}, \mention{be}) anchor low-frequency ones. We learn \mention{ran} alongside \mention{walked}, and the pattern generalises.
    \item \textbf{Alignment:} In conversation, speakers converge on the same structures. Use an unusual construction, and your interlocutor is more likely to repeat it back. Categories are maintained in real time.
    \item \textbf{Transmission:} Each generation must learn language from scratch. Patterns that are hard to learn get simplified or lost. Only robust structures survive the generational bottleneck.
    \item \textbf{Functional pressure:} Categories that serve communicative needs~-- distinguishing agents from patients, old information from new~-- persist because they're useful.
\end{itemize}

The shift is fundamental. Instead of asking \enquote{What is the definition of \term{noun}?} we ask \enquote{What keeps \term{noun} stable?} The answer is a story about mechanisms, not a dictionary entry.

\bigskip

Once you learn to see maintained clusters, you might start seeing them everywhere. \term{Noun}? Maintained. \term{Verb}? Maintained. \term{Subject}? \term{Voice}? \term{Non-finite}? Everything looks like a spinning top.

This is the inflation problem. Not every stable pattern is a genuine kind. Some patterns are just accidents; some labels are just filing conventions. When a framework can explain everything, it stops explaining anything in particular. We need diagnostics.

Two tests separate genuine categories from mere labels:
\begin{itemize}
    \item \textbf{The projectibility test:} If a category is real, learning from some instances should let you predict new ones. Make risky predictions on new data~-- train on half, test on the other half. If the pattern \emph{travels}, the category is doing something. If it doesn't, you've just memorised noise.
    \item \textbf{The homeostasis test:} Can you name the stabilisers? What would change if you removed them? Genuine categories are \emph{perturbation-sensitive}: weaken a mechanism, and the cluster frays. Mere labels are \emph{perturbation-inert}: remove the label, and nothing in the world changes.
\end{itemize}

Categories that fail these tests come in three flavours:
\begin{enumerate}
    \item \textbf{Thin categories} lack stabilisers. Consider \emph{preposition copying}~-- sentences like \enquote{the box from which I got it from.} It exists, but barely: frequency vanishing, no child learning it as a target, no social group marking it as identity. If English lost it tomorrow, nothing would restore it. It's a smoke ring~-- real for a moment, but not maintained.
    \item \textbf{Fat categories} lump distinct clusters under one label. \term{Adverb} is the paradigm case~-- the \enquote{dustbin of the parts of speech,} a category with excellent storage capacity and terrible explanatory power. It groups manner adverbs like \mention{quickly} (which follow verbs), degree words like \mention{very} (which never do), and sentence adverbs like \mention{unfortunately} (which float freely). These are different animals, and learning \mention{quickly} tells you nothing about where \mention{very} can appear~-- the label creates an illusion of unity where there's only a filing decision.
    \item \textbf{Negative categories} are defined by absence. \term{Non-finite clause} groups infinitivals, gerunds, and past participles~-- not because they share something, but because they all lack tense inflection. This is a family portrait taken in silhouette: the shared trait is what you can't see.
\end{enumerate}

Categories are maintained; classes are merely named. The diagnostics tell us which is which.

\bigskip

If grammatical categories are maintained clusters, several debates dissolve.

\textbf{The universals debate:} Linguists have argued for decades about whether categories like \term{subject} or \term{adjective} are universal. The maintenance view reframes the question. Instead of asking \enquote{Does Language $X$ have subjects?} we ask \enquote{Is the category that Language $X$ calls `subject' maintained by the same mechanisms as in English?}

Usually, the answer is no. English subjects are maintained by agreement morphology and fixed word order. Tagalog \enquote{subjects} are maintained by voice morphology and discourse-role assignment~-- entirely different mechanisms. They may look alike, but they're distinct clusters, convergent solutions to different functional pressures. The sameness is in the label, not the kind.

\textbf{The gradience debate:} Linguists have long puzzled over gradient judgments. Some sentences feel \enquote{sort of} grammatical; some category memberships feel \enquote{kind of} true. Is grammar discrete or continuous?

The maintenance view says: both. The underlying substrates~-- articulation, perception, processing~-- are continuous, but feedback loops create discrete basins. Categories are like attractor states: the mechanisms pull deviant instances back toward the centre, so gradience at the edges is real while discreteness is also real. The apparent contradiction is a false dichotomy.

\textbf{The methodology debate:} What should linguists actually \emph{do}? The standard approach is to collect judgments, find patterns, and assign labels. The maintenance view adds a step: identify the mechanisms. A label without a mechanism is descriptive, not explanatory. If you claim \term{voice} is a category in Language $L$, you need to show what maintains it~-- the acquisition pathways, the frequency effects, the functional pressures. This isn't a higher bar; it's a different bar.

\bigskip

The maintenance framework extends beyond grammar.

Wherever we have categories that feel real but lack essences~-- personality types, mental disorders, social roles, moral concepts~-- the same logic applies. The question is never \enquote{What is the definition?} but \enquote{What keeps it stable?} The answer will always be a mechanism story.

This is a third way between old philosophical extremes. Platonism held that categories are eternal forms, discovered not invented. Nominalism held that they're convenient fictions, invented not discovered. The maintenance view says: categories are \emph{real} in virtue of being \emph{maintained}. They can change; they can blur at the edges; they can be maintained by different mechanisms in different communities~-- but they are not arbitrary.

The practical payoff is a different way of looking at puzzles. When a word like \mention{otherwise} resists classification, the problem isn't that we lack a definition. It's that multiple maintenance regimes are pulling in different directions. Some uses are stabilised by one set of mechanisms (verbal complementation, manner modification); others are stabilised by a different set (predicative function, scalar semantics). The category is real, contested, and alive.

At 3~AM, Rodney Huddleston wasn't stuck because he lacked scholarship. He was stuck because he had spent a career discovering what the textbook version of grammar can't say: that categories are not definitions to be memorised but processes to be traced. The puzzle wasn't a failure~-- it was evidence of what he already knew.

Grammar doesn't hold still because it's fixed. It holds still because something is holding it.
