% Glossary entries for the HPC book
% This file is \input from the main document after the preamble

% Glossary stance: categories are maintained clusters, not essences.
\renewcommand*{\glossarypreamble}{%
  \epigraph{The circle of the English language has a well-defined centre but no discernible circumference.}{James Murray, \textit{OED} Vol.~1 (1888), p.~xvii}
  \noindent \small Many of the entries in this glossary are themselves categories and so are defined in terms of mechanisms and diagnostics where relevant; examples illustrate use, not definitions. For page references, see the Subject Index (p.~\ref{idx:subject}).\par\bigskip}

% Micro-structure helpers (quiet, Bringhurst-style)
\newcommand{\glsdef}[1]{#1}
\newcommand{\glsmech}[1]{#1}
\newcommand{\glsdiag}[1]{#1}
\newcommand{\glsnote}[1]{#1}
\newcommand{\glsxsee}[1]{See also: #1}

% ===== Core framework terms =====

\newglossaryentry{hpc}{
  name={homeostatic property cluster (HPC)},
  text={HPC},
  description={\glsdef{A category whose properties cluster because mechanisms maintain their co-occurrence rather than because of a shared essence.} \glsmech{Maintained by acquisition, entrenchment, alignment, transmission, and functional pressure.} \glsdiag{Passes both projectibility and homeostasis (perturbation sensitivity).} \glsnote{Membership can be graded and internally structured.} \glsxsee{projectibility, homeostasis, stabiliser, two-diagnostic test.}}
}

\newglossaryentry{projectibility}{
  name={projectibility},
  text={projectibility},
  description={\glsdef{A category's capacity to support induction from observed cases to unobserved ones.} \glsmech{It relies on stable correlations enforced by stabilisers.} \glsdiag{Predicts held-out data or generalises across contexts better than chance.} \glsnote{Field-relative: projectible for one purpose may fail for another.} \glsxsee{homeostasis, two-diagnostic test, field-relative projectibility.}}
}

\newglossaryentry{homeostasis}{
  name={homeostasis},
  text={homeostasis},
  description={\glsdef{Active return after perturbation; stability by self-correction rather than mere persistence.} \glsmech{Maintained by alignment, entrenchment, transmission, and other stabilisers across timescales.} \glsdiag{Perturbation sensitivity: weaken a mechanism and the clustering frays.} \glsnote{The ontological anchor for genuine kinds.} \glsxsee{stabiliser, perturbation sensitivity.}}
}

\newglossaryentry{entrenchment}{
  name={entrenchment},
  text={entrenchment},
  description={\glsdef{Strengthening of mental representations through repeated use.} \glsmech{Driven by frequency effects and chunking in memory.} \glsdiag{Resistance to analogical change; faster processing and higher accessibility.} \glsnote{Anchors category cores and stabilises prototypes.} \glsxsee{chunking, prototype.}}
}

\newglossaryentry{essentialism}{
  name={essentialism},
  text={essentialism},
  description={\glsdef{The view that categories are defined by necessary and sufficient conditions.} \glsmech{It is grounded in definitional criteria rather than stabilisers.} \glsdiag{Breaks down when exceptions proliferate without principled repair.} \glsnote{The maintenance view rejects essentialism in favour of mechanism-based kinds.} \glsxsee{nominalism, prototype.}}
}

\newglossaryentry{stabiliser}{
  name={stabiliser},
  text={stabiliser},
  description={\glsdef{A causal process that maintains clustering in a category.} \glsmech{Instances include acquisition, entrenchment, alignment, transmission, and functional pressure.} \glsdiag{Removing it changes the clustering in predictable ways.} \glsnote{Names a role, not a single type of mechanism.} \glsxsee{homeostasis, perturbation sensitivity.}}
}

\newglossaryentry{alignment}{
  name={interactive alignment},
  text={alignment},
  description={\glsdef{Convergence of interlocutors' linguistic choices during interaction.} \glsmech{Realised through accommodation and repair in real-time conversation.} \glsdiag{Local convergence and resistance to innovations under misalignment.} \glsnote{A fast-timescale stabiliser.} \glsxsee{stabiliser, transmission.}}
}

\newglossaryentry{prototype}{
  name={prototype},
  text={prototype},
  description={\glsdef{The most typical member of a category.} \glsmech{Shaped by entrenchment and frequency effects.} \glsdiag{Typicality effects and graded membership.} \glsnote{Descriptive of structure; requires a mechanism story to explain stability.} \glsxsee{entrenchment, real gradience.}}
}

\newglossaryentry{maintenanceview}{
  name={maintenance view},
  text={maintenance view},
  description={\glsdef{The claim that categories are real because they are maintained by stabilising mechanisms.} \glsmech{Supported by a braid of mechanisms across timescales.} \glsdiag{Implies the two-diagnostic test: projectibility plus homeostasis.} \glsnote{Rejects both essentialism and nominalism.} \glsxsee{hpc, two-diagnostic test.}}
}

\newglossaryentry{class}{
  name={class},
  text={class},
  description={\glsdef{A grouping used for description or convenience without a commitment to shared mechanisms.} \glsmech{No stabiliser story required.} \glsdiag{May be projectible locally but need not pass the two-diagnostic test.} \glsnote{Useful for taxonomy; not necessarily a natural kind.} \glsxsee{category, fat class, thin class, comparative concept.}}
}

\newglossaryentry{category}{
  name={category},
  text={category},
  description={\glsdef{A cluster of properties used to organise linguistic phenomena.} \glsmech{When genuine, maintained by stabilisers; when merely convenient, it is a class.} \glsdiag{A category earns kind-status only if it passes projectibility and homeostasis.} \glsnote{The book's core question is which categories are kinds.} \glsxsee{kind, two-diagnostic test, projectibility, homeostasis.}}
}

\newglossaryentry{kind}{
  name={kind},
  text={kind},
  description={\glsdef{A category that supports induction because mechanisms maintain its clustering.} \glsmech{Stabilisers keep the profile coherent across speakers and time.} \glsdiag{Passes the two-diagnostic test.} \glsnote{In this framework, linguistic kinds are HPC kinds rather than definitional essences.} \glsxsee{hpc, copied kind, two-diagnostic test.}}
}

\newglossaryentry{mechanism}{
  name={mechanism},
  text={mechanism},
  description={\glsdef{A causal process with inputs, operations, and outputs that produces or maintains a pattern.} \glsmech{Mechanisms include acquisition, alignment, entrenchment, and transmission.} \glsdiag{Identified by perturbation: weaken the process and the cluster frays.} \glsnote{The explanatory unit of the maintenance view.} \glsxsee{stabiliser, homeostasis, perturbation sensitivity.}}
}

\newglossaryentry{acquisition}{
  name={acquisition},
  text={acquisition},
  description={\glsdef{The learning process by which speakers internalise linguistic categories from input.} \glsmech{Distributional learning and cue integration across contexts.} \glsdiag{Convergence on shared categories despite variable input.} \glsnote{A core stabiliser and transmission bottleneck.} \glsxsee{transmission, entrenchment, stabiliser.}}
}

\newglossaryentry{transmission}{
  name={transmission},
  text={transmission},
  description={\glsdef{Intergenerational passage of linguistic patterns.} \glsmech{Iterated learning filters for stable, learnable variants.} \glsdiag{Convergence toward compressible systems under repeated learning.} \glsnote{A macro-stabiliser of categories and constructions.} \glsxsee{iterated transmission, acquisition.}}
}

\newglossaryentry{properfunction}{
  name={proper function},
  text={proper function},
  description={\glsdef{In Millikan's framework, the function an item has in virtue of its history of selection.} \glsmech{Secured by reproductive success of uses that perform the function.} \glsdiag{Explains why the form persists across generations.} \glsnote{Historical, not intentional or definitional.} \glsxsee{derived proper function, Normal conditions.}}
}

\newglossaryentry{derivedproperfunction}{
  name={derived proper function},
  text={derived proper function},
  description={\glsdef{A function acquired by recruitment beyond an item's original selected function.} \glsmech{Arises through exploitation of stable side effects of a form or construction.} \glsdiag{Depends on the proper function remaining intact elsewhere.} \glsnote{Parasitic rather than foundational.} \glsxsee{proper function, Normal conditions.}}
}

\newglossaryentry{deitality}{
  name={deitality},
  text={deitality},
  description={\glsdef{The morphosyntactic form cluster associated with English determiners, distinct from semantic definiteness.} \glsmech{Formed by grammaticalization of demonstratives plus distributional constraints.} \glsdiag{There-resistance, partitive \textit{of}, and hosting behaviour converge.} \glsnote{Cross-cuts the definiteness cluster.} \glsxsee{definiteness cluster.}}
}

\newglossaryentry{fatclass}{
  name={fat class},
  text={fat class},
  description={\glsdef{A label that lumps distinct causal clusters into a single bin.} \glsmech{Produced by multiple unrelated stabilisers with no shared mechanism.} \glsdiag{Fails projectibility across subtypes.} \glsnote{Useful for pedagogy, not a natural kind.} \glsxsee{thin class, comparative concept.}}
}

\newglossaryentry{adverb}{
  name={adverb},
  text={adverb},
  description={\glsdef{A traditional residual category for modifiers that are not adjectives.} \glsmech{Maintained by multiple unrelated mechanisms across subtypes.} \glsdiag{Fails projectibility across manner, degree, and sentence adverbs.} \glsnote{A paradigmatic fat class; the remedy is decomposition.} \glsxsee{fat class.}}
}

\newglossaryentry{thinclass}{
  name={thin class},
  text={thin class},
  description={\glsdef{A pattern that is weakly maintained or not maintained at all.} \glsmech{Occurs when stabilisers are absent or too weak.} \glsdiag{Fails homeostasis; disappears under perturbation.} \glsnote{Often a byproduct of other mechanisms.} \glsxsee{fat class.}}
}

\newglossaryentry{comparativeconcept}{
  name={comparative concept},
  text={comparative concept},
  description={\glsdef{An analyst's tool constructed for cross-linguistic comparison.} \glsmech{Not guaranteed to be supported by shared stabilisers across languages.} \glsdiag{Does not imply shared homeostasis or projectibility.} \glsnote{Useful without being a kind.} \glsxsee{field-relative projectibility.}}
}

\newglossaryentry{epistemickind}{
  name={epistemic kind},
  text={epistemic kind},
  description={\glsdef{A category that serves epistemic purposes without claiming a shared causal mechanism.} \glsmech{May be institutionally reinforced rather than mechanistically maintained.} \glsdiag{Can be stable without passing the two-diagnostic test.} \glsnote{Khalidi’s term for useful but non-kind groupings.} \glsxsee{class, comparative concept.}}
}

\newglossaryentry{madagascarfallacy}{
  name={Madagascar fallacy},
  text={Madagascar fallacy},
  description={\glsdef{The mistake of treating an ecosystem as if it were a single kind.} \glsmech{It bundles mechanisms that maintain components rather than the whole.} \glsdiag{Fails homeostasis at the wrong grain.} \glsnote{A warning about scale in category analysis.} \glsxsee{two-diagnostic test.}}
}

% ===== Countability (Ch 9) =====

\newglossaryentry{countability}{
  name={countability},
  text={countability},
  description={\glsdef{The coupled system linking semantic individuation and morphosyntactic count marking.} \glsmech{Maintained by bidirectional inference between construal and form.} \glsdiag{Implicational hierarchy of count properties and robust projectibility.} \glsnote{An interface category with two partially distinct clusters.} \glsxsee{individuation cluster, count cluster, bidirectional inference.}}
}

\newglossaryentry{individuationcluster}{
  name={individuation cluster},
  text={individuation cluster},
  description={\glsdef{The semantic profile that enables discrete construal: boundedness, atomicity, enumerability, and homogeneity resistance.} \glsmech{Maintained by perceptual segmentation, object files, and cross-modal integration.} \glsdiag{Supports predictions about count morphosyntax.} \glsnote{Coupled to the count cluster to yield countability.} \glsxsee{count cluster, bidirectional inference.}}
}

\newglossaryentry{countcluster}{
  name={count cluster},
  text={count cluster},
  description={\glsdef{The morphosyntactic properties that travel together in English count nouns (plural, cardinals, \textit{many}/\textit{few}, agreement).} \glsmech{Maintained by acquisition and entrenchment in count frames, plus alignment in discourse.} \glsdiag{Implicational hierarchy: tight properties imply loose ones.} \glsnote{Coupled to individuation by bidirectional inference.} \glsxsee{individuation cluster, bidirectional inference, functional anchoring.}}
}

\newglossaryentry{bidirectionalinference}{
  name={bidirectional inference},
  text={bidirectional inference},
  description={\glsdef{The mechanism coupling semantic construal to morphosyntactic form in comprehension and production.} \glsmech{Realised by inference from morphosyntax to construal and back.} \glsdiag{Mutual predictability of count properties across constructions.} \glsnote{The core homeostatic link in countability.} \glsxsee{count cluster, individuation cluster.}}
}

\newglossaryentry{functionalanchoring}{
  name={functional anchoring},
  text={functional anchoring},
  description={\glsdef{Lexical alternatives absorb pressure to regularise a quasi-count noun.} \glsmech{Implemented through division of labour in the lexicon.} \glsdiag{Intermediate classes persist without extending tight properties.} \glsnote{Explains stability of \textit{cattle}/\textit{police} and drift of \textit{data}.} \glsxsee{count cluster.}}
}

\newglossaryentry{chunking}{
  name={chunking},
  text={chunking},
  description={\glsdef{Storage of high-frequency sequences as units, bypassing compositional assembly.} \glsmech{Driven by frequency-based memory consolidation.} \glsdiag{Fluency and acceptability differences in entrenched frames.} \glsnote{Supports entrenchment and stability at the cluster core.} \glsxsee{entrenchment.}}
}

\newglossaryentry{basin}{
  name={basin (attractor)},
  text={basin},
  description={\glsdef{A dynamical-systems metaphor: an attractor region toward which category states are drawn.} \glsmech{Created by stabiliser-driven pull toward the cluster centre.} \glsdiag{Stable cores with tolerated peripheries.} \glsnote{Explains graded membership with sharp boundaries.} \glsxsee{dynamic discreteness, real gradience.}}
}

% ===== Definiteness (Ch 10) =====

\newglossaryentry{definiteness}{
  name={definiteness},
  text={definiteness},
  description={\glsdef{A semantic category of identifiability, familiarity, and uniqueness.} \glsmech{Maintained by discourse tracking, domain restriction, and Theory of Mind.} \glsdiag{Predicts anaphoric recoverability and discourse behaviour.} \glsnote{Distinct from the morphosyntactic form cluster.} \glsxsee{definiteness cluster, deitality, weak definite.}}
}

\newglossaryentry{definitenesscluster}{
  name={definiteness cluster},
  text={definiteness cluster},
  description={\glsdef{The semantic cluster of familiarity, uniqueness, and identifiability.} \glsmech{Maintained by discourse tracking, domain restriction, and Theory of Mind.} \glsdiag{Anaphoric recoverability and discourse behaviour.} \glsnote{Distinct from the morphosyntactic deitality cluster.} \glsxsee{deitality, Normal conditions.}}
}

\newglossaryentry{formcluster}{
  name={form cluster},
  text={form cluster},
  description={\glsdef{The morphosyntactic bundle associated with English determiners (there-resistance, partitive \textit{of}, hosting).} \glsmech{Grammaticalization of demonstratives and stabilised distributional constraints.} \glsdiag{Convergence of structural diagnostics across determiners.} \glsnote{Also called the deitality cluster.} \glsxsee{deitality, definiteness.}}
}

\newglossaryentry{weakdefinite}{
  name={weak definite},
  text={weak definite},
  description={\glsdef{A definite article use that lacks a unique or familiar referent (institutional/role frames).} \glsmech{A derived function exploiting stable form-cluster properties.} \glsdiag{Form-cluster behaviour without definiteness-cluster satisfaction.} \glsnote{Productive in stereotyped activity frames.} \glsxsee{deitality, definiteness, form cluster.}}
}

\newglossaryentry{normalconditions}{
  name={Normal conditions},
  text={Normal conditions},
  description={\glsdef{The circumstances under which a device performs its proper function.} \glsmech{Sustained by shared context and recoverable referents.} \glsdiag{Breakdown produces miscoordination; derived uses exploit stable side effects.} \glsnote{Frames proper vs. derived function.} \glsxsee{proper function, derived proper function.}}
}

% ===== General framework =====

\newglossaryentry{fieldrelativeprojectibility}{
  name={field-relative projectibility},
  text={field-relative projectibility},
  description={\glsdef{Projectibility indexed to analytic purpose or domain.} \glsmech{Depends on task-specific stabilisers.} \glsdiag{A category may project for one purpose and fail for another.} \glsnote{Explains cross-cutting categories.} \glsxsee{projectibility, crosscutting kinds.}}
}

\newglossaryentry{twodiagnostictest}{
  name={two-diagnostic test},
  text={two-diagnostic test},
  description={\glsdef{The criterion for genuine kinds: projectibility plus homeostasis.} \glsmech{Implemented by pairing the projectibility and homeostasis diagnostics.} \glsdiag{Pass both to count as an HPC; fail one to diagnose thin/fat/negative.} \glsnote{Operational rather than definitional.} \glsxsee{projectibility, homeostasis, perturbation sensitivity.}}
}

\newglossaryentry{noun}{
  name={noun},
  text={noun},
  description={\glsdef{A lexical category associated with nominals and argument structure.} \glsmech{Maintained by distributional cues, morphology, and acquisition biases.} \glsdiag{Cliquish stability across multiple diagnostics (determiners, number, agreement).} \glsnote{Cross-linguistic comparability requires mechanism mapping.} \glsxsee{verb, subject, copied kind.}}
}

\newglossaryentry{verb}{
  name={verb},
  text={verb},
  description={\glsdef{A lexical category associated with predication and event structure.} \glsmech{Maintained by distributional frames, inflectional paradigms, and acquisition.} \glsdiag{Convergent diagnostics (tense/aspect marking, argument structure).} \glsnote{Stability varies by language and construction.} \glsxsee{noun, aspect.}}
}

\newglossaryentry{subject}{
  name={subject},
  text={subject},
  description={\glsdef{A syntactic function cluster defined by converging behavioural and coding properties.} \glsmech{Agreement, word order, case marking, and discourse roles.} \glsdiag{Convergence of multiple diagnostics rather than a single definition.} \glsnote{Cross-linguistic identity is not guaranteed.} \glsxsee{comparative concept, category.}}
}

\newglossaryentry{copiedkind}{
  name={copied kind},
  text={copied kind},
  description={\glsdef{A category whose members are produced from each other or a common template.} \glsmech{Maintained by a transmission lineage with variation.} \glsdiag{Similarity explained by copying rather than essences.} \glsnote{Grammatical categories are copied kinds par excellence.} \glsxsee{replicator, interactor.}}
}

\newglossaryentry{unicept}{
  name={unicept},
  text={unicept},
  description={\glsdef{A concept that tracks sameness through multiple fallible methods (Millikan).} \glsmech{Sustained by convergent cues with no privileged diagnostic.} \glsdiag{Cliquish stability: some cues reliably indicate the cluster.} \glsnote{Matches multi-diagnostic category tracking.} \glsxsee{cliquish stability, two-diagnostic test.}}
}

\newglossaryentry{iteratedtransmission}{
  name={iterated transmission},
  text={iterated transmission},
  description={\glsdef{Structure emerging from multi-generational learning bottlenecks.} \glsmech{Produced by filtering for learnable variants.} \glsdiag{Convergence toward compressible/compositional systems.} \glsnote{A macro-stabiliser of linguistic structure.} \glsxsee{copied kind, entrenchment.}}
}

\newglossaryentry{perturbationsensitivity}{
  name={perturbation sensitivity},
  text={perturbation sensitivity},
  description={\glsdef{The diagnostic that a category frays when a stabiliser is weakened.} \glsmech{Observed when interventions target acquisition, alignment, or transmission.} \glsdiag{Predictable degradation patterns under perturbation.} \glsnote{Operationalises homeostasis.} \glsxsee{homeostasis, stabiliser.}}
}

\newglossaryentry{cliquishstability}{
  name={cliquish stability},
  text={cliquish stability},
  description={\glsdef{Property-correlation: some cues reliably indicate the whole cluster.} \glsmech{Maintained by stabilisers that keep correlations intact despite drift.} \glsdiag{Knowing one property predicts others.} \glsnote{Supports induction without perfect instance stability.} \glsxsee{projectibility.}}
}

% ===== Discreteness and gradience (Ch 5) =====

\newglossaryentry{dynamicdiscreteness}{
  name={dynamic discreteness},
  text={dynamic discreteness},
  description={\glsdef{Sharp boundaries produced by mechanisms rather than definitions.} \glsmech{Maintained by stabilisers that keep tolerance thresholds over time.} \glsdiag{Sharp judgments within speakers, variance across populations.} \glsnote{Explains discreteness without essences.} \glsxsee{relative tolerance, real gradience.}}
}

\newglossaryentry{realgradience}{
  name={real gradience},
  text={real gradience},
  description={\glsdef{Gradient structure as meaningful distance from category centres.} \glsmech{Shaped by variable stabiliser strength across the basin.} \glsdiag{Typicality and marginality effects.} \glsnote{Gradience is signal, not noise.} \glsxsee{basin, dynamic discreteness.}}
}

\newglossaryentry{relativetolerance}{
  name={relative tolerance},
  text={relative tolerance},
  description={\glsdef{Membership depends on scale-relative tolerance of change.} \glsmech{Determined by decision thresholds indexed to context and magnitude.} \glsdiag{Small changes acceptable at one scale but not another.} \glsnote{Explains tolerance intuitions at category edges.} \glsxsee{dynamic discreteness.}}
}

% ===== Projectibility mechanisms (Ch 6) =====

\newglossaryentry{mechanisticdrift}{
  name={mechanistic drift},
  text={mechanistic drift},
  description={\glsdef{Mechanisms that maintain a category can differ from those that originally produced it.} \glsmech{Driven by diachronic reconfiguration of stabilisers.} \glsdiag{Mismatch between current cues and historical rationale.} \glsnote{Explains semantic drift under stable labels.} \glsxsee{grammaticalisation.}}
}

\newglossaryentry{aspect}{
  name={aspect},
  text={aspect},
  description={\glsdef{A grammatical category describing how events are temporally and structurally construed.} \glsmech{Maintained by distributional cue structures and morphological marking.} \glsdiag{Projectible patterns may diverge from textbook semantic definitions.} \glsnote{A comparative concept whose language-specific mechanisms can drift.} \glsxsee{projectibility, mechanistic drift.}}
}

\newglossaryentry{grueproblem}{
  name={grue problem},
  text={grue problem},
  description={\glsdef{Goodman's riddle: why do we project \enquote{green} but not \enquote{grue}?} \glsmech{Explained by the fact that only mechanism-grounded properties are projectible.} \glsdiag{Gerrymandered predicates fail induction.} \glsnote{Applied to language, pseudo-categories lack mechanisms.} \glsxsee{projectibility.}}
}

\newglossaryentry{crosscuttingkinds}{
  name={crosscutting kinds},
  text={crosscutting kinds},
  description={\glsdef{Overlapping causal networks where the same entity belongs to multiple kinds.} \glsmech{Maintained by distinct stabilisers for each network.} \glsdiag{Different inductions for different purposes.} \glsnote{Cross-cutting is expected under field-relative projectibility.} \glsxsee{field-relative projectibility.}}
}

% ===== Essentialism and its alternatives (Ch 1–3) =====

\newglossaryentry{nominalism}{
  name={nominalism},
  text={nominalism},
  description={\glsdef{The view that categories are convenient labels without commitment to natural kinds.} \glsmech{Requires no stabiliser story.} \glsdiag{No homeostasis or projectibility demanded.} \glsnote{Rejected by the maintenance view.} \glsxsee{essentialism, comparative concept.}}
}

\newglossaryentry{grammaticalisation}{
  name={grammaticalisation},
  text={grammaticalisation},
  description={\glsdef{Diachronic shift from lexical item to grammatical marker.} \glsmech{Driven by frequency, reduction, reanalysis, and semantic bleaching.} \glsdiag{Layering and distributional expansion over time.} \glsnote{Creates form–meaning decouplings that later stabilise.} \glsxsee{mechanistic drift.}}
}

\newglossaryentry{replicator}{
  name={replicator},
  text={replicator},
  description={\glsdef{An entity that passes copies of itself through time with variation (Hull).} \glsmech{Sustained by transmission with differential survival.} \glsdiag{Selection on variants across generations.} \glsnote{Supports evolutionary framing of language change.} \glsxsee{interactor, copied kind.}}
}

\newglossaryentry{interactor}{
  name={interactor},
  text={interactor},
  description={\glsdef{An entity that interacts with its environment in ways that cause differential replication (Hull).} \glsmech{Realised in communicative success and failure.} \glsdiag{Selection pressure visible in usage patterns.} \glsnote{Pairs with replicator in an evolutionary account.} \glsxsee{replicator.}}
}

\newglossaryentry{verblessclause}{
  name={verbless clause},
  text={verbless clause},
  description={\glsdef{A clause-like construction with subject–predicate structure but no verb in the predicate.} \glsmech{Arises under competing diagnostics (predication vs. VP-headedness).} \glsdiag{Shows decoupling of form and function.} \glsnote{Illustrates pressure against essentialist definitions.} \glsxsee{two-diagnostic test.}}
}

\newglossaryentry{nonfiniteclause}{
  name={non-finite clause},
  text={non-finite clause},
  description={\glsdef{A residual category for clauses lacking primary tense inflection.} \glsmech{No single shared mechanism maintains the set.} \glsdiag{Fails both projectibility and homeostasis across subtypes.} \glsnote{A negative class; best replaced by positive constructions.} \glsxsee{fat class, category.}}
}

\newglossaryentry{opentexture}{
  name={open texture},
  text={open texture},
  description={\glsdef{Hart's term for the penumbral region where rules underdetermine outcomes.} \glsmech{Emerges when stabilisers are weak or in competition.} \glsdiag{Persistent variability and institutional repair.} \glsnote{Grammar has no Supreme Court; resolution is emergent.} \glsxsee{dynamic discreteness, mechanistic drift.}}
}
