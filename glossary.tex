% Glossary entries for the HPC book
% This file is \input from the main document after the preamble

% Glossary stance: categories are maintained clusters, not essences.
\renewcommand*{\glossarypreamble}{%
  \epigraph{The circle of the English language has a well-defined centre but no discernible circumference.}{James Murray, \textit{OED} Vol.~1 (1888), p.~xvii}
  \noindent \small Many of the entries in this glossary are themselves categories and so are defined in terms of mechanisms and diagnostics where relevant; examples illustrate use, not definitions. For page references, see the Subject Index (p.~\ref{idx:subject}).\par\bigskip}

% Micro-structure helpers (quiet, Bringhurst-style)
\newcommand{\glsdef}[1]{#1}
\newcommand{\glsmech}[1]{#1}
\newcommand{\glsdiag}[1]{#1}
\newcommand{\glsnote}[1]{#1}
\newcommand{\glsxsee}[1]{See also: #1}

% ===== Core framework terms =====

\newglossaryentry{hpc}{
  name={homeostatic property cluster (HPC)},
  text={HPC},
  description={\glsdef{A category whose properties cluster because mechanisms maintain their co-occurrence rather than because of a shared essence.} \glsmech{Maintained by acquisition, entrenchment, alignment, transmission, and functional pressure.} \glsdiag{Passes both projectibility and homeostasis (perturbation sensitivity).} \glsnote{Membership can be graded and internally structured.} \glsxsee{projectibility, homeostasis, stabilizer, two-diagnostic test.}}
}

\newglossaryentry{projectibility}{
  name={projectibility},
  text={projectibility},
  description={\glsdef{A category's capacity to support induction from observed cases to unobserved ones.} \glsmech{It relies on stable correlations enforced by stabilizers.} \glsdiag{Predicts held-out data or generalises across contexts better than chance.} \glsnote{Field-relative: projectible for one purpose may fail for another.} \glsxsee{homeostasis, two-diagnostic test, field-relative projectibility.}}
}

\newglossaryentry{homeostasis}{
  name={homeostasis},
  text={homeostasis},
  description={\glsdef{Active return after perturbation; stability by self-correction rather than mere persistence.} \glsmech{Maintained by alignment, entrenchment, transmission, and other stabilizers across timescales.} \glsdiag{Perturbation sensitivity: weaken a mechanism and the clustering frays.} \glsnote{The ontological anchor for genuine kinds.} \glsxsee{stabilizer, perturbation sensitivity.}}
}

\newglossaryentry{entrenchment}{
  name={entrenchment},
  text={entrenchment},
  description={\glsdef{Strengthening of mental representations through repeated use.} \glsmech{Driven by frequency effects and chunking in memory.} \glsdiag{Resistance to analogical change; faster processing and higher accessibility.} \glsnote{Anchors category cores and stabilizes prototypes.} \glsxsee{chunking, prototype.}}
}

\newglossaryentry{essentialism}{
  name={essentialism},
  text={essentialism},
  description={\glsdef{The view that categories are defined by necessary and sufficient conditions.} \glsmech{It is grounded in definitional criteria rather than stabilizers.} \glsdiag{Breaks down when exceptions proliferate without principled repair.} \glsnote{The maintenance view rejects essentialism in favour of mechanism-based kinds.} \glsxsee{nominalism, prototype.}}
}

\newglossaryentry{stabilizer}{
  name={stabilizer},
  text={stabilizer},
  description={\glsdef{A causal process that maintains clustering in a category.} \glsmech{Instances include acquisition, entrenchment, alignment, transmission, and functional pressure.} \glsdiag{Removing it changes the clustering in predictable ways.} \glsnote{Names a role, not a single type of mechanism.} \glsxsee{homeostasis, perturbation sensitivity.}}
}

\newglossaryentry{alignment}{
  name={interactive alignment},
  text={alignment},
  description={\glsdef{Convergence of interlocutors' linguistic choices during interaction.} \glsmech{Realized through accommodation and repair in real-time conversation.} \glsdiag{Local convergence and resistance to innovations under misalignment.} \glsnote{A fast-timescale stabilizer.} \glsxsee{stabilizer, transmission.}}
}

\newglossaryentry{analogy}{
  name={analogy},
  text={analogy},
  description={\glsdef{Extension of a pattern to new cases based on perceived similarity.} \glsmech{Driven by schema abstraction and entrenchment in acquisition.} \glsdiag{Analogical leveling and productivity for novel forms.} \glsnote{A stabilizer that can also reshape categories.} \glsxsee{entrenchment, transmission.}}
}

\newglossaryentry{errorandrepair}{
  name={error and repair},
  text={error and repair},
  description={\glsdef{Interactive correction of production or interpretation errors in real time.} \glsmech{Self-monitoring, interlocutor feedback, and repair sequences.} \glsdiag{Convergence after misfires; reduced divergence over repeated interactions.} \glsnote{A fast stabilizer operating at discourse timescales.} \glsxsee{alignment, stabilizer.}}
}

\newglossaryentry{processingeconomy}{
  name={processing economy},
  text={processing economy},
  description={\glsdef{Pressure to minimize cognitive and articulatory cost in production and comprehension.} \glsmech{Frequency-driven facilitation and memory constraints.} \glsdiag{Shorter or more predictable forms are preferred and stabilize.} \glsnote{Interacts with social and functional pressures.} \glsxsee{entrenchment, chunking.}}
}

\newglossaryentry{socialindexing}{
  name={social indexing},
  text={social indexing},
  description={\glsdef{Association of linguistic variants with social identities, stances, or registers.} \glsmech{Sociolinguistic signaling reinforced by community norms.} \glsdiag{Variant use correlates with social groupings and style shifts.} \glsnote{A stabilizer that can protect arbitrary variants.} \glsxsee{prestige selection, transmission.}}
}

\newglossaryentry{indexicality}{
  name={indexicality},
  text={indexicality},
  description={\glsdef{The linkage between linguistic forms and social meanings or stances.} \glsmech{Maintained by sociolinguistic association and uptake.} \glsdiag{Variants signal identity, register, or attitude.} \glsnote{Often discussed as social indexing.} \glsxsee{social indexing, prestige selection.}}
}

\newglossaryentry{creativity}{
  name={creativity},
  text={creativity},
  description={\glsdef{The production of novel forms or uses beyond memorized patterns.} \glsmech{Driven by productive schemas and analogical extension.} \glsdiag{Innovations that test and reshape category boundaries.} \glsnote{A source of variation that can become stabilized.} \glsxsee{analogy, transmission.}}
}

\newglossaryentry{standardization}{
  name={standardization},
  text={standardization},
  description={\glsdef{Institutional codification and enforcement of a linguistic norm.} \glsmech{Education, style guides, and prescriptive feedback.} \glsdiag{Reduced variation and convergence toward a prestige standard.} \glsnote{Creates explicit feedback loops outside ordinary transmission.} \glsxsee{social indexing, transmission.}}
}

\newglossaryentry{prototype}{
  name={prototype},
  text={prototype},
  description={\glsdef{The most typical member of a category.} \glsmech{Shaped by entrenchment and frequency effects.} \glsdiag{Typicality effects and graded membership.} \glsnote{Descriptive of structure; requires a mechanism story to explain stability.} \glsxsee{entrenchment, real gradience.}}
}

\newglossaryentry{maintenanceview}{
  name={maintenance view},
  text={maintenance view},
  description={\glsdef{The claim that categories are real because they are maintained by stabilizing mechanisms.} \glsmech{Supported by a braid of mechanisms across timescales.} \glsdiag{Implies the two-diagnostic test: projectibility plus homeostasis.} \glsnote{Rejects both essentialism and nominalism.} \glsxsee{hpc, two-diagnostic test.}}
}

\newglossaryentry{class}{
  name={class},
  text={class},
  description={\glsdef{A grouping used for description or convenience without a commitment to shared mechanisms.} \glsmech{No stabilizer story required.} \glsdiag{May be projectible locally but need not pass the two-diagnostic test.} \glsnote{Useful for taxonomy; not necessarily a natural kind.} \glsxsee{category, fat class, thin class, comparative concept.}}
}

\newglossaryentry{category}{
  name={category},
  text={category},
  description={\glsdef{A cluster of properties used to organize linguistic phenomena.} \glsmech{When genuine, maintained by stabilizers; when merely convenient, it is a class.} \glsdiag{A category earns kind-status only if it passes projectibility and homeostasis.} \glsnote{The book's core question is which categories are kinds.} \glsxsee{kind, two-diagnostic test, projectibility, homeostasis.}}
}

\newglossaryentry{kind}{
  name={kind},
  text={kind},
  description={\glsdef{A category that supports induction because mechanisms maintain its clustering.} \glsmech{Stabilizers keep the profile coherent across speakers and time.} \glsdiag{Passes the two-diagnostic test.} \glsnote{In this framework, linguistic kinds are HPC kinds rather than definitional essences.} \glsxsee{hpc, copied kind, two-diagnostic test.}}
}

\newglossaryentry{mechanism}{
  name={mechanism},
  text={mechanism},
  description={\glsdef{A causal process with inputs, operations, and outputs that produces or maintains a pattern.} \glsmech{Mechanisms include acquisition, alignment, entrenchment, and transmission.} \glsdiag{Identified by perturbation: weaken the process and the cluster frays.} \glsnote{The explanatory unit of the maintenance view.} \glsxsee{stabilizer, homeostasis, perturbation sensitivity.}}
}

\newglossaryentry{acquisition}{
  name={acquisition},
  text={acquisition},
  description={\glsdef{The learning process by which speakers internalise linguistic categories from input.} \glsmech{Distributional learning and cue integration across contexts.} \glsdiag{Convergence on shared categories despite variable input.} \glsnote{A core stabilizer and transmission bottleneck.} \glsxsee{transmission, entrenchment, stabilizer.}}
}

\newglossaryentry{transmission}{
  name={transmission},
  text={transmission},
  description={\glsdef{Intergenerational passage of linguistic patterns.} \glsmech{Iterated learning filters for stable, learnable variants.} \glsdiag{Convergence toward compressible systems under repeated learning.} \glsnote{A macro-stabilizer of categories and constructions.} \glsxsee{iterated transmission, acquisition.}}
}

\newglossaryentry{properfunction}{
  name={proper function},
  text={proper function},
  description={\glsdef{In Millikan's framework, the function an item has in virtue of its history of selection.} \glsmech{Secured by reproductive success of uses that perform the function.} \glsdiag{Explains why the form persists across generations.} \glsnote{Historical, not intentional or definitional.} \glsxsee{derived proper function, Normal conditions.}}
}

\newglossaryentry{derivedproperfunction}{
  name={derived proper function},
  text={derived proper function},
  description={\glsdef{A function acquired by recruitment beyond an item's original selected function.} \glsmech{Arises through exploitation of stable side effects of a form or construction.} \glsdiag{Depends on the proper function remaining intact elsewhere.} \glsnote{Parasitic rather than foundational.} \glsxsee{proper function, Normal conditions.}}
}

\newglossaryentry{deitality}{
  name={deitality},
  text={deitality},
  description={\glsdef{The morphosyntactic form cluster associated with English determiners, distinct from semantic definiteness.} \glsmech{Formed by grammaticalization of demonstratives plus distributional constraints.} \glsdiag{\mention{There}-resistance, partitive \mention{of}, and hosting behaviour converge.} \glsnote{Cross-cuts the definiteness cluster.} \glsxsee{definiteness cluster.}}
}

\newglossaryentry{fatclass}{
  name={fat class},
  text={fat class},
  description={\glsdef{A label that lumps distinct causal clusters into a single bin.} \glsmech{Produced by multiple unrelated stabilizers with no shared mechanism.} \glsdiag{Fails projectibility across subtypes.} \glsnote{Useful for pedagogy, not a natural kind.} \glsxsee{thin class, comparative concept.}}
}

\newglossaryentry{adverb}{
  name={adverb},
  text={adverb},
  description={\glsdef{A traditional residual category for modifiers that are not adjectives.} \glsmech{Maintained by multiple unrelated mechanisms across subtypes.} \glsdiag{Fails projectibility across manner, degree, and sentence adverbs.} \glsnote{A paradigmatic fat class; the remedy is decomposition.} \glsxsee{fat class.}}
}

\newglossaryentry{thinclass}{
  name={thin class},
  text={thin class},
  description={\glsdef{A pattern that is weakly maintained or not maintained at all.} \glsmech{Occurs when stabilizers are absent or too weak.} \glsdiag{Fails homeostasis; disappears under perturbation.} \glsnote{Often a byproduct of other mechanisms.} \glsxsee{fat class.}}
}

\newglossaryentry{inflationproblem}{
  name={inflation problem},
  text={inflation problem},
  description={\glsdef{The tendency to treat every stable pattern as an HPC kind.} \glsmech{Fuelled by overgeneralization from local stability or institutional reinforcement.} \glsdiag{Categories appear projectible until the homeostasis test is applied.} \glsnote{A methodological warning against false positives.} \glsxsee{two-diagnostic test, fat class, thin class.}}
}

\newglossaryentry{comparativeconcept}{
  name={comparative concept},
  text={comparative concept},
  description={\glsdef{An analyst's tool constructed for cross-linguistic comparison.} \glsmech{Not guaranteed to be supported by shared stabilizers across languages.} \glsdiag{Does not imply shared homeostasis or projectibility.} \glsnote{Useful without being a kind.} \glsxsee{field-relative projectibility.}}
}

\newglossaryentry{epistemickind}{
  name={epistemic kind},
  text={epistemic kind},
  description={\glsdef{A category that serves epistemic purposes without claiming a shared causal mechanism.} \glsmech{May be institutionally reinforced rather than mechanistically maintained.} \glsdiag{Can be stable without passing the two-diagnostic test.} \glsnote{Khalidi's term for useful but non-kind groupings.} \glsxsee{class, comparative concept.}}
}

\newglossaryentry{madagascarfallacy}{
  name={Madagascar fallacy},
  text={Madagascar fallacy},
  description={\glsdef{The mistake of treating an ecosystem as if it were a single kind.} \glsmech{It bundles mechanisms that maintain components rather than the whole.} \glsdiag{Fails homeostasis at the wrong grain.} \glsnote{A warning about scale in category analysis.} \glsxsee{two-diagnostic test.}}
}

% ===== HPC extensions and stability taxonomies (Ch 4) =====

\newglossaryentry{etiologicalkind}{
  name={etiological kind},
  text={etiological kind},
  description={\glsdef{A kind defined by its history of production or transmission.} \glsmech{Maintained by copying lineages with variation and selection.} \glsdiag{Similarity is explained by lineage rather than by shared essence.} \glsnote{Khalidi's term for history-defined natural kinds.} \glsxsee{copied kind, transmission.}}
}

\newglossaryentry{simplecausaltheory}{
  name={simple causal theory},
  text={simple causal theory},
  description={\glsdef{A view that kinds are anchored by primary properties that causally generate secondary properties.} \glsmech{One-way causal dependence rather than feedback or self-correction.} \glsdiag{Stable property dependencies without homeostatic return.} \glsnote{Contrasts with homeostatic accounts.} \glsxsee{stable property cluster, homeostasis.}}
}

\newglossaryentry{stablepropertycluster}{
  name={stable property cluster},
  text={stable property cluster},
  description={\glsdef{A cluster of co-occurring properties that remains stable over time or context.} \glsmech{Stability can arise from multiple stabilizers, not only homeostasis.} \glsdiag{Cliquish stability with projectible correlations.} \glsnote{Slater's broadened stability framework.} \glsxsee{cliquish stability, homeostasis.}}
}

\newglossaryentry{instancestability}{
  name={instance stability},
  text={instance stability},
  description={\glsdef{Stability of individual members retaining a cluster's properties over time.} \glsmech{Supported by entrenchment, selection, or institutional reinforcement.} \glsdiag{Members resist drift even under perturbation.} \glsnote{Distinct from cliquish stability.} \glsxsee{cliquish stability, entrenchment.}}
}

\newglossaryentry{symmetrycondition}{
  name={symmetry condition},
  text={symmetry condition},
  description={\glsdef{A sign-language constraint: if both hands move, they share handshape and mirror movement.} \glsmech{Motor planning and perceptual constraints on bimanual coordination.} \glsdiag{Cross-linguistic convergence in two-handed signs.} \glsnote{An example of convergent stability without borrowing.} \glsxsee{dominance condition.}}
}

\newglossaryentry{dominancecondition}{
  name={dominance condition},
  text={dominance condition},
  description={\glsdef{A sign-language constraint: if only one hand moves, the other uses an unmarked handshape.} \glsmech{Motor and perceptual constraints on bimanual articulation.} \glsdiag{Restricted non-dominant handshapes across sign languages.} \glsnote{Pairs with the symmetry condition.} \glsxsee{symmetry condition.}}
}

\newglossaryentry{species}{
  name={species},
  text={species},
  description={\glsdef{A biological grouping of populations maintained by overlapping mechanisms rather than a single essence.} \glsmech{Gene flow, shared selection pressures, and reproductive systems.} \glsdiag{Cohesion with boundary cases (ring species, hybrids).} \glsnote{A paradigm case for mechanism-based kinds.} \glsxsee{hpc, etiological kind.}}
}

% ===== Countability (Ch 9) =====

\newglossaryentry{countability}{
  name={countability},
  text={countability},
  description={\glsdef{The coupled system linking semantic individuation and morphosyntactic count marking.} \glsmech{Maintained by bidirectional inference between construal and form.} \glsdiag{Implicational hierarchy of count properties and robust projectibility.} \glsnote{An interface category with two partially distinct clusters.} \glsxsee{individuation cluster, count cluster, bidirectional inference.}}
}

\newglossaryentry{individuationcluster}{
  name={individuation cluster},
  text={individuation cluster},
  description={\glsdef{The semantic profile that enables discrete construal: boundedness, atomicity, enumerability, and homogeneity resistance.} \glsmech{Maintained by perceptual segmentation, object files, and cross-modal integration.} \glsdiag{Supports predictions about count morphosyntax.} \glsnote{Coupled to the count cluster to yield countability.} \glsxsee{count cluster, bidirectional inference.}}
}

\newglossaryentry{boundedness}{
  name={boundedness},
  text={boundedness},
  description={\glsdef{The property of having discrete edges that separate an entity from its environment.} \glsmech{Grounded in perceptual segmentation and boundary detection.} \glsdiag{Supports exact counting and telic construals.} \glsnote{One component of the individuation cluster.} \glsxsee{individuation cluster, atomicity.}}
}

\newglossaryentry{atomicity}{
  name={atomicity},
  text={atomicity},
  description={\glsdef{The availability of entities as discrete units for reference and tracking.} \glsmech{Maintained by object files and unit-based representation.} \glsdiag{Compatibility with numerals and singular reference.} \glsnote{One component of the individuation cluster.} \glsxsee{individuation cluster, boundedness.}}
}

\newglossaryentry{enumerability}{
  name={enumerability},
  text={enumerability},
  description={\glsdef{Compatibility with exact counting and cardinal quantification.} \glsmech{Depends on stable unitization in conceptual structure.} \glsdiag{Accepts numerals and exact quantifiers.} \glsnote{One component of the individuation cluster.} \glsxsee{individuation cluster, count cluster.}}
}

\newglossaryentry{homogeneityresistance}{
  name={homogeneity resistance},
  text={homogeneity resistance},
  description={\glsdef{The property that proper parts are not of the same kind as the whole.} \glsmech{Tied to construal of individuals rather than substance.} \glsdiag{Fails \enquote{half an N} tests for count nouns.} \glsnote{One component of the individuation cluster.} \glsxsee{individuation cluster, boundedness.}}
}

\newglossaryentry{countcluster}{
  name={count cluster},
  text={count cluster},
  description={\glsdef{The morphosyntactic properties that travel together in English count nouns (plural, cardinals, \textit{many}/\textit{few}, agreement).} \glsmech{Maintained by acquisition and entrenchment in count frames, plus alignment in discourse.} \glsdiag{Implicational hierarchy: tight properties imply loose ones.} \glsnote{Coupled to individuation by bidirectional inference.} \glsxsee{individuation cluster, bidirectional inference, functional anchoring.}}
}

\newglossaryentry{bidirectionalinference}{
  name={bidirectional inference},
  text={bidirectional inference},
  description={\glsdef{The mechanism coupling semantic construal to morphosyntactic form in comprehension and production.} \glsmech{Realized by inference from morphosyntax to construal and back.} \glsdiag{Mutual predictability of count properties across constructions.} \glsnote{The core homeostatic link in countability.} \glsxsee{count cluster, individuation cluster.}}
}

\newglossaryentry{functionalanchoring}{
  name={functional anchoring},
  text={functional anchoring},
  description={\glsdef{Lexical alternatives absorb pressure to regularise a quasi-count noun.} \glsmech{Implemented through division of labour in the lexicon.} \glsdiag{Intermediate classes persist without extending tight properties.} \glsnote{Explains stability of \textit{cattle}/\textit{police} and drift of \textit{data}.} \glsxsee{count cluster.}}
}

\newglossaryentry{massword}{
  name={mass-word},
  text={mass-word},
  description={\glsdef{Jespersen's term for nouns that resist counting and take mass quantifiers.} \glsmech{Maintained by mass construal and distributional patterns.} \glsdiag{Rejects numerals but accepts \textit{much}/\textit{little}.} \glsnote{Highlights grammatical packaging over ontology.} \glsxsee{countability, count cluster.}}
}

\newglossaryentry{quasicountnoun}{
  name={quasi-count noun},
  text={quasi-count noun},
  description={\glsdef{A noun that allows some count diagnostics but resists tight ones.} \glsmech{Stabilized by functional anchoring and lexeme competition.} \glsdiag{Accepts \textit{many} but resists low numerals and articles.} \glsnote{A stable intermediate in the countability system.} \glsxsee{functional anchoring, singulative.}}
}

\newglossaryentry{singulative}{
  name={singulative},
  text={singulative},
  description={\glsdef{A morphological marker deriving a single unit from a collective or mass base.} \glsmech{Grammaticalizes individuation within the count system.} \glsdiag{Enables numerals and tight count diagnostics.} \glsnote{Common in Welsh and Arabic.} \glsxsee{countability, quasi-count noun.}}
}

\newglossaryentry{classifier}{
  name={classifier},
  text={classifier},
  description={\glsdef{A morpheme that mediates numerals and nouns in classifier languages.} \glsmech{Encodes individuation or shape properties in the numeral phrase.} \glsdiag{Obligatory in counting contexts in languages like Mandarin.} \glsnote{Relocates countability coupling to the classifier system.} \glsxsee{countability, count cluster.}}
}

\newglossaryentry{chunking}{
  name={chunking},
  text={chunking},
  description={\glsdef{Storage of high-frequency sequences as units, bypassing compositional assembly.} \glsmech{Driven by frequency-based memory consolidation.} \glsdiag{Fluency and acceptability differences in entrenched frames.} \glsnote{Supports entrenchment and stability at the cluster core.} \glsxsee{entrenchment.}}
}

\newglossaryentry{basin}{
  name={basin (attractor)},
  text={basin},
  description={\glsdef{A dynamical-systems metaphor: an attractor region toward which category states are drawn.} \glsmech{Created by stabilizer-driven pull toward the cluster centre.} \glsdiag{Stable cores with tolerated peripheries.} \glsnote{Explains graded membership with sharp boundaries.} \glsxsee{dynamic discreteness, real gradience.}}
}

% ===== Definiteness (Ch 10) =====

\newglossaryentry{definiteness}{
  name={definiteness},
  text={definiteness},
  description={\glsdef{A semantic category of identifiability, familiarity, and uniqueness.} \glsmech{Maintained by discourse tracking, domain restriction, and Theory of Mind.} \glsdiag{Predicts anaphoric recoverability and discourse behaviour.} \glsnote{Distinct from the morphosyntactic form cluster.} \glsxsee{definiteness cluster, deitality, weak definite.}}
}

\newglossaryentry{definitenesscluster}{
  name={definiteness cluster},
  text={definiteness cluster},
  description={\glsdef{The semantic cluster of familiarity, uniqueness, and identifiability.} \glsmech{Maintained by discourse tracking, domain restriction, and Theory of Mind.} \glsdiag{Anaphoric recoverability and discourse behaviour.} \glsnote{Distinct from the morphosyntactic deitality cluster.} \glsxsee{deitality, Normal conditions.}}
}

\newglossaryentry{familiarity}{
  name={familiarity},
  text={familiarity},
  description={\glsdef{The property of being discourse-old or previously introduced.} \glsmech{Maintained by discourse tracking and common ground updates.} \glsdiag{Supports anaphoric uptake and given/new contrasts.} \glsnote{A core component of the definiteness cluster.} \glsxsee{definiteness cluster, anaphoric recoverability.}}
}

\newglossaryentry{uniqueness}{
  name={uniqueness},
  text={uniqueness},
  description={\glsdef{The property of having exactly one salient candidate in a domain.} \glsmech{Maintained by domain restriction and contextual narrowing.} \glsdiag{Infelicity when multiple candidates are equally salient.} \glsnote{A core component of the definiteness cluster.} \glsxsee{definiteness cluster, identifiability.}}
}

\newglossaryentry{identifiability}{
  name={identifiability},
  text={identifiability},
  description={\glsdef{The property that a hearer can pick out the intended referent.} \glsmech{Maintained by Theory of Mind and discourse modeling.} \glsdiag{First-mention definites are licensed when the description suffices for retrieval.} \glsnote{A core component of the definiteness cluster.} \glsxsee{definiteness cluster, familiarity.}}
}

\newglossaryentry{anaphoricrecoverability}{
  name={anaphoric recoverability},
  text={anaphoric recoverability},
  description={\glsdef{The capacity of a referent to be taken up by pronouns or anaphora.} \glsmech{Maintained by discourse coherence and tracking mechanisms.} \glsdiag{Natural antecedent status in subsequent discourse.} \glsnote{A downstream affordance of definiteness.} \glsxsee{familiarity, definiteness cluster.}}
}

\newglossaryentry{prestigeselection}{
  name={prestige selection},
  text={prestige selection},
  description={\glsdef{Spread of variants due to association with high-status speakers.} \glsmech{Social evaluation and identity signaling reinforce the variant.} \glsdiag{Prestige forms persist despite competing regularization pressures.} \glsnote{A social stabilizer for arbitrary variants.} \glsxsee{social indexing, transmission.}}
}

\newglossaryentry{formcluster}{
  name={form cluster},
  text={form cluster},
  description={\glsdef{The morphosyntactic bundle associated with English determiners (\mention{there}-resistance, partitive \mention{of}, hosting).} \glsmech{Grammaticalization of demonstratives and stabilized distributional constraints.} \glsdiag{Convergence of structural diagnostics across determiners.} \glsnote{Also called the deitality cluster.} \glsxsee{deitality, definiteness.}}
}

\newglossaryentry{weakdefinite}{
  name={weak definite},
  text={weak definite},
  description={\glsdef{A definite article use that lacks a unique or familiar referent (institutional/role frames).} \glsmech{A derived function exploiting stable form-cluster properties.} \glsdiag{Form-cluster behaviour without definiteness-cluster satisfaction.} \glsnote{Productive in stereotyped activity frames.} \glsxsee{deitality, definiteness, form cluster.}}
}

\newglossaryentry{normalconditions}{
  name={Normal conditions},
  text={Normal conditions},
  description={\glsdef{The circumstances under which a device performs its proper function.} \glsmech{Sustained by shared context and recoverable referents.} \glsdiag{Breakdown produces miscoordination; derived uses exploit stable side effects.} \glsnote{Frames proper vs. derived function.} \glsxsee{proper function, derived proper function.}}
}

% ===== General framework =====

\newglossaryentry{fieldrelativeprojectibility}{
  name={field-relative projectibility},
  text={field-relative projectibility},
  description={\glsdef{Projectibility indexed to analytic purpose or domain.} \glsmech{Depends on task-specific stabilizers.} \glsdiag{A category may project for one purpose and fail for another.} \glsnote{Explains cross-cutting categories.} \glsxsee{projectibility, crosscutting kinds.}}
}

\newglossaryentry{twodiagnostictest}{
  name={two-diagnostic test},
  text={two-diagnostic test},
  description={\glsdef{The criterion for genuine kinds: projectibility plus homeostasis.} \glsmech{Implemented by pairing the projectibility and homeostasis diagnostics.} \glsdiag{Pass both to count as an HPC; fail one to diagnose thin/fat/negative.} \glsnote{Operational rather than definitional.} \glsxsee{projectibility, homeostasis, perturbation sensitivity.}}
}

\newglossaryentry{noun}{
  name={noun},
  text={noun},
  description={\glsdef{A lexical category associated with nominals and argument structure.} \glsmech{Maintained by distributional cues, morphology, and acquisition biases.} \glsdiag{Cliquish stability across multiple diagnostics (determiners, number, agreement).} \glsnote{Cross-linguistic comparability requires mechanism mapping.} \glsxsee{verb, subject, copied kind.}}
}

\newglossaryentry{propernoun}{
  name={proper noun},
  text={proper noun},
  description={\glsdef{A syntactic category of nominals with distinctive distributional behaviour.} \glsmech{Maintained by morphosyntactic cues such as article resistance and agreement patterns.} \glsdiag{Distributional diagnostics cluster in specific constructions.} \glsnote{Distinct from the semantic category of proper names.} \glsxsee{proper name, noun, field-relative projectibility.}}
}

\newglossaryentry{propername}{
  name={proper name},
  text={proper name},
  description={\glsdef{A semantic category of expressions that directly refer to individuals.} \glsmech{Maintained by discourse tracking and stable reference to individuals.} \glsdiag{Rigid designation and referential opacity effects.} \glsnote{Cross-cuts the syntactic category of proper nouns.} \glsxsee{proper noun, definiteness cluster, field-relative projectibility.}}
}

\newglossaryentry{verb}{
  name={verb},
  text={verb},
  description={\glsdef{A lexical category associated with predication and event structure.} \glsmech{Maintained by distributional frames, inflectional paradigms, and acquisition.} \glsdiag{Convergent diagnostics (tense/aspect marking, argument structure).} \glsnote{Stability varies by language and construction.} \glsxsee{noun, aspect.}}
}

\newglossaryentry{adjective}{
  name={adjective},
  text={adjective},
  description={\glsdef{A lexical category associated with property attribution.} \glsmech{Maintained by distributional position, morphology, and modification patterns.} \glsdiag{Compatibility with attributive and predicative positions.} \glsnote{Cross-linguistic status is often comparative rather than universal.} \glsxsee{noun, verb, comparative concept.}}
}

\newglossaryentry{auxiliary}{
  name={auxiliary},
  text={auxiliary},
  description={\glsdef{A verbal category that contributes tense, aspect, or modality without full lexical content.} \glsmech{Typically arises through grammaticalization and distributional entrenchment.} \glsdiag{Inversion, \mention{do}-support, or restricted non-finite behaviour.} \glsnote{Boundary cases are common in diachronic change.} \glsxsee{verb, grammaticalization.}}
}

\newglossaryentry{subject}{
  name={subject},
  text={subject},
  description={\glsdef{A syntactic function cluster defined by converging behavioural and coding properties.} \glsmech{Agreement, word order, case marking, and discourse roles.} \glsdiag{Convergence of multiple diagnostics rather than a single definition.} \glsnote{Cross-linguistic identity is not guaranteed.} \glsxsee{comparative concept, category.}}
}

\newglossaryentry{voice}{
  name={voice},
  text={voice},
  description={\glsdef{A morphosyntactic system that maps argument roles to grammatical functions.} \glsmech{Maintained by constructional alternations and discourse pressures.} \glsdiag{Voice morphology and alternations in argument realization.} \glsnote{Cross-linguistic mechanisms vary widely.} \glsxsee{subject, category.}}
}

\newglossaryentry{anaphor}{
  name={anaphor},
  text={anaphor},
  description={\glsdef{An expression that requires a local antecedent for interpretation.} \glsmech{Maintained by binding constraints and acquisition of dependency patterns.} \glsdiag{Obligatory local binding in its domain.} \glsnote{A classic category in binding theory.} \glsxsee{subject, category.}}
}

\newglossaryentry{interrogativephrase}{
  name={interrogative phrase},
  text={interrogative phrase},
  description={\glsdef{A phrase that introduces a question variable (e.g., wh-phrases).} \glsmech{Licensed by interrogative constructions and dependency formation.} \glsdiag{Participates in wh-dependencies and question formation.} \glsnote{Structural diagnostics vary across languages.} \glsxsee{bounding node, category.}}
}

\newglossaryentry{boundingnode}{
  name={bounding node},
  text={bounding node},
  description={\glsdef{A structural node posited to delimit extraction domains in early generative syntax.} \glsmech{Constrains long-distance dependencies by structural configuration.} \glsdiag{Island effects and extraction failures beyond the node.} \glsnote{A theory-internal construct rather than a mechanism.} \glsxsee{interrogative phrase, anaphor.}}
}

\newglossaryentry{canonicalclause}{
  name={canonical clause},
  text={canonical clause},
  description={\glsdef{An idealized benchmark clause used for descriptive comparison.} \glsmech{A regimentation tool rather than a maintained kind.} \glsdiag{Core diagnostics converge most cleanly on canonical cases.} \glsnote{Useful for exposition, not ontologically privileged.} \glsxsee{class, comparative concept.}}
}

\newglossaryentry{animacy}{
  name={animacy},
  text={animacy},
  description={\glsdef{A semantic distinction between animate and inanimate entities.} \glsmech{Grounded in cognitive salience and agency detection.} \glsdiag{Grammatical contrasts like \textit{who} vs.\ \textit{what}.} \glsnote{Often cross-cuts lexical similarity classes.} \glsxsee{personhood cluster.}}
}

\newglossaryentry{factivity}{
  name={factivity},
  text={factivity},
  description={\glsdef{The property of predicates that presuppose the truth of their complements.} \glsmech{Lexical semantics and pragmatic inference patterns.} \glsdiag{Presupposition survives negation and questions.} \glsnote{An inferential grouping that cross-cuts similarity.} \glsxsee{category.}}
}

\newglossaryentry{copiedkind}{
  name={copied kind},
  text={copied kind},
  description={\glsdef{A category whose members are produced from each other or a common template.} \glsmech{Maintained by a transmission lineage with variation.} \glsdiag{Similarity explained by copying rather than essences.} \glsnote{Grammatical categories are copied kinds par excellence.} \glsxsee{replicator, interactor.}}
}

\newglossaryentry{unicept}{
  name={unicept},
  text={unicept},
  description={\glsdef{A concept that tracks sameness through multiple fallible methods (Millikan).} \glsmech{Sustained by convergent cues with no privileged diagnostic.} \glsdiag{Cliquish stability: some cues reliably indicate the cluster.} \glsnote{Matches multi-diagnostic category tracking.} \glsxsee{cliquish stability, two-diagnostic test.}}
}

\newglossaryentry{iteratedtransmission}{
  name={iterated transmission},
  text={iterated transmission},
  description={\glsdef{Structure emerging from multi-generational learning bottlenecks.} \glsmech{Produced by filtering for learnable variants.} \glsdiag{Convergence toward compressible/compositional systems.} \glsnote{A macro-stabilizer of linguistic structure.} \glsxsee{copied kind, entrenchment.}}
}

\newglossaryentry{perturbationsensitivity}{
  name={perturbation sensitivity},
  text={perturbation sensitivity},
  description={\glsdef{The diagnostic that a category frays when a stabilizer is weakened.} \glsmech{Observed when interventions target acquisition, alignment, or transmission.} \glsdiag{Predictable degradation patterns under perturbation.} \glsnote{Operationalizes homeostasis.} \glsxsee{homeostasis, stabilizer.}}
}

\newglossaryentry{cliquishstability}{
  name={cliquish stability},
  text={cliquish stability},
  description={\glsdef{Property-correlation: some cues reliably indicate the whole cluster.} \glsmech{Maintained by stabilizers that keep correlations intact despite drift.} \glsdiag{Knowing one property predicts others.} \glsnote{Supports induction without perfect instance stability.} \glsxsee{projectibility.}}
}

% ===== Discreteness and gradience (Ch 5) =====

\newglossaryentry{dynamicdiscreteness}{
  name={dynamic discreteness},
  text={dynamic discreteness},
  description={\glsdef{Sharp boundaries produced by mechanisms rather than definitions.} \glsmech{Maintained by stabilizers that keep tolerance thresholds over time.} \glsdiag{Sharp judgments within speakers, variance across populations.} \glsnote{Explains discreteness without essences.} \glsxsee{relative tolerance, real gradience.}}
}

\newglossaryentry{realgradience}{
  name={real gradience},
  text={real gradience},
  description={\glsdef{Gradient structure as meaningful distance from category centres.} \glsmech{Shaped by variable stabilizer strength across the basin.} \glsdiag{Typicality and marginality effects.} \glsnote{Gradience is signal, not noise.} \glsxsee{basin, dynamic discreteness.}}
}

\newglossaryentry{relativetolerance}{
  name={relative tolerance},
  text={relative tolerance},
  description={\glsdef{Membership depends on scale-relative tolerance of change.} \glsmech{Determined by decision thresholds indexed to context and magnitude.} \glsdiag{Small changes acceptable at one scale but not another.} \glsnote{Explains tolerance intuitions at category edges.} \glsxsee{dynamic discreteness.}}
}

% ===== Projectibility mechanisms (Ch 6) =====

\newglossaryentry{mechanisticdrift}{
  name={mechanistic drift},
  text={mechanistic drift},
  description={\glsdef{Mechanisms that maintain a category can differ from those that originally produced it.} \glsmech{Driven by diachronic reconfiguration of stabilizers.} \glsdiag{Mismatch between current cues and historical rationale.} \glsnote{Explains semantic drift under stable labels.} \glsxsee{grammaticalization.}}
}

\newglossaryentry{aspect}{
  name={aspect},
  text={aspect},
  description={\glsdef{A grammatical category describing how events are temporally and structurally construed.} \glsmech{Maintained by distributional cue structures and morphological marking.} \glsdiag{Projectible patterns may diverge from textbook semantic definitions.} \glsnote{A comparative concept whose language-specific mechanisms can drift.} \glsxsee{projectibility, mechanistic drift.}}
}

\newglossaryentry{npi}{
  name={negative polarity item (NPI)},
  text={NPI},
  description={\glsdef{An expression whose distribution is restricted to negative or downward-entailing contexts.} \glsmech{Multiple licensing mechanisms yield heterogeneous subtypes.} \glsdiag{Restricted distribution with distinct licensing profiles.} \glsnote{Often a distributional class rather than a single kind.} \glsxsee{class, category.}}
}

\newglossaryentry{perfective}{
  name={perfective},
  text={perfective},
  description={\glsdef{An aspectual category presenting events as bounded or completed.} \glsmech{Maintained by morphological marking and cue–outcome associations.} \glsdiag{Distributional asymmetries across tense frames.} \glsnote{Highlights the gap between semantic definition and usage.} \glsxsee{aspect, boundedness.}}
}

\newglossaryentry{grueproblem}{
  name={grue problem},
  text={grue problem},
  description={\glsdef{Goodman's riddle: why do we project \enquote{green} but not \enquote{grue}?} \glsmech{Explained by the fact that only mechanism-grounded properties are projectible.} \glsdiag{Gerrymandered predicates fail induction.} \glsnote{Applied to language, pseudo-categories lack mechanisms.} \glsxsee{projectibility.}}
}

\newglossaryentry{crosscuttingkinds}{
  name={crosscutting kinds},
  text={crosscutting kinds},
  description={\glsdef{Overlapping causal networks where the same entity belongs to multiple kinds.} \glsmech{Maintained by distinct stabilizers for each network.} \glsdiag{Different inductions for different purposes.} \glsnote{Cross-cutting is expected under field-relative projectibility.} \glsxsee{field-relative projectibility.}}
}

% ===== Essentialism and its alternatives (Ch 1–3) =====

\newglossaryentry{nominalism}{
  name={nominalism},
  text={nominalism},
  description={\glsdef{The view that categories are convenient labels without commitment to natural kinds.} \glsmech{Requires no stabilizer story.} \glsdiag{No homeostasis or projectibility demanded.} \glsnote{Rejected by the maintenance view.} \glsxsee{essentialism, comparative concept.}}
}

\newglossaryentry{grammaticalization}{
  name={grammaticalization},
  text={grammaticalization},
  description={\glsdef{Diachronic shift from lexical item to grammatical marker.} \glsmech{Driven by frequency, reduction, reanalysis, and semantic bleaching.} \glsdiag{Layering and distributional expansion over time.} \glsnote{Creates form–meaning decouplings that later stabilize.} \glsxsee{mechanistic drift.}}
}

\newglossaryentry{replicator}{
  name={replicator},
  text={replicator},
  description={\glsdef{An entity that passes copies of itself through time with variation (Hull).} \glsmech{Sustained by transmission with differential survival.} \glsdiag{Selection on variants across generations.} \glsnote{Supports evolutionary framing of language change.} \glsxsee{interactor, copied kind.}}
}

\newglossaryentry{interactor}{
  name={interactor},
  text={interactor},
  description={\glsdef{An entity that interacts with its environment in ways that cause differential replication (Hull).} \glsmech{Realized in communicative success and failure.} \glsdiag{Selection pressure visible in usage patterns.} \glsnote{Pairs with replicator in an evolutionary account.} \glsxsee{replicator.}}
}

\newglossaryentry{verblessclause}{
  name={verbless clause},
  text={verbless clause},
  description={\glsdef{A clause-like construction with subject–predicate structure but no verb in the predicate.} \glsmech{Arises under competing diagnostics (predication vs. VP-headedness).} \glsdiag{Shows decoupling of form and function.} \glsnote{Illustrates pressure against essentialist definitions.} \glsxsee{two-diagnostic test.}}
}

\newglossaryentry{nonfiniteclause}{
  name={non-finite clause},
  text={non-finite clause},
  description={\glsdef{A residual category for clauses lacking primary tense inflection.} \glsmech{No single shared mechanism maintains the set.} \glsdiag{Fails both projectibility and homeostasis across subtypes.} \glsnote{A negative class; best replaced by positive constructions.} \glsxsee{fat class, category.}}
}

\newglossaryentry{opentexture}{
  name={open texture},
  text={open texture},
  description={\glsdef{Hart's term for the penumbral region where rules underdetermine outcomes.} \glsmech{Emerges when stabilizers are weak or in competition.} \glsdiag{Persistent variability and institutional repair.} \glsnote{Grammar has no Supreme Court; resolution is emergent.} \glsxsee{dynamic discreteness, mechanistic drift.}}
}

% ===== Information structure and lexical categories (Ch 11) =====

\newglossaryentry{bci}{
  name={Backgrounded Constituent Infelicity},
  text={Backgrounded Constituent Infelicity},
  description={\glsdef{The constraint that backgrounded constituents resist focus-sensitive operations.} \glsmech{Information-structural requirements on focus particles.} \glsdiag{Infelicity when a focus operator targets backgrounded material.} \glsnote{A diagnostic from focus-particle research.} \glsxsee{category.}}
}

% ===== Gender and pro-forms (Ch 12) =====

\newglossaryentry{gender}{
  name={gender},
  text={gender},
  description={\glsdef{A system of grammatically relevant contrasts over referents or nouns.} \glsmech{Maintained by agreement systems or pro-form inventories.} \glsdiag{Consistent selection of gender-marked items in anaphora or agreement.} \glsnote{Can be referential or noun-class based.} \glsxsee{personhood cluster, pro-form inventory.}}
}

\newglossaryentry{epicene}{
  name={epicene},
  text={epicene},
  description={\glsdef{A gender category unmarked for sex within the personal domain.} \glsmech{Maintained by personhood-based construal rather than sex marking.} \glsdiag{Compatible with any sexed referent.} \glsnote{Common in pronominal systems.} \glsxsee{gender, personhood cluster.}}
}

\newglossaryentry{designatum}{
  name={designatum},
  text={designatum},
  description={\glsdef{The entity as conceptualized by the speaker in context.} \glsmech{Constraining inference from meaning to form in reference.} \glsdiag{Pro-form choice tracks construal over antecedent form.} \glsnote{Distinct from antecedent and referent.} \glsxsee{gender, pro-form inventory.}}
}

\newglossaryentry{semantictransparency}{
  name={semantic transparency},
  text={semantic transparency},
  description={\glsdef{The predictability of form–meaning mappings in a system.} \glsmech{Stabilized by consistent coupling between semantic and morphosyntactic cues.} \glsdiag{Low mismatch rates and easier acquisition.} \glsnote{Supports tight coupling between clusters.} \glsxsee{pro-form inventory, countability.}}
}

\newglossaryentry{personhoodcluster}{
  name={personhood cluster},
  text={personhood cluster},
  description={\glsdef{The semantic properties that make a referent construable as a person.} \glsmech{Grounded in Theory of Mind and social cognition.} \glsdiag{Triggers personal pro-forms (\textit{who}, \textit{he/she}) over non-personal ones.} \glsnote{A semantic anchor for gender systems.} \glsxsee{gender, animacy.}}
}

\newglossaryentry{proforminventory}{
  name={pro-form inventory},
  text={pro-form inventory},
  description={\glsdef{The set of pro-forms and their distributional constraints in a language.} \glsmech{Maintained by acquisition and usage-based entrenchment.} \glsdiag{Stable patterns in wh-forms and pronominal selection.} \glsnote{A lexico-grammatical cluster in pro-form gender.} \glsxsee{gender, designatum.}}
}
